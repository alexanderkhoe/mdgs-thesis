% 
% Template Laporan Skripsi/Thesis 
%
% @author  Andreas Febrian, Lia Sadita 
% @version 1.03
%
% Dokumen ini dibuat berdasarkan standar IEEE dalam membuat class untuk 
% LaTeX dan konfigurasi LaTeX yang digunakan Fahrurrozi Rahman ketika 
% membuat laporan skripsi. Konfigurasi yang lama telah disesuaikan dengan 
% aturan penulisan thesis yang dikeluarkan UI pada tahun 2008.
%
           
% Tipe dokumen adalah report dengan satu kolom. 
%   
\documentclass[12pt, a4paper, onecolumn, oneside, final]{report}
   
% Load konfigurasi LaTeX untuk tipe laporan thesis
\usepackage{uithesis}

% Load patch tambahan 
\usepackage{patch_madeagus}
 
\newglossaryentry{palpitasi}{
	name=palpitasi,
	description={Sensasi yang tidak menyenangkan dari denyut jantung yang kuat dan tidak teratur}
}

\newglossaryentry{heartrate}{
	name=\emph{heart rate},
	description={laju denyut jantung}
}

\newglossaryentry{arrhytmia}{
	name=\emph{arrhytmia},
	description={kelaian/gangguan urutan
irama jantung atau gangguan kecepatan dari proses depolarisasi, repolarisasi
atau keduanya pada jantung}
}

\newglossaryentry{ecgG}{
	name=Electrocardiogram,
	description={alat yang digunakan untuk mendiagnosa }
}

\newglossaryentry{metric}{
	name=metric,
	description={atau disebut juga fungsi jarak, merupakan fungsi yang digunakan
	untuk menentukan jarak antara elemen dalam himpunan} }
\newacronym{lvq}{LVQ}	{ Learning Vector Quantization }
\newacronym{glvq}{GLVQ}	{ Generelized Learning Vector Quantization }
\newacronym{ecg}{ECG}	{ Electrocardiogramm }
\newacronym{bwr}{BWR}	{ Baseline Wander Removal }
\newacronym{wfdb}{WFDB}	{ WaveForm DataBase }
\newacronym{fnlvq}{FNLVQ}	{ Fuzzy-Neuro Learning Vector Quantization }

 
% Load konfigurasi khusus untuk laporan yang sedang dibuat
% -----------------------------------------------------------------------------%
% Informasi Mengenai Dokumen
% -----------------------------------------------------------------------------%
%  Judul laporan.
\var{\judul}{Metode klasifikasi berdasarkan Prototype untuk Sistem Pengenalan
Signal EKG}

%  Tulis kembali judul laporan, kali ini akan diubah menjadi huruf kapital
\Var{\Judul}{Metode klasifikasi berdasarkan Prototype untuk Sistem Pengenalan
Signal EKG}
 
%  Tulis kembali judul laporan namun dengan bahasa Ingris
\var{\judulInggris}{Prototype base Classification for ECG Signal Recognition}

% 
% Tipe laporan, dapat berisi Skripsi, Tugas Akhir, Thesis, atau Disertasi
\var{\type}{Tugas Akhir}
% 
% Tulis kembali tipe laporan, kali ini akan diubah menjadi huruf kapital
\Var{\Type}{TUGAS AKHIR}
% 
% Tulis nama penulis 
\var{\penulis}{I Made Agus Setiawan}
% 
% Tulis kembali nama penulis, kali ini akan diubah menjadi huruf kapital
\Var{\Penulis}{I MADE AGUS SETIAWAN}
% 
% Tulis NPM penulis
\var{\npm}{0906503793}
% 
% Tuliskan Fakultas dimana penulis berada
\Var{\Fakultas}{ILMU KOMPUTER}
\var{\fakultas}{Ilmu Komputer}
% 
% Tuliskan Program Studi yang diambil penulis
\Var{\Program}{MAGISTER ILMU KOMPUTER}
\var{\program}{Magister Ilmu Komputer}
% 
% Tuliskan tahun publikasi laporan
\Var{\bulanTahun}{Juni 2011}
% 
% Tuliskan gelar yang akan diperoleh dengan menyerahkan laporan ini
\var{\gelar}{Magister Ilmu Komputer}
% 
% Tuliskan tanggal pengesahan laporan, waktu dimana laporan diserahkan ke 
% penguji/sekretariat
\var{\tanggalPengesahan}{20 Juni 2011} 
% 
% Tuliskan tanggal keputusan sidang dikeluarkan dan penulis dinyatakan 
% lulus/tidak lulus
\var{\tanggalLulus}{20 Juni 2011}
% 
% Tuliskan pembimbing 
\var{\pembimbing}{Dr. Wisnu Jatmiko, M.Kom}
% 
% Alias untuk memudahkan alur penulisan paa saat menulis laporan
\var{\saya}{Penulis }

%-----------------------------------------------------------------------------%
% Judul Setiap Bab
%-----------------------------------------------------------------------------%
% 
% Berikut ada judul-judul setiap bab. 
% Silahkan diubah sesuai dengan kebutuhan. 
% 
\Var{\kataPengantar}{Kata Pengantar}
\Var{\babSatu}{Pendahuluan}
\Var{\babDua}{Tinjauan Pustaka}
\Var{\babTiga}{Pengolahan Data}
\Var{\babEmpat}{Rancangan dan Implementasi}
\Var{\babLima}{Analisis dan Pembahasan}
\Var{\kesimpulan}{Kesimpulan dan Saran}

 
% Daftar pemenggalan suku kata dan istilah dalam LaTeX
\include{hype_indonesia}
      
% Daftar istilah yang mungkin perlu ditandai 
%
% @author  Andreas Febrian
% @version 1.00
% 
% Mendaftar seluruh istilah yang mungkin akan perlu dijadikan 
% italic atau bold pada setiap kemunculannya dalam dokumen. 
%  

\varui{\license}{\f{Creative Common License 1.0 Generic}}
\varui{\bslash}{$\setminus$}
      
% Daftar wise quote
\newcommand{\quotepage}[2]{
	\chapter*{} 
	\thispagestyle{empty}
	\addtocounter{page}{-1}
	\vspace{5.5in}
	\begin{flushright}
	``\textit{#1}'' \\
	{\tiny -- #2}	
	\end{flushright}
	\clearpage
% 	\cleardoublepage
% 	\addtocounter{page}{-1}
% 	\thispagestyle{empty}
}


\varui{\QEinstein}{
	\quotepage{If we knew what we were doing, it wouldn't be research, would 
	it?}{Albert Einstein}}

   
% Awal bagian penulisan laporan
\begin{document}

\pagestyle{empty}

% Sampul Laporan
\include{sampul}

% load halaman judul dalam
\addChapter{HALAMAN JUDUL} 
\include{judul_dalam}

\pagestyle{fancy}

% Gunakan penomeran romawi
\pagenumbering{roman}

% setelah bagian ini, halaman dihitung sebagai halaman ke 2
\setcounter{page}{2}

%  load halaman pengesahan
\addChapter{LEMBAR PERSETUJUAN}
\include{pengesahan}

%  load halaman orisinalitas
\addChapter{LEMBAR PERNYATAAN ORISINALITAS}
\include{orisinal}
\addChapter{LEMBAR PENGESAHAN}
\include{pengesahan_sidang}
\addChapter{KATA PENGANTAR}
%-----------------------------------------------------------------------------%
\chapter*{\kataPengantar}
%-----------------------------------------------------------------------------%

\vspace*{0.1cm}
\begin{flushright}
Depok, 1 Juli 2011\\[0.1cm]
\vspace*{1cm}
\penulis

\end{flushright}
\addChapter{LEMBAR PERSETUJUAN PUBLIKASI ILMIAH}
\include{persetujuan_publikasi}
\addChapter{ABSTRAK}
%
% Halaman Abstrak
%
% @author  I Made Agus Setiawan
% @version 1.00
%

\chapter*{Abstrak}

\vspace*{0.2cm}

\noindent \begin{tabular}{l l p{10cm}}
	Nama&: & \penulis \\
	Program Studi&: & \programstudi \\
	Judul&: & \judul \\
\end{tabular} \\ 

\vspace*{0.5cm}

\noindent

Aritmia atau cardiac Aritmia merupakan salah satu penyakit jantung yang dapat
didiagnosa menggunakan standar ECG. dengan menggunakan ECG, para dokter dapat
menganalisis aktifitas elektrik jantung dan menentukan tipe dari Aritmia yang
diderita oleh pasien. 
Pada penelitian ini, proses pengenalan Aritmia dilakukan secara
otomatis menggunakan pendekatan jaringan saraf tiruan. Proses ini dibagi menjadi
tiga tahap yaitu; pemrosesan data, ekstraksi fitur dan proses pengenalan
oleh jaringan saraf. Pada proses pengolahan data awal, sinyal ECG disegmentasi
menjadi satuan beat dengan menggunakan puncak gelombang R sebagai pivot, dan
dilanjutkan dengan proses baseline wander removal dan outlier removal.
Transformasi Wavelet kemudian dilakukan untuk mengekstraksi dan mereduksi fitur.
Setiap beat tunggal kemudian diklasifikasi menjadi 6 dan 12 kelas menggunakan
metode baru yang dikembangkan disebut Fuzzy-Neuro Learning Vector
Quantization (FNGLVQ) yang merupakan adaptasi metode Fuzzy-Neuro kedalam GLVQ
yang dikembangkan oleh A.Sato. 

Hasil dari penelitian ini menunjukkan bahwa rata-rata tingkat pengenalan beat
Aritmia 6 kelas menggunakan metode FNGLVQ sebesar 98.53\% dan untuk 12 kelas
sebesar 96.33\% dimana metode yang dikembangkan memberikan hasil yang lebih
baik daripada GLVQ sebesar 97.03\% dan 94.13\% untuk 6 kelas dan 12 kelas.
Disamping itu, FNGLVQ memberikan hasil yang lebih baik untuk data yang tidak
seimbang dengan nilai rata-rata \emph{recall} mencapai 86.23\% lebih baik dari
GLVQ sebesar 82.12\%.
\\

\vspace*{0.2cm}

\noindent Kata Kunci: \\ 
\noindent Fuzzy-Neuro Generalized Learning Vector Quantization, FNGLVQ,
GLVQ, sistem pengenalan beat Aritmia\\

\newpage
%
% Halaman Abstract
%
% @author  I Made Agus Setiawan
% @version 1.00
%

\chapter*{ABSTRACT}

\vspace*{0.2cm}

\noindent \begin{tabular}{l l p{11.0cm}}
	Name&: & \penulis \\
	Program&: & \programstudi \\
	Title&: & \judulInggris \\
\end{tabular} \\ 

\vspace*{0.5cm}

\noindent
Arrhythmia or Cardiac Arrhythmia is one of heart
disease type that can be diagnosed by a standard electrocardiogram (ECG). By
means of an electrocardiogram, doctors can analyze the electrical activity of
the heart and determine the type of arrhythmia currently suffered. Computerized
process was divided into three steps: data preprocessing, feature extraction and
classification. In preprocessing step, beat by beat signal was segmented using
pivot R peak followed by baseline wander and outlier removal . Wavelet algorithm
was applied for feature extraction and selection. ECG signal is then classified
into 6 and 12 classes using new classification method namely Fuzzy-Neuro
Learning Vector Quantization (FNGLVQ) which is an adaptation on Fuzzy-Neuro
method into GLVQ developed by A.Sato.

Result of this study indicate that the average recognition rate (accuracy) for
arrhytmia beat using FNGLVQ method produce 98.53\% and 96.33\% for 6 and 12
classes. This method produce better result than GLVQ where the recognition rate
are 97.03\% and 94.13\%.  In addition, FNGLVQ gives better results for
unbalance data with the average recall reach 86.23\% better than GLVQ,
82.12\%.
 
\\

\vspace*{0.2cm}

\noindent Keywords: \\ 
\noindent Fuzzy-Neuro Generalized Learning Vector Quantization, FNGLVQ,
GLVQ, Arrhytmia beat detection system\\

\newpage

%
% Daftar isi, gambar, dan tabel
%

\tableofcontents
\clearpage
\listoffigures
\clearpage
\listoftables
\clearpage 

%
% Gunakan penomeran Arab (1, 2, 3, ...) setelah bagian ini.
%
\pagenumbering{arabic}

%-----------------------------------------------------------------------------%
\QEinstein
\chapter{\babSatu}
\label{bab:pendahuluan}
%-----------------------------------------------------------------------------%

%-----------------------------------------------------------------------------%
\section{Latar Belakang}
%-----------------------------------------------------------------------------%
Kesehatan merupakan salah satu bidang sentral yang menjadi fokus dalam 
pembangunan. Tingginya angka penyakit menular seperti DHF dan TBC masih 
menjadi permasalahan, ditambah lagi dengan pergeseran pola hidup yang tidak 
seimbang yang menyebabkan penyakit degeneratif semakin meningkat.

% Sesuai Survei Kesehatan Rumah Tangga Departemen Kesehatan, 2001 proporsi 
% kematian akibat Penyakit Tidak Menular  meningkat  dari 25.41\% di tahun 
% 1990  menjadi 48.53\% di tahun 2001. Sedangkan proporsi kematian karena 
% Penyakit Jantung dan Pembuluh Darah meningkat dari 9.1\% di tahun 1986 
% meningkat menjadi 26.3\% di tahun 2001 
% \footnote{\url{http://id.inaheart.or.id/}}. 
Berdasarkan laporan dari Badan Kesehatan Dunia - WHO, proporsi  kematian akibat
dari penyakit jantung pada tahun 2002 mencapai 29\% dan  menjadi penyebab
kematian no 1 di Indonesia, diatas penyakit menular yang menempati  posisi ke-2,
seperti yang diperlihatkan pada \pic~\ref{fig:cardiodeath}

\addFigure{width=1\textwidth}{pics/cardiodeath.jpg}{fig:cardiodeath}{Proporsi
Beberapa penyebab kematian di Indonesia 2002-WHO.}

Perkembangan teknologi kedokteran dan kesehatan yang digunakan dalam tindakan 
medis saat ini, berkaitan dengan kemajuan ilmu kedokteran itu sendiri yang 
didukung oleh beberapa ilmu-ilmu lainnya seperti; ilmu pengetahuan alam, ilmu
rekayasa serta kemajuan teknologi informasi. Pengembangan perangkat teknologi
informasi, baik perangkat keras maupun perangkat lunak, dapat digunakan untuk
membantu proses penanganan kesehatan yang dilakukan oleh paramedis dengan lebih
baik. Contohnya, deteksi penyakit kanker, deteksi penyakit paru, analisa dan
visualisasi detak jantung dan lain-lain. Dengan pemanfaatan teknologi informasi
ini, penanganan penyakit dapat dilakukan dengan mudah, cepat dan akurat.

Alat yang umum digunakan oleh para pakar dalam mendiagnosa kinerja jantung
adalah electrocardiogram (ECG). Alat ini akan merekam aktifitas elektrik dari
jantung dan merepresentasikannya dalam bentuk gelombang, atau biasa
diistilahkan gelombang ECG. Tidak sembarang orang dapat menerjemahkan gelombang
ECG tersebut, bahkan antara pakar sendiri terkadang terjadi perbedaan
interpretasi. Disamping itu, menurut Perhimpunan Dokter Spesialis
Kardiovaskular Indonesia (PERKI)\footnote{\url{http://www.inaheart.org}}, jumlah
ahli jantung di Indonesia sekitar 452 orang. Sangat sedikit sekali jika 
dibandingkan dengan jumlah penduduk yang berjumlah sekitar 228 juta jiwa, 
atau setiap dokter menangani 550 ribu jiwa
(1:550.000), dan itupun tidak menyebar secara merata, masih terfokus di
kota-kota besar seperti Jakarta. 

Oleh karena itu, sangat dibutuhkan suatu mekanisme untuk membantu proses
pendeteksian awal penyakit jantung yang mudah dan dapat dilakukan tidak hanya
oleh pakar, namun oleh siapa saja, bahkan pasien sekalipun. Salah satu
mekanisme yang dapat dikembangkan adalah suatu sistem cerdas pendeteksi gejala
penyakit jantung secara otomatis berdasarkan gelombang ECG. Dengan sistem ini,
gelombang ECG dapat diinterpretasikan oleh suatu perangkat lunak, yang
sebelumnya sudah dilatih untuk mengenal berbagai macam kemungkinan penyakit
jantung, sehingga proses interpretasi awal dapat dilakukan lebih mudah, lebih
cepat dan lebih akurat. 

Terdapat beberapa tipe penyakit jantung, salah satunya adalah Aritmia
(\emph{Arrhytmia/Cardiac Arrhytmia}). Aritmia adalah kelainan pada irama jantung
atau ketidak-beraturan denyut jantung. Pada kelainan jantung ini, denyut jantung
mungkin terlalu cepat, terlalu lambat atau tidak beraturan. Beberapa pasien
terkadang tidak menyadari akan kondisi tersebut, dan beberapa pasien lainnya
merasakan gejala-gejala klinis yang timbul seperti palpitasi, pusing, nyeri
dada, nafas yang pendek. Di lain pihak, orang normal juga memungkinkan mengalami
perasaan yang sama seperti palpitasi, tapi hal tersebut bukan Aritmia. Oleh
karena itu, kelaian ini tidak cukup hanya di diagnosa berdasarkan gejala klinis
saja, dan pemanfaatan ECG akan sangat membantu dalam proses tersebut.

Berbagai penelitian telah dilakukan dalam kaitan pengenalan berbagai penyakit
Aritmia. Terdapat banyak penelitian dengan mengaplikasikan jaringan saraf tiruan
(JST) dan berbagai variasinya sebagai metode pendeteksian. Yeap \cite{Yeap:1990}
menggunakan backpropagation untuk mendeteksi Aritmia dengan menggunakan AHA
database. Ozbay \cite{Ozbay:2001} memanfaatkan tipe Aritmia dari MIT-BIH sebagai
data untuk pelatihan dan menggunakan data dari institusi-nya sebagai data
testing. Sedangkan Ceylan \cite{Ceylan:2007} melakukan studi perbandingan
beberapa metode extraksi fitur mulai dari Principal Component Analysis(PCA),
Wavelet transform (WT) dan Fuzzy C-Mean Clustering (FCM) untuk reduksi data dan
menggunakan BackPropagation untuk klasifikasi signal. Empat struktur dibangun,
FCM-NN, PCA-NN, FCM-PCA-NN and WT-NN dan menggunakan data Aritmia dari MIT-BIH
database. Studi perbandingan yang lain dilakukan oleh Ghongade dkk.
\cite{Ghongade:2007}. Ghongade melakukan perbandingan beberapa metode ekstraksi
fitur yakni DFT, PCA, DWT terhadap 180 signal sample dan disertai dengan
ekstrasi fitur berdasarkan morfologi signal yakni R-peak, QRS area serta Q-S
distance dengan menggunakan Backpropagation. Data yang digunakan adalah tiga
tipe Aritmia dari MIT-BIH database. Guler \cite{Guler:2005} menggunakan multi
layer perceptron (MLP) dan combine network model untuk mengklasifikasi signal
ECG (beat) dan menggunakan distribusi statistik dari hasil analisis spectral
signal ECG menggunakan DWT sebagai fitur klasifikasi. Elsayad dalam papernya,
\cite{Elsayad:2009} mengaplikasikan competitive based learning, yakni Learning
Vector Quantization - LVQ, untuk mengklasifikasi data ECG Aritmia dari UCI
database. Data diolah menggunakan PCA dan diaplikasikan pada beberapa varian
LVQ. Karraz dkk. dalam papernya \cite{Karraz:2006} mengembangkan klasifikasi
denyut jantung otomatis menggunakan jaringan saraf tiruan berdasarkan Bayesian
Framework. Studi dilakukan terhadap 5 tipe ECG Aritmia dari MIT-BIH database dan
fitur yang digunakan adalah ECG morphology dan time intervals.

Beberapa penelitian mengaplikasikan teori fuzzy untuk mendeteksi berbagai tipe
Aritmia seperti yang dilakukan oleh Anuradha dkk. pada papernya
\cite{Anuradha:2008}. Nonlinier dynamic dari ECG signal digunakan sebagai
karakteristik dari Aritmia diantaranya Spectral entropy, Poincar� plot geometry,
Largest Lyapunov exponent and Detrended fluctuation analysis yang diekstrak dari
heart rate signal. Kemudian linguistik variabel (fuzzy set) digunakan untuk
merepresentasikan ECG fitur dan fuzzy conditional statement digunakan sebagai
rule. Exarchos dkk. \cite{Exarchos:2007} mengembangkan metodologi dimana secara
otomatis menghasilkan fuzzy expert system dari data pelatihan dengan menggunakan
decision tree dimana digunakan untuk mengklasifikasi Aritmia (MIT-BIH database)
dan ischaemic (ST-T database). Yeh dkk. \cite{Yeh:2009} mengembangkan Fuzzy
logic method (FLM) untuk mendeteksi lima tipe Aritmia dari signal ECG  melalui
fuzzy inference engine dan operasi defuzzifikasi dengan menggunakan data dari
MIT-BIH Aritmia database.

Hao dkk. \cite{Hao:2005} mengembangkan metode klasifikasi Aritmia menggunakan
Support Vector Machine (SVM) dengan menggunakan PCA sebagai metode ekstraksi
fitur dari data MIT-BIH database. Studi dilakukan terhadap empat tipe Aritmia
yakni Normal, LBBB, RBBB dan PVC. Nasiri dkk. \cite{Nasiri:2009}
mengembangkan metode klasifikasi Aritmia dengan mengintegrasikan SVM dan
Genetic Algorithm dimana GA digunakan untuk meningkatkan kinerja dari SVM dari
sisi generalisasi-nya. Sedangkan Melgani dkk. \cite{Melgani:2008}
mengintegrasikannya dengan Particle Swarm Optimization (PSO) dengan menggunakan
data dari MIT-BIH database.

Metode pemrosesan ECG otomatis untuk proses klasifikasi denyut jantung
dikembangkan oleh Philip dkk. \cite{Philip:2004} dimana kategori dari Aritmia
yang digunakan adalah lima kategori yang direkomendasikan oleh standar ANSI/AAMI
EC57:1998 yaitu normal beat, ventricular ectopic beat (VEB), supraventricular
ectopic beat (SVEB), fusion of a normal and a VEB, dan unknown beat type. Lima
kategori tersebut merupakan pengelompokkan dari 15 kategori Aritmia yang
terdapat pada MIT-BIH database. Fitur yang digunakan adalah berdasarkan pada
interval denyut jantung, interval RR dan morfologi dari ECG. 

Dari latar belakang tersebut diatas, terlihat belum terlalu banyak peneliti yang
menggunakan pendekatan Competitive Based Learning seperti LVQ digunakan untuk
mendeteksi kelainan jantung khusus-nya Aritmia. Disamping itu, laboratorium
dimana \saya melakukan penelitian, sebelumnya telah mengembangkan metode
pengenalan aroma dengan menggunakan konsep \emph{Competitive Based Learning}
yaitu FNLVQ. Oleh sebab itu pada penelitian ini, \saya akan melakukan studi
tentang klasifikasi signal ECG Aritmia dengan menggunakan pendekatan
\emph{competitive based learning}, yakni Generalized LVQ \cite{Sato:1995}, yang
merupakan generalisasi dari LVQ2.1 , dimana GLVQ memiliki keunggulan dalam hal
menjamin konvergensi dalam proses pelatihan yang tidak dimiliki LVQ2.1 serta
memiliki karakteristik kinerja yang tidak sensitif terhadap inisialisasi bobot
awal \cite{Sato:1999}. Disamping itu juga, LVQ dan varian-nya merupakan jenis
JST yang arsitekturnya paling sederhana  sehingga berimplikasi pada kecepatan
proses pelatihan. Pada tahap berikutnya,  GLVQ yang perhitungan kemiripannya
berdasarkan \emph{euclidean distance} dimodifikasi untuk menghasilkan varian
baru metode klasifikasi LVQ dengan menggunakan teori fuzzy dimana konsep ini
juga diadaptasi dari metode FNLVQ yang telah dikembangkan sebelumnya oleh tim
peneliti lain-nya.

% dengan pengklasifikasian  dengan
% menggunakan mahalanobis distance karena mahalanobis distance memperhitungkan
% tingkat distribusi  dari vektor fitur dan berguna  dalam membandingkan vektor
% fitur dimana tiap  elemen memiliki rentang dan varian yang berbeda-beda.

% Dataset yang akan digunakan pada penelitian ini adalah data yang
% telah tersedia bebas dari physionet\footnote{http://physionet.net} yakni MIT-BIH
% arrhytmia database. Dataset akan diolah terlebih dahulu menggunakan pendekatan
% analisis spectral, Wavelet Transform.
% 
% Namun untuk dapat dikenali oleh sistem, data gelombang ECG perlu diolah lebih
% lanjut sehingga lebih mudah dikenali dan hasil yang diperoleh memiliki tingkat
% akurasi yang tinggi. Oleh karena itu, dalam studi mandiri ini akan dipelajari
% metode untuk mengolah data dasar (\textit{raw preprocessing}), menganalisa,
% memvisualisasikan data gelombang ECG, baik berupa data analog (ECG paper)
% maupun data digital.

%-----------------------------------------------------------------------------%
\section{Permasalahan}
%-----------------------------------------------------------------------------%
Pada bagian ini akan dijelaskan mengenai definisi permasalahan 
yang \saya~hadapi dan ingin diselesaikan serta asumsi dan batasan 
yang digunakan dalam menyelesaikannya. 

%-----------------------------------------------------------------------------%
\subsection{Definisi Permasalahan}
%-----------------------------------------------------------------------------%
% \todo{Tuliskan permasalahan yang ingin diselesaikan. Bisa juga
% 	berbentuk pertanyaan}

Berdasarkan uraian latar belakang diatas, penelitian ini secara umum
terbagi menjadi beberapa permasalahan sebagai berikut;

\begin{enumerate}
  \item Bagaimana pengolahan data awal yang harus dilakukan agar sesuai dengan
  kebutuhan proses pengenalan Aritmia.
  \item Bagaimana penanganan noise data dapat memperbaiki kinerja dari proses
  pengenalan Aritmia.
  \item Bagaimana metode pembelajaran berbasis kompetisi dapat digunakan untuk
  proses pengenalan Aritmia.
  \item Bagaimana teori fuzzy diintegrasikan ke metode Generalized LVQ?
  \item Bagaimana integrasi teori fuzzy dengan GLVQ dapat digunakan untuk
  meningkatkan kinerja dari proses pengenalan Aritmia.
%   \item Bagaimana metode perhitungan kemiripan mahalanobis distance dapat
%   digunakan untuk memperbaiki kinerja dari proses pengenalan Arrhytmia.
\end{enumerate}

%-----------------------------------------------------------------------------%
\subsection{Batasan Permasalahan}
%-----------------------------------------------------------------------------%
% \todo{Umumnya ada asumsi atau batasan yang digunakan untuk 
% 	menjawab pertanyaan-pertanyaan penelitian diatas.}
Ruang lingkup yang digunakan dalam penelitian ini adalah sebagai berikut;
\begin{enumerate}
  \item Dataset yang digunakan sebagai data pelatihan dan evaluasi adalah data
  ECG Aritmia yang didapat dari physionet, yakni data MIT-BIH database dimana
  data tersebut sudah dilengkapi dengan berbagai anotasi dari para ahli jantung
  secara manual.
  \item Jumlah kategori Aritmia yang akan dideteksi dalam penelitian ini
  adalah sebanyak 12 kategori termasuk didalamnya signal denyut jantung hasil
  dari penggunaan alat pacu jantung (pacemaker).
  \item Sinyal ECG yang digunakan pada penelitian ini hanya menggunakan
  sinyal hasil sandapan MLII.
  \item Frekuensi cuplik (sampling rate) dari sinyal ECG yang digunakan adalah
  360Hz.
  \item Jumlah prototype/vektor pewakil untuk setiap kategori kelas yang
  akan diujicoba hanya berjumlah satu.
\end{enumerate} 
%-----------------------------------------------------------------------------%
\section{Tujuan}
%-----------------------------------------------------------------------------%
Tujuan dari penelitian ini adalah sebagai berikut;
\begin{enumerate}
  \item Mencari beberapa kemungkinan pengolahan data ECG agar dapat digunakan
  dalam proses pelatihan dan pengenalan kelainan irama jantung-Aritmia.
  \item Mengembangkan sistem yang mampu mengenali kelainan irama
  jantung-Aritmia berdasarkan data signal ECG dengan menggunakan pendekatan
  competitive based learning.
\end{enumerate}


%-----------------------------------------------------------------------------%
\section{Posisi Penelitian}
%-----------------------------------------------------------------------------%
% \todo{Posisi penelitian Anda jika dilihat secara bersamaan dengan 
% 	peneliti-peneliti lainnya. Akan lebih baik lagi jika ikut menyertakan 
% 	diagram yang menjelaskan hubungan dan keterkaitan antar 
% 	penelitian-penelitian sebelumnya}
Penelitian ini merupakan bagian dari penelitian dan pengembangan alat pendeteksi
kelainan jantung portable oleh Wisnu J dkk, sehingga disamping hasil yang
akurat, dibutuhkan juga sistem yang memiliki waktu eksekusi yang cepat. Oleh 
karena itu, pada penelitian ini dipilih pendekatan competitive based Learning.
Berikut adalah bagan posisi penelitian ini terhadap penelitian-penelitian
lainnya sebagai bagian dari pengembangan alat pendeteksi kelaian jantung
portable.

\addFigure{width=1\textwidth}{pics/baganpospeneliti.png}{fig:bagan}{Diagram
posisi penelitian dan kontribusi.}

Pada bagan \pic~\ref{fig:bagan} dapat dilihat bahwa penelitian ini berfungsi
sebagai penelitian awal dalam menginvestigasi berbagai kemungkinan struktur data
/ fitur yang nantinya dapat digunakan sebagai referensi dalam mengembangkan alat
perekam ECG seperti yang dilakukan oleh Saudara Asep Insani. Berkaitan dengan
alat perekam ECG yang dikembangkan, hal-hal yang perlu diperhatikan adalah
mengenai sampling rate yang digunakan dalam merekam denyut jantung. Karena pada
penelitian ini menggunakan data dari physionet dengan sampling rate tertentu
sebagai data pelatihan, maka ada baiknya sampling rate yang digunakan sama, atau
paling tidak dipertimbangkan langkah-langkah penyesuaian tertentu. Hal ini
dikarenakan data yang dihasilkan dari alat perekam ECG yang dikembangkan tidak
memiliki label, seperti data MIT-BIH dan juga akan mengandung noise yang tinggi.
terkecuali nantinya akan ada pakar jantung yang akan memberikan label pada data
hasil perekaman.  Selain itu penelitian ini juga digunakan sebagai referensi
oleh tim FPGA dalam mengembangkan embeded system pada alat portable.

Secara garis besar, kontribusi \saya pada penelitian ini dapat dilihat
dari dua aspek diantaranya sebagai berikut;
\begin{enumerate}
  \item Internal : kontribusi terhadap penelitian yang dilakukan oleh tim
  peneliti Aritmia Fasilkom Universitas Indonesia
  \begin{itemize}
    \item Sebagai bagian awal penelitian di bidang Aritmia, dimana pada
    penelitian ini dilakukan proses pengolahan data awal dengan melewati
    beberapa tahap pemrosesan.
    \item Pengembangan Java Engine metode klasifikasi yang dapat digunakan
    dengan mudah dan dapat dikembangkan lebih lanjut untuk keperluan yang lebih
    spesifik.
  \end{itemize}
  
  \item Eksternal : kontribusi terhadap dunia penelitian khususnya dibidang
  pengenalan Aritmia.
  \begin{itemize}
    \item Pengembangan metode klasifikasi baru yaitu Fuzzy-Neuro Generalized
    Learning Vector Quantization (FNGLVQ).
    \item Penggunaan metode FNGLVQ dalam pendeteksian pengenalan kelaian
    Aritmia melalui sinyal ECG.
  \end{itemize}
\end{enumerate} 

%-----------------------------------------------------------------------------%
\section{Metodologi Penelitian}
%-----------------------------------------------------------------------------%
Metode yang digunakan dalam penelitian ini adalah sebagai berikut;
\begin{enumerate}
  \item Studi literatur \\
		Pada tahap ini, \saya melakukan kajian literatur mengenai perkembangan terakhir
		penelitian di bidang ECG, mempelajari mengenai pendekatan yang dilakukan para
		peneliti dalam mencari solusi permasalahan, terutama proses pengenalan kelainan
		jantung seperti Aritmia. Literatur ini didapat dari buku, tesis dan artikel
		baik dalam jurnal dalam negeri maupun luar negeri, serta informasi dari berbagai
		sumber yang berkaitan dengan topik tesis ini.
  \item Pengolahan data\\
		Pada tahap ini, \saya mempersiapkan data yang akan digunakan dalam penelitian.
		Data ECG yang digunakan merupakan data yang diambil dari repository
		online, physionet.net. Beberapa pengolahan awal diaplikasikan terhadap data
		tesebut sesuai dengan tujuan yang ingin dicapai.
  \item Perancangan sistem \\
		Tahap perancangan sistem meliputi pembuatan rancangan sistem untuk
		pendeteksian kelainan denyut jantung dengan menggunakan pendekatan
		\emph{competitive base learning}.
  \item Implementasi sistem \\
		Tahap ini merupakan implementasi dari tahap rancangan sebelumnya. pada tahap
		ini diimplementasikan beberapa variasi dari supervised competitive based
		learning mulai dari LVQ (1,21, 3), GLVQ dan Fuzzy Neuro GLVQ.
  \item Uji coba dan analisis sistem \\
		Pada tahap ini dilakukan ujicoba sistem yang telah diimplementasikan terhadap
		data yang telah diolah pada tahap sebelumnya dengan menggunakan berbagai
		skenario percobaan. Kemudian dilanjutkan dengan analisis terhadap hasil  yang
		didapat untuk mengetahui performa dari sistem.
  \item Penulisan laporan \\
		Bagian akhir dari metodologi penelitian ini adalah penulisan tesis yang
		memuat semua hasil penelitian yang telah dilakukan.
\end{enumerate}

%-----------------------------------------------------------------------------%
\section{Sistematika Penulisan}
%-----------------------------------------------------------------------------%
Sistematika penulisan laporan akhir ini adalah sebagai berikut:
\begin{itemize}
	\item Bab 1 \babSatu \\
		Bab ini berisi latar belakang yang memotivasi penelitian ini,  perumusan
		masalah, tujuan, ruang lingkup, metodologi penelitian serta sistematika penulisan.
	\item Bab 2 \babDua \\
		Bab ini berisi pembahasan mengenai sekilas tentang penyakit jantung, dan
		khususnya beberapa variasi kelainan irama jantung-Aritmia. Juga akan
		dibahas mengenai pendekatan dalam machine learning khususnya
		Competitive Base learning mulai dari standard LVQ sampai Generalized LVQ
		(GLVQ) dan bagaimana metode tersebut diaplikasikan terhadap ECG data.
	\item Bab 3 \babTiga \\
		Bab ini berisi pembahasan mengenai pengolahan data MIT-BIH dari physionet.net
		disesuaikan dengan kebutuhan dalam penelitian ini, mulai dari bagaimana
		mekanisme pengambilan data fitur, \emph{baseline wander removal} (BWR),
		reduksi fitur dengan menggunakan wavelet serta filtering dataset dari outlier
		yang dapat mengganggu performa dari sistem nantinya.
	\item Bab 4 \babEmpat \\
		Bab ini berisi pembahasan mengenai rancangan dan implementasi dari algoritma
		yang telah dirumuskan beserta beberapa modifikasi yang dilakukan untuk
		memperbaiki performa dari algoritma.
	\item Bab 5 \babLima \\
		Bab ini berisi pembahasan mengenai ujicoba yang dilakukan beserta
		skenario yang telah ditentukan dilanjutkan dengan evaluasi unjuk kerja
		beberapa algoritma yang dibandingkan satu dengan yang lain. Pada bab
		ini juga dilakukan analisis secara statistik untuk mengetahui efek modifikasi
		yang telah dilakukan terhadap tingkat unjuk kerja sistem.
	\item Bab 6 \kesimpulan \\
		Bab ini berisi kesimpulan yang diperoleh dari hasil ujicoba dan
		analisis penelitian yang telah dilakukan dan saran bagi pengembangan
		selanjutnya.
\end{itemize}

% \newpage
% \section{Revision}
% \begin{itemize}
%   \item Inget, masukkan alasan, Why?, menggunakan data physionet ketimbang UCI,
%   kaitan dengan fitur yang dikandung. UCI, mengandung data interval, dan tiap
%   pasien udah ada anotasi menderita penyakit yang mana, tapi susah kalau
%   disesuaikan dengan keperluan alat. physionet, memberikan data seperti apoa
%   yang akan dihadapi nantinya. hanya saja cuman memberikan informasi beat
%   anotasi nya saja.
%   \item Why? mendeteksi penyakit berdasarkan beat? padahal kan tidak bisa
%   langsung menilai seseorang menderita penyakit arrhytmia. kaitan dengan
%   kemunculan kelainan beat, dengan seseorang mengalami kelainan, data physionet
%   tidak memberikan informasi mengenai seseorang menderita arhytmia, tapi
%   anotasi, yang dikandung pada ECG seseorang
%   \item Why? Competitive Learning? kayaknya gak pas cuman alasan, karena jarang
%   yang make.
% \end{itemize}
% -----------------------------------------------------------------------------%
\chapter{\babDua}
% -----------------------------------------------------------------------------%
% \todo{tambahkan kata-kata pengantar bab 2 disini}
Pada bab ini akan dijelaskan mengenai berbagai literatur yang mendukung
penelitian ini diantaranya mulai dari uraian mengenai penyakit jantung,
khususnya \gls{arrhytmia}, \gls{ecg}, jaringan saraf tiruan, pembelajaran
berbasis kompetisi, algoritma \gls{lvq} dan beberapa varian-nya, seperti
algoritma \gls{glvq}.

% -----------------------------------------------------------------------------%
\section{Penyakit Jantung}
\label{sec:jantung}
Penyakit Jantung (\textit{Heart Disease, Cardiopathy}) adalah segala gangguan
yang timbul dan mempengaruhi jantung \cite{medicinenet.1}. Penyakit jantung
sering juga disebut dengan \textit{Cardiac Disease}. Jika sudah melibatkan
kinerja dari jantung dan pembuluh darah, maka disebut dengan
\textit{Cardiovascular Disease}. Terdapat beberapa jenis penyakit jantung
diantaranya; Angina, Arrhytmia, Congenital heart disease, Coronary artery
disease (CAD), Dilated cardiomyopathy, Heart Attack (Myocardial infarction),
Heart Failure, Hypertrophic cardiomyopathy, Mitral regurgitation, Mitral valve
prolapse, dan Pulmonary stenosis. 

Secara umum, penyakit Jantung disebabkan oleh beberapa faktor, dimana dapat 
dikategorikan menjadi 2 berdasarkan Faktor resikonya yaitu  \textit{Major} dan
\textit{Contributing}.
\begin{enumerate}
    \item \textit{Major Risk Factors} : yaitu faktor yang telah terbukti 
    menyebabkan meningkatnya resiko terhadap penyakit jantung.  Beberapa faktor
    utama yang menyebabkan terjadinya penyakit jantung adalah;
	\begin{itemize}
	    \setlength{\itemsep}{1pt}
  		\setlength{\parskip}{0pt}
  		\setlength{\parsep}{0pt}
	    \item Diabetes (Gula Darah)
	    \item Tekanan darah tinggi (Hypertension)
	    \item Merokok (Smoking)
	    \item Kolesterol tinggi (High cholesterol)
	    \item Jarang untuk beraktifitas fisik seperti Olahraga  (Lack of physical
	    activity)
	    \item Kegemukan (Obesity)
	\end{itemize}

	\item \textit{Contributing risk factors} ; yaitu salah satu  yang
	dipertimbangkan oleh dokter yang dapat menyebabkan penyakit jantung seperti;
	\begin{itemize}
	  	\setlength{\itemsep}{1pt}
		\setlength{\parskip}{0pt}
  		\setlength{\parsep}{0pt}
	    \item Tingkat Stres
	    \item Akibat Pil KB (Birth control pills)
	    \item Hormon Sex (Sex hormones)
	    \item Alkohol
	\end{itemize}
\end{enumerate}

Pada bab ini \saya akan memfokuskan pembahasan
mengenai arrhytmia, definisi, penyebab dan beberapa kategori penyakit arrhytmia
yang nantinya akan dikenali oleh sistem.

% -----------------------------------------------------------------------------%
\subsection{Arrhytmia}
\label{ssec:arrhytmia}
\Gls{arrhytmia} atau juga disebut dengan Dysrhytmia adalah kelaian/gangguan
urutan irama jantung atau gangguan kecepatan dari proses depolarisasi, repolarisasi
atau keduanya pada jantung \cite{karim.1}. Beberapa pasien terkadang tidak
menyadari akan kondisi tersebut, dan beberapa pasien lainnya merasakan
gejala-gejala klinis yang timbul seperti \gls{palpitasi}, pusing, nyeri dada,
nafas yang pendek. Di lain pihak, orang normal juga memungkinkan mengalami perasaan
yang sama seperti \gls{palpitasi}, tapi hal tersebut bukan termasuk arrhytmia.

\Gls{heartrate} atau laju denyut jantung orang normal berkisar
antara 60 sampai dengan 100 denyut per menit\cite{medicinenet.2}. Jika dilihat
dari \gls{heartrate}, terdapat dua kelompok penyakit \gls{arrhytmia} yaitu;
\begin{itemize}
	\item \textbf{Tachycardia} ; yaitu denyut jantung yang cepat, biasanya
	ditentukan lebih dari 100 denyut per menit. Termasuk didalamnya adalah 
	\textit{sinus tachycardia}, \textit{paroxysmal atrial tachycardia} (PAT), dan
	\textit{ventricular tachycardia}. \textit{Tachycardia}  dapat mengakibatkan
	\gls{palpitasi}. Namun, Tachycardia belum tentu sebuah 	\textit{Arrhythmia}.
	Peningkatan denyut jantung adalah respon  normal pada latihan fisik atau stres
	secara emosional.

	\item \textbf{Bradycardia} ; yaitu denyut jantung yang lambat, biasanya kurang
	dari 60 denyut per menit. Termasuk didalamnya \textit{sinus bradycardia}.
\end{itemize}
 
Menurut Dr. Sjukri Karim dalam bukunya \cite{karim.1}, \gls{arrhytmia}
disebabkan oleh faktor aritmogenik yaitu diantaranya;
\begin{itemize}
  \item Hipoksia : semua penyakit yang menyebabkan defisiensi oksigen pada
  miokard misalnya penyakit paru, kardiomiopati, atau penyakit jantung koroner
  dapat menyebabkan \gls{arrhytmia}.
  \item Iskemia : miokard yang iskemik oleh sebab apa saja merupakan faktor
  pencetus timbulnya \gls{arrhytmia}
  \item Rangsamgam sisimam saraf otonom : rangsangan berlebihan pada saraf
  simpatis mapun parasimpatis dapat menimbulkan \gls{arrhytmia}.
  \item Obat-obatan : semua antiaritmia mempengaruhi fase depolarisasi dan
  repolarisasi jantung, sehingga obat-bat tersebut memiliki efek aritmogenik.
  Selain itu obat-obatan seperti kafein, aminofilin, antidepresan trisiklik dan
  digitalis juga memiliki efek aritmogenik.
  \item Gangguan keseimbangan elektrolit dan gas darah : fase depolarisasi dan
  repolarisasi otot jantung ditimbulkan oleh perpindahan berbagai ion elektrolit
  melalui membran sel.
  \item Regangan dinding otot jantung : dinding jantung yang teregang seperti
  pada dilatasi atrium atau ventrikel akibat gagal jantung, kardiomiopati atau
  penyakit-penyakit katub dapat menyebabkan aritmia.
  \item Kelainan struktur sistem konduksi : penderita yang memiliki fetal
  despersi di AV-node dan fasciculo-ventricular connection, atau yang memiliki
  jalur tambahan (accessory pathway) seperti sindrom Wolff-Parkinson-White
  sangat mudah mengalami aritmia melalui mekanisme preeksitasi.
  \item Interval QT yang memanjang : pada penderita penyakit jantung koroner,
  kelainan struktur jantung atau gangguan elektrolit yang disertai interval QT
  memanjang akan lebih sering terjadi aritmia dibandingkan dengan individu
  normal.
\end{itemize} 

Beberapa penyakit/kelainan yang termasuk dalam \gls{arrhytmia} adalah sebagai
berikut;
\begin{enumerate}
  \item Atrial Fibrilation/Flutter (AF) \\
  AF merupakan irama jantung tidak teratur yang umum yang menyebabkan atrial,
  bagian atas bilik jantung, berkontraksi tidak normal. Penyebab tersering dari
  AF adalah infark miokard, dilatasi atrium kiri, penyakit paru kronis, gagal
  jantung, pasca pembedahan kardiotorasik dan tirotoksikosis.
  
  \item Supraventricular tachycardia (SVT) \\
  Denyut jantung cepat yang	tidak  normal secara teratur yang disebabkan oleh
  tembakan secara cepat impuls listrik  dari atas \textit{atrioventricular} node
  (AV node) di dalam jantung. Termasuk didalamnya adalah takikardia atrium,
  takikardia atrium multifokal, takikardia supraventrikular paroksismal.

  \item Sindrom Wolff-Parkinson-White (WPW) \\
  Merupakan kumpulan gejala yang ditimbulkan oleh impuls dari atrium yang
  dikonduksi ke ventrikel lebih cepat dari biasanya (pre-eksitasi) melalui jalur
  tambahan.

  \item Sick-Sinus Syndrome (SSS) \\
  merupakan kelainan dimana nodus SA tidak dapat mencetuskan impuls secara
  normal akibat dari fibrosis di nodus SA. Gejala klinis yang muncul berupa
  bradikardia sinus, episode henti sinus yang intermiten, dan sindrom
  bradi-takikardia sehingga penderita mengeluh palpitasi, presinkope atau
  sinkope. Biasanya diagnosis SSS dapat dilakukan dengan menggunakan Holter
  Monitor ECG (ECG 24 jam atau lebih).
  
  \item Premature Ventricular Contractions (PVC) \\
  Atau juga disebut Ventricular Premature beat, merupakan kejadian umum  dimana
  denyut jantung diinisiasi oleh ventrikel jantung,   bukan  oleh
  \textit{sinoatrial node} yang merupakan  inisiator denyut jantung yang normal.
  Akibatnya, muncul denyut tambahan yang tidak normal sebelum denyut normal
  mumcul.

  \item Ventricular Tachycardia (VT) \\
  Merupakan suatu irama jantung yang cepat yang berasal dari ruang bawah (atau
  ventrikel) jantung. Laju denyut yang cepat mencegah jantung mengisi cukup
  darah , sehingga sejumlah kecil darah dipompa ke seluruh tubuh. Ini bisa
  menjadi \textit{arrhytmia} yang serius,  terutama pada orang dengan penyakit
  jantung, dan dapat berhubungan dengan banyak  gejala. Seorang dokter harus
  mengevaluasi \textit{arrhytmia} ini.
		
  \item Ventricular fibrillation (VF) \\
  Tembakan impuls yang tidak teratur dan tidak menentu dari ventrikel. Ventrikel
  bergetar dan tidak bisa berkontraksi atau memompa darah ke tubuh.  Ini adalah
  keadaan darurat medis yang harus diobati  dengan \textit{cardiopulmonary
  resuscitation} (CPR) dan \textit{defibrillation} sesegera mungkin.
    
  \item Blok \\
  Blok atau hambatan konduksi dapat dibagi menadi 3 jenis menurut lokasi
  kejadiannya yaitu:
  \begin{itemize}
    \item Blok nodus SA : pada kondisi ini, serabut sinus adalah normal, hanya
    saja gelombang depolarisasi yang dicetuskan terhambat sebelum mencapai
    atrium. Blok SA biasanya tidak memberi gejala sehingga tidak memerlukan
    pengobatan.
    \item Blok nodus AV : semua hambatan konduksi yang terjadi antara nodus SA
    sampai pada berkas HIS-Sistem Purkinje. Blok AV dibagi menjadi Blok AV
    derajat I, II dan III. Mungkin jantung berdetak tidak teratur dan kadang sering
	lebih lambat. Jika serius, \textit{Heart Block} diobati dengan alat pacu
	jantung (PaceMaker). Sebagian besar muncul dari patologi di simpul
	atrioventrikular dimana penyakit ini penyebab paling umum dari \textit{Bradycardia}.
    \item Blok Infranodal : hambatan konduksi yang terjadi pada sistem
    infranodal. Sistem infranodal terdiri dari berkas His dan 3 cabang berkas
    infra-ventrikular yaitu satu right bundle branch (RBB), dan 2 fasikulus dari
    left bundle branch (LBB).\\
	Jika konduksi terhambat pada bagian berkas cabang kanan (RBB), maka disebut
	sebagai Right Bundle Branch Block (RBBB), dan jika konduksi terhambat pada
	bagian berkas cabang kiri (LBB), maka disebut Left Bundle Branch Block (LBBB).
  \end{itemize}
  
  \item Sudden Arrhythmia Death Syndrome (SADS) \\
  Merupakan kematian mendadak yang yang tidak diharapkan yang disebabkan oleh
  kehilangan fungsi jantung secara tiba-tiba \textit{sudden cardiac arrest}.
	
\end{enumerate}

Penanggulangan arrhytmia dapat dilakukan secara farmakologik, kardioversi,
ablasi, pemasangan alat pacu jantung atau dengan tindakan operasi. Alat pacu
jantung atau biasa disebut PaceMaker dapat dipasang secara permanen (PPM) atau
secara temporer (TPM). Pemasangan PPM maupun TPM dapapt dilakukan berdasarkan
indikasi terjadinya SSS,  AV block, gagal jantung. Pemasangan alat pacu jantung
akan menimbulkan denyut jantung yang sedikit berbeda dengan denyut jantung
normal, dan terkadang bergabung dengan denyut normal.

\glsreset{ecg}
 
Pendekatan yang dilakukan untuk mendiagnosa \gls{arrhytmia} adalah
mendengarkan denyut jantung dengan menggunakan stetoskop dimana dengan cara ini
dapat memberikan indikasi-indikasi yang umum dari denyut jantung dan apakah
teratur atau tidak. Namun tidak semua impuls elektrik dari jantung dapat
didengar, seperti di banyak cardiac arrhytmia. Cara yang sederhana dilakukan
untuk mendiagnosa irama jantung adalah dengan \gls{ecg}.

% -----------------------------------------------------------------------------%
% ECG section
\section{Electrocardiogram (ECG)}
\label{sec:ecg}

\subsection{Apa itu ECG?}

Secara etimologi, istilah \textit{Electrocardiograph} disusun dari 3 kata
(yunani) yaitu electro; aktifitas elektrik, cardio; jantung dan graph;
menulis/mencatat. jadi \textit{Electrocardiograph} bisa diartikan mencatat
aktifitas elektrik dari jantung.

Menurut medicinenet.com \cite{medicinenet.1},\textit{Electrocardiography}
(ECG/EKG) merupakan pengujian yang bersifat noninvasif yang dilakukan untuk
mengetahui kondisi dari jantung dengan cara mengukur aktifitas elektrik dari
jantung tersebut.

\addFigure{height=0.3\textheight}{pics/ecgmeasure.png}{fig:ecglead}{Pasien
dengan 10 elektrode}

Pengukuran itu dilakukan dengan memasang leads (alat sensor elektrik) pada
tubuh dilokasi-lokasi tertentu, sehingga didapatkan berbagai informasi tentang
kondisi jantung yang dapat dipelajari dengan melihat pola karakteristik dari
ECG.

ECG dilakukan dengan menggunakan suatu sensor elektrik (12-\textit{leads}) yang
dipasangkan dibagian tubuh tertentu seperti dibagian ujung tubuh 
(\textit{Extremity}), yang berjumlah total 4, dan 6 posisi yang telah ditetapkan
sebelumnya, yaitu dibagian dada. Sejumlah kecil gel dioleskan pada kulit, yang
memungkinkan impuls listrik dari jantung lebih mudah terkirim ke lead ECG.
Ilustrasi dapat dilihat seperti pada gambar \ref{fig:ecglead}. 

ECG dapat memberikan informasi mengenai irama jantung  secara keseluruhan dan
kelemahan diberbagai bagian dari otot jantung. Dengan menggunakan ECG, dapat
diketahui;

\begin{itemize}
   \item mekanisme laju dan irama jantung, 
   \item orientasi dari jantung didalam rongga dada, 
   \item gejala peningkatan ketebalan (\textit{hypertrophy}) dari otot jantung,
   \item gejala kerusakan dari berbagai bagian otot jantung, 
   \item gejala gangguan akut aliran darah ke otot jantung
   \item informasi pola-pola aktivitas elektrik yang tidak normal yang dapat
   mempengaruhi pasien ke arah gangguan irama jantung yang abnormal
   (\textit{Abnormal Cardiac Rhythm Disturbances}).
\end{itemize}

\noindent ECG dapat mendiagnosa kondisi-kondisi seperti dibawah ini;
\begin{itemize}
   \item irama jantung cepat atau tidak teratur yang tidak normal.
   \item irama jantung lambat yang tidak normal.
	\item konduksi impuls jantung yang tidak normal, yang mungkin dapat memberikan
	saran terhadap gangguan jantung maupun metabolisme.
	\item petunjuk tentang kemunculan serangan jantung yang terjadi sebelumnya
	\item petunjuk yang berkembang ke arah serangan jantung akut.
	\item petunjuk kerusakan akut dari aliran darah ke  jantung selama episode
	ancaman serangan jantung (angina tidak stabil).
	\item Efek merugikan pada jantung dari berbagai  penyakit jantung atau penyakit
	sistemik (seperti tekanan darah tinggi, kondisi tiroid, dll).
	\item Efek merugikan pada jantung dari kondisi  paru-paru tertentu (seperti
	emfisema, paru embolus (gumpalan darah ke paru-paru), dll).
	\item Petunjuk elektrolit darah tidak normal (kalium, kalsium, magnesium).
	\item Petunjuk peradangan jantung atau lapisannya.
\end{itemize}


\noindent
Keterbatasan dari ECG adalah sebagai berikut;
\begin{itemize}
    \item ECG adalah gambaran statis dan mungkin  tidak menunjukkan permasalahan
    jantung (parah) ketika si pasien tidak menunjukkan gejala apapun. Contoh
    yang paling umum dari kasus ini adalah pada pasien dengan riwayat nyeri dada
    intermiten yang parah yang disebabkan oleh penyakit arteri koroner. Pasien
    ini mungkin memiliki ECG normal ketika pasien tidak mengalami gejala-gejala
    sakit. Namun mungkin saja pada ECG yang tercatat melalui proses
    \textit{stress test} dapat saja menunjukkan suatu kelainan, sedangkan ECG
    yang diambil pada kondisi yang lainnya terlihat normal.

	\item Banyak pola  abnormal yang tidak spesifik muncul pada ECG,  yang berarti
	bahwa ECG dapat diamati pada berbagai kondisi yang berbeda. Bahkan ECG mungkin
	menunjukkan varian yang normal dan tidak mencerminkan suatu kelainan apapun.
	Kondisi ini sering ditemukan oleh seorang dokter, dengan melakukan pemeriksaan
	yang lebih terperinci, dan kadang-kadang tes jantung lainnya (misalnya,
	\textit{echocardiogram, exercise stress test}) mungkin akan menemukan suatu
	kelainan.

	\item ECG tidak dapat mengukur kemampuan pompa jantung secara handal, dimana
	dalam kasus ini sering digunakan \textit{echocardiogram}.

	\item Dalam beberapa kasus, ECG dapat sepenuhnya normal meskipun kemunculan
	kondisi jantung yang normal akan tercermin dalam ECG. Dan hal ini sebagian besar
	tidak diketahui penyebabnya. Namun yang perlu diingat adalah dengan ECG yang
	normal tidak menutup kemungkinan munculnya penyakit jantung. Selain itu, seorang
	pasien dengan gejala-gejala jantung kadang kala memerlukan evaluasi dan
	pengujian tambahan.

\end{itemize}

\subsection{Kertas ECG} 
Interpretasi waktu dari ECG ditunjukkan dengan suatu kertas bertanda
(\textit{paper speed}) yaitu suatu  kertas milimeter block yang berkorelasi
dengan waktu pencatatan dari denyut jantung. Biasanya electrocardiograph bekerja
pada paper speed 25 mm/s, meskipun paper speed yang lebih cepat terkadang
digunakan. luasan block kecil pada paper speed berukuran 1mm2. Pada paper speed
berukuran 25mm/s, satu block kecil ECG diterjemahkan menjadi 40ms. 5 blok kecil
yang disusun membentuk 1 blok besar, diterjemahkan menjadi 200ms, oleh karena
itu ada 5 blok besar untuk setiap detik-nya. Kualitas diagnostik 12-lead ECG
dikalibrasi pada 10 m/V, sehingga 1 mm diterjemahkan menjadi 0,1 mV. Sebuah
sinyal kalibrasi harus disertakan untk setiap record. Sebuah sinyal standar 1 mV
harus menggerakkan jarum 1 cm secara vertikal, yaitu dua kotak besar di kertas
ECG. ilustrasi kertas ECG (paper speed) dapat dilihat pada
\pic~\ref{fig:ecgpaper}

\addFigure{height=0.6\textwidth}{pics/ecgpaper.jpg}{fig:ecgpaper}{representasi
kertas ECG}

\subsection{ECG Signal} 
Siklus dari denyut jantung yang ada pada ECG terdiri dari gelombang P (P-wave),
QRS Complex, T-wave, dan U-wave yang mana biasanya terlihat pada hampir
50\%-70\% dari keseluruhan ECG. tegangan dasar dari ECG biasa dikenal dengan
nama \textit{isoelectric line}. Biasanya  \textit{isoelectric line} diukur
sebagai bagian dari pelacakan yang mengikuti T-wave dan mendahului P-wave
berikutnya. Ilustrasi dapat dilihat pada gambar \ref{fig:ecgwave}.


\addFigure{width=1\textwidth}{pics/ecgwave.jpg}{fig:ecgwave}{representasi
skematik dari ECG Normal}

Seperti yang telah dijabarkan pada bagian sebelumnya, ada beberapa macam alat
rekam ECG berdasarkan lead yang digunakan seperti ECG \textit{12-leads}, ECG
\textit{5-leads} dan ECG \textit{3-leads}. ECG recorder yang standar, umum
digunakan dirumah sakit adalah ECG 12-leads dengan 10 elektrode. Masing-masing
lead  dari 10 elektrode/sensor tersebut akan menghasilkan gelombang ECG
tersendiri dimana dari gelombang inilah diagnosa kemudian dilakukan. setiap lead
dapat memberikan informasi yang saling mendukung dengan lead-lead yang lain.
Pada umumnya, alat perekam ECG akan menghasilkan gelombang untuk setiap lead
secara bergantian tiap interval tertentu seperti pada gambar
\ref{fig:ecg12lead}.


\addFigure{width=1\textwidth}{pics/ecg12lead.jpg}{fig:ecg12lead}{representasi
gelombang ECG 12-lead untuk \textit{Unstable Angina}}


Dalam usaha mendeteksi suatu ketidaknormalan pada jantung melalui ECG,  para
ahli melakukannya  dengan melihat beberapa ciri yang dapat dibandingkan  dengan
ECG kondisi pada kondisi normal diantaranya ; irama, Rate QRS, Aksis QRS,
Morfologi Gelombang P, Interval PR, Durasi QRS, Morfologi QRS, Deviasi Segmen
ST, Morfologi Gelombang T, Morfologi Gelombang U, Lain-lain (LVH,LV Strain,BBB,
QT interval).


% -----------------------------------------------------------------------------%
\section{Jaringan Saraf Tiruan}
\label{sec:jst}

Jaringan Saraf Tiruan (JST) atau Artificial Neural Networks (ANN)
merupakan suatu sistem yang dibangun atas dasar cara kerja jaringan saraf
manusia. Awal perkembangannya dimotivasi oleh kemampuan pengenalan dari manusia
(otak) dimana cara perhitungannya sangat jauh berbeda dengan sistem komputer
digital. JSt merupakan sistem adaptif yang dapat mengubah strukturnya
untuk memecahkan masalah berdasarkan informasi eksternal maupun internal
yang mengalir melalui jaringan tersebut. 

Menurut S. Haykin \cite{haykin-1994}, sebuah jaringan saraf adalah sebuah
prosesor yang terdistribusi paralel dan mempuyai kecenderungan untuk menyimpan
pengetahuan yang didapatkannya dari pengalaman dan membuatnya tetap tersedia
untuk digunakan. Hal ini menyerupai kerja otak dalam dua hal yaitu: (1)
Pengetahuan diperoleh oleh jaringan melalui suatu proses belajar. (2) Kekuatan
hubungan antar sel saraf yang dikenal dengan bobot sinapsis digunakan untuk
menyimpan pengetahuan.

Jaringan saraf merupakan suatu mesin yang digunakan untuk memodelkan kerja otak
dalam menyelesaikan suatu permasalahan. Jaringan tersebut disusun dari
sekumpulan unit pemroses yang disebut neuron dan untuk meningkatkan
kemampuan-nya, dilakukan proses pembelajaran dengan menggunakan suatu algoritma
tertentu (learning algorithm) dimana tujuannya adalah untuk memodifikasi
kekuatan hubungan antar neuron (bobot) dalam jaringan sesuai dengan goal yang
telah ditentukan.

Keuntungan dari penggunaan JST adalah kemampuannya dalam beradaptasi melalui
proses pembelajaran dan kemampuan generalisasi, dalam artian jaringan saraf
mampu memberikan hasil dari input yang tidak diketahui sebelumnya. Berikut
adalah beberapa kemampuan yang dapat diberikan melalui penggunaan JST menurut S.
Haykin \cite{haykin-1994}:
\begin{enumerate}
  \item Non Linier : jaringan saraf dapat menangani permasalahan baik linier
  maupun non linier.
  \item Pemetaan Input-Output : dalam paradigma pembelajaran dengan arahan
  (supervised learning), modifikasi bobot disesuaikan dengan output yang
  diinginkan sebelumnya (label pada data sampel).
  \item Adaptif : jaringan saraf memiliki kemampuan untuk mengadaptasi bobot
  sinapsisnya sesuai dengan lingkungannya. Jaringan saraf pada umumnya melalui
  proses pembelajaran terhadap suatu lingkungan tertentu, dan dapat diajarkan
  kembali (re-train) untuk melakukan penyesuaian terhadap lingkungannya. 
  \item Toleransi terhadap kesalahan.
\end{enumerate}

Konsep JST dimodelkan secara matematis dan direpresentasikan melalui suatu unit
pemrosesan, yaitu neuron. terdapat tiga elemen dasar pada model neuron,
seperti yang terlihat pada \pic~\ref{fig:neuron} yaitu
\begin{itemize}
  \item Sinapsis, koneksi antar neuron dimana direpresentasikan dengan suatu
  bobot untuk menunjukkan kekuatan dari koneksi tersebut.
  \item Penjumlah, yang berfungsi untuk menjumlahkan sinyal, yang
  biasanya dalam hal ini perkalian antara bobot dengan sinyal masukan.
  \item Setiap neuron menerapkan fungsi aktivasi terhadap jumlah dari perkalian
  antara sinyal input dengan bobot neuron sebelumnya, untuk menentukan nilai
  output. Fungsi aktivasi ini pada umumnya membatasi nilai output dari neuron,
  menormalisasi output dalam range [0,1] atau [-1,1].
\end{itemize}

\addFigure{width=0.5\textwidth}{pics/neuron.png}{fig:neuron}{Model neuron non
linier \cite[p.~33]{haykin-1994}}

Paradigma pembelajaran JST secara umum dibagi menjadi dua kelompok utama yaitu
pembelajaran dengan pengarahan (supervised) dan pembelajaran tidak dengan
pengarahan (unsupervised).
\begin{itemize}
  \item Supervised learning : pembelajaran dengan pengarahan adalah hasil
  keluaran komputasi dari JST akan dibandingkan dengan hasil keluaran
  sesungguhnya, sehingga dengan selisih antara keduanya; proses penyesuaian
  bobot dalam jaringan dapat dilakukan. Untuk itu tipe ini memerlukan suatu data
  pelatihan yang berisikan data masukan serta target keluaran dari latihan. JST,
  tipe ini misalnya Multi Layer Perceptron, Learning Vector Quantization (LVQ), dll.

  \item Unsupervised learning : pembelajaran dengan tanpa pelatihan
  adalah proses pembelajaran JST dimana tidak memerlukan
  informasi target, cara pembelajarannya adalah jaringan akan menyesuaikan
  bobotnya tanpa campur tangan dari faktor luar dan berusaha menentukan sendiri
  masuk kedalam kelompok mana. Jaringan macam ini misalnya Kohonen
  Self-Organizing Maps (SOM).
\end{itemize}

\subsection{Pembelajaran berbasis kompetisi}

\subsubsection{LVQ}


% -----------------------------------------------------------------------------%
\section{Wavelet Transform}
\label{sec:wt}

Metode transformasi berbasis wavelet merupakan sarana yang dapat digunakan untuk
menganalisis sinyal non-stasioner, yaitu sinyal dengan kandungan frequensi yang
bervariasi terhadap waktu. Metode ini sangat populer dalam beberapa tahun
terakhir. Analisis wavelet dapat digunakan untuk menunjukkan kelakuan sementara
pada suatu sinyal, atau dapat juga digunakan untuk mem-filter, menghilangkan
sinyal data yang tidak diinginkan (\emph{noise}) dan atau meningkatkan kualitas
dari data. Selain itu, transformasi wavelet juga dapat digunakan untuk
memampatkan (\emph{compressing}) data \cite{Agfi:2006}.

Transformasi wavelet merupakan transformasi yang menggunakan kernel terintegrasi
yang dinamakan dengan wavelet. Wavelet dapat digunakan sebagai kernel
terintegrasi untuk analisis serta ekstraksi informasi suatu data dan juga
sebagai basis penyajian/karakteristik dari suatu data. Kelebihan dari analisis
sinyal menggunakan wavelet adalah terletak dari sifat terpenting wavelet yaitu
lokalisasi waktu-frekuensi (\emph{time-frequency localization}) yang
diistilahkan dengan \emph{compact support} \cite{guler:2005}. Dengan menggunakan
wavelet dapat dipelajari karakteristik sinyal secara lokal dan detil sesuai dengan skala-nya,
dimana hal ini menunjukkan kegunaan dari analisis pada sinyal non-stasioner,
karakteristik berbeda pada skala yang berbeda. Pada dasarnya, cara kerja dari
wavelet dalam mengekstraksi data adalah dengan menggunakan proses penguraian
(\emph{decomposition}) atau ekspansi deret, yaitu dengan cara ekspansi tak
berhingga dari wavelet yang diulur atau \emph{dilated} dan digeser atau
\emph{translated}.


Transformasi wavelet dikembangkan sebagi suatu alternatif pendekatan pada
Transformasi Fourier Waktu Pendek (Short Time Fourier Transform = STFT) dalam
mengatasi masalah resolusi. Pada STFT, skala dari jendela yang
digunakan bersifat konstan, sedangkan pada Transformasi wavelet, lebar dari
jendela akan berubah-ubah selama proses transformasi dilakukan dalam menghitung
masing-masing komponen spektrum. Dimana hal ini merupakan ciri khas dari
Transformasi Wavelet. Dengan karakteristik seperti itu, dengan transformasi
wavelet akan dapat diperoleh resolusi waktu dan frekuensi yang jauh lebih baik
daripada metode-metode yang lain.

Persamaan transformasi wavelet (WT) kontinu pada signal $f(x)$ dapat
didefinisikan sebagai berikut;
\begin{align}
\label{eq:wavelet_transform}
	W_{s}f(x) &= f(x) * \Psi_{s}(x) = \frac{1}{s}\int^{+\infty}_{-\infty}f(t)\Psi(\cfrac{x - t}{s})dt
\end{align}

\noindent dimana $s$ adalah faktor skala,  $\Psi_{s}(x) =
\cfrac{1}{s}\Psi(\cfrac{x}{s})$ adalah dilatasi dari fungsi jendela, yang
kemudian dikenal dengan wavelet penganalisis $\Psi(x)$, $t$ adalah  faktor
translasi/pergeseran dari fungsi jendela tersebut. Dari persamaan
\ref{eq:wavelet_transform} ditunjukkan bahwa fungsi jendela tersebut terdilatasi
maupun termampatkan berdasarkan faktor skala. Dengan skala yang rendah, maka
frekuensi tinggi akan terlokalisasi, sedangkan pada skala tinggi, yang
terlokalisasi adalah frekuensi rendah. 

\addFigure{width=1\textwidth}{pics/waveletcwt.png}{fig:cwt}{Contoh sinyal
nonstasioner dan hasil dari WT (kontinu)} 

Pada \pic~\ref{fig:cwt} ditunjukkan contoh sinyal nonstasioner dengan kandungan
frekuensi 250, 500, 750 dan 1000Hz dengan tambahan noise yang kemudian
ditransformasi menggunakan WT. Dari gambar tersebut terlihat bahwa frekuensi
250Hz muncul pertama kali yang diikuti dengan frekuensi yang lain. Ketika kita
akan menganalisis menggunakan wavelet, terdapat tradeoff resolusi antara waktu
dan frekuensi(pada dasarnya pada WT, direpresetnasikan dengan waktu-skala,
dimana skala dan frekuensi berbanding terbalik, semakin tinggi skala, maka
dapat melokalisasi frekuensi rendah dan sebaliknya). Jika skala tinggi
maka maka semakin tidak jelas resolusi frekuensi, namun
semakin jelas resolusi waktunya, dan sebaliknya.

\subsection{Descrete Wavelet Transform}
Pada dasarnya, nilai koefisien dari wavelet untuk jendela wavelet dan
signal tertentu menunjukkan seberapa dekat korelasi antara wavelet 
dengan bagian tertentu dari signal tersebut. semakin tinggi koefisien, maka akan
semakin mirip dimana hasilnya akan sangat tergantung dari bentuk wavelet yang
dipilih\cite{wavelet:matlab}. Namun jika perhitungan nilai koefisien dilakukan
untuk semua skala dan posisi (kontinu), maka akan dihasilkan jumlah data yang
sangat besar. Mallat \cite{Mallat:1989} mengembangkan suatu cara untuk
menghitung koefisien wavelet dengan mengambil sebagian saja dari skala dan
posisi berdasarkan pangkat-dua, yang kemudian dikenal dengan istilah
\emph{dyadic WT}. Hal ini juga dikenal dengan \emph{Discrete Wavelet
Transform (DWT)}. Dengan menggunakan \emph{dyadic WT} maka analisis akan lebih
efisien dan cukup  akurat. Algoritma Mallat ini menggunakan filter seperti dalam
skema \emph{two channel subband coder} dan dikenal juga dengan Transformasi
Wavelet Cepat (\emph{Fast Wavelet Transform}).

Dalam banyak kasus pemrosesan sinyal, kandungan frekuensi rendah adalah hal yang
penting karena memberikan indentitas dari sinyal yang bersangkutan. Kandungan
frekuensi tinggi hanya sebagai ``nuansa sinyal'' tambahan. Hal ini dapat
dianalogikan seperti sinyal suara manusia, jika komponen frekuensi tinggi
dihilangkan, maka suara akan berubah, namun masih mampu untuk mengetahui apa
yang diucapkan \cite{wavelet:matlab}. Hal inilah yang mendasari mengapa dalam
analisis berbasis wavelet banyak digunakan istilah aproksimasi dan detail.

Diberikan $s = 2^j$ dengan $j \in Z, Z$ set integer, maka dyadic WT dari sinyal
$f(x)$  dapat dihitung menggunakan algoritma Mallat sebagai berikut;

\begin{align}
	\label{eq:sfn}
	S_{2^{j}}f(x) &= \sum_{k \in~Z} h_{k}S_{2^{j-1}}f(x - 2^{j-1}k) \\
	\label{eq:wfn}
	W_{2^{j}}f(x) &= \sum_{k \in~Z} g_{k}S_{2^{j-1}}f(x - 2^{j-1}k)
\end{align}

\noindent dimana $S_{2^{j}}$ merupakan operator penghalus, $S_{2^{j}} f(n) =
a_{j}$. $a_{j}$  adalah koefisien dari frekuensi rendah (skala tinggi) yang
mengaproksimasi sinyal yang asli, sedangkan $W_{2^{j}} f(n) = d_{j}$, $d_{j}$ 
adalah koefisien frekuensi tinggi (skala rendah) yang merepresentasikan detail
dari sinyal asli.

\addFigure{width=0.8\textwidth}{pics/dwtdecompos.png}{fig:decompos}{Ilustrasi
proses dekomposisi, (a) Dekomposisi satu tingkat, (b) dekomposisi multi tingkat}

Pada \pic~\ref{fig:decompos}a dapat dilihat proses filtering wavelet dimana
$f(x)$ adalah sinyal asli, kemudian dilewatkan ke filter lolos-rendah
(\emph{lowpass})  lolos-tinggi (\emph{highpass}) dan menghasilkan dua sinyal,
\textbf{A}-proksimasi dan \textbf{D}-etail. 

Jika dekomposisi sinyal diteruskan secara iteratif untuk bagian aproksimasinya,
sehingga suatu sinyal dapat dibagi-bagi kedalam banyak komponen resolusi rendah,
maka proses ini dinamakan sebagai dekomposisi banyak tingkat atau
\emph{multilple-level decomposition} seperti yang ditunjukkan pada
\pic~\ref{fig:decompos}b.
 
Pemilihan fungsi jendela \emph{mother wavelet} dan jumlah level dekomposisi
dalam analisis menggunakan transformasi wavelet adalah sangat penting.
Pemilihan jumlah level dekomposisi berdasarkan atas komponen frekuensi
yang dominan dalam sinyal. Pemilihan yang tepat bertujuan untuk menjaga agar
bagian signal yang memiliki korelasi yang baik dengan frekuensi yang dibutuhkan
dalam proses pengenalan tetap ada pada koefisien wavelet. Terdapat berbagai
macam \emph{mother wavelet} yang dapat digunakan dalam analisis wavelet. Menurut
Senhadji \cite{Senhadji:1995}, proses dekomposisi dengan menggunakan wavelet
yang orthonormal akan memberikan informasi yang \emph{non-redundant}. Kelompok
wavelet orthonormal dan juga \emph{compactly supported} diantaranya adalah
Daubechies, Symlet dan Coiflet \cite{Agfi:2006}.


     
 
% calculated with Mallat algorithm as follows:
% 
% Informasi yang dapat
% diekstraksi menggunakan wavelet adalah 
% 
% 
% wavelet digunakan untuk melakukan filter frekuensi tinggi,
% dimana setelah didekomposisi, kemudian signal direkonstruksi kembali.
% 
% pada penelitian ini, wavelet digunakan untuk mereduksi fitur dari setiap beat
% dengan menggunakan coefisien aproksimasi-nya sebagai fitur, sesuai dengan
% tingkat level dekomposisi dan mengabaikan coefisien detail, dimana mengandung
% signal frekuensi tinggi.
% 
% 
% Transformasi Fourier
% Sampai sekarang transformasi Fourier mungkin masih menjadi transformasi yang
% paling populer di area pemrosesan sinyal digital (PSD). Transformasi Fourier
% memberitahu kita informasi frekuensi dari sebuah sinyal, tapi tidak informasi
% waktu (kita tidak dapat tahu di mana frekuensi itu terjadi). Karena itulah
% transformasi Fourier hanya cocok untuk sinyal stationari (sinyal yang informasi
% frekuensinya tidak berubah menurut waktu). Untuk menganalisa sinyal yang
% frekuensinya bervariasi di dalam waktu, diperlukan suatu transformasi yang dapat
% memberikan resolusi frekuensi dan waktu disaat yang bersamaan, biasa disebut
% analisis multi resolusi (AMR). AMR dirancang untuk memberika resolusi waktu yang
% baik dan resolusi frekuensi yang buruk pada frekuensi tinggi suatu sinyal, serta
% resolusi frekuensi yang baik dan resolusi waktu yang buruk pada frekuensi rendah
% suatu sinyal. Pendekatan ini sangat berguna untuk menganalisa sinyal dalam
% aplikasi-aplikasi praktis yang memang memiliki lebih banyak frekuensi rendah.
% Transformasi wavelet adalah suatu AMR yang dapat merepresentasikan informasi
% waktu dan frekuensi suatu sinyal dengan baik. Transformasi wavelet menggunakan
% sebuah jendela modulasi yang fleksibel, ini yang paling membedakannya dengan
% transformasi Fourier waktu-singkat (STFT), yang merupakan pengembangan dari
% transformasi Fourier. STFT menggunakan jendela modulasi yang besarnya tetap, ini
% menyebabkan dilema karena jendela yang sempit akan memberikan resolusi frekuensi
% yang buruk dan sebaliknya jendela yang lebar akan menyebabkan resolusi waktu
% yang buruk.
% -----------------------------------------------------------------------------%
\section{Himpunan Fuzzy}

Himpunan Fuzzy adalah suatu himpunan dimana elemennya memiliki derajat
keanggotaan. Himpunan Fuzzy merupakan generalisasi dari teori himpunan 
klasik. Secara formal, definisi dari himpunan fuzzy adalah sebagai berikut;

\begin{quotation}
	Suatu himpunan fuzzy $A$ pada semesta $X$ dengan $A \subset X$ dikarakterik
	oleh suatu fungsi keanggotaan (\emph{membership function}) $f_A(x)$  dimana
	setiap titik di $X$ dipetakan ke suatu bilangan real $[0,1]$, dengan nilai
	$f_A(x)$ menunjukkan derajat keanggotaan dari $x$ pada himpunan $A$. Sehingga
	semakin dekat nilai $f_A(x)$ dengan pusat (\emph{unity}), semakin tinggi
	derajatnya.
	\cite{Zadeh:1965}
\end{quotation}
  
Suatu himpunan fuzzy adalah kosong (\emph{empty}) jika dan hanya jika nilai
fungsi keanggotaannya adalah 0 pada $X$. Dua himpunan fuzzy $A$ dan $B$ adalah
sama, $A = B$ jika dan hanya jika $f_A(x) = f_B(x)$ untuk semua $x \in X$.
Sedangkan komplemen dari suatu himpunan fuzzy $A$ dinotasikan dengan $A'$
didefinisikan sebagai $f_{A'} = 1 - f_A$.

Seperti yang diuraikan diatas, suatu himpunan fuzzy dikarakterisasi dengan suatu
fungsi keanggotaan. Terdapat beberapa fungsi keanggotaan yang sering digunakan
dalam berbagai aplikasi sebagai berikut;

\begin{enumerate}
  \item Fungsi Keanggotaan segitiga.\\
  Fungsi keanggotaan segituga sangat umum dan banyak digunakan dalam pembuatan
  suatu sistem, karena kesederhanaannya dan untuk menyusunnya hanya membutuhkan
  3parameter semisal nilai minimum, rata-rata dan maksimum. Definisi dari fungsi
  keanggotaan segitiga adalah sebagai berikut;
  \begin{align}
	\label{eq:mftrim}
	f(x, a, b, c) = \left\{ 
	\begin{array}{ll}
	0 & , x \leq a\\
	\frac{x - a}{b - a} & , a < x \leq b \\
	\frac{c - x}{c - b} & , b < x < c \\
	0 & , x \geq c
	\end{array}
  \end{align}
   
   Bentuk fungsi keanggotaan segitiga dapat dilihat pada \pic~\ref{fig:mftrim}.
   \addFigure{width=0.6\textwidth}{pics/mftrim.png}{fig:mftrim}{Bentuk fungsi
   keanggotaan segitiga}
   
   \item Fungsi Keanggotaan Trapesium.\\
   Fungsi keanggotaan trapesium didefinisikan sebagai berikut;
   \begin{align}
	\label{eq:mftrap}
	f(x, a, b, c, d) = \left\{ 
	\begin{array}{ll}
	0 & , x \leq a\\
	\frac{x - a}{b - a} & , a < x \leq b \\
	1 & , b < x \leq c \\
	\frac{d - x}{d - c} & , c < x < d \\
	0 & , x \geq c
	\end{array}
  \end{align}
  
  Bentuk fungsi keangotaan segitiga dapat dilihat pada \pic~\ref{fig:mftrap}.
  \addFigure{width=0.6\textwidth}{pics/mftrap.png}{fig:mftrap}{Bentuk fungsi
   keanggotaan trapesium}
   
   \item Fungsi keanggotaan Gaussian.\\
   Fungsi keanggotaan Gaussian didefinisikan sebagai berikut;
	\begin{align}
	\label{eq:mfgaus}
		f(x, m, \sigma) = \exp\left(\frac{(x-x)^2}{2\sigma^2}\right)
	\end{align}

	dimana parameter $m$ dan $\sigma$ adalah pusat dan lebar dari fungsi
	keanggotaan. Bentuk dari fungsi keanggotaan Gaussian dapat dilihat pada
	\pic~\ref{fig:mfgaus}.
	\addFigure{width=0.6\textwidth}{pics/mfgaus.png}{fig:mfgaus}{Bentuk fungsi
	keanggotaan Gaussian}
	    
\end{enumerate}
	

% -----------------------------------------------------------------------------%
\section{Analisis Statistik}




\newpage
\section{Revision}
\begin{itemize}
  \item Cari paper yang menyatakan GLVQ menjamin konvergensi bobot
  \item Cari paper yang menyatakan GLVQ tidak sensitif terhadap inisialisasi
  bobot awal. ketemu -> \cite{Sato-1999}
  
\end{itemize}

\newpage
\section{Hasil presentasi Prof, Rudi di seminar reboan}

differentiable method like sigmoid is used for possible to find the minimization
of error.

error prediction
\begin{align}
\end{align}

cross entropy error function
\begin{align}
	E(W,V) = - \sum d_i \log p_i + (1 - d_i) \log(1 - p_i)
\end{align}
$d_i$ is desired output

Optimization method
\begin{itemize}
  \item Gradient descent / error backprop
  \item Conjugate gradient
  \item Quasi newton method
  \item Geneticalgorithm
\end{itemize}

neural network considered as well trained if it can predict training data and
cross validation separtely


Network Pruning : the way to remove unrelevant connection, so at the end we have
network with connection that is relevant
try to set a weight = 0 and if the performance is not affected, then it is more
likely tobe unrelevant

Rule extraction

Re-RX
Discreate / continous varaible
algorithm Re-Rx(S, D, C)

Card Dataset


AUC , ACC, 
\begin{align}
=\sum_{i=1}^{m}\sum_{i=1}^{m}
\end{align}

$AUC_d$ = $AUC$ = 1 - fp + tp / 2

%=============================================
% -----------------------------------------------------------------------------%

%-----------------------------------------------------------------------------%
\chapter{\babTiga}
%-----------------------------------------------------------------------------%
% \todo{tambahkan kata-kata pengantar bab 1 disini}
Pada bab ini akan dijelaskan mengenai langkah-langkah yang dilakukan dalam
mempersiapkan data fitur yang akan digunakan sebagai data pelatihan dan evaluasi
dari classifier yang dikembangkan. Dijabarkan pula mengenai dataset MIT-BIH
database yang digunakan sebagai sumber data dari keseluruhan penelitian ini.

%-----------------------------------------------------------------------------%
\section{ECG Dataset}
%-----------------------------------------------------------------------------%
Pada penelitian ini, data yang digunakan bersumber dari data yang tersedia
dengan bebas di physionet, yakni data MIT-BIH arrhytmia database. Data ini telah
banyak digunakan oleh para peneliti dalam melakukan investigasi mengenai kelaian
aritmia seperti yang telah diuraikan sebagian pada Bab~\ref{bab:pendahuluan}. Database
ini terdiri dari 48 data record yang didapat dari 47 subjek yang diteliti oleh
BIH arrhytmia Laboratory yang dilakukan antara tahun 1975 sampai 1979. 

\addFigure{width=0.6\textwidth}{pics/ecgsignalphysionet100.jpg}{fig:ecgphy}{Contoh
representasi Signal ECG.}

Setiap record data merupakan hasil perekaman dari 2 sandapan (lead) ECG dengan
durasi masing-masing $\pm$ 30 menit. Sandapan yang digunakan sebagian besar
adalah sandapan MLII dan sebagian sandapan V1/V2/V4/V5. Frekuensi cuplik
(sampling rate) yang digunakan dalam perekaman ECG ini adalah 360Hz.

Data MIT-BIH database direpresentasikan menggunakan suatu standar
format - \gls{wfdb}. sebuah rekam data ECG direpresentasikan menjadi 3 file
dengan tipe \textit{annotation}, \textit{signal data} dan \textit{header}. Cara untuk
membaca maupun menulis format data tersebut juga disediakan dalam situs
resminya. Pada \pic~\ref{fig:ecgphy} adalah contoh data signal yang dibaca
dengan menggunakan \gls{wfdb} toolbox di matlab.
 
\addFigure{width=0.7\textwidth}{pics/ecgsignalphysionetdata.jpg}{fig:ecgphydata}{Contoh
representasi data ECG.}

Seperti yang terlihat pada \pic~\ref{fig:ecgphy}, merupakan sinyal ECG selama
10 detik yang berasal dari data $100.dat$ dimana isi dari data tersebut dapat
dilihat pada \pic~\ref{fig:ecgphydata}. Secara keseluruhan, data pada $100.dat$
tersebut merupakan hasil rekam selama 30 menit 5.556 detik dari seorang pasien
berumur 69 tahun , Laki-Laki dengan diagnosa Aldomet, Inderal, atau sekitar
650.000 sample interval. Terdapat 3 kolom dimana yg pertama merupakan sample
capture interval time dari signal dengan frequensi 360, sehingga untuk 10 detik
mengandung 3600 sample. kolom kedua adalah besar amplitude (mV) dari lead II
(MLII) dan kolom ketiga adalah besar amplitude dari lead V5.

%-----------------------------------------------------------------------------%
\section{Baseline Wander Removal}
%-----------------------------------------------------------------------------%
Tahapan pertama dalam pemrossesan sinyal ECG adalah \gls{bwr}. Baseline wander
adalah suatu kondisi dimana signal ECG yang dihasilkan tidak berada pada garis
isoelectrik (garis sumbu), melainkan mengalami pergeseran keatas maupun
kebawah. hal ini dikarenakan aktifitas frekuensi rendah yang muncuk ketika 
proses perekaman dilakukan yaitu dari proses pernafasan maupun dari pergerakan
bagian tubuh. Hal ini dapat mengganggu proses analisis sinyal dan memungkinkan
terjadinya kesalahan dalam proses penerjemahan sinyal ECG,  salah satu
contoh-nya adalah perubahan gelombang ST-T pada ECG dimana  proses
penerjemahannya sangat tergantung pada garis isoelektrik.

Terdapat beberapa teknik yang digunakan untuk mereduksi noise frekuensi rendah
(baseline wander) mulai dari linear filtering seperti yang dikembangkan oleh
J.A. {Alst\'{e}} van \cite{Alst:1986} maupun polynomial fitting atau cubic
spline filtering seperti yang dikembangkan oleh Meyer dkk. \cite{Meyer:1977}.
Penggunaan linear filter mengakibatkan terjadinya distorsi gelombang ECG,
terutama dibagian antara PQ interval dan ST segment. Dengan menggunakan
nonlinear cubic spline interpolation, digabungkan dengan teknik pengurangan
(substraction technique) dapat mereduksi noise tanpa mempengaruhi gelombang ECG
secara signifikan.

Konsep dasar dari teknik polynomial fitting ini adalah dengan melakukan proses
estimasi pergeseran sumbu utama dengan menggunakan titik perwakilan pada
gelombang ECG, dengan satu titik untuk setiap beat. Pilihan terbaik yang
digunakan sebagai titik acuan (knot) adalah PQ interval. Estimasi kemudian
dilakukan sedemikian sehingga menginterpolasi setiap titik secara halus. Oleh
karena itu, sebelum langkah ini dilakukan, QRS complex harus dideteksi  terlebih
dahulu dan PQ interval sudah ditentukan.

\addFigure{width=0.7\textwidth}{pics/bwrsample.png}{fig:bwrsample}{Sinyal ECG
asli dan hasil proses BWR serta garis isoelectrik (estimasi)}

Metode interpolasi cubic spline adalah salah satu cara untuk fitting kurva pada
data eksperimental yang bentuk dari fungsinya maupun turunannya tidak diketahui.
Metode ini menggunakan polinomial pangkat tiga yang diasumsikan berlaku pada
titik-titik yang terletak di antara dua titik data yang diketahui. Fungsi yang
bersangkutan kemudian diaplikasikan pada semua titik-titik data yang ada,
sehingga didapatkan persamaan simultan, yang selanjutnya dapat diselesaikan
dengan menggunakan metode matriks. Estimasi pergeseran sumbu utama dilakukan
dengan interpolasi cubic spline karena interpolasi cubic spline menghasilkan
suatu pendekatan yang lebih halus dibandingkan dengan linier spline ataupun
kuadratik spline karena ada jaminan bahwa turunan pertama dan kedua adalah
kontinu pada seluruh selang. Interpolasi cubic spline merupakan pendekatan
sungsi yang diperoleh dengan mengunakan polinomial derajat tiga pada
masing-masing sub selang. Dalam kasus sinyal ECG sub selang yang kita gunakan
adalah PQ interval dari tiap-tiap beat.
 
\noindent Definisi dari cubic spline adalah sebagai berikut: \\
Diberikan titik-titik data $(t_1, A_1), (t_2, A_2),...(t_n, A_n)$. Dimana
titik data tersebut merupakan PQ interval dari sinyal ECG. Suatu cubic spline
$S$ yang menginterpolasi data yang diberikan memenuhi sifat-sifat berikut: 
\begin{enumerate}[(i)] 
  \item Dalam setiap selang $[x, x_{i+1}]$, dimana $i=1, 2, \dots, n-1$ , dan
  $S$ adalah polinomial derajat tiga. 
  \item $S(x_i)=f_i$, $i=1,2,...,n.$ 
  \item $S, S'$ dan $S"$ adalah kontinu di titik-titik dalam $x_2, x_3, \dots,
  x_{n-1}$.
\end{enumerate}
 
Dari definisi di atas, persamaan-persamaan yang terjadi pada cubic spline
diberikan sebagai berikut, pada selang yang ke-$i$ yaitu terletak diantara titik
$(x_i, f_i)$ dan $(x_{i+1}, f_{i+1})$, polinomial berderajat tiga yang memenuhi
adalah: 
\begin{align}
	\label{eq:polybwr}
	f = a_i(x-x_i)^3 + b_i(x-x_i)^2 + c_i(x-x_i) + d_i
\end{align}

Setelah diperoleh persamaan tersebut maka pergeseran sumbu utama bisa diestimasi
dan langkah selanjutnya adalah mencari sumbu utama yang sebenarnya atau iso
elektrik sinyal ECG dengan menentukan pivot isoelektrik untuk digunakan sebagai
pengurang sinyal ECG. 

Pada penelitian ini, proses BWR dilakukan menggunakan matlab package yang
dikembangkan oleh Gari Clifford \cite{Clifford:2005} yang bersifat open source.
Untuk lebih jelasnya, hasil dari proses BWR dapat dilihat pada
\pic~\ref{fig:bwrsample}.

%-----------------------------------------------------------------------------%
\section{Ekstraksi Beat}
\label{sec:ekstractbeat}
%-----------------------------------------------------------------------------%
Untuk melakukan pengenalan kelaianan aritmia, terdapat beberapa pendekatan yang
umum dilakukan oleh para peneliti dimana dapat dikelompokkan menjadi dua sebagai
berikut;
\begin{enumerate}
  \item Pendekatan beat : dimana pada metode ini rangkaian sinyal ECG 
  disegmentasi menjadi sekumpulan beat tunggal seperti yang dilakukan pada
  \cite{Zhao:2005, Ghongade:2007}. 
  \item Pendekatan interval waktu : pengenalan dilakukan dengan mendeteksi
  karakteristik interval waktu dari sinyal ECG seperti yang umum
  dilakukan oleh dokter seperti RR interval, PR interval, PR segment, QRS
  interval, ST segment, ST interval, QT interval.
\end{enumerate}

\addFigure{width=0.6\textwidth}{pics/ecgcutoff.png}{fig:ecgcutoff}{Ilustrasi
teknik segmentasi beat.}

Pada penelitian ini, pendekatan beat dengan segmentasi sinyal ECG dipilih karena
dataset sudah menyediakan anotasi tambahan mengenai posisi puncak gelompang R
(R-peak). Dengan informasi anotasi tersebut, asumsi terhadap lebar setiap beat
dibuat dengan pendekatan sekitar 300 sampel data dan ekstraksi beat dilakukan
dengan memposisikan puncak R sebagai pivot untuk setiap beat. Untuk setiap
puncak R, sinyal awal dipotong dimulai dari posisi $R-150$ sampai $R+149$,
sehingga didapat beat dengan lebar 300 sampel data. Ilustrasi dapat dilihat pada
\pic~\ref{fig:ecgcutoff}. Hasil dari proses segmentasi tersebut dapat dilihat
pada \pic~\ref{fig:ecgplot}.

\addFigure{width=0.6\textwidth}{pics/ecgplot.png}{fig:ecgplot}{Contoh  beat dari
5 jenis aritmia dan 1 beat normal.}

Dari proses segmentasi yang telah dilakukan didapat sejumlah beat yang
dilengkapi dengan label anotasi nya seperti yang ditunjukkan pada
\tab~\ref{tbl:beatstat}.

Dari \tab~\ref{tbl:beatstat} dapat dilihat jumlah beat untuk tiap label anotasi
tidak berimbang dan bahkan ada yang berjumlah sangat kecil, yaitu 2 beat untuk
jenis Supraventrikular premature beat. 



Sesuai dengan standar AAMI, seperti yang terlihat pada \tab~\ref{tbl:aami}, maka
jenis beat yang akan digunakan adalah 15 beat pertama yang ditunjukkan dengan
kotak hitam pada \tab~\ref{tbl:beatstat}. Karena jumlah tiga beat terakhir
sangat kecil, maka untuk saat ini hanya 12 kelas saja yang nantinya akan digunakan
dalam proses pengenalan. 

\addTableFromFigures{width=0.7\textwidth}{pics/beatstat.png}{tbl:beatstat}{Jumlah
Beat untuk setiap jenis anotasi pada MIT-BIH database}

\addTableFromFigures{width=1\textwidth}{pics/tblaami.png}{tbl:aami}{Pemetaan
jenis beat pada MIT-BIH Arrhytmia database dengan kelas beat pada
AAMI\cite{Philip:2004}}

%-----------------------------------------------------------------------------%
\section{Noise/Outlier Removal}
%-----------------------------------------------------------------------------%
Untuk menganalisis hasil ekstraksi beat yang dilakukan pada tahap sebelumnya
(sub-bab \ref{sec:ekstractbeat}), beat yang dihasilkan kemudian dikelompokkan
berdasarkan tipe masing-masing aritmia dan diplot menggunakan aplikasi matlab.
Empat jenis aritmia diplot seperti yang diilustrasikan pada
\pic~\ref{fig:plotecgoutlier}.

\addFigure{width=1\textwidth}{pics/plotecgoutlier.jpg}{fig:plotecgoutlier}{Plot
Empat tipe aritmia hasil proses ekstraksi fitur}

Dari hasil pengamatan terlihat bahwa terdapat beat ECG pada tiap-tiap tipe
aritmia yang berada diluar distribusi masing-masing kategori kelas, dimana
dalam hal ini diistilahkan dengan outlier. Outlier ini jika digunakan untuk
melatih jaringan saraf yang dibangun dapat menyebabkan ketidak-akuratan data
karena model akan mencoba mengakomodasi outlier tersebut, namun akan berimbas
pada kinerja dari jaringan saraf karena akan menurunkan tingkat pengenalan
terhadap beat tersebut. Oleh karena itu, beat yang terindikasi menjadi outlier
harus dihilangkan.

\begin{algorithm}
\scriptsize 
\caption{Mencari dan menghilangkan outlier beat}          
\label{alg:outlier}                           
\begin{algorithmic}                    % enter the algorithmic environment
	\FORALL {$Data_c$ in $C_{aritmia}$}
		\FOR{$k = 1 \to d$}
			\STATE $Q_1 = lower\ quartile $ \COMMENT{percentile ke-25} 
			\STATE $Q_3 = upper\ quartile $ \COMMENT{percentile ke-75}
			
			\STATE
			\STATE $IQR = Q_3 - Q_1$ \COMMENT{Hitung IQR}
			
			\STATE
			\STATE \COMMENT{Hitung tingkat ekstrimitas data sebagai pembatas outlier}
			\STATE $Low = Q_1 - 1.5 * IQR $ 
			\STATE $Up  = Q_3 + 1.5 * IQR $
			
			\STATE
			\STATE \COMMENT{Cari kandidat outlier untuk dimensi k}
			\FOR {$x = 1 \to N$}
				\IF {$x_k < Low\ and\ x_k > Up$}
				 	\STATE $Outlier \Leftarrow  x$ 
				\ENDIF
			\ENDFOR
		\ENDFOR
		\STATE $Data_c \Leftarrow Data_c \ni Outlier$
	\ENDFOR
\end{algorithmic}
\end{algorithm}

Untuk menghilangkan outlier pada tiap kelas, ada beberapa teknik yang dapat
dilakukan salah satunya adalah teknik Inter-Quartile Range (IQR). Pada umumnya
IQR digunakan untuk univariate data, tapi pada penelitian ini, teknik IQR
digunakan untuk multivariate data dengan asumsi penerapannya dilakukan tanpa
memperhitungkan korelasi antar fitur yang biasanya didapat dengan covariance
matrix. Algoritma untuk menghilangkan outlier dapat dilihat pada
algoritma~\ref{alg:outlier}.

\addFigure{width=0.7\textwidth}{pics/detectoutlier.png}{fig:detectoutlier}{Ilustrasi
Outlier detection menggunakan IQR}

Pada \pic~\ref{fig:detectoutlier}  dapat dilihat ilustrasi pendeteksian outlier
dengan menggunakan teknik IQR. Sampel data yang memiliki fitur berada diluar
rentang quartil yang ditentukan akan dianggap outlier, dalam hal ini sampel data
dengan fitur bertanda $+$ berwarna merah.

\addFigure{width=0.7\textwidth}{pics/hasiloutlierremoval.png}{fig:hasilremove}{Plot
beat RBBB setelah dilakukan proses penghapusan outlier beat}

Setelah dilakukan proses pendeteksian dan penghapusan outlier beat dengan
menggunakan algoritma~\ref{alg:outlier}, maka didapatkan hasil beat yang jika
diplot seperti yang ditunjukkan pada \pic~\ref{fig:hasilremove}

\addTableFromFigures{width=0.7\textwidth}{pics/beatstat-nonoutlier.png}
{tbl:beatstat-nonoutlier}{Data Statistik jumlah beat untuk setiap  kategori
arrhymia setelah dilakukan proses penghilangan outlier.}

Dari \pic~\ref{fig:hasilremove} dapat dilihat beat outlier yang terdapat pada
sinyal ECG di-kecualikan/dihapus dan menghasilkan kumpulan beat dengan sebaran
sesuai dengan rentang quartil yang ditentukan. Pada
\tab~\ref{tbl:beatstat-nonoutlier} dapat dilihat data statistik jumlah beat yang
dihasilkan untuk setiap tipe aritmia setelah dilakukan penghilangan outlier
pada data. Seperti yang telah disebutkan sebelumnya, karena jumlah beat yang
sedikit, untuk saat ini, proses pengenalan akan dilakukan hanya pada 12 kategori
pertama yang terdapat pada tabel tersebut (dapat dilihat pada bagian yang
didalam kotak hitam).

%-----------------------------------------------------------------------------%
\section{Ekstraksi dan Reduksi Fitur}
\label{sec:ekstrakwt}
%-----------------------------------------------------------------------------%
Ekstraksi fitur merupakan suatu bagian yang penting dalam sistem pngenalan pola
karena pemilihan fitur yang digunakan sangat mempengaruhi performa dari sistem
yang dikembangkan. Terdapat berbagai cara yang bisa digunakan untuk melakukan
ekstraksi fitur, salah satunya adalah dengan menggunakan \emph{Discrete Wavelet
Trasform} (DWT). 

Seperti yang telah diuraikan pada sub-bab \ref{sec:wt}, disebutkan
bahwa kegunaan dari wavelet itu adalah (1) Untuk mengkstraksi informasi dari
suatu data sinyal, (2) Untuk melakukan kompresi data, (3) Untuk
memfilter/membersihkan data sinyal. Yang pertama dilakukan untuk mendapat
informasi karakteristik dari sinyal tersebut seperti waktu kemunculan suatu
frekuensi tertentu pada sinyal. Yang kedua sangat jelas adalah untuk memampatkan
data, umumnya untuk keperluan transmisi dll. Yang ketiga adalah untuk memfilter,
misal frekuensi tinggi, yang terdapat pada sinyal tersebut. Umumnya langkah yang
dilakukan adalah dengan melakukan dekomposisi, aproksimasi dan detil, dimana
aproksimasi mengandung komponen frekuensi rendah dan detil mengandung komponen
frekuensi tinggi. Setelah dilakukan sampai level tertentu, kemudian sinyal
direkonstruksi kembali, sehingga akan menghasilkan sinyal dengan mereduksi
kandungan frekuensi tinggi. Jadi disini terdapat langkah
dekomposisi-rekonstruksi.

\addFigure{width=0.5\textwidth}{pics/db8.png}{fig:db8}{Ilustrasi dekomposisi 5
level dengan menggunakan \emph{db8}.}

Di bidang biomedical, transformasi wavelet dapat digunakan untuk mengekstraksi
dan mendeteksi kemunculan gelombang P, Q, R, S, T maupun time interval antar
gelombang seperti yang dilakukan Zheng dkk\cite{Zheng:1995} dan Haque dkk
\cite{Haque:2002}. Namun, pada penelitian ini transformasi wavelet diskrit 
akan digunakan untuk mengekstraksi dan mereduksi fitur dari data dasar
yang dihasilkan dari ekstraksi beat pada tahap sebelumnya, dimana tujuannya
adalah untuk mencari fitur yang dapat merepresentasikan pola dengan baik.
\emph{Mother wavelet} yang akan digunakan adalah wavelet daubechies dimana Guler
dkk.\cite{Guler:2005} telah menunjukkan bahwa proses ekstraksi fitur menggunakan
wavelet daubechies memberikan hasil yang lebih baik dibandingkan dengan wavelet
orthogonal yang lain.

Pada proses ekstraksi fitur ini, ECG beat akan didekomposisi secara bertahap
dari level 1 sampai dengan level 5, yang artinya proses dekomposisi akan
menghasilkan lima komponen detail $d_1,\dots, d_5$ dan salah satu dari
aproksimasi $a_1, \dots, a_5$, tergantung dari level dekomposisi yang akan
dipakai. Jika kita memilih dekomposisi level 5, maka akan didapat $a_5,
d_1,\dots,d_5$, dan kemudian dipilih koefisien yang tepat untuk
merepresentasikan sinyal dengan baik dimana pada dasarnya koefisien wavelet ini
merupakan representasi distribusi energi dari sinyal dalam dimensi waktu dan
frekuensi. Ilustrasi proses dekomposisi dengan jumlah koefisien yang dihasilkan
dengan menggunakan wavelet daubechies \emph{db8} dapat dilihat pada
\pic~\ref{fig:db8}.

Pada \pic~\ref{fig:norm-wavelet} dapat dilihat komponen aproksimasi tiap
level dari hasil dekomposisi sinyal ECG dengan menggunakan wavelet daubechies
\emph{db8} sebanyak 5 level. Sedangkan pada \pic~\ref{fig:norm-wavelet-det}
adalah komponen detail dari proses dekomposisi tersebut.

\addFigure{width=0.5\textwidth}{pics/db8-stat.png}{fig:db8-stat}{Ilustrasi
  penentuan fitur dengan pendekatan statistik.}
  
Pada penelitian ini, ekstraksi fitur menggunakan dua pendekatan yang nantinya
akan diujicobakan pada tahap selanjutnya, diantaranya adalah;
\begin{enumerate}
  \item Fitur Aproksimasi: fitur akan diekstrak dari setiap level dekomposisi,
  dan hanya memilih komponen aproksimasi pada setiap level. Sehingga akan
  dihasilkan 5 model data fitur yaitu 
  \begin{itemize}
    \item Fit1($a_1$) dengan 157 fitur , 
    \item Fit2($a_2$) dengan 86 fitur, 
    \item Fit3($a_3$) dengan 50 fitur,
    \item Fit4($a_4$) dengan 32 fitur, 
    \item Fit5($a_5$) dengan 23 fitur 
  \end{itemize}
   
  \item Fitur Statistik: dari 5 level dekomposisi, akan dihasilkan
  $a_5,d_1,\dots,d_5$. Semua komponen koefisien yang dihasilkan akan digunakan.
  Karena jumlah fitur menjadi 371 (23+157+86+50+32+23) maka untuk mereduksi
  jumlah fitur, dicari fitur statistik untuk setiap komponen,  yakni \emph{min},
  \emph{mean}, \emph{max} dan \emph{standard deviasion}. Karena  ada enam
  komponen wavelet, satu aproksimasi dan lima detail, maka akan  didapat Fit6
  dengan jumlah 24 fitur data. Ilustrasi lebih jelas dapat dilihat pada
  \pic~\ref{fig:db8-stat}.
\end{enumerate} 

\clearpage

\addFigure{height=0.7\textheight}{pics/norm-wavelet.png}{fig:norm-wavelet}{Contoh
sinyal Normal ECG hasil dekomposisi (aproksimasi) tiap level dengan menggunakan
wavelet daubechies8} 

\clearpage

\addFigure{height=0.7\textheight}{pics/norm-wavelet-det.png}{fig:norm-wavelet-det}{
Contoh sinyal Normal ECG hasil dekomposisi (detail) tiap level dengan
menggunakan  wavelet daubechies8}

\clearpage

% Input selection has two
% meanings: (1) which components of a pattern, or (2) which
% set of inputs best represent a given pattern. The computed
% discrete wavelet coefficients provide a compact representation
% that shows the energy distribution of the signal in time
% and frequency

% \newpage
% \begin{enumerate}
%   \item Apakah mungkin penyebab kegagalan pendeteksian unknown karena outlier
%   nya menggunakan IQR
%   \item Dari buku   Statistical Pattern Recognition (p414)
% 	We now consider the problem of detecting outliers in multivariate data. This is one of
% 	the aims of robust statistics. Outliers are observations that are not consistent with the
% 	rest of the data. They may be genuine observations (termed discordant observations by
% 	Beckman and Cook, 1983) that are surprising to the investigator. Perhaps they may be
% 	the most valuable, indicating certain structure in the data that shows deviations from
% 	normality. Alternatively, outliers may be contaminants, errors caused by copying and
% 	transferring the data. In this situation, it may be possible to examine the original data
% 	source and correct for any transcription errors. 
% \end{enumerate}

%-----------------------------------------------------------------------------%
\chapter{\babEmpat}
%-----------------------------------------------------------------------------%
Pada bab ini akan diuraikan mengenai sekilas metode pelatihan LVQ, metode
klasifikasi yang dikembangkan, modifikasi yang dilakukan serta perancangan dan
implementasi dari metode tersebut.

%-----------------------------------------------------------------------------%
\section{Metode Pelatihan LVQ}
%-----------------------------------------------------------------------------%
\subsection{Sistem Epoch}
Dalam penelitian ini, studi dilakukan terhadap algoritma LVQ yang dikembangkan
oleh Kohonen dkk \cite{Kohonen92lvqpak}. Dari hasil analisis terhadap
implementasi algoritma tersebut, yakni LVQ PAK package, \saya menyadari bahwa
mekanisme iterasi (epoch) yang penulis pahami dan kebanyakan digunakan
dalam JST dengan yang ada pada paket program tersebut sedikit berbeda.
\begin{enumerate}
  \item Iterasi pada LVQ PAK, iterasi yang dilakukan untuk sekali proses
  pembelajaran (satu kali proses update bobot) dengan menggunakan satu data
  sampel. Hal ini akan berelasi dengan jumlah maksimal iterasi yang ditentukan
  sebagai parameter pembelajaran. Artinya jika jumlah data sampel yang digunakan
  sebanyak 100, sedangkan maksimal iterasi ditentukan sebanyak 50 kali, maka
  hanya 50 data sampel saja yang akan digunakan untuk proses pelatihan, dengan
  pemilihan data yang random maupun sequensial dan hanya akan terjadi 50 kali
  proses update bobot.
  \item Iterasi pada sistem epoch, makna dari satu iterasi adalah
  proses pelatihan dilakukan untuk semua data sampel yang diberikan. Jika data
  sampel yang diberikan sebanyak 100, maka jaringan saraf akan dilatih sebanyak
  100 kali untuk satu iterasi. Jika maksimum iterasi ditentukan sebanyak 50
  kali, maka akan setara dengan 1500 iterasi pada metode sebelumnya, atau 1500
  kali proses update bobot.
\end{enumerate}

Selain sensitif terhadap inisial bobot awal, LVQ standar tersebut juga sangat
sensitif terhadap jumlah iterasi yang dilakukan dalam proses pembelajaran. Jika
terlalu banyak iterasi pelatihan dilakukan, maka kecenderungannya adalah bobot
yang dihasilkan menjauhi bobot optimal (divergen)\cite{Sato:1995}.

\noindent Dalam penelitian ini, sistem iterasi yang akan digunakan adalah
mengikuti cara yang kedua diatas.
% 
% 
\subsection{Iterasi dengan data Round Robin}
Melihat karakteristik dari proses pembelajaran LVQ dimana aturan update pada
proses pembelajaran dilakukan secara sequensial, maka perilaku proses
pembelajaran akan dimodifikasi dari sisi urutan data training. Pada iterasi metode sebelumnya,
proses iterasi pembelajaran dalam satu epoch dilakukan sesuai  dengan urutan
data masukan yang diberikan, bisa berupa urutan sejumlah N sampel data kategori
1 diikuti dengan  M sampel data kategori 2 dan seterusnya, atau bisa juga berupa
urutan data yang diambil secara random terhadap kategori data. Pada penelitian ini, \saya
mencoba untuk menerapkan mekanisme round-robin dalam iterasi proses pembelajaran
LVQ dimana dalam mekanisme ini, urutan data sampel yang diberikan pada proses
pembelajaran jaringan saraf dipastikan selang seling untuk setiap kategori data
sejumlah $k$ kategori. Jadi urutan data sampel akan menjadi; $X_{1,c_1},
X_{1,c_2}, \dots, X_{1,c_k}, X_{2,c_1}, X_{2,c_2}, \dots, X_{2,c_k},
X_{n_1,c_1},X_{n_2,c_2}, \dots,X_{n_3,c_k}$ dimana $n_i$ adalah jumlah data
untuk kelas ke-$i$.

Dengan menggunakan mekanisme round-robin diharapkan dapat memperbaiki proses
pembelajaran secara keseluruhan.


% . Berikut pada algoritma \ref{alg:opsi3} dapat dilihat pseudocode dari
% mekanisme iterasi round-robin ini.
% 
% % \begin{algorithm}
% % \scriptsize
% % \caption{Mekanisme iterasi secara round robin}
% % \label{alg:opsi3}
% % \begin{algorithmic}                    % enter the algorithmic environment
% % 	\STATE \ldots
% % 	\STATE $max\_N \leftarrow findMaxNumOfDataInCategory()$
% % 	\FOR {$i=1 \to max\_N$}
% % 		\FORALL {$C$ in $Category$}
% % 			\IF {$i > nC$} continue \ENDIF
% % 
% % 			\STATE $sample \leftarrow X_{c,i}$
% % 			\STATE $train(codebook, sample)$
% % 			\STATE \ldots
% % 		\ENDFOR
% % 	\ENDFOR
% % 	\STATE \ldots
% % \end{algorithmic}
% % \end{algorithm}

%-----------------------------------------------------------------------------%
\section{Metode Fuzzy Neuro GLVQ}
%-----------------------------------------------------------------------------%
% Metode Fuzzy-Neuro GLVQ yang dikembangkan merupakan penggabungan dari metode
% Fuzzy-Neuro LVQ dengan Generalized LVQ, dimana hal ini dimotivasi oleh 
% Kelemahan dari FNLVQ yang sensitif terhadap inisialisasi bobot awal
% dan keunggulan dari GLVQ yang menjamin konvergensi dari prototipe selama proses
% pelatihan, dan juga tidak sensitif terhadap inisialisasi bobot awal. Sedangkan
% keunggulan dari FNLVQ adalah memiliki kemampuan untuk mengenali data
% \emph{unknown}.  Dengan penggabungan ini diharapkan dihasilkan metode yang tidak
% sensitif terhadap inisialisasi data awal dan juga memiliki kemampuan untuk
% mengenali \emph{unknown} data. Berikut akan diuraikan mengenai metode FNGLVQ.

\subsection{Konsep dasar}
Perbedaan yang mendasar dari karakteristik pengenalan arrhytmia berdasarkan beat
dengan pengenalan aroma adalah dari sisi data yang akan menjadi masukan
jaringan LVQ. Pada pengenalan aroma, data masukan merupakan data himpunan
fuzzy, dimana merupakan representasi dari ketidakpastian (\emph{fuzziness})
sensor dalam membaca informasi aroma, dimana hal ini berimplikasi pada model
dari vektor pewakil yang diimplementasikan juga dengan himpunan fuzzy. Sedangkan
pada pengenalan arrhytmia, distribusi data masing-masing kategori saling tumpang
tindih (\emph{overlaping}) satu sama lain, seperti yang dapat ditunjukkan pada
\pic~\ref{fig:overlap} sehingga diharapkan dengan menggunakan vektor pewakil
fuzzy, maka ketidakpastian pola suatu kelas dapat diturunkan. Selain itu, kunci
utama proses pengenalan \emph{unknown} data pada \gls{fnlvq} adalah terletak
pada model vektor pewakil dengan menggunakan fuzzy, dimana jika nilai
similarity antara vektor masukan dengan vektor pewakil adalah 0 (nol), maka
vektor masukkan tersebut polanya belum pernah diketahui oleh jaringan saraf
(\emph{unknown}).

\addFigure{width=0.6\textwidth}{pics/overlap.png}{fig:overlap}{Ilustrasi data
arrhytmia yang tumpang tindih (\emph{overlap}) antar kategori.}

Pada pengenalan arrhytmia yang dilakukan pada penelitian ini, yang dikenali
adalah beat dalam ECG, dimana kemunculan suatu kelainan beat ditandai dengan
morfologinya. Menurut dr. Jolanda Jonas, tidak semua kemunculan kelainan beat
menandakan seorang pasien menderita arrhytmia. Terdapat kelainan beat yang hanya muncul
sekali dalam data ECG dimana hal ini sering digunakan sebagai indikasi untuk
pemeriksaan lanjutan, seperti pemasangan alat observasi ECG 24 jam
(\emph{Holter ECG}). Sehingga himpunan fuzzy sebagai masukan
sistem, seperti pada sistem pengenalan aroma, tidak cocok untuk digunakan. Oleh
karena itu pada pengenalan arrhytmia ini, digunakan masukan berupa data
\emph{crisp}.
 
Seperti yang sudah diuraikan pada sub-bab \ref{ssec:glvq}, fungsi
diskriminan yang digunakan \gls{glvq} adalah menggunakan \emph{\gls{metric}}, 
yakni jarak euclidean, sehingga semakin kecil jarak antara input dengan vektor
pewakil, maka kedua vektor akan semakin mirip. Pada metode yang dikembangkan,
fungsi diskriminan yang digunakan adalah dengan menggunakan \emph{fuzzy similarity},
seperti yang digunakan \gls{FNLVQ} dalam aplikasi pengenalan aroma.

\addFigure{width=0.6\textwidth}{pics/overlap.png}{fig:overlap}{Ilustrasi
Perhitungan similarity crisp data dengan menggunakan fungsi keanggotaan
segitiga}

Pada fuzzy similarity, perhitungan kemiripan dilakukan dengan mencari derajat
keanggotaan setiap fitur ($x_i$) terhadap fungsi keanggotaannya ($h_{ij}(x)$), 
dengan $i=$ fitur ke-$i$ dan $j=$ vektor pewakil kategori ke-$j$.
\begin{align}
	\mu_{ij} = h_{ij}(x_i)
\end{align}

Kemudian nilai derajat keanggotaan ($\mu_{j}$) vektor pewakil
(\emph{cluster}) dipropagasi ke neuron keluaran dengan menggunakan operasi
rata-rata (\emph{mean}).

\begin{align}
	\mu_j = \text{mean}_{\substack{j}} [\mu_{ij}]
\end{align}

Untuk menentukan pemenang (\emph{winner-take-all}), dipilih vektor
pewakil dengan nilai similarity ($\mu_j$) terbesar (\emph{max}).  
\begin{align}
	w_p = \max_j ( \mu_j )
\end{align}

dimana vektor pewakil pemenang akan di-update selama proses pembelajaran 
tergantung dari vektor masukan yang diberikan. Namun pada metode ini, vektor
pewakil yang akan di-update tidak hanya berdasarkan vektor pemenang saja,
melainkan ditentukan oleh \emph{missclassification error} dengan menghitung
jarak relatif antara jarak vektor masukkan($x$) dengan vektor pewakil dari
kelas yang sama ($C_x = C_w$) dan jarak terbesar vektor masukan dengan vektor
pewakil yang tidak berasal dari kelas yang sama ($C_x \neq C_{\max_{j}(w_j)}$).
Lebih jelas dapat dilihat kembali pada sub-bab \ref{ssec:glvq}.

\subsection{Integrasi GLVQ dan FNLVQ}
Pada fuzzy similarity, semakin besar nilainya, maka kedua vektor akan semakin
mirip. Untuk dapat menggunakan konsep similarity pada \gls{glvq}, maka dicari
nilai \emph{disimilarity}  dengan menggunakan persamaan $d = 1 - \mu$, dimana
$d$ adalah nilai jarak (\emph{disimilarity}). Kemudian disubstitusikan ke
\equ~\ref{eq:mce} didapat;

\begin{align}
\label{eq:}
	\varphi(x) &= \frac{(1 - \mu_1) - (1 - \mu_2)}{(1 - \mu_1) + (1 -
	\mu_2)}\nonumber\\
	&= \frac{\mu_2 - \mu_1}{2 - \mu_1 - \mu_2}
\end{align}

dengan $\varphi(x)$ adalah nilai \emph{miss-classification error (MCE)},
$\mu_1$ adalah nilai similarity vektor masukkan($x$) dengan vektor pewakil dari
kelas yang sama ($C_x = C_w$), dan $\mu_2$ adalah nilai similarity
terbesar antara vektor masukan dengan vektor pewakil yang tidak berasal dari
kelas yang sama ($C_x \neq C_{\max_{j}(w_j)}$).  Untuk dapat mengintegrasikan
teori fuzzy dengan \gls{glvq}, maka akan dilakukan penurunan  cost function
terhadap bobot $w$ sebagai berikut; (dengan mengacu pada cost function
\equ~\ref{eq:costS} dan update rule pada \equ~\ref{eq:genuprule})

\begin{align}
\label{eq:turunancostS}
	\frac{\delta S}{\delta w_i} =  
	\frac{\delta S}{\delta \varphi} . \frac{\delta \varphi}{\delta \mu}.
	\frac{\delta \mu}{\delta w_i}
\end{align}

% \begin{align}
% \label{eq:}
% 	\Psi(x) &= \frac{f(x)}{g(x)} \nonumber \\
% 	\frac{\delta \Psi}{\delta x} &=  
% 	\frac{f'(x)g(x) - f(x)g'(g)}{g(x)^2}
% \end{align}

\noindent Berikut adalah turunan dari masing-masing bagian turunan berantai pada
\equ~\ref{eq:turunancostS} adalah sebagai berikut; 
\begin{align}
\label{eq:turunanmce1}
	\frac{\delta \varphi}{\delta \mu_1} &= 
	\frac{\delta \left(\frac{\mu_2-\mu_1}{2-\mu_1-\mu_2}\right)}{\delta \mu_1} 
	\nonumber \\ \frac{\delta \varphi}{\delta \mu_1} &=  
	\frac{-1 . (2 - \mu_1 - \mu_2) - (-1).(\mu_2-\mu_1)}
	{(2 - \mu_1 - \mu_2)^2} \nonumber \\
	\frac{\delta \varphi}{\delta \mu_1} &=  
	-2.\frac{(1 - \mu_2)}{(2 - \mu_1 - \mu_2)^2}
\end{align}

\begin{align}
\label{eq:turunanmce2}
	\frac{\delta \varphi}{\delta \mu_2} &= 
	\frac{\delta \frac{\mu_2-\mu_1}{2-\mu_1-\mu_2}}{\delta \mu_2}  \nonumber \\
	\frac{\delta \varphi}{\delta \mu_2} &=  
	\frac{1 . (2 - \mu_1 - \mu_2) - (-1).(\mu_2-\mu_1)}
	{(2 - \mu_1 - \mu_2)^2} \nonumber \\
	\frac{\delta \varphi}{\delta \mu_2} &=  
	2.\frac{(1 - \mu_1)}{(2 - \mu_1 - \mu_2)^2}
\end{align}

$\frac{\delta \varphi}{\delta \mu_1}$ dan $\frac{\delta \varphi}{\delta
\mu_2}$ merupakan turunan MCE terhadap nilai similarity $\mu_1$ dan $\mu_2$.
Untuk mencari $\frac{\delta \mu}{\delta w_i}$, maka perhitungannya tergantung
dari fungsi keanggotaan yang digunakan pada setiap vektor pewakil. Pada
penelitian ini, fungsi keanggotaan $h(x)$ yang dipakai adalah fungsi segitiga,
seperti yang juga digunakan pada algoritma \gls{fnlvq}, karena fungsi segitiga adalah
fungsi yang paling sederhana untuk diimplementasikan, dan untuk mendapatkan
parameter-nya hanya membutuhkan nilai minimum, rata-rata dan maksimum yang
dihitung dari sebaran data pelatihan. Oleh karena itu, elemen vektor pewakil
pada algoritma ini akan direpresentasikan sebagai berikut;

\begin{align}
\label{eq:trimbobot}
	w_{ij} = (w_{min,ij}, w_{mean,ij}, w_{max,ij})
\end{align}

\noindent dimana $w_{ij}$ adalah vektor pewakil untuk fitur ke-$i$ dengan
kategori $j$, dan $w_{min,ij}, w_{mean,ij}, w_{max,ij}$ secara
berturut-turut nilai minimum, rata-rata dan maksimum dari distribusi data sampel
fitur ke-$i$ dengan kategori $j$. Untuk lebih menyederhanakan notasi, akan
digunakan $w_{min}, w_{mean}, w_{max}$ untuk mewakili notasi diatas.

\noindent Jika fungsi keanggotaan segitiga didefinisikan sebagai;
\begin{align}
\label{eq:trim}
	\mu = h(x, w_{min}, w_{mean}, w_{max}) = \left\{ 
	\begin{array}{ll}
	0 & , x \leq w_{min}\\
	\frac{x - w_{min}}{w_{mean} - w_{min}} & , w_{min} < x \leq w_{mean} \\
	\frac{w_{max} - x}{w_{max} - w_{mean}} & , w_{mean} < x < w_{max} \\
	0 & , x \geq w_{max}
	\end{array}
\end{align}

\noindent maka turunan dari fungsi segitiga diatas, dalam hal ini akan
diturunkan terhadap nilai rata-rata bobot ($w_{mean}$), didapat sebagai berikut;
\begin{itemize}
  \item Untuk nilai $x$ dengan kondisi $w_{min} < x \leq w_{mean}$ :
  \begin{align}
	\label{eq:trim1}
		\mu &= \frac{x - w_{min}}{w_{mean} - w_{min}} \nonumber \\
			&= (x - w_{min}) . (w_{mean} - w_{min})^{-1} \nonumber \\
		\frac{\delta \mu}{\delta w_{mean}} &=
		(x - w_{min}).(-1).(w_{mean} - w_{min})^{-2}.(1) \nonumber \\
		 &=
		- \frac{x - w_{min}}{(w_{mean} - w_{min})^2}
	\end{align}

	\item Untuk nilai $x$ dengan kondisi  $w_{mean} < x < w_{max}$ :
	\begin{align}
	\label{eq:trim2}
		\mu &= \frac{w_{max} - x}{w_{max} - w_{mean}} \nonumber \\
			&= (w_{max} - x) . (w_{max} - w_{mean})^{-1} \nonumber \\
		\frac{\delta \mu}{\delta w_{mean}} &=
		(w_{max} - x).(-1).(w_{max} - w_{mean})^{-2}.(-1) \nonumber \\
		 &=
		+ \frac{w_{max} - x}{(w_{max} - w_{mean})^2}
	\end{align}
	
	\item Untuk nilai $x$ dengan kondisi $x <= w_{min}\ \text{AND}\ x >= w_{max}$ :
	\begin{align}
	\label{eq:trim3}
		\mu &= 0 \nonumber \\
		\frac{\delta \mu}{\delta w_{mean}} &= 0
	\end{align}
\end{itemize}

Dari \equ~\ref{eq:trim1}, \ref{eq:trim2} dan \ref{eq:trim3} kemudian
disubstitusikan ke \equ~\ref{eq:genuprule} didapat aturan update pada proses
pembelajaran sebagai berikut;
\begin{itemize}
  \item Untuk nilai $x$ dengan kondisi $w_{min} < x \leq w_{mean}$ :
  	\begin{align}
	\label{eq:uprule11}
		w_1(t+1) \leftarrow & w_1(t) - \alpha .  
		\frac{\delta f}{\delta \varphi} . \frac{2.(1 - \mu_2)}{(2 - \mu_1 - \mu_2)^2}.
		\Bigg(\frac{x - w_{min}}{(w_{mean} - w_{min})^2}\Bigg) \\
	\label{eq:uprule12}
		w_2(t+1) \leftarrow & w_2(t) + \alpha .  
		\frac{\delta f}{\delta \varphi} . \frac{2.(1 - \mu_1)}{(2 - \mu_1 - \mu_2)^2}.
		\Bigg(\frac{x - w_{min}}{(w_{mean} - w_{min})^2}\Bigg) 
	\end{align}
  \item Untuk nilai $x$ dengan kondisi  $w_{mean} < x < w_{max}$ :
	\begin{align}
	\label{eq:uprule21}
		w_1 \leftarrow & w_1 + \alpha .  
		\frac{\delta f}{\delta \varphi} . \frac{2.(1 - \mu_2)}{(2 - \mu_1 - \mu_2)^2}.
		\Bigg(\frac{w_{max} - x}{(w_{max} - w_{mean})^2}\Bigg) \\
	\label{eq:uprule22}
		w_2 \leftarrow & w_2 - \alpha .  
		\frac{\delta f}{\delta \varphi} . \frac{2.(1 - \mu_1)}{(2 - \mu_1 - \mu_2)^2}.
		\Bigg(\frac{w_{max} - x}{(w_{max} - w_{mean})^2})\Bigg)
	\end{align}  
  \item Untuk nilai $x$ dengan kondisi  $x \leq w_{min}$ dan $x \geq w_{max}$ :
  	\begin{align}
	\label{eq:}
		w_i(t+1) \leftarrow & w_i(t) \qquad, i=1, 2     
	\end{align}  
\end{itemize}

\noindent dengan $w_1$ adalah vektor pewakil dari kelas yang sama dengan vektor
masukkan $x$ ($C_x = C_w$), dan $w_2$ adalah vektor pewakil dari kelas yang
berbeda dengan vektor masukkan dengan nilai similarity terbesar ($C_x \neq
C_{\max_{j}(w_j)}$). Proses update pada persamaan diatas (\ref{eq:uprule11},
\ref{eq:uprule12},\ref{eq:uprule21},\ref{eq:uprule22}) dilakukan pada $w_{mean}$
sedangkan $w_{min}, w_{max}$ mengikuti pergeseran dari $w_{mean}$. 
\begin{align}
\label{eq:upruleminmax}
	w_{min} \leftarrow & w_{mean}(t+1) - (w_{mean}(t) - w_{min}(t)) \\
	w_{max} \leftarrow & w_{mean}(t+1) + (w_{max}(t) - w_{mean}(t)) \nonumber \\
\end{align} 

Fungsi Monoton naik yang dipakai pada algoritma ini akan tetap sama dengan yang
dipakai pada \glvq standar, yakni menggunakan fungsi sigmoid, sehingga
$\frac{\delta \varphi}{\delta \mu}$ akan sama seperti pada
\equ~\ref{eq:deltasigmoid}. Sedangkan nilai laju pembelajaran $\alpha$ yang
digunakan adalah berkisar [0, 1] dan menurun seiring bertambahnya iterasi proses
pembelajaran. 
\begin{align}
\label{eq:alpha}
	\alpha(t+1) = \alpha_0 - (1 - \frac{t}{t_{max}})
\end{align}


Sebagai bagian dari karakteristik algoritma \gls{glvq} dimana terlepas dari
apakah jaringan saraf benar mengenali vektor masukan maupun tidak, vektor
pewakil $w_1, w_2$ keduanya secara simultan akan di-update. Namun pada algoritma
ini, selain melakukan penyesuaian $w_1$ dan $w_2$, dilakukan penyesuaian
tambahan seperti yang dilakukan pada \gls{fnlvq} yaitu proses penyesuaian
lebar interval dari fungsi keanggotaan tiap vektor pewakil. 
\begin{enumerate}
  \setlength{\itemsep}{1pt}
  \setlength{\parskip}{0pt}
  \setlength{\parsep}{0pt}
  \item Jika jaringan bisa mengenali dengan benar vektor masukan yang diberikan,
  maka fungsi keanggotaan akan diperlebar dengan harapan tingkat pengenalan-nya
  meningkat. 
  \item Sebaliknya jika jaringan salah mengenali vektor masukan, maka fungsi
  keanggotaan akan dipersempit dengan harapan tingkat pengenalan terhadap 
  vektor masukan menurun. 
\end{enumerate}

\noindent Kedua langkah ini hanya akan dilakukan jika nilai $\mu_1 > 0$ atau
$\mu_2 > 0$. Jika  kedua nilai similarity adalah 0, $\mu_1=0$ dan $\mu_2=0$, 
maka hal ini berarti semua vektor pewakil sama sekali tidak mengenali vektor 
masukan yang diberikan. Terdapat 2 kemungkinan; 
\begin{enumerate}
  \setlength{\itemsep}{1pt}
  \setlength{\parskip}{0pt}
  \setlength{\parsep}{0pt}
  \item Vektor masukan memang berada diluar distribusi dari kategori yang
  dikenali, 
  \item Interval dari fungsi keanggotaan (fuzzy) dari vektor pewakil
  terlalu sempit, sehingga tingkat pengenalan-nya rendah. 
\end{enumerate}

\noindent Karena ini merupakan proses pelatihan, maka asumsi adalah
yang ke-2, sehingga untuk membuat jaringan mengenali vektor pewakil, semua
interval dari fungsi keanggotaan vektor pewakil diperlebar. Berikut adalah
aturan perlebaran/penyempitan fungsi keanggotaan vektor pewakil yang telah
dijelaskan pada uraian diatas;
\begin{itemize}
  \setlength{\itemsep}{1pt}
  \setlength{\parskip}{0pt}
  \setlength{\parsep}{0pt}
  \item Jika  $\mu_1 > 0$ atau $\mu_2 > 0$, minimal salah satu dari kedua
  vektor pewakil mengenali vektor masukan:
  \begin{itemize}
  \setlength{\itemsep}{1pt}
  \setlength{\parskip}{0pt}
  \setlength{\parsep}{0pt}
    \item Jika pengenalan-nya benar ($\varphi < 0$), maka interval
    ketidakpastian (\emph{fuzziness}) dari vektor pewakil diperlebar
    	\begin{align}
    	w_{min} &\leftarrow w_{mean} - (w_{mean} - w_{min}) . (1 + (\beta .
    	\alpha)) \nonumber \\
    	w_{max} &\leftarrow w_{mean} + (w_{max} - w_{mean}) . (1 + (\beta .
    	\alpha)) 
    	\end{align}
    \item Jika pengenalan-nya salah ($\varphi \geq 0$), maka interval
    ketidakpastian (\emph{fuzziness}) dari vektor pewakil dipersempit
    	\begin{align}
    	w_{min} &\leftarrow w_{mean} - (w_{mean} - w_{min}) . (1 - (\beta .
    	\alpha)) \nonumber \\
    	w_{max} &\leftarrow w_{mean} + (w_{max} - w_{mean}) . (1 - (\beta .
    	\alpha)) 
    	\end{align}
  \end{itemize}
  
  Disini, nilai $\beta$ adalah diantara [0,1]. Pada studi kasus yang dilakukan
  disini, dipilih nilai $\beta = 0.00005$.
  
  \item Jika  $\mu_1=0$ dan $\mu_2=0$, yang artinya kedua vektor pewakil
  tidak mengenali vektor masukan, maka semua fungsi keanggotaan pada vektor
  pewakil diperlebar dengan aturan sebagai berikut:
  	\begin{align}
	w_{min} &\leftarrow w_{mean} - (w_{mean} - w_{min}) . (1 - (\alpha .
	\gamma)) \nonumber \\ 
	w_{max} &\leftarrow w_{mean} + (w_{max} - w_{mean})
	. (1 + (\alpha . \gamma)) 
	\end{align}
	
	Disini, nilai $\gamma$ adalah diantara [0,1]. Pada studi kasus yang dilakukan
	disini, dipilih nilai $\gamma = 0.1$.
\end{itemize}  


\noindent Berikut ini adalah algoritma FNGLVQ secara keseluruhan dalam bentuk
pseudocode;

\begin{algorithm}  
\scriptsize 
\caption{Algoritma FNGLVQ}          
\label{alg:lvq3}                           
\begin{algorithmic}                    % enter the algorithmic environment
	\STATE Initialize weight vector $W$
	\STATE Initialize learning rate $\alpha_0$
	\STATE Initialize maximum iteration $t_{max}$
	\STATE $t \leftarrow 0$
	\WHILE {$\alpha_t \neq 0$ or $t < t_{max}$}
		\STATE $x \leftarrow $ getNextSample()
		\STATE $train(W, x) \Downarrow$
		\STATE $\qquad \leadsto w \leftarrow $ getClosestPrototipe()
		\STATE $\qquad \leadsto $updatePrototipe($w$)
		\STATE $t \leftarrow t + 1$
	\ENDWHILE
	\STATE $w_1 \leftarrow $ ClosesDistanceWeighht($x, W$)
	\STATE $w_2 \leftarrow $ RunnerUpClosestDistanceWeight($x, W$)
	\STATE $d_1 \leftarrow distance(x, w_1)$
	\STATE $d_2 \leftarrow distance(x, w_2)$
	
	\IF {$C_{w_1} \neq C_{w_2}$}
		\IF {$\min \left(\frac{d_1}{d_2},\frac{d_2}{d_1}\right) > 
			 \frac{(1 - \omega)}{(1 + \omega)}$}
			\STATE $w_{1,t+1} \leftarrow w_{1,t} + \alpha_t . (x - w_{1,t})$
			\STATE $w_{2,t+1} \leftarrow w_{2,t} - \alpha_t . (x - w_{2,t})$
		\ENDIF
	\ELSE
		\STATE $w_{1,t+1} \leftarrow w_{1,t} + \epsilon.\alpha_t . (x - w_{1,t})$
		\STATE $w_{2,t+1} \leftarrow w_{2,t} + \epsilon.\alpha_t . (x - w_{2,t})$
	\ENDIF
	\STATE $\alpha \leftarrow $ getNextLearningRate()
\end{algorithmic}
\end{algorithm}  

%-----------------------------------------------------------------------------%
\section{Implementasi Sistem}
%-----------------------------------------------------------------------------%
\subsection{Lingkungan pengembangan}

\subsection{Arsitektur}

\subsection{Antar Muka Sistem}

%-----------------------------------------------------------------------------%
\chapter{\babLima}
%-----------------------------------------------------------------------------%
Pada bab ini akan dijelaskan mengenai percobaan, simulasi berbagai skenario
terhadap sistem yang dikembangkan yang dilanjutkan dengan analisis terhadap
hasil yang diperoleh untuk mengukur kinerja dari sistem, dibandingkan dengan
beberapa metode lain yang telah dicoba dan memperhitungkan tingkat kevalidan
percobaan terhadap hasil yang diperoleh.

%-----------------------------------------------------------------------------%
\section{Percobaan algoritma terhadap data fitur}
%-----------------------------------------------------------------------------%
Pada subbab ini akan diuraikan mengenai beberapa skenario ujicoba yang dilakukan
untuk mengetahui pengaruh ekstraksi fitur yang sudah dilakukan dan kinerja dari
algoritma pengenalan yakni LVQ1, LVQ21, GLVQ, FPGLVQ dan MGLVQ.

%-----------------------------------------------------------------------------%
\subsection{Percobaan terhadap data hasil ekstraksi beat}
%-----------------------------------------------------------------------------%
Pada percobaan ini, data hasil ekstraksi beat akan dicoba digunakan untuk
melatih algoritma pengenalan yang ada dan 

%-----------------------------------------------------------------------------%
\subsection{Percobaan terhadap data tanpa outlier}
%-----------------------------------------------------------------------------%

%-----------------------------------------------------------------------------%
\subsection{Percobaan terhadap fitur yang direduksi}
%-----------------------------------------------------------------------------%

%-----------------------------------------------------------------------------%
\subsection{Percobaan variasi urutan data terhadap algoritma}
%-----------------------------------------------------------------------------%
Pada percobaan ini, data pelatihan yang akan digunakan diurutkan secara
round robin terhadap kelas. Variasi yang dilakukan adalah dengan
membedakan jumlah data tiap kelas yang akan di saling silang, diantaranya;
\begin{enumerate}
  \item Pola#1 : jumlah data tiap kelas yang di saling silang adalah 1 data
  \item Pola#3 : jumlah data tiap kelas yang di saling silang adalah 3 data
  \item Pola#5 : jumlah data tiap kelas yang di saling silang adalah 5 data  
\end{enumerate}

Tujuan dari percobaan ini adalah untuk mengetahui bagaimana pengaruh pola
pelatihan terhadap kinerja dari classifier. Untuk uji coba ini 

urutan
data pelatihan yang akan digunakan yang round robin divariasikan mulai dari round robin 1 data tiap kelas

%-----------------------------------------------------------------------------%
\subsection{Percobaan algoritma terhadap variasi parameter}
%-----------------------------------------------------------------------------%

%-----------------------------------------------------------------------------%
\section{Analisis Hasil}
%-----------------------------------------------------------------------------%
1. Analisis bisa menggunakan T-Test dari 
Buku Combining Pattern Classifier
Error Calculation (Counting estimator p8)

2. McNemar Test
	- tentukan bobot terbaik antara LVQ1m LVQ21, GLVQ dan FPGLVQ
	- train dengan data 86 fitur, 6 kelas, simpan bobot nya
	- testing dengan mencatat setiap single data, catat untuk tiap classifier,
	dikenali atau tidak.


Untuk memberikan ranking terhadap algoritma, bisa menggunakan amalisis ROC 
\newpage
\section{Revision}
\begin{itemize}
  \item Why? wavelet
  \item Why? daubhechies
  \item Why? proses Outlier , baik baseline wander, maupun Outlier removal,
  knapa pake interquartile range? bukan mahalanobis?
  \item 
  \item 
\end{itemize}
%---------------------------------------------------------------
\chapter{\kesimpulan}
%---------------------------------------------------------------
% \todo{Tambahkan kesimpulan dan saran terkait dengan perkerjaan 
% 	yang dilakukan.}
Pada bab ini akan dijelaskan mengenai kesimpulan yang dapat diambil dari
penelitian yang telah dilakukan oleh \saya dan beberapa saran yang dapat
dipertimbangan dalam penelitian berikutnya.

%---------------------------------------------------------------
\section{Kesimpulan}
%---------------------------------------------------------------
Sebuah metode jaringan saraf tiruan berbasis kompetisi dengan nama
FN-GLVQ berhasil dikembangkan dan diuji-coba dengan menggunakan data arrhytmia
dari data MIT-BIH database. Pengujian dilakukan dengan menggunakan beberapa
skenario dan dilakukan uji statistik untuk menunjukkan keunggulan metode FNLVQ
terhadap GLVQ khususnya pada data kelainan arrhytmia. Dari hasil pengujian dan
analisis dapat ditarik kesimpulan sebagai berikut;
\begin{enumerate}
  \item Penelitian ini berhasil mengembangkan engine pengenalan arrhytmia
  dalam bentuk java library.
  \item Pengolahan data awal menggunakan wavelet daubechies order 8 (db8)
  memberikan hasil rata-rata terbaik dibandingkan dengan db2, db4 maupun
  db6, yakni 94.92\%. 
  \item Penghilangan Noise pada data awal arrhytmia dapat meningkatkan kinerja
  algoritma dalam proses pengenalan arrhytmia.
  \item Penggunaan mekanisme pengurutan data training menggunakan metode
  round-robin dapat meningkatkan level akurasi dengan nilai $i < 30$.
  \item Penggabungan teori fuzzy yang diadaptasi dari metode FNLVQ dengan
  GLVQ menjadi FN-GLVQ dapat meningkatkan kinerja dari pengenalan kelainan
  arrhytmia sampai 98.53\% untuk 6 kelas arrhytmia dan 96.33\% untuk 12 kelas.
  \item Metode FN-GLVQ masih sensitif terhadap inisialisasi bobot awal
  dibandingkan dengan metode GLVQ, namun dengan bobot yang tepat, misal bobot
  diambil dari data training, dapat memberikan hasil yang lebih baik.
  \item Penggunaan metode LVQ khususnya GLVQ dan FN-GLVQ pada data arrhytmia
  dapat memberikan tingkat pengenalan rata-rata diatas 97\% untuk 6 kelas dan
  sampai rata-rata 95\% untuk 12 kelas
\end{enumerate}

%---------------------------------------------------------------
\section{Saran}
%---------------------------------------------------------------
Pemrosesan data awal yang dilakukan pada penelitian ini masih menyisakan
beberapa kemungkinan-kemungkinan pelenitian lanjutan diataranya;
\begin{enumerate}
  \item Pada penelitian ini fitur diambil berdasarkan beat, kedepan, pengambilan
  data fitur ECG dapat dilakukan dengan mengambil fitur-fitur time  interval dan
  morfologi dengan menggunakan transformasi wavelet.
  \item Metode FN-GLVQ membutuhkan analisis lanjutan untuk mempelajari tingkat
  sensitifitas inisalisasi bobot awal yang belum bisa dicapai pada penelitian
  ini.
\end{enumerate}


%
% Daftar Pustaka
%
% Daftar Pustaka 
% 

% 
% Tambahkan pustaka yang digunakan setelah perintah berikut. 
% 

\bibliographystyle{ieeetr}
\bibliography{pustaka}

% 
% Lampiran 
%
% \setcounter{page}{102}    
%\begin{appendix}
% 	\pagestyle{empty}
	\include{markLampiran} 
	\chapter{Daftar Istilah}
\printglossary[type=main,title={Daftar Istilah},toctitle={Daftar Istilah}]
 
\chapter{Daftar Singkatan}
\printglossary[type=\acronymtype,title={Daftar Singkatan}, toctitle={Daftar
Singkatan}]

	\addChapter{Errata}
\chapter*{Errata}

Pada bagian ini akan diuraikan beberapa kesalahan penulisan yang muncul
pada buku tesis ini dan disertai dengan perbaikannya.

\subsubsection*{\textit{12 Agustus 2011}}
\begin{enumerate}
	\item Pada Persamaan \ref{eq:turunan1b} sub-bab 2.3.3, halaman 29, terjadi
	kesalahan penulisan tanda operasi ($+,-$). Pada persamaan tersebut tertulis :
	\begin{align}
	\label{eq-errataN:turunan1b}
	\frac{\delta S}{\delta w_2} =  
	\frac{\delta S}{\delta \varphi} \frac{\delta \varphi}{\delta d_2} \frac{\delta
	d_2}{\delta w_2} =
	- \frac{\delta f}{\delta \varphi} \frac{4d_1}{(d_1 + d_2)^2} (x - w_2) \nonumber
	\end{align}
 
 	Seharusnya tertulis (perhatikan tanda $-$, seharusnya $+$):
 	\begin{align}
	\label{eq-errataY:turunan1b}
	\frac{\delta S}{\delta w_2} =  
	\frac{\delta S}{\delta \varphi} \frac{\delta \varphi}{\delta d_2} \frac{\delta
	d_2}{\delta w_2} =
	+ \frac{\delta f}{\delta \varphi} \frac{4d_1}{(d_1 + d_2)^2} (x - w_2) \nonumber
	\end{align}
 	
 	Ini terjadi karena hasil penurunan persamaan \ref{eq:mce} terhadap $d_2$
 	menghasilkan $-\frac{4d_1}{(d_1 + d_2)^2}$ dan hasil penurunan $d_2$ terhadap
 	$w_2$ menghasilkan $-1$, sehingga menghasilkan persamaan seperti diatas 
 	dengan tanda $+$.
 	
 	\textbf{status} : sudah diperbaiki.
 	
 	\item Pada sub-bab \ref{ssec:glvq}, terjadi kesalahan penulisan 
 	mengenai konsep MCE. yang tertulis adalah : 
 	
 	``\ldots dari cost function, \emph{miss-classification error}, \ldots''
 	
 	dan 
 	
 	``\ldots \emph{miss-classification error} $\varphi(x)$ dapat dihitung \ldots''
 	
 	seharusnya adalah :
 	
 	``\ldots dari cost function, \emph{minimum classification error}, \ldots''
 	
 	dan 
 	
 	``\ldots \emph{minimum classification error} $\varphi(x)$ dapat dihitung
 	\ldots''
 	 
 	Hal ini juga terjadi pada bab \ref{ch:fnglvq}, khususnya pada sub-bab
 	\ref{ssec:konsep-dasar} dimana tertulis :
 	
 	``\ldots ditentukan oleh \emph{miss classification error} dengan menghitung
 	\ldots''
 	
 	dan juga tertulis pada sub-bab \ref{ssec:metode-fnglvq} sbb:
 	
 	``\ldots dengan $\varphi(x)$ adalah nilai \emph{miss-classification error
 	(MCE)} \ldots''
 	
 	seharusnya adalah :
 	
 	``\ldots ditentukan oleh \emph{minimum classification error} dengan
 	menghitung \ldots''
 	
	dan
	
	``\ldots dengan $\varphi(x)$ adalah nilai \emph{minimum classification error
 	(MCE)} \ldots''
	 	
	\textbf{status} : sudah diperbaiki.
\end{enumerate}
 
	%-----------------------------------------------------------------------------%

\addChapter{Tabel Distribusi t}
\chapter*{Tabel Distribusi t}
%-----------------------------------------------------------------------------%
\vspace{-1cm}
\begin{center}
\includegraphics[width=1\textwidth]{pics/tdist-1.png}
\end{center}
\clearpage


\begin{center}
\includegraphics[width=1\textwidth]{pics/tdist-2.png}
\end{center}
\clearpage
   
\addChapter{Hasil Ujicoba}
\chapter*{Hasil Ujicoba}

%\end{appendix}

\end{document}