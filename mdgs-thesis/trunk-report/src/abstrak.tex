%
% Halaman Abstrak
%
% @author  Andreas Febrian
% @version 1.00
%

\chapter*{Abstrak}

\vspace*{0.2cm}

\noindent \begin{tabular}{l l p{10cm}}
	Nama&: & \penulis \\
	Program Studi&: & \programstudi \\
	Judul&: & \judul \\
\end{tabular} \\ 

\vspace*{0.5cm}

\noindent

Aritmia atau cardiac aritmia merupakan salah satu penyakit jantung yang dapat
didiagnosa menggunakan standar ECG. dengan menggunakan ECG, para dokter dapat
menganalisis aktifitas elektrik jantung dan menentukan tipe dari aritmia yang
diderita oleh pasien. 
Pada penelitian ini, proses pengenalan aritmia dilakukan secara
otomatis menggunakan pendekatan jaringan saraf tiruan. Proses ini dibagi menjadi
tiga tahap yaitu; pemrosesan data, ekstraksi fitur dan proses pengenalan
oleh jaringan saraf. Pada proses pengolahan data awal, sinyal ECG disegmentasi
menjadi satuan beat dengan menggunakan puncak gelombang R sebagai pivot, dan
dilanjutkan dengan proses baseline wander removal dan outlier removal.
Transformasi Wavelet kemudian dilakukan untuk mengekstraksi dan mereduksi fitur.
Setiap beat tunggal kemudian diklasifikasi menjadi 6 dan 12 kelas menggunakan
metode baru yang dikembangkan disebut Fuzzy-Neuro Learning Vector
Quantization (FNGLVQ) yang merupakan adaptasi metode Fuzzy-Neuro kedalam GLVQ
yang dikembangkan oleh A.Sato. 

Dengan menggunakan validasi silang, hasil dari penelitian ini menunjukkan bahwa
rata-rata tingkat pengenalan beat aritmia 6 kelas menggunakan metode FN-GLVQ
sebesar 98.53\% dan untuk 12 kelas sebesar 96.33\% dimana metode yang
dikembangkan ini memberikan hasil yang lebih baik daripada GLVQ sebesar 97.03\%
dan 94.13\% untuk 6 kelas dan 12 kelas.
\\

\vspace*{0.2cm}

\noindent Kata Kunci: \\ 
\noindent Fuzzy-Neuro Generalized Learning Vector Quantization, FNGLVQ,
GLVQ, sistem pengenalan beat aritmia\\

\newpage