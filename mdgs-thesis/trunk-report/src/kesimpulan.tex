%---------------------------------------------------------------
\chapter{\kesimpulan}
%---------------------------------------------------------------
% \todo{Tambahkan kesimpulan dan saran terkait dengan perkerjaan 
% 	yang dilakukan.}
Pada bab ini akan dijelaskan mengenai kesimpulan yang dapat diambil dari
penelitian yang telah dilakukan oleh \saya dan beberapa saran yang dapat
dipertimbangan dalam penelitian berikutnya.

%---------------------------------------------------------------
\section{Kesimpulan}
%---------------------------------------------------------------
Sebuah metode jaringan saraf tiruan berbasis kompetisi dengan nama
FNGLVQ berhasil dikembangkan dan diuji-coba dengan menggunakan data Aritmia
dari data MIT-BIH database. Pengujian dilakukan dengan menggunakan beberapa
skenario dan dilakukan uji statistik untuk menunjukkan keunggulan metode FNLVQ
terhadap GLVQ khususnya pada data kelainan Aritmia. Dari hasil pengujian dan
analisis dapat ditarik kesimpulan sebagai berikut;
\begin{enumerate}
  \item Penelitian ini berhasil mengembangkan suatu metode pengenalan baru,
  FNGLVQ yang diimplementasikan dengan menggunakan java.

  \item Pengolahan data awal menggunakan wavelet daubechies order 8 (db8)
  memberikan hasil rata-rata terbaik dibandingkan dengan db2, db4 maupun
  db6, yakni 94.92\%. 

  \item Penggunaan mekanisme pengurutan data latih menggunakan metode
  \emph{round-robin} dapat meningkatkan tingkat akurasi dengan rentang jumlah
  data per kategori kelas $i < 30$.

  \item Penggabungan teori fuzzy yang diadaptasi dari metode FNLVQ dengan
  GLVQ menjadi FNGLVQ dapat meningkatkan kinerja dari pengenalan kelainan
  Aritmia mencapai 98.53\% untuk data 6 kelas dan 96.33\% untuk data 12
  kelas.

  \item Penggunaan metode LVQ pada data Aritmia dapat memberikan tingkat
  pengenalan rata-rata diatas 97\% untuk data 6 kelas dan 95\%
  untuk 12 kelas.

%   \item Penggabungan teori fuzzy yang diadaptasi dari metode FNLVQ dengan
%   GLVQ menjadi FNGLVQ dapat meningkatkan kinerja dari pengenalan kelainan
%   Aritmia rata-rata tingkat akurasi mencapai diatas 97\% untuk 6 kelas dan
%   rata-rata diatas 95\% untuk 12 kelas.

  \item Metode FNGLVQ masih sensitif terhadap inisialisasi bobot awal
  dibandingkan dengan metode GLVQ, namun dengan bobot yang tepat, misal
  diinisialisasi dengan data dari data latih, dapat memberikan hasil yang
  lebih baik.

  \item Data dengan sebaran yang tidak seimbang antar kelas dapat ditangani
  lebih baik dengan menggunakan FNGLVQ dibandingkan dengan metode LVQ1, LVQ21,
  GLVQ dan Backpropagation dengan nilai rata-rata \emph{recall, f-measure} dan
  \emph{g-mean} berturut-turut 86.23\%, 89.31\% dan 92.33\%.
\end{enumerate}

%---------------------------------------------------------------
\section{Saran}
%---------------------------------------------------------------
Pemrosesan data awal yang dilakukan pada penelitian ini masih menyisakan
beberapa kemungkinan-kemungkinan pelenitian lanjutan diataranya;
\begin{enumerate}
  \item Pada penelitian ini pendekatan analisis ECG yang digunakan adalah
  pendekatan pengenalan Aritmia berdasarkan beat dimana fitur
  diekstraksi menjadi kumpulan beat tunggal. Penelitian lanjutan yang dapat
  dilakukan adalah menggunakan pendekatan fitur time interval dan
  morfologi yang bisa didapat dengan menggunakan transformasi
  wavelet.
  \item Metode penghilangan pencilan (outlier) dapat dilakukan dengan
  menggunakan Mahalanobis distance.
  \item Metode FNGLVQ membutuhkan analisis lanjutan untuk mempelajari tingkat
  sensitifitas inisalisasi bobot awal yang belum bisa dicapai pada penelitian
  ini serta kemampuan dalam mengenal \emph{unknown} data.
\end{enumerate}
