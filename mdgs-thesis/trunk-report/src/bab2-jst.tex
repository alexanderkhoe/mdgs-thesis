\section{Jaringan Saraf Tiruan}
\label{sec:jst}

Jaringan Saraf Tiruan (JST) atau Artificial Neural Networks (ANN)
merupakan suatu sistem yang dibangun atas dasar cara kerja jaringan saraf
manusia. Awal perkembangannya dimotivasi oleh kemampuan pengenalan dari manusia
(otak) dimana cara perhitungannya sangat jauh berbeda dengan sistem komputer
digital. JSt merupakan sistem adaptif yang dapat mengubah strukturnya
untuk memecahkan masalah berdasarkan informasi eksternal maupun internal
yang mengalir melalui jaringan tersebut. 

Menurut S. Haykin \cite{haykin-1994}, sebuah jaringan saraf adalah sebuah
prosesor yang terdistribusi paralel dan mempuyai kecenderungan untuk menyimpan
pengetahuan yang didapatkannya dari pengalaman dan membuatnya tetap tersedia
untuk digunakan. Hal ini menyerupai kerja otak dalam dua hal yaitu: (1)
Pengetahuan diperoleh oleh jaringan melalui suatu proses belajar. (2) Kekuatan
hubungan antar sel saraf yang dikenal dengan bobot sinapsis digunakan untuk
menyimpan pengetahuan.

Jaringan saraf merupakan suatu mesin yang digunakan untuk memodelkan kerja otak
dalam menyelesaikan suatu permasalahan. Jaringan tersebut disusun dari
sekumpulan unit pemroses yang disebut neuron dan untuk meningkatkan
kemampuan-nya, dilakukan proses pembelajaran dengan menggunakan suatu algoritma
tertentu (learning algorithm) dimana tujuannya adalah untuk memodifikasi
kekuatan hubungan antar neuron (bobot) dalam jaringan sesuai dengan goal yang
telah ditentukan.

Keuntungan dari penggunaan JST adalah kemampuannya dalam beradaptasi melalui
proses pembelajaran dan kemampuan generalisasi, dalam artian jaringan saraf
mampu memberikan hasil dari input yang tidak diketahui sebelumnya. Berikut
adalah beberapa kemampuan yang dapat diberikan melalui penggunaan JST menurut S.
Haykin \cite{haykin-1994}:
\begin{enumerate}
  \item Non Linier : jaringan saraf dapat menangani permasalahan baik linier
  maupun non linier.
  \item Pemetaan Input-Output : dalam paradigma pembelajaran dengan arahan
  (supervised learning), modifikasi bobot disesuaikan dengan output yang
  diinginkan sebelumnya (label pada data sampel).
  \item Adaptif : jaringan saraf memiliki kemampuan untuk mengadaptasi bobot
  sinapsisnya sesuai dengan lingkungannya. Jaringan saraf pada umumnya melalui
  proses pembelajaran terhadap suatu lingkungan tertentu, dan dapat diajarkan
  kembali (re-train) untuk melakukan penyesuaian terhadap lingkungannya. 
  \item Toleransi terhadap kesalahan.
\end{enumerate}

Konsep JST dimodelkan secara matematis dan direpresentasikan melalui suatu unit
pemrosesan, yaitu neuron. terdapat tiga elemen dasar pada model neuron,
seperti yang terlihat pada \pic~\ref{fig:neuron} yaitu
\begin{itemize}
  \item Sinapsis, koneksi antar neuron dimana direpresentasikan dengan suatu
  bobot untuk menunjukkan kekuatan dari koneksi tersebut.
  \item Penjumlah, yang berfungsi untuk menjumlahkan sinyal, yang
  biasanya dalam hal ini perkalian antara bobot dengan sinyal masukan.
  \item Setiap neuron menerapkan fungsi aktivasi terhadap jumlah dari perkalian
  antara sinyal input dengan bobot neuron sebelumnya, untuk menentukan nilai
  output. Fungsi aktivasi ini pada umumnya membatasi nilai output dari neuron,
  menormalisasi output dalam range [0,1] atau [-1,1].
\end{itemize}

\addFigure{width=0.5\textwidth}{pics/neuron.png}{fig:neuron}{Model neuron non
linier \cite[p.~33]{haykin-1994}}

Paradigma pembelajaran JST secara umum dibagi menjadi dua kelompok utama yaitu
pembelajaran dengan pengarahan (supervised) dan pembelajaran tidak dengan
pengarahan (unsupervised).
\begin{itemize}
  \item Supervised learning : pembelajaran dengan pengarahan adalah hasil
  keluaran komputasi dari JST akan dibandingkan dengan hasil keluaran
  sesungguhnya, sehingga dengan selisih antara keduanya; proses penyesuaian
  bobot dalam jaringan dapat dilakukan. Untuk itu tipe ini memerlukan suatu data
  pelatihan yang berisikan data masukan serta target keluaran dari latihan. JST,
  tipe ini misalnya Multi Layer Perceptron, Learning Vector Quantization (LVQ), dll.

  \item Unsupervised learning : pembelajaran dengan tanpa pelatihan
  adalah proses pembelajaran JST dimana tidak memerlukan
  informasi target, cara pembelajarannya adalah jaringan akan menyesuaikan
  bobotnya tanpa campur tangan dari faktor luar dan berusaha menentukan sendiri
  masuk kedalam kelompok mana. Jaringan macam ini misalnya Kohonen
  Self-Organizing Maps (SOM).
\end{itemize}

\subsection{Pembelajaran berbasis kompetisi}
Pembelajaran berbasis kompetisi atau competitive based learning, adalah suatu
metode pembelajaran dimana neuron pada output layer berkompetisi satu sama lain
untuk menjadi aktif, diupdate dalam proses pembelajarannya. Dimana
dalam jaringan saraf berdasarkan Hebbian learning, beberapa neuron bisa aktif
secara simultan, dalam pembelajaran jenis ini, hanya satu neuron output yang
aktif dalam satu waktu. Terdapat satu neuron pemenang, dimana aturan ini dikenal
dengan istilah \textit{winner-take-all}. Terdapat beberapa metode JST yang
mengadopsi aturan ini diantaranya adalah SOM dan LVQ. 

\subsection{LVQ}
\glsreset{LVQ}
\Gls{LVQ}  

\subsubsection{LVQ1}

\subsubsection{LVQ2}

\subsubsection{LVQ2.1}

\subsubsection{LVQ3}

\subsection{GLVQ}

Dari beberapa uraian diatas, dapat diberikan ilustrasi mengenai pohon algoritma
seperti yang terlihat pada \pic~\ref{fig:pohon-alg}

\addFigure{width=0.8\textwidth}{pics/pohon-alg.png}{fig:pohon-alg}{Ilustrasi
pohon algoritma}
