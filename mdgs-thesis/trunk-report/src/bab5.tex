%-----------------------------------------------------------------------------%
\chapter{\babLima}
%-----------------------------------------------------------------------------%
Pada bab ini akan dijelaskan mengenai percobaan, simulasi berbagai skenario
terhadap sistem yang dikembangkan yang dilanjutkan dengan analisis terhadap
hasil yang diperoleh untuk mengukur kinerja dari sistem, dibandingkan dengan
beberapa metode lain yang telah dicoba dan memperhitungkan tingkat kevalidan
percobaan terhadap hasil yang diperoleh.

%-----------------------------------------------------------------------------%
\section{Ujicoba algoritma terhadap data fitur}
%-----------------------------------------------------------------------------%
Pada subbab ini akan diuraikan mengenai beberapa skenario ujicoba yang dilakukan
untuk mengetahui pengaruh ekstraksi fitur yang sudah dilakukan dan kinerja dari
algoritma pengenalan yakni LVQ1, LVQ21, GLVQ, FN-GLVQ. Algoritma LVQ3 tidak
diujicobakan pada bab ini karena fokus dari penelitian pada saat ini adalah
hanya untuk satu vektor pewakil untuk setiap kategori kelas. Sedangkan LVQ3,
seperti yang telah diuraikan pada sub-bab sebelumnya akan sama dengan LVQ2.1.
Ujicoba dilakukan dilingkungan dengan spesifikasi sebagai berikut; 
\begin{itemize}
  \setlength{\itemsep}{1pt}
  \item Komputer Intel i7-9500
  \item Memory 6Gb
  \item Sistem Operasi Windows 7
\end{itemize} 

Jenis Data yang akan diujicobakan ada dua kelompok yaitu data 6 kelas dan data
12 kelas. Berikut adalah rincian untuk masing-masing kelompok;
\begin{itemize}
  \setlength{\itemsep}{1pt}
  \item Data 6 kelas terdiri dari kategori penyakit arrhytmia;
  \begin{enumerate}
    \setlength{\itemsep}{1pt}
    \item Normal beat (NOR)
    \item Left Bundle Branch Block (LBBB)
    \item Right Bundle Branch Block (RBBB)
    \item Premature Ventricular Contraction (PVC)
    \item Paced beat (PACE)
    \item Atrial Premature beat (AP) 
  \end{enumerate}
  \item Data 12 kelas terdiri dari kategori 6 kelas diatas ditambah dengan;
  \begin{enumerate}
    \setlength{\itemsep}{1pt}
    \item Fusion of Ventricular and Normal beat (fVN)
    \item Fusion of Paced and Normal beat (fPN)
    \item Nodal (junctional) Escape beat (NE)
    \item Aberrated Atrial Premature beat (aAP) 
    \item Ventricular escape beat (VE)
    \item Nodal (juctional) Premature beat (NP)
  \end{enumerate}
\end{itemize}

Dari data hasil ekstraksi yang telah didapat berdasarkan
\tab~\ref{tbl:beatstat-nonoutlier}, untuk keperluan ujicoba ini hanya akan
dipakai maksimal 2000 sampel data untuk setiap kategori kelas arrhytmia, untuk
menurunkan faktor ketidak-seimbangan data. Meskipun demikian, untuk data
kategori 12 kelas, ketidak-seimbangan data akan tetap ada dan akan diujicobakan
terhadap algoritma yang ada. 

Pada uji coba ini, tingkat akurasi akan digunakan untuk menunjukkan kinerja dari
algoritma terhadap data fitur yang ada.
%-----------------------------------------------------------------------------%
\subsection{Ujicoba Skenario 1: mother wavelet daubhecies}
%-----------------------------------------------------------------------------%
Ujicoba ini dilakukan untuk mengetahui fitur terbaik yang dapat
merepresentasikan sinyal ECG dari 4 buah varian daubechies berdasarkan
tingkatannya yakni db2, sb4,sb6 dan db8. Mother wavelet daubhecies digunakan
karena seperti yang telah disampaikan diawal, banyak penelitian pada bidang ECG
telah menunjukkan bahwa daubechies lebih baik daripada wavelet ortogonal yang
lainnya.

Ujicoba ini akan dilakukan dengan mengggunakan 2 algoritma yaitu algoritma GLVQ
dan FN-GLVQ. Parameter ujicoba dapat dilihat pada \tab~\ref{tbl:paramset1}.
\begin{table}
\small  
\label{tbl:paramset1}
\caption{Parameter untuk uji kinerja dengan\\ fitur hasil berbagai jenis
wavelet daubechies}
\begin{center}
	\begin{tabular}{ccc}
	\hline
	 & GLVQ & FN-GLVQ\\ \hline
	 $\alpha$ & 0.05 & 0.05\\
	 epoch 	  & 150  & 150 \\
	 inisial bobot & random & random \\
	\hline
	\end{tabular}
\end{center} 
\end{table}

Data yang diujicobakan adalah data arrhytmia 6 kelas, tanpa outlier dengan
total jumlah data 11.134 dimana akan dibagi menjadi dua dengan porsi yang  sama
untuk training dan testing dan telah dilakukan proses round-robin 1 data. Sedangkan pada
\tab~\ref{tbl:statdatask1} ditunjukkan detail jumlah fitur untuk masing-masing
order daubechies setiap level dekomposisi.
\begin{table}
\small
\label{tbl:statdatask1}
\caption{Jumlah fitur untuk setiap dekomposisi\\ dari order daubechies yang
berbeda.} 
\begin{center}
	\begin{tabular}{ccccc}
	\hline
	& \multicolumn{4}{c}{daubechies wavelet}\\
	Level & db2 & db4 & db6 & db8\\\hline
	L1 & 151 & 153 & 155 & 157\\
	L2 & 77 & 80 & 83 & 86\\
	L3 & 40 & 43 & 47 & 50\\
	L4 & 21 & 25 & 29 & 32\\
	L5 & 12 & 16 & 20 & 23\\
	\hline
	\end{tabular}
\end{center}
\end{table}

Ujicoba dilakukan sebanyak 10 kali untuk setiap data dan akurasi dihitung dengan
mencari rata-rata dari 10 akurasi hasil percobaan. Dari ujicoba skenario
diatas didapatkan hasil akurasi untuk setiap algoritma dan data adalah seperti
yang ditunjukkan pada \tab~\ref{tbl:sk1-tab}.
\begin{table}
\small
\label{tbl:sk1-tab}
\caption{Hasil ujicoba dengan tingkat akurasi (dalam \%) dengan algoritma GLVQ
dan FN-GLVQ.}
\begin{center}
	\begin{tabular}{lccccclcccc}
	\cline{1-5} \cline{7-11}
	\multicolumn{5}{c}{GLVQ} & & \multicolumn{5}{c}{FN-GLVQ}\\
	Level & db2 & db4 & db6 & db8 & & Level & db2 & db4 & db6 & db8\\
	\cline{1-5} \cline{7-11}
	L1 & 95.24  &  96.51  &  96.27  &  96.68  &&  L1  &  97.49  &  98.13  &  97.86  &  98.09 \\
	L2 & 96.05  &  96.38  &  96.06  &  96.30  &&  L2  &  97.98  &  98.09  &  97.99  &  97.96 \\
	L3 & 93.60  &  93.22  &  93.43  &  93.67  &&  L3  &  97.52  &  97.74  &  97.38  &  97.64 \\
	L4 & 89.95  &  92.55  &  93.30  &  92.77  &&  L4  &  96.41  &  96.56  &  96.95  &  96.72 \\
	L5 & 86.99  &  81.39  &  85.75  &  85.07  &&  L5  &  93.89  &  94.35  &  91.67  &  94.28 \\ 
	\cline{1-5} \cline{7-11}
	avg & 92.37  &  92.01  &  92.96  &  92.90  &    & avg   &  96.66  &  96.97  &  96.37  &  96.94 \\
	\cline{1-5} \cline{7-11}
	\multicolumn{11}{c}{} \\
	 GLVQ  &  92.37  &  92.01  &  92.96  &  92.90  &    &    &  &  &  &  \\
  	 FN-GLVQ  &  96.66  &  96.97  &  96.37  &  96.94  &    &    &  &  &  &  \\
  	\cline{1-5}
     avg  &  94.51  &  94.49  &  94.67  &  94.92  &    &    & &  &  &   \\
	\cline{1-5}
	\end{tabular}
\end{center}
\end{table}

Dari \tab~\ref{tbl:sk1-tab} dapat dilihat bahwa rata-rata akurasi untuk
masing-masing tipe daubechies untuk setiap algoritma hampir sebandidng. Akurasi
untuk algoritma GLVQ rata-rata 92\% dan untuk FN-GLVQ adalah 96\%. Dari kedua
akurasi kemudian dicari rata-rata akurasi dan didapatkan rata-rata tertinggi
untuk daubechies order 8 (\emph{db8}) kemudian diikuti oleh \emph{db6}, 
\emph{db2} dan \emph{db4}. Namun pada dasarnya hasil yang diperoleh tidak
terlalu jauh berbeda. Pemilihan daubhecies order yang akan digunakan
nantinya tergantung dari resource yang tersedia nantinya. Pada
\pic~\ref{fig:sk1chart} ditunjukkan perbandingan hasil  ujicoba dalam bentuk
chart.

\addFigure{width=1\textwidth}{pics/sk1-chart.png}{fig:sk1chart}{Perbandingan
tingkat akurasi untuk beberapa family daubechies wavelet}

Berdasarkan hasil ujicoba diatas, untuk ujicoba berikutnya, akan digunakan
wavelet \emph{db8} sebagai default.

%---------------------------------------- -------------------------------------%
\subsection{Ujicoba Skenario 2 : 5 level dekomposisi db5 + data statstik}
%awalnya skenario 4
%-----------------------------------------------------------------------------%
Ujicoba ini dilakukan untuk mengetahui kinerja terbaik dari beberapa model
fitur yang sudah diuraikan pada sub-bab \ref{sec:ekstrakwt} khususnya untuk
daubechies order 8 (\emph{db8}) sesuai dengan yang dipilih pada skenario
sebelumnya. Ujicoba akan dilakukan dengan 4 algoritma yaitu LVQ1, LVQ21, GLVQ
dan FN-GLVQ. Pada \tab~\ref{tbl:paramsk2} dapat dilihat parameter yang digunakan
untuk keempat algoritma tersebut.

\begin{table}
\small  
\label{tbl:paramsk2}
\caption{Parameter untuk uji kinerja dengan\\ pada data fitur  hasil dekomposisi
daubechies 5 level + fitur statistik + sinyal asli}
\begin{center}
	\begin{tabular}{ccccc}
	\hline
	 & LVQ1 & LVQ21 & GLVQ & FN-GLVQ\\ \hline
	 $\alpha$ & 0.05 & 0.075  & 0.05 & 0.05\\
	 window & 0.005 & -  & - & -\\
	 epoch 	  & 20 & 100 & 150  & 150 \\
	 Init bobot & random & random & random & random \\
	 porsi init dtset   & ALL & ALL & ALL & 0.5 \\
	\hline 
	\end{tabular}
\end{center} 
\end{table}

Seperti pada ujicoba sebelumnya, eksekusi masing-masing algoritma dilakukan
sebanyak 10 kali untuk setiap data dan akurasi dihitung dengan mencari rata-rata 
dari 10 akurasi hasil percobaan. Ujicoba dilakukan terhadap data dengan 6 kelas
tanpa outlier dengan total jumlah data 11.134 dimana akan dibagi  menjadi dua
dengan porsi yang  sama untuk training dan testing dan telah dilakukan  proses
round-robin 1 data. Hasil dari ujicoba dapat dilihat pada
\tab~\ref{tbl:sk2-hasil}.

\begin{table}
\small
\label{tbl:sk2-hasil}
\caption{Hasil ujicoba dengan tingkat akurasi (dalam \%) dengan empat algoritma
LVQ1, LVQ21, GLVQ dan FN-GLVQ pada data fitur hasil dekomposisi daubechies 5
level + fitur statistik + sinyal asli.}
\begin{center}
	\begin{tabular}{cccccccc}
	\hline
	& \multicolumn{7}{c}{Fitures Model}\\
	  &  300  &  157  &  86  &  50  &  32  &  23  &  24 \\
	\hline
	LVQ1  &  77.39  &  78.15  &  77.50  &  77.72  &  75.23  &  74.57  &  77.92 \\
	LVQ21  &  86.82  &  91.15  &  92.10  &  89.23  &  88.65  &  85.19  &  92.21 \\
	GLVQ  &  96.39  &  96.20  &  95.39  &  94.69  &  92.58  &  85.24  &  93.58 \\
	FN-GLVQ  &  95.06  &  97.66  &  98.15  &  98.00  &  97.42  &  93.20  &  93.12\\
	\hline
 	avg  &  88.92  &  90.79  &  90.79  &  89.91  &  88.47  &  84.55  &  89.21 \\
	\hline
	\end{tabular}
\end{center}
\end{table}

Dari \tab~\ref{tbl:sk2-hasil} dapat dilihat nilai akurasi rata-rata antara
masing-masing model fitur dimana akurasi tertinggi dicapai oleh fitur dengan
jumlah 157 dan 86, sedangkan terendah pada fitur 23. Namun perbedaan antara
masing-masing model fitur, terutama untuk algoritma FN-GLVQ tidak terlalu
besar antara fitur 157, 86, 50 dan 32, yakni rata-rata 97-98\%. sedangkan
algoritma yang lain stabil di fitur 157 dan 86. Yang menarik disini adalah 3
algoritma menghasilkan fitur diatas 93\% untuk fitur 24, yakni LVQ21, GLVQ dan
FNGLVQ. 

Secara keseluruhan dapat ditunjukkan bahwa ekstraksi dan reduksi fitur dengan
dekomposisi menggunakan daubechies wavelet dapat mempertahankan tingkat akurasi
pengenalan arrhytmia, bahkan pada beberapa kasus diatas dapat meningkatkan
akurasi. Fitur dengan jumlah 86 dapat dipilih untuk digunakan dalam proses pengenalan
selanjutnya, karena dari \tab~\ref{tbl:sk2-hasil} dapat dilihat fitur 157 dan 86
memiliki rata-rata akurasi yang sama, namun fitur 86 memiliki hampir setengah
jumlah fitur 157. Namun seandainya jumlah fitur yang dibutuhkan nantinya
mensyaratkan jumlah yang kecil, Fitur 24 dapat dipertimbangkan untuk digunakan.

%-----------------------------------------------------------------------------%
\subsection{Ujicoba Skenario 3 : Round-robin data}
%awalnya skenario 5
%-----------------------------------------------------------------------------%
Pada percobaan ini, data pelatihan yang akan digunakan diurutkan secara
round robin terhadap kategori kelas arrhytmia seperti yang telah diuraikan
pada sub-bab \ref{ssec:round-robin}. Tujuan dari percobaan ini adalah  untuk
mengetahui bagaimana pengaruh round-robin data training terhadap proses
pembelajaran dari algoritma. Untuk uji coba ini, data yang akan digunakan adalah
data 6 kelas dengan fitur 300.  dan algoritma yang akan digunakan untuk
ujicoba adalah LVQ1, LVQ21, GLVQ dan FN-GLVQ. Berikut pada
\tab~\ref{tbl:sk3-param1} adalah parameter yang digunakan pada ujicoba ini;

\begin{table}
\small  
\label{tbl:sk3-param1}
\caption{Parameter untuk ujicoba round-robin data}
\begin{center}
	\begin{tabular}{ccccc}
	\hline
	 & LVQ1 & LVQ21 & GLVQ & FN-GLVQ\\ \hline
	 $\alpha$ & 0.05 & 0.05  & 0.05 & 0.05\\
	 window & 0.005 & -  & - & -\\
	 epoch 	  & 150 & 150 & 150  & 150 \\
	 Init bobot & random & random & random & random \\
	 porsi init dtset   & ALL & ALL & ALL & 0.5 \\
	\hline 
	\end{tabular}
\end{center} 
\end{table}

Pada ujicoba ini, dataset dipartisi menjadi 2 untuk training dan testing, dimana
jumlah maksimal data tiap kelas untuk data training adalah 1000. Sehingga untuk 
ujicoba ini dilakukan sebanyak 1000 iterasi, dimana setiap iterasi $i$ dilakukan
ujicoba round-robin dengan jumlah data $i$ yang saling silang. Pada
\pic~\ref{fig:sk4-hasil} ditunjukkan hasil ujicoba untuk keempat algoritma
tersebut diatas.

\addFigure{\width=0.7\textwidth}{pics/efekroundrobin.png}{fig:sk4-hasil}{Hasil
ujicoba data round-robin 1000 kombinasi.}

Dari \pic~\ref{fig:sk4-hasil} dapat dilihat masing-masing algoritma memberikan
hasil yang bervariasi terhadap kombinasi round-robin yang diujocobakan.
Algoritma LVQ1 masih tetap stabil untuk nilai approximasi $i$ dari 1 - 30,
setelah itu tingkat akurasi berfluktuasi cenderung menurun, bahkan sangat
signifikan. Untuk LVQ21, tingkat akurasi sangar bervariasi hampir disetiap nilai
$i$, sehingga dibutuhkan kehati-hatian dalam menentukan nilai dari $i$. Untuk
GLVQ, tingkat akurasi tetap stabil untuk nilai $i$ dari 1 - 130. kemudian secara
gradual menurun seiring bertambahnya nilai $i$. Sedangkan untuk FN-GLVQ, tingkat
akurasi stabil di rentang nilai $i$ yang hampir sama dengan GLVQ, yakni antara
1 - 130. Namun seiring dengan bertambahnya nilai $i$, tingkat akurasi
berfluktuasi naik turun, seperti yang ditunjukkan pada nilai $i=460$, tingkat
akurasi tinggi (94\%-95\%), sedangkan pada nilai $i=485$, akurasi kembali turun
ke level 75\%, dan pada nilai $i=881$, tingkat akurasi kembali naik ke level
92\% - 94\%.

Dari kondisi diatas dapat disimpulkan, dengan menggunakan mekanisme round-robin
dapat meningkatkan level akurasi pengenalan arrhytmia, khususnya untuk nilai $i$
yang tidak terlalu besar $i < 30$, hampir keempat algoritma memberikan nilai
akurasi yang stabil di level tertinggi.

%---------------------------------------- -------------------------------------%
\subsection{Ujicoba Skenario 4 : Parameter terbaik}
\label{ssec:paramterbaik}
%awalnya skenario 1
%-----------------------------------------------------------------------------%
Ujicoba ini dilakukan untuk mengetahui parameter yang dapat memberikan
kinerja pengenalan terbaik untuk setiap algoritma yang ada. Pada ujicoba ini,
dataset yang akan digunakan adalah data 6 kelas dan 12 kelas dari fitur 300
(fitur sinyal asli), fitur 86 dan fitur 24. Parameter yang akan digunakan,
dipilih sedemikian rupa sehingga dapat mewakili sebaran parameter yang ada dan
tidak terlalu presisi. Berikut pada \tab~\ref{tbl:sk4-param1} dapat dilihat
parameter yang akan diujicobakan pada empat algoritma.

\begin{table}
\small  
\label{tbl:sk4-param1}
\caption{Parameter yang akan diuji untuk proses pencarian parameter terbaik}
\begin{center}
	\begin{tabular}{cp{0.5\textwidth}}
	\hline
	 Parameter & nilai \\ \hline
	 $\alpha$ 	& 0.1, 0.075, 0.05, 0.01, 0.005, 0.001 \\
	 window 	& 0.1, 0.75, 0.05, 0.01, 0.005, 0.001\\
	 epoch 	  	& 20, 50, 100, 150\\
	 max-attempt& 1 \\
	 Init bobot & LVQ1, LVQ21, GLVQ (knn 1) - \newline 
	              FNLVQ (statistik) \\
	 porsi init dtset  & ALL \\
	\hline 
	\end{tabular}
\end{center} 
\end{table}

Pada ujicoba ini, setiap algoritma dicoba sebanyak 1 kali untuk setiap kombinasi
paramater. INisialisasi bobot yang digunakan adalah dengan menggunakan knn-1
pada semua data training yang digunakan, dimana artinya selama proses ujicoba
ini, vektor pewakil untuk LVQ1, LVQ21 dan GLVQ akan tetap sama. Sedangkan untuk
FN-GLVQ, inisialisasi bobot dilakukan dengan mencari nilai statistik (min, mean
dan max) untuk semua data training, sehingga vektor pewakil akan tetap sama
untuk semua kombinasi parameter. Dataset dipartisi menjadi 2 untuk
training dan sebagian untuk testing. Berikut pada \tab~\ref{tbl:sk4-hasil}
ditunjukkan hasil parameter terbaik yang didapatkan  disertai dengan akurasi
terhadap data testing.

\begin{table}
\small
\label{tbl:sk4-hasil}
\caption{Hasil ujicoba dengan tingkat akurasi (acc dalam \%) dengan empat
algoritma LVQ1, LVQ21, GLVQ dan FN-GLVQ dalam menentukan parameter terbaik
untuk 6 macam data fitur}
\begin{center}
	\begin{tabular}{cc|ccccccccc}
	\hline
	\multirow{2}{*}{Kelas} &  
	\multirow{2}{*}{Fit}  &  
	\multicolumn{2}{c}{LVQ1}  &
	\multicolumn{3}{c}{LVQ21}  &
	\multicolumn{2}{c}{GLVQ}  &
	\multicolumn{2}{c}{FN-GLVQ} \\ \cline{3-11}
    &    &  $\alpha$  &  acc  &  $\alpha$  &  width  &  acc  &  $\alpha$  &  acc  &  $\alpha$  & acc\\ 
    \hline
    \multirow{3}{*}{6}  &  300  &  0.05  &  76.04  &  0.075  &  0.01  &  90.03  &  0.05  &  96.69  &  0.005  &  98.01 \\
  &  86  &  0.05  &  75.70  &  0.075  &  0.005  &  90.30  &  0.05  &  96.57  &  0.05  &  98.02 \\
  &  24  &  0.075  &  77.47  &  0.05  &  0.005  &  93.50  &  0.1  &  92.69  &  0.001  &  93.89 \\
    \hline
    \multirow{3}{*}{12}  &  300  &  0.05  &  80.16  &  0.1  &  0.005  &  87.01  &  0.075  &  93.73  &  0.01  &  94.88 \\
  &  86  &  0.001  &  75.99  &  0.075  &  0.01  &  83.23  &  0.075  &  94.43  &  0.01  &  92.95 \\
  &  24  &  0.1  &  70.72  &  0.075  &  0.005  &  87.48  &  0.075  &  87.26  &  0.001  &  88.08 \\
	\hline
	\end{tabular}
\end{center}
\end{table}
 
Dari \tab~\ref{tbl:sk4-hasil} dapat dilihat parameter terbaik untuk setiap jenis
dataset terhadap algoritma yang digunakan. Dari setiap kombinasi parameter
terlihat menghasilkan tingkat akurasi yang berbeda-beda. Ini menunjukkan bahwa
karakteristik dari data juga akan menentukan parameter terbaik untuk setiap
algoritma. Sehingga pemilihan suatu jenis dataset diharapkan diikuti
dengan penyesuaian parameter yang digunakan untuk melakukan proses pembelajaran.

%---------------------------------------- -------------------------------------%
%\subsection{Ujicoba Skenario 5 : Data outlier}
%awalnya skenario 6
%-----------------------------------------------------------------------------%

%---------------------------------------- -------------------------------------%
\subsection{Ujicoba Skenario 5 : Inisialisasi Bobot Awal}
%awalnya skenario 2 dan 3
%-----------------------------------------------------------------------------%
Ujicoba ini dilakukan untuk mengetahui tingkat sensitifitas inisialisasi bobot
awal terhadap kinerja dari algoritma, yaitu GLVQ dan FN-GLVQ, dimana akan diukur
dari tingkat akurasi yang dihasilkan. Tingkat sensitifitas disini tidak
dimaksudkan bahwa bobot awal diinisialisasi dengan nilai sembarang jauh dari
distribusi data training nya, melainkan diinisialisasi dengan nilai random
dengan nilai yang tidak jauh dari distribusi dataset-nya. Pada ujicoba ini,
masing-masing algoritma akan diinisialisasi berdasarkan;
\begin{enumerate}
  \item GLVQ \\
  \textbf{Random Internal} : bobot awal diinisialisasi dari data training dengan
  dipilih secara random. \\
  \textbf{Random External} : bobot awal diinisialisasi dari nilai random
  interval [0, 1], dengan harapan bahwa jika diinisialisasi dengan cara seperti ini, tingkat
  akurasi tidak begitu jauh berubah.
  \item FN-GLVQ \\
  \textbf{Random Internal} : bobot awal diinisialisasi dari data training dengan
  menghitung agregasi data statistik (min, mean, max) dari setengah data
  training yang dipilih secara random. \\ 
  \textbf{Random External} :  bobot awal diinisialisasi dengan cara memilih nilai
  pusat fungsi keanggotaan ($w_{mean}$) secara random pada interval distribusi
  dari data training.
\end{enumerate}

Dataset yang akan digunakan pada ujicoba ini adalah data 6 kelas dan 12 kelas
dari fitur 300 (fitur sinyal asli), fitur 86 dan fitur 24. Parameter yang akan 
digunakan mengacu pada parameter terbaik yang dihasilkan pada skenario
sebelumnya yang dapat dilihat pada \tab~\ref{tbl:sk4-hasil}.

LVQ1 dan LVQ21 tidak diujicobakan disini karena sudah sangat jelas dari karakteristik
yang dimiliki dimana inisialisasi bobot awal sangat berpengaruh terhadap level 
akurasi algoritma. 

Dari Ujicoba yang telah dilakukan didapat hasil seperti yang ditunjukkan pada
\tab~\ref{tbl:sk6-hasil}.


\begin{table}
\small
\label{tbl:sk4-hasil}
\caption{Hasil ujicoba pengaruh inisialisasi data awal terhadap tingkat
akurasi (acc dalam \%) dengan algoritma GLVQ dan FN-GLVQ}
\begin{center}
	\begin{tabular}{cc|cccc}
	\hline
	\multirow{2}{*}{Kelas} &  
	\multirow{2}{*}{Fit}  &  
	\multicolumn{2}{c}{GLVQ}  &
	\multicolumn{2}{c}{FN-GLVQ} \\ \cline{3-11}
    &    &    RandInt  &  RandEks  &  RandInt  &  RandEks \\
    \hline
    \multirow{3}{*}{6}  &  300    & 96.62  &  96.03  &  98.10  &  97.61 \\
  &  86  & 96.02  &  96.72  &  98.06  &  97.05 \\
  &  24  & 93.18  &  92.74  &  94.41  &  83.70 \\
    \hline
    \multirow{3}{*}{12}  &  300  &  93.57  &  93.46  &  96.89  &  95.67 \\
  &  86  & 94.25  &  94.09  &  95.70  &  93.03 \\
  &  24  &  89.61  &  89.91  &  90.98  &  75.38 \\
	\hline
	\end{tabular}
\end{center}
\end{table}

 Untuk lebih memudahkan analisis, berikut pada \pic~\ref{fig:fig:sk6-hasil}
 dapat dilihat efek inisialisasi data awal terhadap kinerja dari algoritma.
 
 \addFigure{width=1\textwidth}{pics/sk6-hasil.png}{fig:sk6-hasil}{Hasil ujicoba
 inisialisasi bobot awal terhadap kinerja algoritma}.
 
 Dari \pic~\ref{fig:fig:sk6-hasil} dapat dilihat untuk algoritma GLVQ,
 inisialisasi random tidak mempengaruhi kinerja algoritma dalam melakukan
 pengenalan kelainan arrhytmia dimana di setiap model fitur data, tingkat
 akurasi tidak jauh berbeda (rata-rata error 0.0005). Namun berbeda dengan
 algoritma FN-GLVQ, untuk data dengan fitur 300 dan 86, tingkat akurasi
 antara kedua kondisi yang dibuat masih dapat mengimbangi dengan tingkat
 error rata-rata (0.013475). Namun untuk data fitur 24,  tingkat akurasi
 turun sangat drastis, sampai lebih dari 10\% (error rata-rata
 0.13155). Karena Tidak semua kondisi data dapat dipertahankan tingkat
 akurasinya, maka hal ini menunjukkan bahwa FN-GLVQ belum bisa
 menjamin ketidak sensitifan bobot terhadap kinerja algoritma untuk 
 semua kondisi data, hanya untuk kondisi-kondisi tertentu saja.  

%-----------------------------------------------------------------------------%
%\subsection{Ujicoba Skenario 7 : Pengenalan Unknown data}
%-----------------------------------------------------------------------------%

%-----------------------------------------------------------------------------%
\section{Uji Statistik}
%-----------------------------------------------------------------------------%
Untuk menunjukkan apakah modifikasi yang dilakukan dalam mengembangkan
metode baru FN-GLVQ memiliki kemampuan yang yang lebih baik secara signifikan
terhadap algoritma GLVQ, maka pada sub-bab ini akan dilakukan pengujian secara
statistik menggunakan 4 macam uji t-test seperti yang telah diuraikan pada
sub-bab \ref{sec:ujittest}. Untuk melakukan uji test statistik ini, data yang
akan digunakan adalah data %dengan fitur 86 untuk 6 kelas dan 12 kelas.  
6 dan 12 kelas dengan fitur 86. Pada uji test ini, dilakukan dengan menggunakan
parameter terbaik yang telah didapatkan pada skenario sebelumnya pada 
\tab~\ref{sk4-hasil}. 
 
\subsection{McNemar Test}
Pada uji ini, dataset dibagi menjadi dua dengan porsi yang sama untuk
testing dan training. Sedangkan inisialisasi bobot akan menggunakan cara yang
sama dengan ujicoba dalam mencari parameter terbaik pada sub-bab
\ref{ssec:paramterbaik}. Setelah dilakukan ujicoba didapatkan hasil
seperti yang ditunjukkan pada \tab~\ref{tbl:mcnemar-hasil};

\begin{table}
\small  
\label{tbl:mcnemar-hasil}
\caption{Hasil tabulasi uji test McNemar untuk 6 dan 12 kategori kelas, $D_1=$
GLVQ, $D_2=$ FN-GLVQ}
\begin{center}
	\begin{tabular}{ccccccc} 
	\cline{1-3} \cline{5-7}
	\multicolumn{3}{c}{INPUT/OUTPUT MATRIX} & & 
	\multicolumn{3}{c}{INPUT/OUTPUT MATRIX} \\
	\multicolumn{3}{c}{6 Kelas} & & 
	\multicolumn{3}{c}{12 Kelas} \\
	\cline{1-3} \cline{5-7}
	   $D_1$  &     $D_2$  &  N  &    &     $D_1$  &     $D_2$  &  N \\
	   \cline{1-3} \cline{5-7} 
		0  &  0  &  47  &    &  0  &  0  &  168 \\
		0  &  1  &  151  &    &  0  &  1  &  199 \\
		1  &  0  &  63  &    &  1  &  0  &  255 \\
		1  &  1  &  5306  &    &  1  &  1  &  5376 \\\cline{1-3} \cline{5-7}
		96.44  &  98.02  &    &    &  93.88  &  92.95  &   \\
		  &    &    &    &    &    &  \\
		  \cline{1-3} \cline{5-7}
		\multicolumn{3}{c}{CONFUSION MATRIX} & & 
		\multicolumn{3}{c}{CONFUSION MATRIX} \\
		\multicolumn{3}{c}{6 Kelas} & & 
		\multicolumn{3}{c}{12 Kelas} \\		
		\cline{1-3} \cline{5-7}  
		 &  $D_2$(+)  &  $D_2$(-)  &    &    &  $D_2$(+)  &  $D_2$(-) \\
		 \cline{1-3} \cline{5-7}       
		$D_1$(+)  &  5306  &  63  &    &  $D_1$(+)  &  5376  &  255 \\
		$D_1$(-)  &  151  &  47  &    &   $D_1$(-)  &  199  &  168 \\
		\cline{1-3} \cline{5-7} 
		  &    &    &    &    &    &  \\
		\multicolumn{3}{l}{McNemar Score 6 Kelas} & & 
		\multicolumn{3}{l}{McNemar Score 12 Kelas} \\  
		$X_6^2$[$D_1$,$D_2$] = 35.3692  &    &    &    &  $X_{12}^2$[$D_1$,$D_2$] = 
		6.6630 & & \\
	\end{tabular}
\end{center} 
\end{table}

Dari data McNemar Score pada tabulasi \ref{tbl:mcnemar-hasil}, didapat $X_6^2$
untuk 6 kelas adlah 35.4692, sedangkan $X_{12}^2$ untuk 12 kelas didapat 
6.6630. Level Signifikan yang digunakan untuk uji McNemar adalah 0.05 (95\%) 
dengan satu derajat kebebasan. Nilai tabulasi yang diperoleh dari tabel adalah
$12.706$. Untuk 6 Kelas, $H_0$ dapat ditolak, karena $\lvert X_6^2 \rvert >
12.706$, Namun untuk 12 kelas, nilai $\lvert X_6^2 \rvert < 12.706$, sehingga
$H_0$ tidak dapat ditolak. 

Dari uji ini, belum bisa ditarik kesimpulan salah satu dari algoritma lebih baik
dari yang lainnya.
 
\subsection{K-Hold Out paired t-test}
Pada uji ini, jumlah iterasi dilakukan sebanyak $K=30$ dimana dataset dipartisi
menjadi dua sama besar (50:50) untuk training dan testing.  Sedangkan
inisialisasi bobot menggunakan cara random internal dari dataset.  Setelah
dilakukan ujicoba didapat data tabulasi seperti yang ditunjukkan pada
\tab~\ref{tbl:khold-hasil};

Dari data KHold Score pada tabulasi \ref{tbl:khold-hasil}, didapat $t_6$
untuk 6 kelas adlah -76.672141, sedangkan $t_{12}$ untuk 12 kelas didapat 
-70.245154. 

Level Signifikan yang digunakan untuk uji KHold adalah 0.05 (95\%) 
dengan 29 ($K-1$) derajat kebebasan. Nilai tabulasi yang diperoleh dari tabel
adalah $2.045$. Untuk 6 Kelas, $H_0$ dapat ditolak, karena $\lvert t_6 \rvert >
2.045$. Dan untuk 12 kelas, nilai $\lvert t_{12} \rvert > 2.045$, sehingga $H_0$
juga dapat ditolak. Sehingga ini menunjukan bahwa perbedaan
akurasi dari kedua algoritma sangat signifikan dan dapat disimpulkan algoritma
FN-GLVQ memberikan hasil lebih baik dibandingkan dengan GLVQ.


\subsection{K-Fold Cross Validation paired t-test}
Pada uji ini, dataset yang digunakan dipartisi menjadi 10 ($K=10$)
kemudian dilakukan 10 kali iterasi dengan komposisi data training dan
testing (90:10). Sedangkan inisialisasi bobot menggunakan cara random internal 
dari dataset.  Setelah dilakukan ujicoba didapat data tabulasi seperti yang 
ditunjukkan pada \tab~\ref{tbl:kfold-hasil}.

Dari data KFold Score pada tabulasi \ref{tbl:kfold-hasil}, didapat $t_6$
untuk 6 kelas adlah -6.664332, sedangkan $t_{12}$ untuk 12 kelas didapat 
-10.255489. 


%============================================================================
% TABEL K HOLD
%============================================================================
\begin{table}
\small  
\label{tbl:khold-hasil}
\caption{Hasil tabulasi uji K-Hold untuk 6 dan 12 kategori kelas, $P_A=$
GLVQ, $P_B=$ FN-GLVQ, dengan nilai $K=30$}
\begin{center}
	\begin{tabular}{ccccccc} 
	\cline{1-3} \cline{5-7}
	\multicolumn{3}{c}{Akurasi Data} & & 
	\multicolumn{3}{c}{Akurasi Data} \\
	\multicolumn{3}{c}{6 Kelas} & & 
	\multicolumn{3}{c}{12 Kelas} \\
	\cline{1-3} \cline{5-7}
	   $K$  &     $P_A$  &  $P_B$  &    &     $K$  &     $P_A$  &  P_B \\
	   \cline{1-3} \cline{5-7} 
  K:0  &  97.32  &  98.82  &    &    K:0  &  94.52  &  97.28 \\
  K:1  &  97.49  &  98.80  &    &    K:1  &  94.30  &  97.33 \\
  K:2  &  97.43  &  98.76  &    &    K:2  &  94.42  &  97.12 \\
  K:3  &  97.56  &  98.83  &    &    K:3  &  94.23  &  97.25 \\
  K:4  &  97.56  &  98.83  &    &    K:4  &  94.60  &  97.10 \\
  K:5  &  97.50  &  98.78  &    &    K:5  &  94.47  &  97.53 \\
  K:6  &  97.59  &  98.87  &    &    K:6  &  94.13  &  97.42 \\
  K:7  &  97.36  &  98.87  &    &    K:7  &  94.30  &  97.47 \\
  K:8  &  97.50  &  98.83  &    &    K:8  &  94.50  &  97.28 \\
  K:9  &  97.50  &  98.71  &    &    K:9  &  94.58  &  97.47 \\
 K:10  &  97.58  &  98.89  &    &   K:10  &  94.15  &  97.43 \\
 K:11  &  97.50  &  98.92  &    &   K:11  &  94.17  &  97.32 \\
 K:12  &  97.41  &  98.78  &    &   K:12  &  94.50  &  97.33 \\
 K:13  &  97.49  &  98.85  &    &   K:13  &  94.52  &  97.25 \\
 K:14  &  97.54  &  98.78  &    &   K:14  &  94.35  &  97.23 \\
 K:15  &  97.50  &  98.76  &    &   K:15  &  94.43  &  97.30 \\
 K:16  &  97.50  &  98.83  &    &   K:16  &  94.68  &  97.07 \\
 K:17  &  97.47  &  98.83  &    &   K:17  &  94.58  &  97.38 \\
 K:18  &  97.49  &  98.78  &    &   K:18  &  94.78  &  97.43 \\
 K:19  &  97.49  &  98.82  &    &   K:19  &  94.48  &  97.18 \\
 K:20  &  97.56  &  98.85  &    &   K:20  &  94.42  &  97.40 \\
 K:21  &  97.58  &  98.92  &    &   K:21  &  94.47  &  97.18 \\
 K:22  &  97.11  &  98.80  &    &   K:22  &  94.20  &  97.27 \\
 K:23  &  97.56  &  98.90  &    &   K:23  &  94.48  &  97.12 \\
 K:24  &  97.49  &  98.87  &    &   K:24  &  94.28  &  97.08 \\
 K:25  &  97.52  &  98.76  &    &   K:25  &  94.15  &  97.30 \\
 K:26  &  97.52  &  98.87  &    &   K:26  &  94.23  &  97.10 \\
 K:27  &  97.43  &  98.80  &    &   K:27  &  94.40  &  97.53 \\
 K:28  &  97.52  &  98.82  &    &   K:28  &  94.23  &  97.37 \\
 K:29  &  97.58  &  98.82  &    &   K:29  &  94.55  &  97.30 \\
 \cline{1-3} \cline{5-7} 
 avg  &  97.49  &  98.83  &    &    &  94.40  &  97.29  \\
 \cline{1-3} \cline{5-7} 
  &    &    &    &    &    &   \\
		\multicolumn{3}{l}{KHold Score 6 Kelas} & & 
		\multicolumn{3}{l}{KHold Score 12 Kelas} \\ 
\cline{1-3} \cline{5-7}   
pBar: -0.013357  &    &    &    &  pBar: -0.028906  &    &  \\
$t_6$  : -76.672141  &    &    &    &  $t_{12}$   : -70.245154  &    & \\  
	\end{tabular}
\end{center} 
\end{table}
  

%============================================================================
% TABEL K FOLD
%============================================================================
Level Signifikan yang digunakan untuk uji KFold adalah 0.05 (95\%) 
dengan 9 ($K-1$) derajat kebebasan. Nilai tabulasi yang diperoleh dari tabel
adalah $2.262$. Untuk 6 Kelas, $H_0$ dapat ditolak, karena $\lvert t_6 \rvert >
2.262$. Dan untuk 12 kelas, nilai $\lvert t_{12} \rvert > 2.262$, sehingga $H_0$
juga dapat ditolak. Sehingga ini menunjukan bahwa perbedaan
akurasi dari kedua algoritma juga sangat signifikan dan dapat disimpulkan
algoritma FN-GLVQ memberikan hasil lebih baik dibandingkan dengan GLVQ.

\begin{table}
\small  
\label{tbl:kfold-hasil}
\caption{Hasil tabulasi uji K-Fold untuk 6 dan 12 kategori kelas, $P_A=$
GLVQ, $P_B=$ FN-GLVQ, dengan nilai $K=10$ dan porsi training dan testing
$90:10$}
\begin{center}
	\begin{tabular}{ccccccc} 
	\cline{1-3} \cline{5-7}
	\multicolumn{3}{c}{Akurasi Data} & & 
	\multicolumn{3}{c}{Akurasi Data} \\
	\multicolumn{3}{c}{6 Kelas} & & 
	\multicolumn{3}{c}{12 Kelas} \\
	\cline{1-3} \cline{5-7}
	   $K$  &     $P_A$  &  $P_B$  &    &     $K$  &     $P_A$  &  P_B \\
	   \cline{1-3} \cline{5-7} 
  K:0  &  96.77  &  98.47  &    &    K:0  &  93.50  &  96.50 \\
  K:1  &  95.78  &  98.65  &    &    K:1  &  96.17  &  97.08 \\
  K:2  &  97.76  &  98.92  &    &    K:2  &  94.33  &  96.42 \\
  K:3  &  97.76  &  98.56  &    &    K:3  &  93.75  &  96.08 \\
  K:4  &  97.67  &  98.56  &    &    K:4  &  94.42  &  96.33 \\
  K:5  &  96.77  &  98.38  &    &    K:5  &  93.42  &  96.25 \\
  K:6  &  97.04  &  97.85  &    &    K:6  &  94.67  &  96.25 \\
  K:7  &  97.40  &  99.01  &    &    K:7  &  94.17  &  96.00 \\
  K:8  &  95.69  &  98.11  &    &    K:8  &  93.33  &  95.75 \\
  K:9  &  97.67  &  98.65  &    &    K:9  &  93.58  &  96.67 \\
 \cline{1-3} \cline{5-7}
  avg  &  97.03  &  98.52  &    &    avg &  94.13  &  96.33 \\
  \cline{1-3} \cline{5-7}
  &    &    &    &    &    &  \\  
		\multicolumn{3}{l}{Kfold Score 6 Kelas} & & 
		\multicolumn{3}{l}{Kfold Score 12 Kelas} \\ 
		\cline{1-3} \cline{5-7}     
pBar: -0.014901  &    &    &    &  pBar: -0.022000  &    & \\  
$t_6$   : -6.664332  &    &    &    &  $t_{12}$   : -10.255489  &    &   \\
	\end{tabular}
\end{center} 
\end{table}

\subsection{Dietterich�s 5 x 2-Fold Cross-Validation Paired t-Test}
Pada uji ini, dataset yang digunakan dipartisi menjadi 2-Fold di iterasi
sebanyak 5 kali seperti yang telah diuraikan pada sub-bab
\ref{sec:ujittest}. Inisialisasi bobot menggunakan cara random  internal dari
dataset.  Setelah dilakukan ujicoba didapat data  tabulasi seperti yang
ditunjukkan pada \tab~\ref{tbl:52fold-hasil}.
%============================================================================
% TABEL Dietterich�s
%============================================================================
\begin{table}
\small  
\label{tbl:52fold-hasil}
\caption{Hasil tabulasi uji Dietterich�s untuk 6 dan 12 kategori kelas, $P_A=$
GLVQ, $P_B=$ FN-GLVQ}
\begin{center}
	\begin{tabular}{ccccc} 
	\hline
	\multicolumn{5}{c}{Tabulasi Data 6Kelas} \\
	\hline
     K  &  $P_{A_1}$  &      $P_{B_1}$  &      $P_{A_2}$  &      $P_{B_2}$ \\
     \hline
     K:0  &  96.39  &  98.13  &  96.37  &  98.29 \\
  K:1  &  96.57  &  98.06  &  96.44  &  98.47\\
  K:2  &  96.73  &  98.17  &  96.35  &  98.40\\
  K:3  &  96.80  &  98.13  &  96.30  &  98.37\\
  K:4  &  96.61  &  98.24  &  96.37  &  98.24\\
  \hline
  &    &    &    &  \\
  \multicolumn{3}{l}{Dietterich�s Score 6 Kelas} & &  \\   
  $t_6$   : -4.846367  &    &    &    &   \\
  &    &    &    &   \\
  \hline
	\multicolumn{5}{c}{Tabulasi Data 12 Kelas} \\
     K  &  $P_{A_1}$  &      $P_{B_1}$  &      $P_{A_2}$  &      $P_{B_2}$ \\
     \hline
  K:0  &  93.08  &  95.73  &  93.45  &  95.22 \\
  K:1  &  92.88  &  95.90  &  93.56  &  95.52 \\
  K:2  &  93.00  &  95.72  &  93.51  &  95.43 \\
  K:3  &  92.81  &  95.70  &  93.43  &  95.28 \\
  K:4  &  92.83  &  95.62  &  93.40  &  95.20 \\
  \hline
  &    &    &    &   \\
   \multicolumn{3}{l}{Dietterich�s Score 6 Kelas} & &  \\   
  $t_12$   : -3.910283  &    &    &    & \\  
	\end{tabular}
\end{center} 
\end{table}
   
Dari data Dietterich�s Score pada tabulasi \ref{tbl:52fold-hasil}, didapat
$t_6$ untuk 6 kelas adlah -4.846367, sedangkan $t_{12}$ untuk 12 kelas didapat 
-3.910283. 

Level Signifikan yang digunakan untuk uji KHold adalah 0.05 (95\%) 
dengan 5 derajat kebebasan. Nilai tabulasi yang diperoleh dari tabel
adalah $2.571$. Untuk 6 Kelas, $H_0$ dapat ditolak, karena $\lvert t_6 \rvert >
2.571$. Dan untuk 12 kelas, nilai $\lvert t_{12} \rvert > 2.571$, sehingga $H_0$
juga dapat ditolak. Sehingga ini menunjukan juga bahwa perbedaan
akurasi dari kedua algoritma sangat signifikan dan dapat disimpulkan algoritma
FN-GLVQ memberikan hasil lebih baik dibandingkan dengan GLVQ.


Dari empat uji test yang dilakukan terhadap algoritma GLVQ dan FN-GLVQ pada
data arrhytmia 6 dan 12 Kelas, 86 fitur; 3 uji test menunjukkan bahwa FN-GLVQ
memberikan hasil yang lebih baik secara signifikan daripada GLVQ, sedangkan 1
uji test memberikan kesimpulan yang seimbang. Sehingga dapat ditarik kesimpulan
bahwa FN-GLVQ memberikan hasil yang lebih baik dari pada GLVQ pada pengenalan
kelainan Arrhytmia.
% %-----------------------------------------------------------------------------%
% \section{Analisis Hasil}
% %-----------------------------------------------------------------------------%
% 1. Analisis bisa menggunakan T-Test dari 
% Buku Combining Pattern Classifier
% Error Calculation (Counting estimator p8)
% 
% 2. McNemar Test
% 	- tentukan bobot terbaik antara LVQ1m LVQ21, GLVQ dan FPGLVQ
% 	- train dengan data 86 fitur, 6 kelas, simpan bobot nya
% 	- testing dengan mencatat setiap single data, catat untuk tiap classifier,
% 	dikenali atau tidak.
% 
% 
% Untuk memberikan ranking terhadap algoritma, bisa menggunakan amalisis ROC 
% \newpage
% \section{Revision}
% \begin{itemize}
%   \item Why? wavelet
%   \item Why? daubhechies
%   \item Why? proses Outlier , baik baseline wander, maupun Outlier removal,
%   knapa pake interquartile range? bukan mahalanobis?
%   \item 
%   \item 
% \end{itemize}