\section{Wavelet Transform}
\label{sec:wt}


wavelet digunakan untuk melakukan filter frekuensi tinggi,
dimana setelah didekomposisi, kemudian signal direkonstruksi kembali.

pada penelitian ini, wavelet digunakan untuk mereduksi fitur dari setiap beat
dengan menggunakan coefisien aproksimasi-nya sebagai fitur, sesuai dengan
tingkat level dekomposisi dan mengabaikan coefisien detail, dimana mengandung
signal frekuensi tinggi.


Transformasi Fourier
Sampai sekarang transformasi Fourier mungkin masih menjadi transformasi yang
paling populer di area pemrosesan sinyal digital (PSD). Transformasi Fourier
memberitahu kita informasi frekuensi dari sebuah sinyal, tapi tidak informasi
waktu (kita tidak dapat tahu di mana frekuensi itu terjadi). Karena itulah
transformasi Fourier hanya cocok untuk sinyal stationari (sinyal yang informasi
frekuensinya tidak berubah menurut waktu). Untuk menganalisa sinyal yang
frekuensinya bervariasi di dalam waktu, diperlukan suatu transformasi yang dapat
memberikan resolusi frekuensi dan waktu disaat yang bersamaan, biasa disebut
analisis multi resolusi (AMR). AMR dirancang untuk memberika resolusi waktu yang
baik dan resolusi frekuensi yang buruk pada frekuensi tinggi suatu sinyal, serta
resolusi frekuensi yang baik dan resolusi waktu yang buruk pada frekuensi rendah
suatu sinyal. Pendekatan ini sangat berguna untuk menganalisa sinyal dalam
aplikasi-aplikasi praktis yang memang memiliki lebih banyak frekuensi rendah.
Transformasi wavelet adalah suatu AMR yang dapat merepresentasikan informasi
waktu dan frekuensi suatu sinyal dengan baik. Transformasi wavelet menggunakan
sebuah jendela modulasi yang fleksibel, ini yang paling membedakannya dengan
transformasi Fourier waktu-singkat (STFT), yang merupakan pengembangan dari
transformasi Fourier. STFT menggunakan jendela modulasi yang besarnya tetap, ini
menyebabkan dilema karena jendela yang sempit akan memberikan resolusi frekuensi
yang buruk dan sebaliknya jendela yang lebar akan menyebabkan resolusi waktu
yang buruk.
