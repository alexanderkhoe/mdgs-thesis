\section{Penurunan Rumus Fuzzy GLVQ}

\subsection{Penurunan GLVQ}
\begin{align}
\label{eq:}
	\varphi(x) = \frac{d_1 - d_2}{d_1 + d_2}
\end{align}

\begin{align}
\label{eq:}
	S = \sum_{i=1}^{N} f(\varphi(x)), 
\end{align}

\begin{align}
\label{eq:}
	w_i \leftarrow w_i - \alpha \frac{\delta S}{\delta w_i}, i = 1, 2
\end{align}

$d_i = \lvert x - w_i\rvert ^2$

\begin{align}
\label{eq:}
	\frac{\delta S}{\delta w_1} =  
	\frac{\delta S}{\delta \varphi} \frac{\delta \varphi}{\delta d_1} \frac{\delta
	d_1}{\delta w_1} =
	- \frac{\delta f}{\delta \varphi} \frac{4d_2}{(d_1 + d_2)^2} (x - w_1)
\end{align}

\begin{align}
\label{eq:}
	\frac{\delta S}{\delta w_2} =  
	\frac{\delta S}{\delta \varphi} \frac{\delta \varphi}{\delta d_2} \frac{\delta
	d_2}{\delta w_2} =
	- \frac{\delta f}{\delta \varphi} \frac{4d_1}{(d_1 + d_2)^2} (x - w_2)
\end{align}

Update rules;
\begin{align}
\label{eq:}
	w_1 \leftarrow w_1 + \alpha   
	\frac{\delta f}{\delta \varphi} \frac{d_2}{(d_1 + d_2)^2} (x - w_1)
\end{align}

\begin{align}
\label{eq:}
	w_2 \leftarrow w_2 - \alpha   
	\frac{\delta f}{\delta \varphi} \frac{d_1}{(d_1 + d_2)^2} (x - w_2)
\end{align}

Fungsi Monoton naik, fungsi sigmoid:
\begin{align}
\label{eq:}
	f(\varphi, \xi) = \frac{1}{1 + e^{-\varphi \xi}}
\end{align}

\begin{align}
\label{eq:deltasigmoid}
	\frac{\delta f}{\delta \varphi} = f(\varphi, \xi) (1 - f(\varphi , 
	\xi))
\end{align}
	
	
\subsection{Penurunan FPGLVQ}
Pada GLVQ, kemiripan dihitung dengan menggunakan metric, yakni jarak euclid,
sehingga semakin kecil jarak antara input dengan prototipe, maka akan semakin
mirip. Pada FPGLVQ, tingkat kemiripan antara vektor masukan dengan prototipe
menggunakan fuzzy similarity , dimana pada perhitungan ini dicari
derajat keanggotaan setiap fitur ($x_i$) terhadap fungsi keanggotaannya
($h_{ij}(x)$), $i=$ fitur, $j=$ prototipe.
\begin{align}
	\mu_{ij} = h_{ij}(x_i)
\end{align}

Kemudian nilai derajat keanggotaan dipropagasi ke neuron keluaran dengan
menggunakan operasi rata-rata (mean). 

\begin{align}
	\mu_j = \text{mean}_i [\mu_i]
\end{align}

Untuk menentukan prototipe pemenang, dipilih prototipe dengan nilai kemiripan
terbesar (max).  
\begin{align}
	w_p = \max_j [\mu_j]
\end{align}

Ketidakjelasan dari prototipe akan disesuaikan selama proses pembelajaran
tergantung dari vektor masukan-nya.

Pada fuzzy similarity, semakin besar nilainya , maka kedua vektor akan semakin
mirip. Untuk dapat menggunakan konsep similarity pada GLVQ, maka dicari
negasi-nya menjadi \textit{disimilarity} dengan menggunakan persamaan $d = 1 -
\mu$. Sehingga;

\begin{align}
\label{eq:}
	\varphi(x) &= \frac{(1 - \mu_1) - (1 - \mu_2)}{(1 - \mu_1) + (1 -
	\mu_2)}\nonumber\\
	&= \frac{\mu_2 - \mu_1}{2 - \mu_1 - \mu_2}
\end{align}

Untuk dapat mengimplementasikan konsep fuzzy dalam GLVQ, maka akan dilakukan
penurunan rumus sebagai berikut;

\begin{align}
\label{eq:}
	\frac{\delta S}{\delta w} =  
	\frac{\delta S}{\delta \varphi} . \frac{\delta \varphi}{\delta \mu}.
	\frac{\delta \mu}{\delta w}
\end{align}

\begin{align}
\label{eq:}
	\Psi(x) &= \frac{f(x)}{g(x)} \nonumber \\
	\frac{\delta \Psi}{\delta x} &=  
	\frac{f'(x)g(x) - f(x)g'(g)}{g(x)^2}
\end{align}

\begin{align}
\label{eq:}
	\frac{\delta \varphi}{\delta \mu_1} &=  
	\frac{-1 . (2 - \mu_1 - \mu_2) - (-1).(\mu_2-\mu_1)}
	{(2 - \mu_1 - \mu_2)^2} \nonumber \\
	 &=  
	-2.\frac{(1 - \mu_2)}{(2 - \mu_1 - \mu_2)^2}
\end{align}

\begin{align}
\label{eq:}
	\frac{\delta \varphi}{\delta \mu_2} &=  
	\frac{1 . (2 - \mu_1 - \mu_2) - (-1).(\mu_2-\mu_1)}
	{(2 - \mu_1 - \mu_2)^2} \nonumber \\
	 &=  
	2.\frac{(1 - \mu_1)}{(2 - \mu_1 - \mu_2)^2}
\end{align}

Karena dalam menghitung kemiripan kita menggunakan fuzzy similarity, maka harus
diturunkan juga. Pada saat ini, kita menggunakan fungsi segitiga
sebagai fungsi keanggotaan fuzzy-nya.

\begin{align}
\label{eq:}
	f(x, a, b, c) = \left\{ 
	\begin{array}{ll}
	0 & , x \leq a\\
	\frac{x - a}{b - a} & , a < x \leq b \\
	\frac{c - x}{c - b} & , b < x < c \\
	0 & , x \geq c
	\end{array}
\end{align}

dimana $a = minimum, b = rata-rata, c = maximum$. Pada algoritma ini,
vektor dari prototipe bobot akan disusun sbb;

\begin{align}
\label{eq:}
	w = (w_{min}, w_{mean}, w_{max})
\end{align}
 
Sehingga turunan dari fungsi segitiga diatas didapat sebagai berikut;
\begin{itemize}
  \item Untuk $w_{min} < x \leq w_{mean}$ :
  \begin{align}
	\label{eq:}
		\mu &= \frac{x - w_{min}}{w_{mean} - w_{min}} \nonumber \\
			&= (x - w_{min}) . (w_{mean} - w_{min})^{-1} \nonumber \\
		\frac{\delta \mu}{\delta w_{mean}} &=
		(x - w_{min}).(-1).(w_{mean} - w_{min})^{-2}.(1) \nonumber \\
		 &=
		- \frac{x - w_{min}}{(w_{mean} - w_{min})^2}
	\end{align}

	\item Untuk $w_{mean} < x < w_{max}$ :
	\begin{align}
	\label{eq:}
		\mu &= \frac{w_{max} - x}{w_{max} - w_{mean}} \nonumber \\
			&= (w_{max} - x) . (w_{max} - w_{mean})^{-1} \nonumber \\
		\frac{\delta \mu}{\delta w_{mean}} &=
		(w_{max} - x).(-1).(w_{max} - w_{mean})^{-2}.(-1) \nonumber \\
		 &=
		+ \frac{w_{max} - x}{(w_{max} - w_{mean})^2}
	\end{align}
\end{itemize}
  
\noindent Sehingga didapat update rule sebagai berikut;
\begin{itemize}
  \item Untuk $w_{min} < x \leq w_{mean}$ :
  	\begin{align}
	\label{eq:}
		w_1 \leftarrow & w_1 - \alpha .  
		\frac{\delta f}{\delta \varphi} . \frac{2.(1 - \mu_2)}{(2 - \mu_1 - \mu_2)^2}.
		\Bigg(\frac{x - w_{min}}{(w_{mean} - w_{min})^2}\Bigg) \\
		w_2 \leftarrow & w_2 + \alpha .  
		\frac{\delta f}{\delta \varphi} . \frac{2.(1 - \mu_1)}{(2 - \mu_1 - \mu_2)^2}.
		\Bigg(\frac{x - w_{min}}{(w_{mean} - w_{min})^2}\Bigg)
	\end{align}
  \item Untuk $w_{mean} < x < w_{max}$ :
	\begin{align}
	\label{eq:}
		w_1 \leftarrow & w_1 + \alpha .  
		\frac{\delta f}{\delta \varphi} . \frac{2.(1 - \mu_2)}{(2 - \mu_1 - \mu_2)^2}.
		\Bigg(\frac{w_{max} - x}{(w_{max} - w_{mean})^2}\Bigg) \\
		w_2 \leftarrow & w_2 - \alpha .  
		\frac{\delta f}{\delta \varphi} . \frac{2.(1 - \mu_1)}{(2 - \mu_1 - \mu_2)^2}.
		\Bigg(\frac{w_{max} - x}{(w_{max} - w_{mean})^2})\Bigg)
	\end{align}  
  \item Untuk $x \leq w_{min}$ dan $x \geq w_{max}$ :
  	\begin{align}
	\label{eq:}
		w_1(t+1) \leftarrow & w_1(t)     
	\end{align}  
\end{itemize}

Fungsi Monoton naik yang dipakai tetap sama dengan yang dipakai pada GLVQ yang
awal, yakni menggunakan fungsi sigmoid, seperti yang pada \equ~\ref{eq:deltasigmoid}

Satu langkah tambahan yang diadaptasi dari FNLVQ adalah proses penyesuaian
lebar interval dari fungsi keanggotaan tiap prototipe. Jika jaringan bisa
mengenali dengan benar vektor masukan yang diberikan, maka fungsi keanggotaan
akan diperlebar dengan harapan tingkat pengenalan-nya meningkat. Sebaliknya jika
jaringan salah mengenali vektor masukan, maka fungsi keanggotaan akan
dipersempit dengan harapan tingkat pengenalan terhadap vektor masukan menurun.
Kedua langkah ini hanya akan dilakukan jika nilai $\mu_1 > 0$ atau $\mu_2 > 0$.

Jika  kedua nilai similarity adalah 0, $\mu_1=0$ dan $\mu_2=0$, maka hal ini
berarti semua prototipe sama sekali tidak mengenali vektor masukan yang
diberikan. Terdapat 2 kemungkinan; (1) Vektor masukan memang berada diluar
distribusi dari kategori yang dikenali, (2) interval dari fungsi
keanggotaan (fuzzy) dari Vektor prototipe terlalu sempit, sehingga tingkat
pengenalan-nya rendah. Karena ini merupakan proses pelatihan, maka asumsi adalah
yang ke-2, sehingga untuk membuat jaringan mengenali vektor pewakil, semua
interval dari fungsi keanggotaan prototipe diperlebar.

\noindent
Berikut adalah aturan perlebaran/penyempitan fungsi keanggotaan prototipe lebih
jelas;
\begin{itemize}
  \item Jika  $\mu_1 > 0$ atau $\mu_2 > 0$:
  \begin{itemize}
    \item Jika pengenalan-nya benar ($\varphi < 0$), diperlebar
    	\begin{align}
    	w_{min} &\leftarrow w_{mean} - (w_{mean} - w_{min}) . (1 + (\beta .
    	\alpha)) \nonumber \\
    	w_{max} &\leftarrow w_{mean} + (w_{max} - w_{mean}) . (1 + (\beta .
    	\alpha)) 
    	\end{align}
    \item Jika pengenalan-nya salah ($\varphi \geq 0$), dipersempit
    	\begin{align}
    	w_{min} &\leftarrow w_{mean} - (w_{mean} - w_{min}) . (1 - (\beta .
    	\alpha)) \nonumber \\
    	w_{max} &\leftarrow w_{mean} + (w_{max} - w_{mean}) . (1 - (\beta .
    	\alpha)) 
    	\end{align}
  \end{itemize}
  Disini, nilai $\beta$ adalah diantara [0,1]. Pada studi kasus yang dilakukan
  disini, dipilih nilai $\beta = 0.00005$.
  
  \item Jika  $\mu_1=0$ dan $\mu_2=0$, semua fungsi keanggotaan pada prototipe
  diperlebar dengan aturan sbb:
  	\begin{align}
	w_{min} &\leftarrow w_{mean} - (w_{mean} - w_{min}) . (1 - (\alpha .
	\gamma)) \nonumber \\ 
	w_{max} &\leftarrow w_{mean} + (w_{max} - w_{mean})
	. (1 + (\alpha . \gamma)) 
	\end{align}
	
	Disini, nilai $\gamma$ adalah diantara [0,1]. Pada studi kasus yang dilakukan
	disini, dipilih nilai $\gamma = 0.1$.
\end{itemize}  
