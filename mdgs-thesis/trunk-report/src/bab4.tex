%-----------------------------------------------------------------------------%
\chapter{\babEmpat}
%-----------------------------------------------------------------------------%
Pada bab ini akan dijelaskan mengenai metode klasifikasi yang dikembangkan,
modifikasi yang dilakukan serta perancangan dan implementasi dari sistem
pengenalan \gls{arrhytmia}. 

%-----------------------------------------------------------------------------%
% \section{Metode Pelatihan LVQ}
%-----------------------------------------------------------------------------%
% \subsection{Epoch}
% Dalam penelitian ini, studi dilakukan terhadap algoritma LVQ yang dikembangkan
% oleh Kohonen et.al. \cite{Kohonen92lvqpak}. Dari hasil analisis terhadap
% implementasi algoritma tersebut, yakni LVQ PAK package, \saya menyadari bahwa
% mekanisme iterasi (epoch) yang penulis pahami dan kebanyakan digunakan
% dalam JST dengan yang ada pada paket program tersebut sedikit berbeda.
% \begin{itemize}
%   \item Iterasi pada LVQ PAK, iterasi yang dilakukan untuk sekali proses
%   pembelajaran (satu kali proses update bobot) dengan menggunakan satu data
%   sampel. Hal ini akan berelasi dengan jumlah maksimal iterasi yang ditentukan
%   sebagai parameter pembelajaran. Artinya jika jumlah data sampel yang digunakan
%   sebanyak 100, sedangkan maksimal iterasi ditentukan sebanyak 50 kali, maka
%   hanya 50 data sampel saja yang akan digunakan untuk proses pelatihan, dengan
%   pemilihan data yang random maupun sequensial dan hanya akan terjadi 50 kali
%   proses update bobot.
%   \item Iterasi pada sistem epoch, makna dari satu iterasi adalah
%   proses pelatihan dilakukan untuk semua data sampel yang diberikan. Jika data
%   sampel yang diberikan sebanyak 100, maka jaringan saraf akan dilatih sebanyak
%   100 kali untuk satu iterasi. Jika maksimum iterasi ditentukan sebanyak 50
%   kali, maka akan setara dengan 1500 iterasi pada metode sebelumnya, atau 1500
%   kali proses update bobot.
% \end{itemize} 
% 
% Selain sensitif terhadap inisial bobot awal, LVQ original tersebut juga sangat
% sensitif terhadap jumlah iterasi yang dilakukan dalam proses pembelajaran. Jika
% terlalu banyak iterasi pelatihan dilakukan, maka kecenderungannya adalah bobot
% yang dihasilkan melewati dan menjauhi bobot optimal (divergen). 
% 
% 
% \subsection{Round Robin iteration}
% Melihat karakteristik dari proses pembelajaran LVQ yang berbasis
% winner-take-all dan proses update rule secara sequensial, maka muncul suatu ide
% untuk merubah perilaku proses pembelajaran dari sisi iterasi data training. Pada
% iterasi metode sebelumnya, proses iterasi pembelajaran dalam satu epoch
% dilakukan sesuai dengan urutan data masukan yang diberikan, bisa berupa urutan
% sejumlah N sampel data kategori 1 dan M sampel data kategori 2 dan seterusnya,
% atau bisa juga berupa urutan data yang diambil secara random terhadap kategori
% data. Pada penelitian ini, \saya mencoba untuk menerapkan mekanisme round-robin
% dalam iterasi proses pembelajaran LVQ dimana dalam mekanisme ini, urutan data
% sampel yang diberikan pada proses pembelajaran jaringan saraf dipastikan selang
% seling untuk setiap kategori data sejumlah $k$ kategori. Jadi urutan data
% sampelnya adalah $X_{1,c_1}, X_{1,c_2}, \dots, X_{1,c_k}, X_{2,c_1}, X_{2,c_2},
% \dots, X_{2,c_k}, X_{n_1,c_1},X_{n_2,c_2}, \dots,X_{n_3,c_k}$.
% 
% Dengan menggunakan mekanisme round robin, diharapkan proses update rule secara
% sequensial mempengaruhi proses pembelajaran secara merata untuk semua kategori
% data. Berikut pada algoritma \ref{alg:opsi3} dapat dilihat pseudocode dari
% mekanisme iterasi round-robin ini.
% 
% \begin{algorithm}  
% \caption{Mekanisme iterasi secara round robin}          
% \label{alg:opsi3}                           
% \begin{algorithmic}                    % enter the algorithmic environment
% 	\STATE \ldots
% 	\STATE $max\_N \leftarrow findMaxNumOfDataInCategory()$
% 	\FOR {$i=1 \to max\_N$}
% 		\FORALL {$C$ in $Category$}
% 			\IF {$i > nC$} continue \ENDIF
% 			
% 			\STATE $sample \leftarrow X_{c,i}$
% 			\STATE $train(codebook, sample)$
% 			\STATE \ldots
% 		\ENDFOR
% 	\ENDFOR
% 	\STATE \ldots
% \end{algorithmic}
% \end{algorithm}


%-----------------------------------------------------------------------------%
\section{Metode Fuzzy Neuro GLVQ}
%-----------------------------------------------------------------------------%


%-----------------------------------------------------------------------------%
\section{Metode Mahalanobis GLVQ}
%-----------------------------------------------------------------------------%


%-----------------------------------------------------------------------------%
\section{Implementasi Sistem}
%-----------------------------------------------------------------------------%
\subsection{Lingkungan pengembangan}

\subsection{Arsitektur}

\subsection{Antar Muka Sistem}
