\section{Wavelet Transform}
\label{sec:wt}

Metode transformasi berbasis wavelet merupakan sarana yang dapat digunakan untuk
menganalisis sinyal non-stasioner, yaitu sinyal dengan kandungan frequensi yang
bervariasi terhadap waktu. Metode ini sangat populer dalam beberapa tahun
terakhir. Analisis wavelet dapat digunakan untuk menunjukkan kelakuan sementara
pada suatu sinyal, atau dapat juga digunakan untuk mem-filter, menghilangkan
sinyal data yang tidak diinginkan (\emph{noise}) dan atau meningkatkan kualitas
dari data. Selain itu, transformasi wavelet juga dapat digunakan untuk
memampatkan (\emph{compressing}) data \cite{Agfi:2006}.

Transformasi wavelet merupakan transformasi yang menggunakan kernel terintegrasi
yang dinamakan dengan wavelet. Wavelet dapat digunakan sebagai kernel
terintegrasi untuk analisis serta ekstraksi informasi suatu data dan juga
sebagai basis penyajian/karakteristik dari suatu data. Kelebihan dari analisis
sinyal menggunakan wavelet adalah terletak dari sifat terpenting wavelet yaitu
lokalisasi waktu-frekuensi (\emph{time-frequency localization}) yang
diistilahkan dengan \emph{compact support} \cite{guler:2005}. Dengan menggunakan
wavelet dapat dipelajari karakteristik sinyal secara lokal dan detil sesuai dengan skala-nya,
dimana hal ini menunjukkan kegunaan dari analisis pada sinyal non-stasioner,
karakteristik berbeda pada skala yang berbeda. Pada dasarnya, cara kerja dari
wavelet dalam mengekstraksi data adalah dengan menggunakan proses penguraian
(\emph{decomposition}) atau ekspansi deret, yaitu dengan cara ekspansi tak
berhingga dari wavelet yang diulur atau \emph{dilated} dan digeser atau
\emph{translated}.


Transformasi wavelet dikembangkan sebagi suatu alternatif pendekatan pada
Transformasi Fourier Waktu Pendek (Short Time Fourier Transform = STFT) dalam
mengatasi masalah resolusi. Pada STFT, skala dari jendela yang
digunakan bersifat konstan, sedangkan pada Transformasi wavelet, lebar dari
jendela akan berubah-ubah selama proses transformasi dilakukan dalam menghitung
masing-masing komponen spektrum. Dimana hal ini merupakan ciri khas dari
Transformasi Wavelet. Dengan karakteristik seperti itu, dengan transformasi
wavelet akan dapat diperoleh resolusi waktu dan frekuensi yang jauh lebih baik
daripada metode-metode yang lain.

Persamaan transformasi wavelet (WT) kontinu pada signal $f(x)$ dapat
didefinisikan sebagai berikut;
\begin{align}
\label{eq:wavelet_transform}
	W_{s}f(x) &= f(x) * \Psi_{s}(x) = \frac{1}{s}\int^{+\infty}_{-\infty}f(t)\Psi(\cfrac{x - t}{s})dt
\end{align}

\noindent dimana $s$ adalah faktor skala,  $\Psi_{s}(x) =
\cfrac{1}{s}\Psi(\cfrac{x}{s})$ adalah dilatasi dari fungsi jendela, yang
kemudian dikenal dengan wavelet penganalisis $\Psi(x)$, $t$ adalah  faktor
translasi/pergeseran dari fungsi jendela tersebut. Dari persamaan
\ref{eq:wavelet_transform} ditunjukkan bahwa fungsi jendela tersebut terdilatasi
maupun termampatkan berdasarkan faktor skala. Dengan skala yang rendah, maka
frekuensi tinggi akan terlokalisasi, sedangkan pada skala tinggi, yang
terlokalisasi adalah frekuensi rendah. 

\addFigure{width=1\textwidth}{pics/waveletcwt.png}{fig:cwt}{Contoh sinyal
nonstasioner dan hasil dari WT (kontinu)} 

Pada \pic~\ref{fig:cwt} ditunjukkan contoh sinyal nonstasioner dengan kandungan
frekuensi 250, 500, 750 dan 1000Hz dengan tambahan noise yang kemudian
ditransformasi menggunakan WT. Dari gambar tersebut terlihat bahwa frekuensi
250Hz muncul pertama kali yang diikuti dengan frekuensi yang lain. Ketika kita
akan menganalisis menggunakan wavelet, terdapat tradeoff resolusi antara waktu
dan frekuensi(pada dasarnya pada WT, direpresetnasikan dengan waktu-skala,
dimana skala dan frekuensi berbanding terbalik, semakin tinggi skala, maka
dapat melokalisasi frekuensi rendah dan sebaliknya). Jika skala tinggi
maka maka semakin tidak jelas resolusi frekuensi, namun
semakin jelas resolusi waktunya, dan sebaliknya.

\subsection{Descrete Wavelet Transform}
Pada dasarnya, nilai koefisien dari wavelet untuk jendela wavelet dan
signal tertentu menunjukkan seberapa dekat korelasi antara wavelet 
dengan bagian tertentu dari signal tersebut. semakin tinggi koefisien, maka akan
semakin mirip dimana hasilnya akan sangat tergantung dari bentuk wavelet yang
dipilih\cite{wavelet:matlab}. Namun jika perhitungan nilai koefisien dilakukan
untuk semua skala dan posisi (kontinu), maka akan dihasilkan jumlah data yang
sangat besar. Mallat \cite{Mallat:1989} mengembangkan suatu cara untuk
menghitung koefisien wavelet dengan mengambil sebagian saja dari skala dan
posisi berdasarkan pangkat-dua, yang kemudian dikenal dengan istilah
\emph{dyadic WT}. Hal ini juga dikenal dengan \emph{Discrete Wavelet
Transform (DWT)}. Dengan menggunakan \emph{dyadic WT} maka analisis akan lebih
efisien dan cukup  akurat. Algoritma Mallat ini menggunakan filter seperti dalam
skema \emph{two channel subband coder} dan dikenal juga dengan Transformasi
Wavelet Cepat (\emph{Fast Wavelet Transform}).

Dalam banyak kasus pemrosesan sinyal, kandungan frekuensi rendah adalah hal yang
penting karena memberikan indentitas dari sinyal yang bersangkutan. Kandungan
frekuensi tinggi hanya sebagai ``nuansa sinyal'' tambahan. Hal ini dapat
dianalogikan seperti sinyal suara manusia, jika komponen frekuensi tinggi
dihilangkan, maka suara akan berubah, namun masih mampu untuk mengetahui apa
yang diucapkan \cite{wavelet:matlab}. Hal inilah yang mendasari mengapa dalam
analisis berbasis wavelet banyak digunakan istilah aproksimasi dan detail.

Diberikan $s = 2^j$ dengan $j \in Z, Z$ set integer, maka dyadic WT dari sinyal
$f(x)$  dapat dihitung menggunakan algoritma Mallat sebagai berikut;

\begin{align}
	\label{eq:sfn}
	S_{2^{j}}f(x) &= \sum_{k \in~Z} h_{k}S_{2^{j-1}}f(x - 2^{j-1}k) \\
	\label{eq:wfn}
	W_{2^{j}}f(x) &= \sum_{k \in~Z} g_{k}S_{2^{j-1}}f(x - 2^{j-1}k)
\end{align}

\noindent dimana $S_{2^{j}}$ merupakan operator penghalus, $S_{2^{j}} f(n) =
a_{j}$. $a_{j}$  adalah koefisien dari frekuensi rendah (skala tinggi) yang
mengaproksimasi sinyal yang asli, sedangkan $W_{2^{j}} f(n) = d_{j}$, $d_{j}$ 
adalah koefisien frekuensi tinggi (skala rendah) yang merepresentasikan detail
dari sinyal asli.

\addFigure{width=0.8\textwidth}{pics/dwtdecompos.png}{fig:decompos}{Ilustrasi
proses dekomposisi, (a) Dekomposisi satu tingkat, (b) dekomposisi multi tingkat}

Pada \pic~\ref{fig:decompos}a dapat dilihat proses filtering wavelet dimana
$f(x)$ adalah sinyal asli, kemudian dilewatkan ke filter lolos-rendah
(\emph{lowpass})  lolos-tinggi (\emph{highpass}) dan menghasilkan dua sinyal,
\textbf{A}-proksimasi dan \textbf{D}-etail. 

Jika dekomposisi sinyal diteruskan secara iteratif untuk bagian aproksimasinya,
sehingga suatu sinyal dapat dibagi-bagi kedalam banyak komponen resolusi rendah,
maka proses ini dinamakan sebagai dekomposisi banyak tingkat atau
\emph{multilple-level decomposition} seperti yang ditunjukkan pada
\pic~\ref{fig:decompos}b.
 
Pemilihan fungsi jendela \emph{mother wavelet} dan jumlah level dekomposisi
dalam analisis menggunakan transformasi wavelet adalah sangat penting.
Pemilihan jumlah level dekomposisi berdasarkan atas komponen frekuensi
yang dominan dalam sinyal. Pemilihan yang tepat bertujuan untuk menjaga agar
bagian signal yang memiliki korelasi yang baik dengan frekuensi yang dibutuhkan
dalam proses pengenalan tetap ada pada koefisien wavelet. Terdapat berbagai
macam \emph{mother wavelet} yang dapat digunakan dalam analisis wavelet. Menurut
Senhadji \cite{Senhadji:1995}, proses dekomposisi dengan menggunakan wavelet
yang orthonormal akan memberikan informasi yang \emph{non-redundant}. Kelompok
wavelet orthonormal dan juga \emph{compactly supported} diantaranya adalah
Daubechies, Symlet dan Coiflet \cite{Agfi:2006}.


     
 
% calculated with Mallat algorithm as follows:
% 
% Informasi yang dapat
% diekstraksi menggunakan wavelet adalah 
% 
% 
% wavelet digunakan untuk melakukan filter frekuensi tinggi,
% dimana setelah didekomposisi, kemudian signal direkonstruksi kembali.
% 
% pada penelitian ini, wavelet digunakan untuk mereduksi fitur dari setiap beat
% dengan menggunakan coefisien aproksimasi-nya sebagai fitur, sesuai dengan
% tingkat level dekomposisi dan mengabaikan coefisien detail, dimana mengandung
% signal frekuensi tinggi.
% 
% 
% Transformasi Fourier
% Sampai sekarang transformasi Fourier mungkin masih menjadi transformasi yang
% paling populer di area pemrosesan sinyal digital (PSD). Transformasi Fourier
% memberitahu kita informasi frekuensi dari sebuah sinyal, tapi tidak informasi
% waktu (kita tidak dapat tahu di mana frekuensi itu terjadi). Karena itulah
% transformasi Fourier hanya cocok untuk sinyal stationari (sinyal yang informasi
% frekuensinya tidak berubah menurut waktu). Untuk menganalisa sinyal yang
% frekuensinya bervariasi di dalam waktu, diperlukan suatu transformasi yang dapat
% memberikan resolusi frekuensi dan waktu disaat yang bersamaan, biasa disebut
% analisis multi resolusi (AMR). AMR dirancang untuk memberika resolusi waktu yang
% baik dan resolusi frekuensi yang buruk pada frekuensi tinggi suatu sinyal, serta
% resolusi frekuensi yang baik dan resolusi waktu yang buruk pada frekuensi rendah
% suatu sinyal. Pendekatan ini sangat berguna untuk menganalisa sinyal dalam
% aplikasi-aplikasi praktis yang memang memiliki lebih banyak frekuensi rendah.
% Transformasi wavelet adalah suatu AMR yang dapat merepresentasikan informasi
% waktu dan frekuensi suatu sinyal dengan baik. Transformasi wavelet menggunakan
% sebuah jendela modulasi yang fleksibel, ini yang paling membedakannya dengan
% transformasi Fourier waktu-singkat (STFT), yang merupakan pengembangan dari
% transformasi Fourier. STFT menggunakan jendela modulasi yang besarnya tetap, ini
% menyebabkan dilema karena jendela yang sempit akan memberikan resolusi frekuensi
% yang buruk dan sebaliknya jendela yang lebar akan menyebabkan resolusi waktu
% yang buruk.
