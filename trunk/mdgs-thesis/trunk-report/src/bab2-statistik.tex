\section{Analisis Kinerja}
Studi perbandingan umumnya dilakukan berdasarkan eksperimen  dengan
menggunakan data simulasi maupun menggunakan data nyata. Untuk menunjukkan suatu
algoritma memiliki kinerja yang signifikan terhadap yang lain, dapat dilakukan
melalui beberapa ujicoba yang kemudian data yang dihasilkan dapat di analisis
secara statistik untuk menunjukkan apakah suatu algoritma lebih baik secara
signifikan atau tidak. Kuncheva pada bukunya \cite{Kuncheva:2004} menyebutkan
beberapa uji tersebut diantaranya;

\subsection{McNemar Test}
Pada uji test mcNemar, hasil ujicoba dianalisis dengan menghitung peluang
perbedaan kesalahan pengenalan antara algoritma (\emph{Type I Error}). Misal
terdapat dua algoritma $A$ dan $B$;\\
\begin{tabular}{llp{0.7\textwidth}}
$n_{00}$ &=& jumlah data yang salah dikenali oleh $A$ dan $B$ \\
$n_{01}$ &=& jumlah data yang salah dikenali oleh $A$ tapi tidak $B$ \\
$n_{10}$ &=& jumlah data yang salah dikenali oleh $B$ tapi tidak $A$ \\
$n_{11}$ &=& jumlah data yang benar dikenali oleh $A$ dan $B$ \\
\end{tabular}

\noindent Maka perbedaan antara yang diharapkan dengan hasil observasi dapat
dihitung secara statistik dengan persamaan \ref{};

\begin{align}
\label{eq:mcnemar}
x^2 = \frac{(\lvert n_{01} - n_{10}\rvert - 1)^2}{n_{01} + n_{10}}
\end{align}

\noindent dimana nilai dari $x^2$ mendekati distribusi dari $\chi^2$ dengan 1
degree of freedom. null hypothesis ($H_0$=algoritma memiliki tingkat error yang
sama) dapat ditolak jika nilai $\lvertx^2\rvert > z$ dimana $z$ adalah nilai
tabular pada level signifikan tertentu, misal 0.05 dengan 1 derajat kebebasan.
Hal ini berarti kedua algoritma memiliki perbedaan tingkat error yang
signifikan.

\subsection{Cross-Validation}
Validasi silang (\emph{cross-validation}) merupakan metode statistik yang
digunakan untuk uji coba dengan memperhitungkan tingkat variasi dari data
set. Variasi tersebut dilakukan dengan membagi data menjadi beberapa sub bagian
dan menggunakannya baik sebagai data training maupun testing berdasarkan porsi
tertentu. Terdapat beberapa variasi \emph{cross validation} yang dapat dilakukan
diataranya;

\subsubsection*{K-Hold Out paired t-test}
Uji K-Hold out sangat umum digunakan untuk membandingkan algoritma dalam machine
learning. Ujicoba dilakukan dengan menggunakan data set $Z$ dimana data tersebut
akan dipartisi menjadi training dan testing set dengan porsi biasanya
$\frac{2}{3}$ training dana $\frac{1}{3}$ untuk testing (\emph{hold-out
method}), atau mungkin juga dengan mebagi data sama rata $\frac{1}{2}$ untuk
training dan testing. Kemudian algoritma diujicobakan dengan kedua data
tersebut, pelatihan dengan training set, dan di uji dengan testing set. 

Proses ujicoba akan dilakukan berulang sebanyak $K$ kali dimana umumnya nilai
 $K=30$.Jika akurasi dari hasil eksperimen dinotasikan $P_A$ untuk algoritma $A$
 dan $P_B$ untuk algoritma $B$, maka untuk $K$ ujicoba didapat $P^1 = P_A^1 -
 P_B^1$ sampai $P^K = P_A^K - P_B^K$. Asumsi yang dipakai disini adalah bahwa
 perbedaan akurasi disini adalah independent dari distribusi normal.
 
Untuk menguji tingkat signifikansi kinerja algoritma $A$ dan $B$, digunakan
\emph{paired t-test}, dimana null hypothesis ($H_0$ = akurasi tidak berbeda
secara signifikan), maka persamaan berikut memiliki \emph{t-distribution} dengan
$K-1$ \emph{degree of freedom}

\begin{align}
\label{eq:t-test}
t = \frac{\bar{P}\sqrt{K}}{\sqrt{\sum_{i=1}^K (P^i - \bar{P})^2 / (K - 1)}}
\end{align}

\noindent dimana $\bar{P} = \frac{1}{K} \sum_{i=1}^K P^i$. Untuk menarik
kesimpulan apakah null hypothesis di tolak atau diterima, nilai $t$ test yang
didapat dari persamaan \ref{equ:t-test} dibandingkan dengan nilai pada
tabel distribusi $t$ pada tingkat signifikansi yang diinginkan pada $K-1$ degree
of freedom. Umumnya pada perbandingan algoritma ini, tingkat signifikasi 0.05
digunakan. Jika nilai $\lvert t \rvert > z$ dengan $z$ adalah nilai tabular yang
didapat, maka $H_0$ dapat ditolak, dan menerima bahwa memang terdapat perbedaan
yang signifikan antara kedua algoritma yang dibandingkan.

\subsubsection*{K-Hold Cross Validation paired t-test}
Metode validasi silang \emph{K-Hold} berusaha untuk menghindari overlap pada
testing data, dimana pada metode ini, data sampel di partisi menjadi $K$ bagian
dengan jumlah yang hampir sama, dan setiap bagian digunakan untuk testing
algoritma yang telah dilatih menggunakan $K-1$ bagian yang lain-nya. Pada metode
ini diasumsikan perbedaan yang dihasilkan adalah independen dari distribusi
normal. Dengan menggunakan perhitungan statistik $t$ yang sama pada persamaan
\ref{eq:t-test}, kemudian nilai $t$ dibandingkan dengan nilai dari tabel
distribusi $t$ untuk menarik kesimpulan.

\subsubsection*{Dietterich�s 5 3 2-Fold Cross-Validation Paired t-Test}
Pada metode ini, prosedur ujicoba dilakukan dengan mengulang \emph{2-fold cross
validation} sebanyak 5 kali, dimana disetiap eksekusi validasi silang, data
sampel dipartisi menjadi dua sama banyak. Algoritma $A$ dan $B$ dilatih
menggunakan setengah data #1 dan diuji dengan setengah data #2, kemudian
dihitung akurasi $P_A^1$ dan $P_B^1$. kemudian data training dan testing di
silang, dan dihitung $P_A^2$ dan $P_B^2$. Perbedaannya kemudian dihitung sebagai
berikut;
\begin{align}
P^1 = P_A^1 - P_B^1 \nonumber \\
\text{dan} \nonumber \\
P^2 = P_A^2 - P_B^2 \nonumber 
\end{align}

\noindent estimasi dari rata-rata dan varian dari perbedaan diatas, untuk
tow-fold cross validation dihitung sebagai berikut;
\begin{align}
\bar{P} = \frac{P^1 + P^2}{2}; \qquad s^2 = (P^1 - \bar{P})^2 + (P^2 -
\bar{P})^2
\end{align}

\noindent dengan $P_i^1$ adalah selisih $P^1$ pada eksekusi ke-$i$, dan $s_i^2$
adalah estimasi varian dari setiap eksekusi $i, i = 1, \dots, 5$, maka
perhitungan $\tilde{t}$ adalah sebagai berikut;
\begin{align}
\tilde{t} = \frac{P_1^1}{\sqrt{\frac{1}{5}\sum_{i=1}^5 s_i^2}}
\end{align}

\noindent Hanya satu dari 10 nilai perbedaan yang didapat akan dipergunakan
dalam perhitungan statistik diatas. Pada metode ini, $\tilde{t}$ mendekati 
distribusi $t$ dengan 5 derajat kebebasan. Penarikan kesimpulan sama dengan
metode-metode sebelumnya.

\newpage
\section{Revision}
\begin{itemize}
  \item Cari paper yang menyatakan GLVQ menjamin konvergensi bobot
  \item Cari paper yang menyatakan GLVQ tidak sensitif terhadap inisialisasi
  bobot awal. ketemu -> \cite{Sato-1999}
  
\end{itemize}

% \newpage
% \section{Hasil presentasi Prof, Rudi di seminar reboan}
% 
% differentiable method like sigmoid is used for possible to find the minimization
% of error.
% 
% error prediction
% \begin{align}
% \end{align}
% 
% cross entropy error function
% \begin{align}
% 	E(W,V) = - \sum d_i \log p_i + (1 - d_i) \log(1 - p_i)
% \end{align}
% $d_i$ is desired output
% 
% Optimization method
% \begin{itemize}
%   \item Gradient descent / error backprop
%   \item Conjugate gradient
%   \item Quasi newton method
%   \item Geneticalgorithm
% \end{itemize}
% 
% neural network considered as well trained if it can predict training data and
% cross validation separtely
% 
% 
% Network Pruning : the way to remove unrelevant connection, so at the end we have
% network with connection that is relevant
% try to set a weight = 0 and if the performance is not affected, then it is more
% likely tobe unrelevant
% 
% Rule extraction
% 
% Re-RX
% Discreate / continous varaible
% algorithm Re-Rx(S, D, C)
% 
% Card Dataset
% 
% 
% AUC , ACC, 
% \begin{align}
% =\sum_{i=1}^{m}\sum_{i=1}^{m}
% \end{align}
% 
% $AUC_d$ = $AUC$ = 1 - fp + tp / 2
% 
% %=============================================