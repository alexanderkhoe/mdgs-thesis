%
% Halaman Abstract
%
% @author  I Made Agus Setiawan
% @version 1.00
%

\chapter*{ABSTRACT}

\vspace*{0.2cm}

\noindent \begin{tabular}{l l p{11.0cm}}
	Name&: & \penulis \\
	Program&: & \programstudi \\
	Title&: & \judulInggris \\
\end{tabular} \\ 

\vspace*{0.5cm}

\noindent
Arrhythmia or Cardiac Arrhythmia is one of heart
disease type that can be diagnosed by a standard electrocardiogram (ECG). By
means of an electrocardiogram, doctors can analyze the electrical activity of
the heart and determine the type of arrhythmia currently suffered. Computerized
process was divided into three steps: data preprocessing, feature extraction and
classification. In preprocessing step, beat by beat signal was segmented using
pivot R peak followed by baseline wander and outlier removal . Wavelet algorithm
was applied for feature extraction and selection. ECG signal is then classified
into 6 and 12 classes using new classification method namely Fuzzy-Neuro
Learning Vector Quantization (FNGLVQ) which is an adaptation on Fuzzy-Neuro
method into GLVQ developed by A.Sato.

10-Fold Cross Validation was applied to verify the system. The result of this
study indicate that average recognition rate for Arrhytmia beat using
FNGLVQ method produce 98.53\% and 96.33\% for 6 and 12 classes. This method
produce better result than GLVQ where the recognition rate are 97.03\% and
94.13\%.
\\


\vspace*{0.2cm}

\noindent Keywords: \\ 
\noindent Fuzzy-Neuro Generalized Learning Vector Quantization, FNGLVQ,
GLVQ, Arrhytmia beat detection system\\

\newpage