%-----------------------------------------------------------------------------%
\chapter{\babSatu}
\label{bab:1}
%-----------------------------------------------------------------------------%

%-----------------------------------------------------------------------------%
\section{Latar Belakang}
%-----------------------------------------------------------------------------%
Kesehatan merupakan salah satu bidang sentral yang menjadi fokus dalam 
pembangunan. Tingginya angka penyakit menular seperti DHF dan TBC masih 
menjadi permasalahan, ditambah lagi dengan pergeseran pola hidup yang tidak 
seimbang yang menyebabkan penyakit degeneratif semakin meningkat.

% Sesuai Survei Kesehatan Rumah Tangga Departemen Kesehatan, 2001 proporsi 
% kematian akibat Penyakit Tidak Menular  meningkat  dari 25.41\% di tahun 
% 1990  menjadi 48.53\% di tahun 2001. Sedangkan proporsi kematian karena 
% Penyakit Jantung dan Pembuluh Darah meningkat dari 9.1\% di tahun 1986 
% meningkat menjadi 26.3\% di tahun 2001
% \footnote{\url{http://id.inaheart.or.id/}}. 
Berdasarkan laporan dari Badan Kesehatan Dunia - WHO, proporsi  kematian akibat
dari penyakit jantung pada tahun 2002 mencapai 29\% dan  menjadi penyebab kematian no
1 di Indonesia, diatas penyakit menular yang menempati  posisi ke-2, seperti
yang diperlihatkan pada \pic~\ref{fig:cardiodeath}  

\addFigure{width=1\textwidth}{pics/cardiodeath.jpg}{fig:cardiodeath}{Proporsi
Beberapa penyebab kematian di Indonesia 2002-WHO.}

Perkembangan teknologi kedokteran dan kesehatan yang digunakan dalam tindakan 
medis saat ini, berkaitan dengan kemajuan ilmu kedokteran itu sendiri yang 
didukung oleh beberapa ilmu-ilmu lainnya seperti; ilmu pengetahuan alam, ilmu
rekayasa serta kemajuan teknologi informasi. Pengembangan perangkat teknologi
informasi, baik perangkat keras maupun perangkat lunak, dapat digunakan untuk
membantu proses penanganan kesehatan yang dilakukan oleh paramedis dengan lebih
baik. Contohnya, deteksi penyakit kanker, deteksi penyakit paru, analisa dan
visualisasi detak jantung dan lain-lain. Dengan pemanfaatan teknologi informasi
ini, penanganan penyakit dapat dilakukan dengan mudah, cepat dan akurat.

Alat yang umum digunakan oleh para pakar dalam mendiagnosa kinerja jantung
adalah electrocardiogram (ECG). Alat ini akan merekam aktifitas elektrik dari
jantung dan merepresentasikannya dalam bentuk gelombang, atau biasa
diistilahkan gelombang ECG. Tidak sembarang orang dapat menerjemahkan gelombang
ECG tersebut, bahkan antara pakar sendiri terkadang terjadi perbedaan
interpretasi. Disamping itu, menurut Perhimpunan Dokter Spesialis
Kardiovaskular Indonesia (PERKI)\footnote{\url{http://www.inaheart.org}}, jumlah
ahli jantung di Indonesia sekitar 452 orang. Sangat sedikit sekali jika 
dibandingkan dengan jumlah penduduk yang berjumlah sekitar 228 juta jiwa, 
atau setiap dokter menangani 550 ribu jiwa
(1:550.000), dan itupun tidak menyebar secara merata, masih terfokus di
kota-kota besar seperti Jakarta. 

Oleh karena itu, sangat dibutuhkan suatu mekanisme untuk membantu proses
pendeteksian awal penyakit jantung yang mudah dan dapat dilakukan tidak hanya
oleh pakar, namun oleh siapa saja, bahkan pasien sekalipun. Salah satu
mekanisme yang dapat dikembangkan adalah suatu sistem cerdas pendeteksi gejala
penyakit jantung secara otomatis berdasarkan gelombang ECG. Dengan sistem ini,
gelombang ECG dapat diinterpretasikan oleh suatu perangkat lunak, yang
sebelumnya sudah dilatih untuk mengenal berbagai macam kemungkinan penyakit
jantung, sehingga proses interpretasi awal dapat dilakukan lebih mudah, lebih
cepat dan lebih akurat. 

Terdapat beberapa tipe penyakit jantung, salah satunya adalah Arrhytmia
atau Cardiac Arrhytmia. Arrhytmia adalah kelainan pada irama jantung atau
ketidak-beraturan denyut jantung. Pada kelainan jantung ini, denyut jantung
mungkin terlalu cepat, terlalu lambat atau tidak beraturan. Beberapa pasien
terkadang tidak menyadari akan kondisi tersebut, dan beberapa pasien lainnya
merasakan gejala-gejala klinis yang timbul seperti palpitasi, pusing, nyeri
dada, nafas yang pendek. Di lain pihak, orang normal juga memungkinkan mengalami
perasaan yang sama seperti palpitasi, tapi hal tersebut bukan Arrhytmia. Oleh
karena itu, kelaian ini tidak cukup hanya di diagnosa berdasarkan gejala klinis
saja, dan pemanfaatan ECG akan sangat membantu dalam proses tersebut. 

Berbagai penelitian telah dilakukan dalam kaitan pengenalan berbagai
penyakit Arrhytmia. Terdapat banyak penelitian dengan mengaplikasikan jaringan
saraf tiruan (JST) dan berbagai variasinya sebagai metode pendeteksian. Yeap
\cite{yeap.1} menggunakan backpropagation untuk mendeteksi arrhytmia dengan
menggunakan AHA database. Ozbay \cite{ozbay.1} memanfaatkan tipe arrhytmia dari
MIT-BIH sebagai data untuk pelatihan dan menggunakan data dari institusi-nya
sebagai data testing. Sedangkan Ceylan \cite{Ceylan.1} melakukan studi 
perbandingan beberapa metode extraksi fitur mulai dari Principal Component
Analysis(PCA), Wavelet transform (WT) dan Fuzzy C-Mean Clustering (FCM) untuk
reduksi data dan menggunakan BackPropagation untuk klasifikasi signal. Empat
struktur dibangun, FCM-NN, PCA-NN, FCM-PCA-NN and WT-NN dan menggunakan data 
arrhytmia dari MIT-BIH database. Studi perbandingan yang lain dilakukan oleh
Ghongade et.al \cite{Ghongade.1}. Ghongade melakukan perbandingan beberapa
metode ekstraksi fitur yakni DFT, PCA, DWT terhadap 180 signal sample dan
disertai dengan ekstrasi fitur berdasarkan morfologi signal yakni R-peak, QRS
area serta Q-S distance dengan menggunakan Backpropagation. Data yang digunakan adalah tiga
tipe arrhytmia dari MIT-BIH database. Guler \cite{guler.1} menggunakan multi
layer perceptron (MLP) dan combine network model untuk mengklasifikasi signal ECG (beat) 
dan menggunakan distribusi statistik dari hasil analisis spectral signal ECG
menggunakan DWT sebagai fitur klasifikasi. Elsayad dalam papernya,
\cite{Elsayad.1} mengaplikasikan competitive based learning, yakni Learning
Vector Quantization - LVQ, untuk mengklasifikasi data ECG arrhytmia dari UCI
database. Data diolah menggunakan PCA dan diaplikasikan pada beberapa
varian LVQ. Karraz et.al. dalam papernya \cite{Karraz.1} mengembangkan
klasifikasi denyut jantung otomatis menggunakan jaringan saraf tiruan
berdasarkan Bayesian Framework. Studi dilakukan terhadap 5 tipe ECG
arrhytmia dari MIT-BIH database dan fitur yang digunakan adalah ECG morphology
dan time intervals.

Beberapa penelitian mengaplikasikan teori fuzzy untuk mendeteksi berbagai tipe
arrhytmia seperti yang dilakukan oleh Anuradha et.al pada
papernya \cite{Anuradha.1}. Nonlinier dynamic dari ECG signal digunakan sebagai
karakteristik dari arrhytmia diantaranya Spectral entropy, Poincar� plot
geometry, Largest Lyapunov exponent and Detrended fluctuation analysis yang
diekstrak dari heart rate signal. Kemudian linguistik variabel (fuzzy set)
digunakan untuk merepresentasikan ECG fitur dan fuzzy conditional statement
digunakan sebagai rule. Exarchos et.al \cite{Exarchos.1} mengembangkan
metodologi dimana secara otomatis membuat fuzzy expert system dari data training
dengan menggunakan decision tree dimana digunakan untuk mengklasifikasi
arrhytmia (MIT-BIH database) dan ischaemic (ST-T database). Yeh et.al
\cite{Yeh.1} mengembangkan Fuzzy logic method (FLM) untuk mendeteksi lima
tipe arrhytmia dari signal ECG  melalui fuzzy inference engine dan operasi
defuzzifikasi dengan menggunakan data dari MIT-BIH arrhytmia database.

Hao et.al \cite{Hao.1} mengembangkan metode klasifikasi arrhytmia menggunakan
Support Vector Machine (SVM) dengan menggunakan PCA sebagai metode ekstraksi
fitur dari data MIT-BIH database. Studi dilakukan terhadap empat tipe arrhytmia
yakni Normal, LBBB, RBBB dan PVC. Nasiri et.al. \cite{Nasiri.1}
mengembangkan metode klasifikasi arrhytmia dengan mengintegrasikan SVM dan
Genetic Algorithm dimana GA digunakan untuk meningkatkan kinerja dari SVM dari
sisi generalisasi-nya. Sedangkan Melgani et.al. \cite{Melgani.1}
mengintegrasikannya dengan Particle Swarm Optimization (PSO) dengan menggunakan
data dari MIT-BIH database.

Metode pemrosesan ECG otomatis untuk proses klasifikasi denyut jantung
dikembangkan oleh Philip et.al.\cite{Philip.1} dimana kategori dari arrhytmia
yang digunakan adalah lima kategori yang direkomendasikan oleh standar ANSI/AAMI
EC57:1998 yaitu normal beat, ventricular ectopic beat (VEB), supraventricular
ectopic beat (SVEB), fusion of a normal and a VEB, dan unknown beat type. Lima
kategori tersebut merupakan pengelompokkan dari 15 kategori arrhytmia yang
terdapat pada MIT-BIH database. Fitur yang digunakan adalah berdasarkan pada
interval denyut jantung, interval RR dan morfologi dari ECG. 

Dari beberapa literatur yang dipelajari, belum terlalu banyak peneliti yang
menggunakan pendekatan Competitive Based Learning seperti LVQ digunakan untuk
mendeteksi kelainan jantung khusus-nya Arrhytmia. Oleh sebab itu pada penelitian
ini, \saya akan melakukan studi tentang klasifikasi signal ECG arrhytmia dengan
menggunakan pendekatan competitive based learning, yakni Generalized LVQ
\cite{Sato.1}, yang merupakan generalisasi dari LVQ2.1 , dimana GLVQ memiliki
keunggulan dalam hal menjamin konvergensi dalam proses pelatihan yang tidak
dimiliki LVQ2.1. Disamping itu juga, LVQ dan varian-nya merupakan jenis JST yang
arsitekturnya paling sederhana sehingga berimplikasi pada kecepatan proses
pelatihan. Pada tahap berikutnya, GLVQ yang perhitungan kemiripannya
berdasarkan euclidean distance dimodifikasi dengan menggunakan mahalanobis
distance  karena mahalanobis distance memperhitungkan tingkat distribusi  dari
vektor fitur dan berguna  dalam membandingkan vektor fitur dimana tiap  elemen
memiliki rentang dan varian yang berbeda-beda.

% Dataset yang akan digunakan pada penelitian ini adalah data yang
% telah tersedia bebas dari physionet\footnote{http://physionet.net} yakni MIT-BIH
% arrhytmia database. Dataset akan diolah terlebih dahulu menggunakan pendekatan
% analisis spectral, Wavelet Transform.
% 
% Namun untuk dapat dikenali oleh sistem, data gelombang ECG perlu diolah lebih
% lanjut sehingga lebih mudah dikenali dan hasil yang diperoleh memiliki tingkat
% akurasi yang tinggi. Oleh karena itu, dalam studi mandiri ini akan dipelajari
% metode untuk mengolah data dasar (\textit{raw preprocessing}), menganalisa,
% memvisualisasikan data gelombang ECG, baik berupa data analog (ECG paper)
% maupun data digital.

%-----------------------------------------------------------------------------%
\section{Permasalahan}
%-----------------------------------------------------------------------------%
Pada bagian ini akan dijelaskan mengenai definisi permasalahan 
yang \saya~hadapi dan ingin diselesaikan serta asumsi dan batasan 
yang digunakan dalam menyelesaikannya. 

%-----------------------------------------------------------------------------%
\subsection{Definisi Permasalahan}
%-----------------------------------------------------------------------------%
% \todo{Tuliskan permasalahan yang ingin diselesaikan. Bisa juga
% 	berbentuk pertanyaan}

Berdasarkan uraian latar belakang diatas, penelitian ini secara umum
terbagi menjadi beberapa permasalahan sebagai berikut;

\begin{enumerate}
  \item Bagaimana pengolahan data awal yang harus dilakukan agar sesuai dengan
  kebutuhan proses pengenalan Arrhytmia.
  \item Bagaimana penanganan noise data dapat memperbaiki kinerja dari proses
  pengenalan Arrhytmia.
  \item Bagaimana metode pembelajaran berbasis kompetisi dapat digunakan untuk
  proses pengenalan Arrhytmia.
  \item Bagaimana metode perhitungan kemiripan mahalanobis distance dapat
  digunakan untuk memperbaiki kinerja dari proses pengenalan Arrhytmia.
\end{enumerate}

%-----------------------------------------------------------------------------%
\subsection{Batasan Permasalahan}
%-----------------------------------------------------------------------------%
% \todo{Umumnya ada asumsi atau batasan yang digunakan untuk 
% 	menjawab pertanyaan-pertanyaan penelitian diatas.}
Ruang lingkup yang digunakan dalam penelitian ini adalah sebagai berikut;
\begin{enumerate}
  \item Dataset yang digunakan sebagai data pelatihan dan evaluasi adalah data
  ECG Arrhytmia yang didapat dari physionet, yakni data MIT-BIH database dimana
  data tersebut sudah dilengkapi dengan berbagai anotasi dari para ahli jantung
  secara manual.
  \item Jumlah kategori arrhytmia yang akan dideteksi dalam penelitian ini
  adalah sebanyak 12 kategori termasuk didalamnya signal denyut jantung hasil
  dari penggunaan alat pacu jantung (pacemaker).
  \item Sinyal ECG yang digunakan pada penelitian ini hanya menggunakan
  sinyal hasil sandapan MLII.
  \item Frekuensi cuplik (sampling rate) dari sinyal ECG yang digunakan adalah
  360Hz.
  \item 
\end{enumerate} 
%-----------------------------------------------------------------------------%
\section{Tujuan}
%-----------------------------------------------------------------------------%
Tujuan dari penelitian ini adalah sebagai berikut;
\begin{enumerate}
  \item Mencari beberapa kemungkinan pengolahan data ECG agar dapat digunakan
  dalam proses pelatihan dan pengenalan kelainan irama jantung (Arrhytmia).
  \item Mengembangkan sistem yang mampu mengenali kelainan irama jantung
  (arrhytmia) berdasarkan data signal ECG dengan menggunakan pendekatan
  competitive based learning.
\end{enumerate}


%-----------------------------------------------------------------------------%
\section{Posisi Penelitian}
%-----------------------------------------------------------------------------%
% \todo{Posisi penelitian Anda jika dilihat secara bersamaan dengan 
% 	peneliti-peneliti lainnya. Akan lebih baik lagi jika ikut menyertakan 
% 	diagram yang menjelaskan hubungan dan keterkaitan antar 
% 	penelitian-penelitian sebelumnya}
Penelitian ini merupakan bagian dari penelitian dan pengembangan alat pendeteksi
kelainan jantung portable oleh Wisnu J et.al, sehingga disamping hasil yang
akurat, dibutuhkan juga sistem yang memiliki waktu eksekusi yang cepat. Oleh 
karena itu, pada penelitian ini dipilih pendekatan competitive based Learning.
Berikut adalah bagan posisi penelitian ini terhadap penelitian-penelitian
lainnya sebagai bagian dari pengembangan alat pendeteksi kelaian jantung
portable.

\addFigure{width=1\textwidth}{pics/baganpospeneliti.png}{fig:bagan}{Diagram
posisi penelitian dan kontribusi.}

Pada bagan diatas dapat dilihat bahwa penelitian ini berfungsi sebagai
penelitian awal dalam menginvestigasi berbagai kemungkinan struktur data /
fitur yang nantinya dapat digunakan sebagai referensi dalam mengembangkan alat
perekam ECG seperti yang dilakukan oleh Pak Asep. Berkaitan dengan alat
perekam ECG yang dikembangkan, hal-hal yang perlu diperhatikan adalah mengenai
sampling rate yang digunakan dalam merekam denyut jantung. Karena pada
penelitian ini menggunakan data dari physionet dengan sampling rate tertentu
sebagai data pelatihan, maka ada baiknya sampling rate yang digunakan
sama, atau paling tidak dipertimbangkan langkah-langkah penyesuaian
tertentu. Hal ini dikarenakan data yang dihasilkan dari alat perekam ECG
yang dikembangkan tidak memiliki label, seperti data MIT-BIH dan juga akan
mengandung noise yang tinggi. terkecuali nantinya akan ada pakar jantung
yang akan memberikan label pada data hasil perekaman.

Selain itu penelitian ini juga digunakan sebagai referensi oleh tim FPGA dalam
mengembangkan embeded system pada alat portable. 

Secara garis besar, kontribusi \saya pada penelitian ini adalah sebagai berikut;
\begin{enumerate}
  \item 
\end{enumerate} 

%-----------------------------------------------------------------------------%
\section{Metodologi Penelitian}
%-----------------------------------------------------------------------------%
Metode yang digunakan dalam penelitian ini adalah sebagai berikut;
\begin{enumerate}
  \item Studi literatur \\
		Pada tahap ini, \saya melakukan kajian literatur mengenai perkembangan
		terakhir penelitian di bidang ECG, mempelajari mengenai pendekatan
		yang dilakukan para peneliti dalam mencari solusi permasalahan, terutama
		proses pengenalan kelainan jantung seperti Arrhytmia. Literatur ini didapat
		dari buku, thesis dan paper baik dalam jurnal dalam negeri maupun luar negeri,
		serta informasi dari berbagai sumber yang berkaitan dengan topik thesis ini. 
  \item Pengolahan data\\
		Pada tahap ini, \saya mempersiapkan data yang akan digunakan dalam penelitian
		ini. Data ECG yang digunakan merupakan data yang diambil dari repository
		online, physionet.net. Beberapa pengolahan awal diaplikasikan terhadap data
		tesebut sesuai dengan tujuan yang ingin dicapai.
  \item Perancangan sistem \\
		Tahap perancangan sistem meliputi pembuatan rancangan sistem untuk
		pendeteksian kelainan denyut jantung dengan menggunakan pendekatan competitive
		base learning.
  \item Implementasi sistem \\
		Tahap ini merupakan implementasi dari tahap rancangan sebelumnya. pada tahap
		ini diimplementasikan beberapa variasi dari supervised competitive based
		learning mulai dari LVQ (1,21,3), GLVQ, GLVQ-PSO beserta beberapa varian-nya. 
  \item Uji coba dan analisis sistem \\
		Pada tahap ini dilakukan ujicoba sistem yang telah diimplementasikan terhadap
		data yang telah diolah pada tahap sebelumnya dengan menggunakan berbagai
		skenario percobaan. Kemudian dilanjutkan dengan analisis terhadap hasil  yang
		didapat untuk mengetahui performa dari sistem.
  \item Penulisan laporan \\
		Bagian akhir dari metodologi penelitian ini adalah penulisan thesis yang
		memuat semua hasil penelitian yang telah dilakukan.
\end{enumerate}

%-----------------------------------------------------------------------------%
\section{Sistematika Penulisan}
%-----------------------------------------------------------------------------%
Sistematika penulisan laporan akhir ini adalah sebagai berikut:
\begin{itemize}
	\item Bab 1 \babSatu \\
		Bab ini berisi latar belakang yang memotivasi penelitian ini,  perumusan
		masalah, tujuan, ruang lingkup, metodologi penelitian serta sistematika penulisan.
	\item Bab 2 \babDua \\
		Bab ini berisi pembahasan mengenai sekilas tentang penyakit jantung, dan
		khususnya beberapa variasi kelainan irama jantung (Arrhytmia). Juga akan
		dibahas mengenai pendekatan dalam machine learning khususnya
		Competitive Base learning mulai dari standard LVQ sampai Generalized LVQ
		(GLVQ) dan bagaimana metode tersebut diaplikasikan terhadap ECG data.
	\item Bab 3 \babTiga \\
		Bab ini berisi pembahasan mengenai pengolahan data MIT-BIH dari physionet.net
		disesuaikan dengan kebutuhan dalam penelitian ini, mulai dari bagaimana
		mekanisme pengambilan data fitur (per beat), baseline wander removal (BWR),
		reduksi fitur dengan menggunakan wavelet serta filtering dataset dari outlier
		yang dapat mengganggu performa dari sistem nantinya.
	\item Bab 4 \babEmpat \\
		Bab ini berisi pembahasan mengenai rancangan dan implementasi dari algoritma
		yang telah ditentukan beserta beberapa modifikasi yang dilakukan untuk
		memperbaiki performa dari algoritma.
	\item Bab 5 \babLima \\
		Bab ini berisi pembahasan mengenai ujicoba yang dilakukan beserta
		skenario yang telah ditentukan dilanjutkan dengan evaluasi unjuk kerja
		beberapa algoritma yang dibandingkan satu dengan yang lain. Pada bab
		ini juga dilakukan analisis secara statistik untuk mengetahui efek modifikasi
		yang telah dilakukan terhadap tingkat unjuk kerja sistem.
	\item Bab 6 \kesimpulan \\
		Bab ini berisi kesimpulan yang diperoleh dari hasil ujicoba dan
		analisis penelitian yang telah dilakukan dan saran bagi pengembangan
		selanjutnya.
\end{itemize}