\section{Himpunan Fuzzy}

Himpunan Fuzzy adalah suatu himpunan dimana elemennya memiliki derajat
keanggotaan. Himpunan Fuzzy merupakan generalisasi dari teori himpunan 
klasik. Secara formal, definisi dari himpunan fuzzy adalah sebagai berikut;

\begin{quotation}
	Suatu himpunan fuzzy $A$ pada semesta $X$ dengan $A \subset X$ dikarakterik
	oleh suatu fungsi keanggotaan (\emph{membership function}) $f_A(x)$  dimana
	setiap titik di $X$ dipetakan ke suatu bilangan real $[0,1]$, dengan nilai
	$f_A(x)$ menunjukkan derajat keanggotaan dari $x$ pada himpunan $A$. Sehingga
	semakin dekat nilai $f_A(x)$ dengan pusat (\emph{unity}), semakin tinggi
	derajatnya.
	\cite{Zadeh:1965}
\end{quotation}
  
Suatu himpunan fuzzy adalah kosong (\emph{empty}) jika dan hanya jika nilai
fungsi keanggotaannya adalah 0 pada $X$. Dua himpunan fuzzy $A$ dan $B$ adalah
sama, $A = B$ jika dan hanya jika $f_A(x) = f_B(x)$ untuk semua $x \in X$.
Sedangkan komplemen dari suatu himpunan fuzzy $A$ dinotasikan dengan $A'$
didefinisikan sebagai $f_{A'} = 1 - f_A$.

Seperti yang diuraikan diatas, suatu himpunan fuzzy dikarakterisasi dengan suatu
fungsi keanggotaan. Terdapat beberapa fungsi keanggotaan yang sering digunakan
dalam berbagai aplikasi sebagai berikut;

\begin{enumerate}
  \item Fungsi Keanggotaan segitiga.\\
  Fungsi keanggotaan segituga sangat umum dan banyak digunakan dalam pembuatan
  suatu sistem, karena kesederhanaannya dan untuk menyusunnya hanya membutuhkan
  3parameter semisal nilai minimum, rata-rata dan maksimum. Definisi dari fungsi
  keanggotaan segitiga adalah sebagai berikut;
  \begin{align}
	\label{eq:mftrim}
	f(x, a, b, c) = \left\{ 
	\begin{array}{ll}
	0 & , x \leq a\\
	\frac{x - a}{b - a} & , a < x \leq b \\
	\frac{c - x}{c - b} & , b < x < c \\
	0 & , x \geq c
	\end{array}
  \end{align}
   
   Bentuk fungsi keanggotaan segitiga dapat dilihat pada \pic~\ref{fig:mftrim}.
   \addFigure{width=0.6\textwidth}{pics/mftrim.png}{fig:mftrim}{Bentuk fungsi
   keanggotaan segitiga}
   
   \item Fungsi Keanggotaan Trapesium.\\
   Fungsi keanggotaan trapesium didefinisikan sebagai berikut;
   \begin{align}
	\label{eq:mftrap}
	f(x, a, b, c, d) = \left\{ 
	\begin{array}{ll}
	0 & , x \leq a\\
	\frac{x - a}{b - a} & , a < x \leq b \\
	1 & , b < x \leq c \\
	\frac{d - x}{d - c} & , c < x < d \\
	0 & , x \geq c
	\end{array}
  \end{align}
  
  Bentuk fungsi keangotaan segitiga dapat dilihat pada \pic~\ref{fig:mftrap}.
  \addFigure{width=0.6\textwidth}{pics/mftrap.png}{fig:mftrap}{Bentuk fungsi
   keanggotaan trapesium}
   
   \item Fungsi keanggotaan Gaussian.\\
   Fungsi keanggotaan Gaussian didefinisikan sebagai berikut;
	\begin{align}
	\label{eq:mfgaus}
		f(x, m, \sigma) = \exp\left(\frac{(x-x)^2}{2\sigma^2}\right)
	\end{align}

	dimana parameter $m$ dan $\sigma$ adalah pusat dan lebar dari fungsi
	keanggotaan. Bentuk dari fungsi keanggotaan Gaussian dapat dilihat pada
	\pic~\ref{fig:mfgaus}.
	\addFigure{width=0.6\textwidth}{pics/mfgaus.png}{fig:mfgaus}{Bentuk fungsi
	keanggotaan Gaussian}
	    
\end{enumerate}
	
