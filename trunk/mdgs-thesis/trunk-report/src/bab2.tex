% -----------------------------------------------------------------------------%
\chapter{\babDua}
% -----------------------------------------------------------------------------%
% \todo{tambahkan kata-kata pengantar bab 2 disini}
Pada bab ini akan dijelaskan mengenai berbagai literatur yang mendukung
penelitian ini diantaranya mulai dari uraian mengenai penyakit jantung,
khususnya \gls{arrhytmia}, \gls{ecg}, jaringan saraf tiruan, pembelajaran
berbasis kompetisi, algoritma \gls{lvq} dan beberapa varian-nya, seperti
algoritma \gls{glvq}.

% -----------------------------------------------------------------------------%
\section{Penyakit Jantung}
\label{sec:jantung}
Penyakit Jantung (\textit{Heart Disease, Cardiopathy}) adalah segala gangguan
yang timbul dan mempengaruhi jantung \cite{medicinenet.1}. Penyakit jantung
sering juga disebut dengan \textit{Cardiac Disease}. Jika sudah melibatkan
kinerja dari jantung dan pembuluh darah, maka disebut dengan
\textit{Cardiovascular Disease}. Terdapat beberapa jenis penyakit jantung
diantaranya; Angina, Arrhytmia, Congenital heart disease, Coronary artery
disease (CAD), Dilated cardiomyopathy, Heart Attack (Myocardial infarction),
Heart Failure, Hypertrophic cardiomyopathy, Mitral regurgitation, Mitral valve
prolapse, dan Pulmonary stenosis. 

Secara umum, penyakit Jantung disebabkan oleh beberapa faktor, dimana dapat 
dikategorikan menjadi 2 berdasarkan Faktor resikonya yaitu  \textit{Major} dan
\textit{Contributing}.
\begin{enumerate}
    \item \textit{Major Risk Factors} : yaitu faktor yang telah terbukti 
    menyebabkan meningkatnya resiko terhadap penyakit jantung.  Beberapa faktor
    utama yang menyebabkan terjadinya penyakit jantung adalah;
	\begin{itemize}
	    \item Diabetes (Gula Darah)
	    \item Tekanan darah tinggi (Hypertension)
	    \item Merokok (Smoking)
	    \item Kolesterol tinggi (High cholesterol)
	    \item Jarang untuk beraktifitas fisik seperti Olahraga  (Lack of physical
	    activity)
	    \item Kegemukan (Obesity)
	\end{itemize}

	\item \textit{Contributing risk factors} ; yaitu salah satu  yang
	dipertimbangkan oleh dokter yang dapat menyebabkan penyakit jantung seperti;
	\begin{itemize}
	    \item Tingkat Stres
	    \item Akibat Pil KB (Birth control pills)
	    \item Hormon Sex (Sex hormones)
	    \item Alkohol
	\end{itemize}
\end{enumerate}

Pada bab ini \saya akan memfokuskan pembahasan
mengenai arrhytmia, definisi, penyebab dan beberapa kategori penyakit arrhytmia
yang nantinya akan dikenali oleh sistem.

% -----------------------------------------------------------------------------%
\subsection{Arrhytmia}
\label{ssec:arrhytmia}
\Gls{arrhytmia} atau juga disebut dengan Dysrhytmia adalah kelaian/gangguan
urutan irama jantung atau gangguan kecepatan dari proses depolarisasi, repolarisasi
atau keduanya pada jantung \cite{karim.1}. Beberapa pasien terkadang tidak
menyadari akan kondisi tersebut, dan beberapa pasien lainnya merasakan
gejala-gejala klinis yang timbul seperti \gls{palpitasi}, pusing, nyeri dada,
nafas yang pendek. Di lain pihak, orang normal juga memungkinkan mengalami perasaan
yang sama seperti \gls{palpitasi}, tapi hal tersebut bukan termasuk arrhytmia.

\Gls{heartrate} atau laju denyut jantung orang normal berkisar
antara 60 sampai dengan 100 denyut per menit\cite{medicinenet.2}. Jika dilihat
dari \gls{heartrate}, terdapat dua kelompok penyakit \gls{arrhytmia} yaitu;
\begin{itemize}
	\item \textbf{Tachycardia} ; yaitu denyut jantung yang cepat, biasanya
	ditentukan lebih dari 100 denyut per menit. Termasuk didalamnya adalah 
	\textit{sinus tachycardia}, \textit{paroxysmal atrial tachycardia} (PAT), dan
	\textit{ventricular tachycardia}. \textit{Tachycardia}  dapat mengakibatkan
	\gls{palpitasi}. Namun, Tachycardia belum tentu sebuah 	\textit{Arrhythmia}.
	Peningkatan denyut jantung adalah respon  normal pada latihan fisik atau stres
	secara emosional.

	\item \textbf{Bradycardia} ; yaitu denyut jantung yang lambat, biasanya kurang
	dari 60 denyut per menit. Termasuk didalamnya \textit{sinus bradycardia}.
\end{itemize}
 
Menurut Dr. Sjukri Karim dalam bukunya \cite{karim.1}, \gls{arrhytmia}
disebabkan oleh faktor aritmogenik yaitu diantaranya;
\begin{itemize}
  \item Hipoksia : semua penyakit yang menyebabkan defisiensi oksigen pada
  miokard misalnya penyakit paru, kardiomiopati, atau penyakit jantung koroner
  dapat menyebabkan \gls{arrhytmia}.
  \item Iskemia : miokard yang iskemik oleh sebab apa saja merupakan faktor
  pencetus timbulnya \gls{arrhytmia}
  \item Rangsamgam sisimam saraf otonom : rangsangan berlebihan pada saraf
  simpatis mapun parasimpatis dapat menimbulkan \gls{arrhytmia}.
  \item Obat-obatan : semua antiaritmia mempengaruhi fase depolarisasi dan
  repolarisasi jantung, sehingga obat-bat tersebut memiliki efek aritmogenik.
  Selain itu obat-obatan seperti kafein, aminofilin, antidepresan trisiklik dan
  digitalis juga memiliki efek aritmogenik.
  \item Gangguan keseimbangan elektrolit dan gas darah : fase depolarisasi dan
  repolarisasi otot jantung ditimbulkan oleh perpindahan berbagai ion elektrolit
  melalui membran sel.
  \item Regangan dinding otot jantung : dinding jantung yang teregang seperti
  pada dilatasi atrium atau ventrikel akibat gagal jantung, kardiomiopati atau
  penyakit-penyakit katub dapat menyebabkan aritmia.
  \item Kelainan struktur sistem konduksi : penderita yang memiliki fetal
  despersi di AV-node dan fasciculo-ventricular connection, atau yang memiliki
  jalur tambahan (accessory pathway) seperti sindrom Wolff-Parkinson-White
  sangat mudah mengalami aritmia melalui mekanisme preeksitasi.
  \item Interval QT yang memanjang : pada penderita penyakit jantung koroner,
  kelainan struktur jantung atau gangguan elektrolit yang disertai interval QT
  memanjang akan lebih sering terjadi aritmia dibandingkan dengan individu
  normal.
\end{itemize} 

Beberapa penyakit/kelainan yang termasuk dalam \gls{arrhytmia} adalah sebagai
berikut;
\begin{enumerate}
  \item Atrial Fibrilation/Flutter (AF) \\
  AF merupakan irama jantung tidak teratur yang umum yang menyebabkan atrial,
  bagian atas bilik jantung, berkontraksi tidak normal. Penyebab tersering dari
  AF adalah infark miokard, dilatasi atrium kiri, penyakit paru kronis, gagal
  jantung, pasca pembedahan kardiotorasik dan tirotoksikosis.
  
  \item Supraventricular tachycardia (SVT) \\
  Denyut jantung cepat yang	tidak  normal secara teratur yang disebabkan oleh
  tembakan secara cepat impuls listrik  dari atas \textit{atrioventricular} node
  (AV node) di dalam jantung. Termasuk didalamnya adalah takikardia atrium,
  takikardia atrium multifokal, takikardia supraventrikular paroksismal.

  \item Sindrom Wolff-Parkinson-White (WPW) \\
  Merupakan kumpulan gejala yang ditimbulkan oleh impuls dari atrium yang
  dikonduksi ke ventrikel lebih cepat dari biasanya (pre-eksitasi) melalui jalur
  tambahan.

  \item Sick-Sinus Syndrome (SSS) \\
  merupakan kelainan dimana nodus SA tidak dapat mencetuskan impuls secara
  normal akibat dari fibrosis di nodus SA. Gejala klinis yang muncul berupa
  bradikardia sinus, episode henti sinus yang intermiten, dan sindrom
  bradi-takikardia sehingga penderita mengeluh palpitasi, presinkope atau
  sinkope. Biasanya diagnosis SSS dapat dilakukan dengan menggunakan Holter
  Monitor ECG (ECG 24 jam atau lebih).
  
  \item Premature Ventricular Contractions (PVC) \\
  Atau juga disebut Ventricular Premature beat, merupakan kejadian umum  dimana
  denyut jantung diinisiasi oleh ventrikel jantung,   bukan  oleh
  \textit{sinoatrial node} yang merupakan  inisiator denyut jantung yang normal.
  Akibatnya, muncul denyut tambahan yang tidak normal sebelum denyut normal
  mumcul.

  \item Ventricular Tachycardia (VT) \\
  Merupakan suatu irama jantung yang cepat yang berasal dari ruang bawah (atau
  ventrikel) jantung. Laju denyut yang cepat mencegah jantung mengisi cukup
  darah , sehingga sejumlah kecil darah dipompa ke seluruh tubuh. Ini bisa
  menjadi \textit{arrhytmia} yang serius,  terutama pada orang dengan penyakit
  jantung, dan dapat berhubungan dengan banyak  gejala. Seorang dokter harus
  mengevaluasi \textit{arrhytmia} ini.
		
  \item Ventricular fibrillation (VF) \\
  Tembakan impuls yang tidak teratur dan tidak menentu dari ventrikel. Ventrikel
  bergetar dan tidak bisa berkontraksi atau memompa darah ke tubuh.  Ini adalah
  keadaan darurat medis yang harus diobati  dengan \textit{cardiopulmonary
  resuscitation} (CPR) dan \textit{defibrillation} sesegera mungkin.
    
  \item Blok \\
  Blok atau hambatan konduksi dapat dibagi menadi 3 jenis menurut lokasi
  kejadiannya yaitu:
  \begin{itemize}
    \item Blok nodus SA : pada kondisi ini, serabut sinus adalah normal, hanya
    saja gelombang depolarisasi yang dicetuskan terhambat sebelum mencapai
    atrium. Blok SA biasanya tidak memberi gejala sehingga tidak memerlukan
    pengobatan.
    \item Blok nodus AV : semua hambatan konduksi yang terjadi antara nodus SA
    sampai pada berkas HIS-Sistem Purkinje. Blok AV dibagi menjadi Blok AV
    derajat I, II dan III. Mungkin jantung berdetak tidak teratur dan kadang sering
	lebih lambat. Jika serius, \textit{Heart Block} diobati dengan alat pacu
	jantung (PaceMaker). Sebagian besar muncul dari patologi di simpul
	atrioventrikular dimana penyakit ini penyebab paling umum dari \textit{Bradycardia}.
    \item Blok Infranodal : hambatan konduksi yang terjadi pada sistem
    infranodal. Sistem infranodal terdiri dari berkas His dan 3 cabang berkas
    infra-ventrikular yaitu satu right bundle branch (RBB), dan 2 fasikulus dari
    left bundle branch (LBB).\\
	Jika konduksi terhambat pada bagian berkas cabang kanan (RBB), maka disebut
	sebagai Right Bundle Branch Block (RBBB), dan jika konduksi terhambat pada
	bagian berkas cabang kiri (LBB), maka disebut Left Bundle Branch Block (LBBB).
  \end{itemize}
  
  \item Sudden Arrhythmia Death Syndrome (SADS) \\
  Merupakan kematian mendadak yang yang tidak diharapkan yang disebabkan oleh
  kehilangan fungsi jantung secara tiba-tiba \textit{sudden cardiac arrest}.
	
\end{enumerate}

Penanggulangan arrhytmia dapat dilakukan secara farmakologik, kardioversi,
ablasi, pemasangan alat pacu jantung atau dengan tindakan operasi. Alat pacu
jantung atau biasa disebut PaceMaker dapat dipasang secara permanen (PPM) atau
secara temporer (TPM). Pemasangan PPM maupun TPM dapapt dilakukan berdasarkan
indikasi terjadinya SSS,  AV block, gagal jantung. Pemasangan alat pacu jantung
akan menimbulkan denyut jantung yang sedikit berbeda dengan denyut jantung
normal, dan terkadang bergabung dengan denyut normal.

\glsreset{ecg}
 
Pendekatan yang dilakukan untuk mendiagnosa \gls{arrhytmia} adalah
mendengarkan denyut jantung dengan menggunakan stetoskop dimana dengan cara ini
dapat memberikan indikasi-indikasi yang umum dari denyut jantung dan apakah
teratur atau tidak. Namun tidak semua impuls elektrik dari jantung dapat
didengar, seperti di banyak cardiac arrhytmia. Cara yang sederhana dilakukan
untuk mendiagnosa irama jantung adalah dengan \gls{ecg}.

% -----------------------------------------------------------------------------%
% ECG section
\section{Electrocardiogram (ECG)}
\label{sec:ecg}

\subsection{Apa itu ECG?}

Secara etimologi, istilah \textit{Electrocardiograph} disusun dari 3 kata
(yunani) yaitu electro; aktifitas elektrik, cardio; jantung dan graph;
menulis/mencatat. jadi \textit{Electrocardiograph} bisa diartikan mencatat
aktifitas elektrik dari jantung.

Menurut medicinenet.com \cite{medicinenet.1},\textit{Electrocardiography}
(ECG/EKG) merupakan pengujian yang bersifat noninvasif yang dilakukan untuk
mengetahui kondisi dari jantung dengan cara mengukur aktifitas elektrik dari
jantung tersebut.

\addFigure{height=0.3\textheight}{pics/ecgmeasure.png}{fig:ecglead}{Pasien
dengan 10 elektrode}

Pengukuran itu dilakukan dengan memasang leads (alat sensor elektrik) pada
tubuh dilokasi-lokasi tertentu, sehingga didapatkan berbagai informasi tentang
kondisi jantung yang dapat dipelajari dengan melihat pola karakteristik dari
ECG.

ECG dilakukan dengan menggunakan suatu sensor elektrik (12-\textit{leads}) yang
dipasangkan dibagian tubuh tertentu seperti dibagian ujung tubuh 
(\textit{Extremity}), yang berjumlah total 4, dan 6 posisi yang telah ditetapkan
sebelumnya, yaitu dibagian dada. Sejumlah kecil gel dioleskan pada kulit, yang
memungkinkan impuls listrik dari jantung lebih mudah terkirim ke lead ECG.
Ilustrasi dapat dilihat seperti pada gambar \ref{fig:ecglead}. 

ECG dapat memberikan informasi mengenai irama jantung  secara keseluruhan dan
kelemahan diberbagai bagian dari otot jantung. Dengan menggunakan ECG, dapat
diketahui;

\begin{itemize}
   \item mekanisme laju dan irama jantung, 
   \item orientasi dari jantung didalam rongga dada, 
   \item gejala peningkatan ketebalan (\textit{hypertrophy}) dari otot jantung,
   \item gejala kerusakan dari berbagai bagian otot jantung, 
   \item gejala gangguan akut aliran darah ke otot jantung
   \item informasi pola-pola aktivitas elektrik yang tidak normal yang dapat
   mempengaruhi pasien ke arah gangguan irama jantung yang abnormal
   (\textit{Abnormal Cardiac Rhythm Disturbances}).
\end{itemize}

\noindent ECG dapat mendiagnosa kondisi-kondisi seperti dibawah ini;
\begin{itemize}
   \item irama jantung cepat atau tidak teratur yang tidak normal.
   \item irama jantung lambat yang tidak normal.
	\item konduksi impuls jantung yang tidak normal, yang mungkin dapat memberikan
	saran terhadap gangguan jantung maupun metabolisme.
	\item petunjuk tentang kemunculan serangan jantung yang terjadi sebelumnya
	\item petunjuk yang berkembang ke arah serangan jantung akut.
	\item petunjuk kerusakan akut dari aliran darah ke  jantung selama episode
	ancaman serangan jantung (angina tidak stabil).
	\item Efek merugikan pada jantung dari berbagai  penyakit jantung atau penyakit
	sistemik (seperti tekanan darah tinggi, kondisi tiroid, dll).
	\item Efek merugikan pada jantung dari kondisi  paru-paru tertentu (seperti
	emfisema, paru embolus (gumpalan darah ke paru-paru), dll).
	\item Petunjuk elektrolit darah tidak normal (kalium, kalsium, magnesium).
	\item Petunjuk peradangan jantung atau lapisannya.
\end{itemize}


\noindent
Keterbatasan dari ECG adalah sebagai berikut;
\begin{itemize}
    \item ECG adalah gambaran statis dan mungkin  tidak menunjukkan permasalahan
    jantung (parah) ketika si pasien tidak menunjukkan gejala apapun. Contoh
    yang paling umum dari kasus ini adalah pada pasien dengan riwayat nyeri dada
    intermiten yang parah yang disebabkan oleh penyakit arteri koroner. Pasien
    ini mungkin memiliki ECG normal ketika pasien tidak mengalami gejala-gejala
    sakit. Namun mungkin saja pada ECG yang tercatat melalui proses
    \textit{stress test} dapat saja menunjukkan suatu kelainan, sedangkan ECG
    yang diambil pada kondisi yang lainnya terlihat normal.

	\item Banyak pola  abnormal yang tidak spesifik muncul pada ECG,  yang berarti
	bahwa ECG dapat diamati pada berbagai kondisi yang berbeda. Bahkan ECG mungkin
	menunjukkan varian yang normal dan tidak mencerminkan suatu kelainan apapun.
	Kondisi ini sering ditemukan oleh seorang dokter, dengan melakukan pemeriksaan
	yang lebih terperinci, dan kadang-kadang tes jantung lainnya (misalnya,
	\textit{echocardiogram, exercise stress test}) mungkin akan menemukan suatu
	kelainan.

	\item ECG tidak dapat mengukur kemampuan pompa jantung secara handal, dimana
	dalam kasus ini sering digunakan \textit{echocardiogram}.

	\item Dalam beberapa kasus, ECG dapat sepenuhnya normal meskipun kemunculan
	kondisi jantung yang normal akan tercermin dalam ECG. Dan hal ini sebagian besar
	tidak diketahui penyebabnya. Namun yang perlu diingat adalah dengan ECG yang
	normal tidak menutup kemungkinan munculnya penyakit jantung. Selain itu, seorang
	pasien dengan gejala-gejala jantung kadang kala memerlukan evaluasi dan
	pengujian tambahan.

\end{itemize}

\subsection{Kertas ECG} 
Interpretasi waktu dari ECG ditunjukkan dengan suatu kertas bertanda
(\textit{paper speed}) yaitu suatu  kertas milimeter block yang berkorelasi
dengan waktu pencatatan dari denyut jantung. Biasanya electrocardiograph bekerja
pada paper speed 25 mm/s, meskipun paper speed yang lebih cepat terkadang
digunakan. luasan block kecil pada paper speed berukuran 1mm2. Pada paper speed
berukuran 25mm/s, satu block kecil ECG diterjemahkan menjadi 40ms. 5 blok kecil
yang disusun membentuk 1 blok besar, diterjemahkan menjadi 200ms, oleh karena
itu ada 5 blok besar untuk setiap detik-nya. Kualitas diagnostik 12-lead ECG
dikalibrasi pada 10 m/V, sehingga 1 mm diterjemahkan menjadi 0,1 mV. Sebuah
sinyal kalibrasi harus disertakan untk setiap record. Sebuah sinyal standar 1 mV
harus menggerakkan jarum 1 cm secara vertikal, yaitu dua kotak besar di kertas
ECG. ilustrasi kertas ECG (paper speed) dapat dilihat pada
\pic~\ref{fig:ecgpaper}

\addFigure{height=0.6\textwidth}{pics/ecgpaper.jpg}{fig:ecgpaper}{representasi
kertas ECG}

\subsection{ECG Signal} 
Siklus dari denyut jantung yang ada pada ECG terdiri dari gelombang P (P-wave),
QRS Complex, T-wave, dan U-wave yang mana biasanya terlihat pada hampir
50\%-70\% dari keseluruhan ECG. tegangan dasar dari ECG biasa dikenal dengan
nama \textit{isoelectric line}. Biasanya  \textit{isoelectric line} diukur
sebagai bagian dari pelacakan yang mengikuti T-wave dan mendahului P-wave
berikutnya. Ilustrasi dapat dilihat pada gambar \ref{fig:ecgwave}.


\addFigure{width=1\textwidth}{pics/ecgwave.jpg}{fig:ecgwave}{representasi
skematik dari ECG Normal}

Seperti yang telah dijabarkan pada bagian sebelumnya, ada beberapa macam alat
rekam ECG berdasarkan lead yang digunakan seperti ECG \textit{12-leads}, ECG
\textit{5-leads} dan ECG \textit{3-leads}. ECG recorder yang standar, umum
digunakan dirumah sakit adalah ECG 12-leads dengan 10 elektrode. Masing-masing
lead  dari 10 elektrode/sensor tersebut akan menghasilkan gelombang ECG
tersendiri dimana dari gelombang inilah diagnosa kemudian dilakukan. setiap lead
dapat memberikan informasi yang saling mendukung dengan lead-lead yang lain.
Pada umumnya, alat perekam ECG akan menghasilkan gelombang untuk setiap lead
secara bergantian tiap interval tertentu seperti pada gambar
\ref{fig:ecg12lead}.


\addFigure{width=1\textwidth}{pics/ecg12lead.jpg}{fig:ecg12lead}{representasi
gelombang ECG 12-lead untuk \textit{Unstable Angina}}


Dalam usaha mendeteksi suatu ketidaknormalan pada jantung melalui ECG,  para
ahli melakukannya  dengan melihat beberapa ciri yang dapat dibandingkan  dengan
ECG kondisi pada kondisi normal diantaranya ; irama, Rate QRS, Aksis QRS,
Morfologi Gelombang P, Interval PR, Durasi QRS, Morfologi QRS, Deviasi Segmen
ST, Morfologi Gelombang T, Morfologi Gelombang U, Lain-lain (LVH,LV Strain,BBB,
QT interval).


% -----------------------------------------------------------------------------%
\section{Jaringan Saraf Tiruan}
\label{sec:jst}

Jaringan Saraf Tiruan (JST) atau Artificial Neural Networks (ANN)
merupakan suatu sistem yang dibangun atas dasar cara kerja jaringan saraf
manusia. Awal perkembangannya dimotivasi oleh kemampuan pengenalan dari manusia
(otak) dimana cara perhitungannya sangat jauh berbeda dengan sistem komputer
digital. JSt merupakan sistem adaptif yang dapat mengubah strukturnya
untuk memecahkan masalah berdasarkan informasi eksternal maupun internal
yang mengalir melalui jaringan tersebut. 

Menurut S. Haykin \cite{haykin-1994}, sebuah jaringan saraf adalah sebuah
prosesor yang terdistribusi paralel dan mempuyai kecenderungan untuk menyimpan
pengetahuan yang didapatkannya dari pengalaman dan membuatnya tetap tersedia
untuk digunakan. Hal ini menyerupai kerja otak dalam dua hal yaitu: (1)
Pengetahuan diperoleh oleh jaringan melalui suatu proses belajar. (2) Kekuatan
hubungan antar sel saraf yang dikenal dengan bobot sinapsis digunakan untuk
menyimpan pengetahuan.

Jaringan saraf merupakan suatu mesin yang digunakan untuk memodelkan kerja otak
dalam menyelesaikan suatu permasalahan. Jaringan tersebut disusun dari
sekumpulan unit pemroses yang disebut neuron dan untuk meningkatkan
kemampuan-nya, dilakukan proses pembelajaran dengan menggunakan suatu algoritma
tertentu (learning algorithm) dimana tujuannya adalah untuk memodifikasi
kekuatan hubungan antar neuron (bobot) dalam jaringan sesuai dengan goal yang
telah ditentukan.

Keuntungan dari penggunaan JST adalah kemampuannya dalam beradaptasi melalui
proses pembelajaran dan kemampuan generalisasi, dalam artian jaringan saraf
mampu memberikan hasil dari input yang tidak diketahui sebelumnya. Berikut
adalah beberapa kemampuan yang dapat diberikan melalui penggunaan JST menurut S.
Haykin \cite{haykin-1994}:
\begin{enumerate}
  \item Non Linier : jaringan saraf dapat menangani permasalahan baik linier
  maupun non linier.
  \item Pemetaan Input-Output : dalam paradigma pembelajaran dengan arahan
  (supervised learning), modifikasi bobot disesuaikan dengan output yang
  diinginkan sebelumnya (label pada data sampel).
  \item Adaptif : jaringan saraf memiliki kemampuan untuk mengadaptasi bobot
  sinapsisnya sesuai dengan lingkungannya. Jaringan saraf pada umumnya melalui
  proses pembelajaran terhadap suatu lingkungan tertentu, dan dapat diajarkan
  kembali (re-train) untuk melakukan penyesuaian terhadap lingkungannya. 
%   \item Toleransi terhadap kesalahan.
\end{enumerate}

Konsep JST dimodelkan secara matematis dan direpresentasikan melalui suatu unit
pemrosesan, yaitu neuron. terdapat tiga elemen dasar pada model neuron,
seperti yang terlihat pada \pic~\ref{fig:neuron} yaitu
\begin{itemize}
  \item Sinapsis, koneksi antar neuron dimana direpresentasikan dengan suatu
  bobot untuk menunjukkan kekuatan dari koneksi tersebut.
  \item Penjumlah, yang berfungsi untuk menjumlahkan sinyal, yang
  biasanya dalam hal ini perkalian antara bobot dengan sinyal masukan.
  \item Setiap neuron menerapkan fungsi aktivasi terhadap jumlah dari perkalian
  antara sinyal input dengan bobot neuron sebelumnya, untuk menentukan nilai
  output. Fungsi aktivasi ini pada umumnya membatasi nilai output dari neuron,
  menormalisasi output dalam range [0,1] atau [-1,1].
\end{itemize}

\addFigure{width=0.5\textwidth}{pics/neuron.png}{fig:neuron}{Model neuron non
linier \cite[p.~33]{haykin-1994}}

Paradigma pembelajaran JST secara umum dibagi menjadi dua kelompok utama yaitu
pembelajaran dengan pengarahan (supervised) dan pembelajaran tidak dengan
pengarahan (unsupervised).
\begin{itemize}
  \item Supervised learning : pembelajaran dengan pengarahan adalah hasil
  keluaran komputasi dari JST akan dibandingkan dengan hasil keluaran
  sesungguhnya, sehingga dengan selisih antara keduanya; proses penyesuaian
  bobot dalam jaringan dapat dilakukan. Untuk itu tipe ini memerlukan suatu data
  pelatihan yang berisikan data masukan serta target keluaran dari latihan. JST,
  tipe ini misalnya Multi Layer Perceptron, Learning Vector Quantization (LVQ), dll.

  \item Unsupervised learning : pembelajaran dengan tanpa pelatihan
  adalah proses pembelajaran JST dimana tidak memerlukan
  informasi target, cara pembelajarannya adalah jaringan akan menyesuaikan
  bobotnya tanpa campur tangan dari faktor luar dan berusaha menentukan sendiri
  masuk kedalam kelompok mana. Jaringan macam ini misalnya Kohonen
  Self-Organizing Maps (SOM).
\end{itemize}

\subsection{Pembelajaran berbasis kompetisi}
Pembelajaran berbasis kompetisi atau \textit{competitive based learning}, adalah
suatu metode pembelajaran dimana neuron pada output layer berkompetisi satu sama lain
untuk menjadi aktif, diupdate dalam proses pembelajarannya. Dimana
dalam jaringan saraf berdasarkan Hebbian learning, beberapa neuron bisa aktif
secara simultan, dalam pembelajaran jenis ini, hanya satu neuron output yang
aktif dalam satu waktu. Terdapat satu neuron pemenang, dimana aturan ini dikenal
dengan istilah \textit{winner-take-all}. Penentuan neuron pemenang dapat
dilakukan dengan menghitung relasi antara 2 vector, bisa
dengan menggunakan jarak (\textit{distance metric}), dimana semakin kecil
jaraknya, maka relasi semakin tinggi, atau dengan menggunakan tingkat
kemiripan (\textit{similarity measures}), dimana semakin besar nilai
\textit{similarity}, maka relasi akan semakin tinggi. 

Jika $x$ adalah vektor masukan, dan  $w_i$ adalah vektor keluaran ke-$i$, maka
$w_p$, neuron pemenang adalah;

\begin{align}
\label{eq:lvqwin}
	w_p = \arg \min_i d(x, w_i)
\end{align}

Terdapat beberapa metode JST yang mengadopsi aturan ini diantaranya adalah SOM
dan LVQ.

\subsection{Learning Vector Quantization (LVQ)}
\glsreset{lvq}

\Gls{lvq} yang dikembangkan oleh Teuvo Kohonen (1986) \cite{Kohonen-1986b}
merupakan suatu metode pengenalan pola di mana setiap unit output
merepresentasikan suatu kelas atau kategori. Vektor bobot untuk suatu unit
output sering dirujuk sebagai vektor pewakil (\textit{vector reference},
\textit{codebook}, \textit{prototype}) untuk kelas yang direpresentasikan unit
output tersebut. Dalam suatu jaringan LVQ, beberapa unit output vektor pewakil
dapat digunakan untuk setiap kelas. 

Diasumsikan bahwa satu set pola pembelajaran dengan klasifikasi yang diketahui,
diberikan pada jaringan, bersama dengan distribusi awal dari vektor
reference-nya. Setelah pembelajaran, jaringan LVQ mengklasifikasikan suatu
vektor input dengan memasukkannya pada kelas yang sama dengan unit output yang
vektor bobot-nya paling dekat ke vektor input.

Dari sisi arsitektur, karakteristik dari jaringan \gls{lvq} memiliki jaringan
lapis tunggal tanpa \textit{hidden layers}, dimana arsitekturnya serupa dengan
\textit{Self-Organized Map} tanpa adanya asumsi topologi tertentu. Terdiri dari
satu lapis input dengan satu lapis output untuk komputasi. Dalam lapisan output,
setiap unit neuron merepresentasikan suatu kelas atau cluster tertentu.

Dalam JST, pada lapisan output umumnya terdapat suatu fungsi yang digunakan
untuk menentukan level aktivasi dari neuron, dimana fungsi tersebut akan
membatasi nilai keluaran pada suatu interval tertentu. Pada \gls{lvq}, fungsi
aktivasi yang digunakan adalah fungsi identitas yang artinya keluaran input sama
dengan masukkannya, $f(x) = x$.

\noindent
Secara umum, algoritma LVQ dapat ditunjukkan pada algoritma \ref{alg:lvq}.

\begin{algorithm}  
\scriptsize 
\caption{Pseudocode Algoritma LVQ}          
\label{alg:lvq}                           
\begin{algorithmic}                    % enter the algorithmic environment
	\STATE Initialize weight vector $W$
	\STATE Initialize learning rate $\alpha_0$
	\STATE Initialize maximum iteration $t_{max}$
	\STATE $t \leftarrow 0$
	\WHILE {$\alpha_t \neq 0$ or $t < t_{max}$}
		\STATE $x \leftarrow $ getNextSample()
		\STATE $train(W, x) \Downarrow$
		\STATE $\qquad \leadsto w \leftarrow $ getClosestPrototipe()
		\STATE $\qquad \leadsto $updatePrototipe($w$)
		\STATE $t \leftarrow t + 1$
	\ENDWHILE
\end{algorithmic}
\end{algorithm}
 
Terdapat beberapa versi dari \gls{lvq} dimana setiap versi merupakan
penyempurnaan dari versi sebelumnya, baik dari sisi konvergensi 
maupun inisialisasi awal prototipe. Berikut akan diuraikan beberapa versi
dari \gls{lvq};

\subsubsection*{LVQ1}
Pada \gls{lvq} versi pertama, setiap pemberian satu sampel data akan
mengakibatkan proses update terhadap satu prototipe. Pada setiap iterasi dari
proses pelatihan, prototipe dengan jarak minimal terhadap input
akan disesuaikan. Proses penyesuaian prototipe tergantung dari hasil proses
klasifikasi. Jika prototipe pemenang adalah sama dengan kategori input, maka
prototipe akan disesuaikan mendekati sampel data. Jika tidak, maka prototipe
pemenang akan disesuaikan menjauhi sampel data. Tahap pembelajaran yang
dilakukan pada LVQ1 dapat diuraikan sebagai berikut;
\begin{enumerate}
  \setlength{\itemsep}{1pt}
  \setlength{\parskip}{0pt}
  \setlength{\parsep}{0pt}
  \item Pilih sampel data $x$,
  \item Tentukan prototipe pemenang $w_p$ seperti pada \ref{eq:lvqwin}.
  \item Sesuaikan prototipe $w_p$ berdasarkan aturan berikut;
  \begin{align}
  \label{eq:lvq1}
  \begin{array}{ll}
  	w_p \leftarrow w_p + \alpha . (x - w_p), &\quad \text{if}\ C_{w_p} = C_x \\
  	w_p \leftarrow w_p - \alpha . (x - w_p), &\quad \text{if}\ C_{w_p} \neq C_x
  	\\
  \end{array}
  \end{align} 
\end{enumerate}

\noindent Nilai $\alpha$ disini adalah laju pembelajaran dengan rentang nilai
antara $0 < \alpha < 1$ dimana nilainya selalu menurun seiring iterasi proses
pembelajaran.

\noindent 
Aturan pembelajaran diatas dapat ditunjukkan lebih detail seperti
yang terlihat pada algoritma \ref{alg:lvq1}
\begin{algorithm}  
\scriptsize 
\caption{Aturan pembelajaran LVQ1 $train(W, x)$}          
\label{alg:lvq1}                           
\begin{algorithmic}                    % enter the algorithmic environment
	\REQUIRE W, x
	
	\STATE $w_p \leftarrow $ ClosestDistanceWeight($x, W$)
	\IF {$C_x = C_{w_p} $}
		\STATE $w_{p, t+1} \leftarrow w_{p,t} + \alpha_t . (x - w_{p,t})$
	\ELSIF {$C_x \neq C_{w_p}$}
		\STATE $w_{p, t+1} \leftarrow w_{p,t} - \alpha_t . (x - w_{p,t})$
	\ENDIF	
	\STATE $\alpha \leftarrow $ getNextLearningRate()
\end{algorithmic}
\end{algorithm}

\subsubsection*{LVQ2}
Pada algoritma LVQ1, proses penyesuaian bobot hanya ditentukan berdasarkan
prototipe pemenang saja. Namun, pada algoritma LVQ2, proses pembelajaran
memperhitungkan prototipe tentangga dimana proses pembelajaran ditentukan
berdasarkan ide dimana jika vektor masukan ($x$) memiliki jarak yang hampir sama
antara pemenang ($w_p$) dan runner-up($w_r$), maka kedua prototipe seharusnya
di-update secara simultan jika $x$ berada pada bagian jendela yang salah.

\noindent
Tahap pembelajaran yang dilakukan pada LVQ2 dapat diuraikan sebagai berikut;
\begin{enumerate}
  \setlength{\itemsep}{1pt}
  \setlength{\parskip}{0pt}
  \setlength{\parsep}{0pt}
  \item Pilih sampel data $x$
  \item Tentukan prototipe pemenang $w_p$ dan pemenang kedua (\textit{runner
  up}) $w_r$
  \item Lakukan pengecekan terhadap $w_p$ dan $w_r$;
  \begin{enumerate}
    \item $w_p$ dan $w_r$ harus berasal dari kategori yang berbeda, $C_{w_p}
    \neq C_{w_r}$
    \item Kategori dari vektor masukan ($C_x$) berasal dari kategori yang sama
    dengan prototipe pemenang kedua, $C_x = C_{w_r}$
    \item Jarak antara vektor masukan ke vektor pemenang ($d(x, w_p)$) dan
    ke vektor runner-up ($d(x, w_r)$) hampir sama. 
    Untuk menentukan seberapa dekat/sama antara $d_p$ dan $d_r$, disini
    digunakan suatu jendela (\textit{window}) untuk membatasi kedua jarak
    tersebut. Pada algoritma LVQ2 diperkenalkan satu parameter baru yakni
    $\omega$ yaitu seberapa lebar jendela yang ditentukan.
    \begin{align}
    \frac{d_p}{d_r} > (1 - \omega)\quad \text{AND}\quad \frac{d_r}{d_p} < (1 -
    \omega)
    \nonumber
    \end{align}
    
    Jika misal nilai $\omega=0.3$, maka kondisi nya menjadi $d_p > 0.7 d_r
   \ \text{AND}\ d_r < 1.3 d_p$. ilustrasi dapat dilihat pada
    \pic~\ref{fig:lvq2}\footnote{sumber:
    {http://ccy.dd.ncu.edu.tw/miat/course/Neural/ch4/}}.
    
    \addFigure{width=0.5\textwidth}{pics/lvq2.png}{fig:lvq2}
    {Ilustrasi sistem jendela pada LVQ2} 
  \end{enumerate}
    
  \item Jika ketiga kondisi pada langkah (3) terpenuhi, maka lakukan proses
  penyesuaian prototipe sebagai berikut;
  \begin{align}
  w_p \leftarrow w_p - \alpha . (x - w_p) \nonumber \\
  w_r \leftarrow w_r + \alpha . (x - w_r)
  \end{align}
  
  Aturan ini dapat diartikan, jika $x$ berada dalam rentang jendela yang
  ditentukan, tapi dikenali salah ($C_x \neq C(w_p)$), maka jauhkan $w_p$ dari
  distribusi kelas, dan dekatkan $w_r$ dengan distribusi kelas.
\end{enumerate}

\noindent 
Aturan pembelajaran diatas dapat ditunjukkan lebih detail seperti
yang terlihat pada algoritma \ref{alg:lvq2}

\begin{algorithm}  
\scriptsize 
\caption{Aturan pembelajaran LVQ2 $train(W, x)$}          
\label{alg:lvq2}                           
\begin{algorithmic}                    % enter the algorithmic environment
	\REQUIRE W, x
	\STATE $w_p \leftarrow $ ClosesDistanceWeighht($x, W$)
	\STATE $w_r \leftarrow $ RunnerUpClosestDistanceWeight($x, W$)
	\STATE $d_p \leftarrow distance(x, w_p)$
	\STATE $d_r \leftarrow distance(x, w_r)$
	
	\IF {$C_{w_p} \neq C_{w_r}$}
		\IF {$C_x = C_{w_r}$}
			\IF {$\frac{d_p}{d_r} > (1 - \omega)\quad \text{AND}\quad \frac{d_r}{d_p} <
			(1 + \omega)$}
				\STATE $w_{r,t+1} \leftarrow w_{r,t} + \alpha_t . (x - w_{r,t})$
				\STATE $w_{p,t+1} \leftarrow w_{p,t} - \alpha_t . (x - w_{p,t})$
			\ENDIF
		\ENDIF
	\ENDIF
	\STATE $\alpha \leftarrow $ getNextLearningRate()
\end{algorithmic}
\end{algorithm}

\subsubsection*{LVQ2.1}
Algoritma LVQ2.1 merupakan penyempurnaan dari LVQ2 dimana algoritma ini
mengabaikan aturan (2) dari kondisi update prototipe LVQ2. Pada algoritma
LVQ2.1, kategori dari sampel data ($C_x$) \underline{tidak harus sama} dengan
prototipe pemenang kedua ($C_{w_r}$). Persyaratannya adalah minimal salah
satu dari prototipe ($w_p, w_r$) berasal dari kategori yang sama dengan kategori
input ($C_x$). Sedangkan aturan update yang lain masih tetap sama dengan
sebelumnya. Secara lebih detail dapat ditunjukkan pada algoritma
\ref{alg:lvq21}, dimana diasumsikan $C_{w_1} = C_x$.

\begin{algorithm}  
\scriptsize 
\caption{Aturan pembelajaran LVQ2.1 $train(W, x)$}          
\label{alg:lvq21}                           
\begin{algorithmic}                    % enter the algorithmic environment
	\REQUIRE W, x
	\STATE $w_1 \leftarrow $ ClosesDistanceWeighht($x, W$)
	\STATE $w_2 \leftarrow $ RunnerUpClosestDistanceWeight($x, W$)
	\STATE $d_1 \leftarrow distance(x, w_1)$
	\STATE $d_2 \leftarrow distance(x, w_2)$
	
	\IF {$C_{w_1} \neq C_{w_2}$}
		\IF {$C_x = C_{w_1}$ or $C_x = C_{w_2}$}
			\IF {$\min \left(\frac{d_1}{d_2},\frac{d_2}{d_1}\right) > 
				 \frac{(1 - \omega)}{(1 + \omega)}$}
				\STATE $w_{1,t+1} \leftarrow w_{1,t} + \alpha_t . (x - w_{1,t})$
				\STATE $w_{2,t+1} \leftarrow w_{2,t} - \alpha_t . (x - w_{2,t})$
			\ENDIF
		\ENDIF
	\ENDIF
	\STATE $\alpha \leftarrow $ getNextLearningRate()
\end{algorithmic}
\end{algorithm}

\subsubsection*{LVQ3}
Algoritma LVQ2.1 memiliki kelemahan dimana prototipe kemungkinan mengalami
divergensi selama proses pembelajaran dilakukan\cite{Sato:1998}. Pada algoritma
LVQ3, koreksi dilakukan terhadap LVQ2.1 dimana untuk memastikan prototipe agar
selalu mendekati distribusi dari kelas. Aturan update prototipe sama dengan
LVQ2.1, hanya saja terdapat aturan tambahan dimana jika kedua prototipe ($w_1,
w_2$) \underline{berasal dari kelas yang sama}, maka update prototipe nya adalah
sebagai berikut;
\begin{align}
w_i \leftarrow w_i + \epsilon . \alpha . (x - w_i), \quad \epsilon > 0
\end{align}

\noindent
dimana $i \in {1,2}$, jika $x, w_1, w_2$ berasal dari kelas yang sama.
Berikut pada algoritma \ref{alg:lvq3} untuk metode pembelajaran LVQ3

\begin{algorithm}  
\scriptsize 
\caption{Aturan pembelajaran LVQ3 $train(W, x)$}          
\label{alg:lvq3}                           
\begin{algorithmic}                    % enter the algorithmic environment
	\REQUIRE W, x
	\STATE $w_1 \leftarrow $ ClosesDistanceWeighht($x, W$)
	\STATE $w_2 \leftarrow $ RunnerUpClosestDistanceWeight($x, W$)
	\STATE $d_1 \leftarrow distance(x, w_1)$
	\STATE $d_2 \leftarrow distance(x, w_2)$
	
	\IF {$C_{w_1} \neq C_{w_2}$}
		\IF {$\min \left(\frac{d_1}{d_2},\frac{d_2}{d_1}\right) > 
			 \frac{(1 - \omega)}{(1 + \omega)}$}
			\STATE $w_{1,t+1} \leftarrow w_{1,t} + \alpha_t . (x - w_{1,t})$
			\STATE $w_{2,t+1} \leftarrow w_{2,t} - \alpha_t . (x - w_{2,t})$
		\ENDIF
	\ELSE
		\STATE $w_{1,t+1} \leftarrow w_{1,t} + \epsilon.\alpha_t . (x - w_{1,t})$
		\STATE $w_{2,t+1} \leftarrow w_{2,t} + \epsilon.\alpha_t . (x - w_{2,t})$
	\ENDIF
	\STATE $\alpha \leftarrow $ getNextLearningRate()
\end{algorithmic}
\end{algorithm}  

Yang perlu diperhatikan pada algoritma ini adalah jika masing-masing kategori
hanya direpresentasikan dengan \underline{hanya satu} prototipe, maka algoritma
LVQ3 akan sama dengan algoritma LVQ2.1. Algoritma LVQ3 hanya akan berguna, jika
kategori kelas direpresentasikan dengan lebih dari satu prototipe. 


\subsection{Generalized Learning Vector Quantization (GLVQ)}
\label{ssec:glvq}
\glsreset{glvq}

\Gls{glvq} merupakan algoritma yang dikembangkan oleh A. Sato dan Yamada pada
tahun 1995 \cite{Sato:1995}. Algoritma ini merupakan variasi dari algoritma LVQ
khususnya LVQ2.1 dimana merupakan penurunan dari \textit{cost function} yang
eksplisit, tidak seperti pada algoritma LVQ. Disamping itu algoritma LVQ2.1
juga tidak menjamin konvergensi dari prototipe ke distribusi dari kelas selama
proses pelatihan (\cite{Sato:1995},\cite{Sato:1998}).Metode pembelajaran yang 
digunakan disini berdasarkan atas proses minimisasi dari cost function,
\emph{miss-classification error}, dengan menggunakan metode optimasi gradient
descent.

Diberikan $w_1$ adalah prototipe terdekat berasal dari kategori
yang sama dengan kategori vektor masukan (($C_{w_1} = C_x$)), dan $w_2$ adalah
prototipe terdekat yang bukan berasal dari kategori vektor masukan ($C_{w_2} \neq
C_x$). \emph{miss-classification error} $\varphi(x)$ dapat dihitung dengan
menggunakan persamaan \ref{eq:mce};
 
\begin{align}
\label{eq:mce}
	\varphi(x) = \frac{d_1 - d_2}{d_1 + d_2}
\end{align}

\noindent dimana $d_1$ dan $d_2$ adalah jarak antara $x$ dari $w_1$ dan $w_2$.
nilai dari $\varphi(x)$ diantara $-1$ dan $+1$. Jika $\varphi(x)$ negatif, maka
$x$ dikenali secara benar, sedangkan jika positif, maka $x$ dikenali secara
salah. Untuk memperbaiki error rate, maka $\varphi(x)$ harus diturunkan terhadap
semua vektor masukan. Sehingga, kriteria dari proses pembelajaran adalah
meminimisasi cost function $S$ sebagai berikut;

\begin{align}
\label{eq:costS}
	S = \sum_{i=1}^{N} f(\varphi(x)), 
\end{align}

\noindent dimana $N$ adalah jumlah dari vektor masukan untuk training, dan
$f(\varphi(x))$ adalah fungsi monoton naik. untuk meminimalkan $S$, $w_1$ dan
$w_2$ diupdate berdasarkan metode steepest descent dengan nilai laju
pembelajaran $\alpha$ sebagai berikut;

\begin{align}
\label{eq:genuprule}
	w_i \leftarrow w_i - \alpha \frac{\delta S}{\delta w_i}, i = 1, 2
\end{align}


\noindent Jika fungsi diskriminan yang digunakan adalah menggunakan
\textit{squared euclidean distance} $d_i = \lvert x - w_i\rvert ^2$,  maka akan
didapat;

\begin{align}
\label{eq:turunan1a}
	\frac{\delta S}{\delta w_1} =  
	\frac{\delta S}{\delta \varphi} \frac{\delta \varphi}{\delta d_1} \frac{\delta
	d_1}{\delta w_1} =
	- \frac{\delta f}{\delta \varphi} \frac{4d_2}{(d_1 + d_2)^2} (x - w_1)
\end{align}

\begin{align}
\label{eq:turunan1b}
	\frac{\delta S}{\delta w_2} =  
	\frac{\delta S}{\delta \varphi} \frac{\delta \varphi}{\delta d_2} \frac{\delta
	d_2}{\delta w_2} =
	- \frac{\delta f}{\delta \varphi} \frac{4d_1}{(d_1 + d_2)^2} (x - w_2)
\end{align}

\noindent Sehingga, aturan pembelajaran dari algoritma \gls{glvq} dapat ditulis
sebagai berikut;

\begin{align}
\label{eq:glvq-rulea}
	w_1 \leftarrow w_1 + \alpha   
	\frac{\delta f}{\delta \varphi} \frac{d_2}{(d_1 + d_2)^2} (x - w_1)
\end{align}

\begin{align}
\label{eq:glvq-ruleb}
	w_2 \leftarrow w_2 - \alpha   
	\frac{\delta f}{\delta \varphi} \frac{d_1}{(d_1 + d_2)^2} (x - w_2)
\end{align}

$\frac{\delta f}{\delta \varphi}$ dapat dilihat sebagai \emph{gain factor} untuk
proses update prototipe dan nilai-nya tergantunf pada $x$. Ini artinya, 
$\frac{\delta f}{\delta \varphi}$ merupakan bobot untuk setiap $x$. Untuk
menurunkan error rate, maka akan efektif jika proses update prototipe
menggunakan vektor input yang berada di class boundaries, sehingga decision
boundaries akan digeser menuju batas bayes. Dengan demikian, $f(\varphi)$ harus
merupakan fungsi monoton naik yang non-linear, dan dianggap bahwa kemampuan
pengenalan tergantung pada definisi fungsi $f(\varphi)$ .

Pada \gls{glvq}, fungsi monoton naik yang digunakan adalah fungsi sigmoid;
\begin{align}
\label{eq:sigmoid}
	f(\varphi, t) = \frac{1}{1 + e^{-\varphi t}} \\
\label{eq:deltasigmoid}
	\frac{\delta f}{\delta \varphi} = f(\varphi, t) (1 - f(\varphi , t))
\end{align} 

\noindent dimana $\frac{\delta f}{\delta \varphi}$ memiliki puncak yang
tunggal pada $\varphi=0$, semakin bertambah nilai $t$ maka lebar dari puncak
semakin mengecil dan vektor masukan yang mempengaruhi proses pembelajaran secara
gradual dibatasi pada decision boundaries tersebut. Pada Algoritma
\ref{alg:glvq} dapat dilihat algoritma GLVQ dengan pseudocode.

\begin{algorithm}  
\scriptsize 
\caption{Algoritma GLVQ}          
\label{alg:glvq}                           
\begin{algorithmic}                    % enter the algorithmic environment
	\REQUIRE $X$ in round-robin mode
	\STATE Initialize weight vector $W$
	\STATE Initialize learning rate $\alpha_0$
	\STATE Initialize maximum iteration $t_{max}$
	\STATE $t \leftarrow 0$
	\WHILE {$\alpha_t \neq 0$ or $t < t_{max}$}
		\FOR{$x_i \in X$}
			\STATE $d_1 \leftarrow $ closestDistance1($x, W$) where $C_x = C_{w_i}$
			\STATE $d_2 \leftarrow $ closestDistance2($x, W$) where $d_j = \max_{j
			\wedge j \neq i}(d_j) \wedge (C_x \neq C_{w_j})$
			
			\STATE $mce \leftarrow $ equation \ref{eq:mce}
			\STATE $factor1 \leftarrow \frac{4d_2}{(d_1 + d_2)^2}$ 
			\STATE $factor2 \leftarrow \frac{4d_1}{(d_1 + d_2)^2}$ 
			\STATE
			\STATE \COMMENT{adjust prototype 1}
			\STATE $w_{1} \leftarrow $ equation \ref{eq:glvq-rulea}
			\STATE
			\STATE \COMMENT{adjust prototype 2}
			\STATE $w_{2} \leftarrow $ equation \ref{eq:glvq-ruleb}
		\ENDFOR
	\ENDWHILE
	\STATE $\alpha_{t+1} \leftarrow \alpha \times (1 - \frac{t}{t_{max}})$ 
\end{algorithmic}
\end{algorithm}  
 
Seperti yang telah disebutkan diatas, keunggulan dari \gls{glvq} adalah menjamin
konvergensi dari prototipe akan mendekati distribusi dari kategori kelas selama
pelatihan. Keungulan yang lain adalah bahwa \gls{glvq} tidak sensitif terhadap
inisialisasi bobot awal yang umumnya dialami oleh LVQ standar \cite{Sato:1999}. 

 
\subsection{Fuzzy-Neuro Learning Vector Quantization (FNLVQ)}

\Gls{fnlvq} merupakan algoritma pembelajaran yang berbasis kompetisi yang
dikembangkan oleh Kusumoputro dan Wisnu J \cite{Kusumoputro:2002} dimana
algoritma ini diaplikasikan pada sistem pengenalan aroma.  Algoritma ini dikembangkan
berdasarkan algoritma LVQ dengan menggunakan teori fuzzy dimana aktifasi
dari neuron ditunjukkan dalam bentuk nilai fuzzy karena dimotivasi
oleh ketidakjelasan (\emph{fuzzines}) dari data yang dihasilkan akibat dari
kesalahan pengukuran oleh alat. Proses fuzzifikasi dari semua komponen prototipe
dan vektor masukan dikalkulasi melalui proses normalisasi dengan menggunakan
fungsi keanggotaan segitiga, dengan nilai derajat keanggotaan maksimal adalah 1.
Fungsi keanggotaan segitiga sangat umum digunakan karena sifatnya yang sangat
sederhana dan mudah untuk diimplementasikan. Distribusi data direpresentasikan
pada prototipe dengan nilai min, mean dan max, yaitu $F = (f_l, f, f_r)$. $f$
adalah pusat (\emph{center}) dari distribusi sampel data, sedangkan $f_l$ dan
$f_r$ secara berurutan adalah nilai minimum dan maksimum sampel data. 

Karena neuron pada \gls{fnlvq} menggunakan fuzzy number, maka konsep jarak
euclid yang digunakan pada standar \gls{lvq} dimodifikasi menggunakan fuzzy
similarity dimana dihitung dengan menggunakan operasi MAX-MIN antara vektor
input dengan prototipe. Dengan model seperti ini, arsitektur dari jaringan
disesuaikan untuk mengakomodasi operasi MAX-MIN dari dua vektor. Arsitektur
\gls{fnlvq} dapat dilihat pada \pic~\ref{fig:fnlvqarch}, dimana jaringan terdiri
dari lapisan input , satu lapisan tersembunyi (\emph{hidden layer}) dan lapisan
output. Lapisan tersembunyi disini merupakan representasi dari vektor referensi
yang terdiri dari beberapa fungsi keanggotaan yang berkorespondensi dengan
setiap fitur masukan , dan setiap kategori  kelas direpresentasikan dengan satu
vektor referensi, dalam hal ini disebut dengan \emph{cluster}.

\addFigure{width=0.8\textwidth}{pics/fnlvqarch.png}{fig:fnlvqarch}{Arsitektur
dari fuzzy neuro LVQ digunakan dalam sistem pengenalan aroma.}

Diberikan $x(t)$ adalah vektor masukan. $h_x(t)$ adalah fungsi keanggotaan dari
$x$, sedangkan $w_i$ adalah vektor referensi dari kategori $i$ dan $h_{w_i}$
adalah fungsi keanggotaan untuk $w_i$. Tingkat kemiripan antara setiap
\emph{cluster} ($w_i$) pada \emph{hidden layer} dengan vektor masukan ($x(t)$)
dihitung dengan menggunakan \emph{fuzzy similarity} $\mu_i(t)$ dengan
menggunakan operasi max sebagai berikut;

\begin{align}
\mu_i(t) = \max [h_x(t) \wedge h_{w_i}(t)], \quad i = 1, 2, \dots, m
\end{align}

\noindent dimana $m$ adalah maksimal jumlah kategori dari aroma.

Fungsi aktifasi yang digunakan pada tiap \emph{cluster} adalah dengan
menggunakan operasi minimum untuk semua komponen yang ada didalamnya, 
dimana nantinya nilai fuzzy similarity $\mu(t)$ akan dipropagasikan ke neuron
keluaran.

\begin{align}
\mu(t) = \min [\mu_i(t)]
\end{align}

\noindent kemudian dicari nilai terbesar nilai similarity $\mu(t)$
setiap neuron output untuk menentukan neuron pemenang. Jika neuron pemenang
memiliki $\mu(t)$ sama dengan 1, maka vektor masukan dan vektor
referen adalah sama, sedangkan jika $\mu(t)$ sama dengan 0, maka vektor
masukan dan vektor referen tidak mirip sama sekali.

Aturan pembelajaran dari algoritma \gls{fnlvq} terdiri dari tiga kondisi yang
mungkin terjadi diantaranya; (1) Jika jaringan dapat mengenali masukan dengan
benar, (2) Jika jaringan salah mengenali vektor masukan, dan (3) Jika tidak
terdapat interseksi fuzzy set antara vektor masukan dan prototipe. Berikut
adalah aturan pembelajaran untuk setiap kondisi;
\begin{enumerate}
  \setlength{\itemsep}{1pt}
  \setlength{\parskip}{0pt}
  \setlength{\parsep}{0pt}
  \item Jika jaringan mengenali vektor masukan dengan benar, $C_x = C_{w_p}$
  maka prototipe dari cluster pemenang diupdate sesuai dengan;
  \begin{enumerate}
  	\item Posisi pusat dari prototipe digeser mendekati vektor masukan
  	\begin{align}
  	w_i(t+1) = w_i(t) + \alpha(t) \left((1-\mu(t))\times(x(t) - w_i(t))  \right)
  	\end{align}
  	\item Tingkatkan kemampuan pengenalan prototipe  dengan memperlebar fungsi
  	keanggotaan dari prototipe dengan aturan sebagai berikut;
  	\begin{enumerate}
  	  \item Modifikasi dengan faktor konstan
  	  \begin{align}
  	  f_l(t+1) &= f_l(t) - (1+\beta) (f(t) - f_l(t))  \\
  	  f_r(t+1) &= f_r(t) - (1+\beta) (f_r(t) - f(t)) \nonumber \\
  	  f(t+1)   &= w_i(t+1) \nonumber
  	  \end{align}
  	  
  	  \item Modifikasi dengan faktor variabel
	  \begin{align}
  	  f_l(t+1) &= f_l(t) - (1 - \mu)(1+\eta)(f(t) - f_l(t))  \\
  	  f_r(t+1) &= f_r(t) - (1 + \mu)(1+\eta)(f_r(t) - f(t)) \nonumber \\
  	  f(t+1)   &= w_i(t+1) \nonumber
  	  \end{align}  	  
  	\end{enumerate}
  \end{enumerate}
  
  \item Jika jaringan salah mengenali vektor masukan, $C_x \neq C_{w_p}$ maka
  prototipe dari cluster pemenang diupdate sesuai aturan;
  \begin{enumerate}
    \item Posisi pusat dari prototipe digeser menjauhi vektor masukan
  	\begin{align}
  	w_i(t+1) = w_i(t) - \alpha(t) \left((1-\mu(t))\times(x(t) - w_i(t))  \right)
  	\end{align}
  	\item Turunkan kemampuan pengenalan prototipe dengan mempersempit fungsi
  	keanggotaan dari prototipe dengan aturan sebagai berikut;
  	\begin{enumerate}
  	  \item Modifikasi dengan faktor konstan
  	  \begin{align}
  	  f_l(t+1) &= f_l(t) + (1+\gamma) (f(t) - f_l(t)) \\
  	  f_r(t+1) &= f_r(t) - (1+\gamma) (f_r(t) - f(t)) \nonumber \\
  	  f(t+1)   &= w_i(t+1) \nonumber 
  	  \end{align}
  	  
  	  \item Modifikasi dengan faktor variabel
	  \begin{align}
  	  f_l(t+1) &= f_l(t) + (1 - \mu)(1-\kappa)(f(t) - f_l(t)) \\
  	  f_r(t+1) &= f_r(t) - (1 - \mu)(1-\kappa)(f_r(t) - f(t)) \nonumber \\
  	  f(t+1)   &= w_i(t+1) \nonumber
  	  \end{align}  	  
  	\end{enumerate}
  \end{enumerate}
  
  \item Jika fungsi keanggotaan prototipe tidak memiliki interseksi dengan
  vektor masukan, maka fungsi keanggotaan prototipe diupdate berdasarkan aturan
  sebagai berikut;
  \begin{align}
  w_i(t+1) = \xi(t).w_i(t)
  \end{align}
\end{enumerate}

\noindent dimana nomenklatur yang digunakan adalah sebagai berikut;

\begin{tabular}{llp{0.8\textwidth}}
$w_{i}(t+1)$ &=& prototipe pemenang setelah di-update  \\
$w_{i}(t)$ 	 &=& prototipe pemenang sebelum di-update \\
$\alpha(t)$  &=& laju pembelajaran, nilai monoton turun $(0 < \alpha \le 1))$,
 yang didefinisikan sebagai berikut;
\end{tabular}

\begin{small}
	\begin{align}\label{eq:learning_rate}
		\alpha(t+1) &= 0.9999 \alpha(t) \\[0.2cm]
		\alpha(0) &= 0.05 \nonumber
	\end{align}
\end{small}

\begin{tabular}{llp{0.8\textwidth}}
$\beta, \gamma$ &=& nilai konstan yang digunakan dalam proses pelebaran
dan penyempitan fungsi keanggotaan dengan interval [0,1] \\
$\eta, \kappa$ &=& nilai variabel yang digunakan dalam proses pelebaran
dan penyempitan fungsi keanggotaan dengan definisi;
\end{tabular}

\begin{small}
	\begin{align}\label{eq:fuzziness_value}
	\eta(t+1) &= 1/100 {1-\alpha(t+1)} \\[0.2cm]
	\kappa(t+1) &= 1 - \alpha(t+1) \nonumber
	\end{align}
\end{small}

\noindent dimana nilai $\xi$ adalah konstan dengan = $\xi=1.1$.

Terdapat beberapa penelitian yang dilakukan untuk meningkatkan
kinerja dari algoritma FNLVQ. Kusumoputro et.al. \cite{Kusumoputro:2002b} 
menggunakan Matrix Similarity Analysis (MSA) menangani kelemahan dari FNLVQ
standar dimana sensitif terhadap pemilihan prototipe awal. Untuk
menanggulanginya, dalam proses pelatihan akan dipilih bobot terbaik untuk
setiap iterasi dengan menggunakan matrix similarity sebagai  \emph{fitness
function} nya, sehingga selain menggunakan maksimal epoch, lama proses
pembelajaran juga dapat ditentukan berdasarkan nilai matrix yang didapat 
yang digunakan sebagai nilai threshold. Matrix ideal yang diharapkan untuk 
mendapatkan kinerja terbaik adalah jika menghasilkan matrik identitas.

Selain itu, pendekatan lain yang dilakukan untuk menangani
sensitifitas pemilihan prototipe awal adalah penelitian yang dilakukan oleh
Rohmatullah dalam tesis-nya \cite{Rochmatullah:2009}, dimana pendekatan yang
dilakukan menggunakan konsep Particle Swarm Optimization (PSO) digabungkan
dengan FNLVQ-MSA. Tujuan dari PSO ini adalah untuk menentukan inisialisasi
vektor awal terbaik yang dihasilkan dari proses pencarian dengan menggunakan
faktor kognitif dan sosial sebagai salah satu parameter pencarian. Namun,
kelemahannya adalah, proses komputasi yang dibutuhkan sangat lama.


% Dari beberapa uraian diatas, dapat diberikan
% ilustrasi mengenai pohon algoritma seperti yang terlihat pada \pic~\ref{fig:pohon-alg}
% 
% \addFigure{width=0.8\textwidth}{pics/pohon-alg.png}{fig:pohon-alg}{Ilustrasi
% pohon algoritma}













% -----------------------------------------------------------------------------%
\section{Wavelet Transform}
\label{sec:wt}

Metode transformasi berbasis wavelet merupakan sarana yang dapat digunakan untuk
menganalisis sinyal non-stasioner, yaitu sinyal dengan kandungan frequensi yang
bervariasi terhadap waktu. Metode ini sangat populer dalam beberapa tahun
terakhir. Analisis wavelet dapat digunakan untuk menunjukkan kelakuan sementara
pada suatu sinyal, atau dapat juga digunakan untuk mem-filter, menghilangkan
sinyal data yang tidak diinginkan (\emph{noise}) dan atau meningkatkan kualitas
dari data. Selain itu, transformasi wavelet juga dapat digunakan untuk
memampatkan (\emph{compressing}) data \cite{Agfi:2006}.

Transformasi wavelet merupakan transformasi yang menggunakan kernel terintegrasi
yang dinamakan dengan wavelet. Wavelet dapat digunakan sebagai kernel
terintegrasi untuk analisis serta ekstraksi informasi suatu data dan juga
sebagai basis penyajian/karakteristik dari suatu data. Kelebihan dari analisis
sinyal menggunakan wavelet adalah terletak dari sifat terpenting wavelet yaitu
lokalisasi waktu-frekuensi (\emph{time-frequency localization}) yang
diistilahkan dengan \emph{compact support} \cite{Guler:2005}. Dengan menggunakan
wavelet dapat dipelajari karakteristik sinyal secara lokal dan detil sesuai
dengan skala-nya, dimana hal ini menunjukkan kegunaan dari analisis pada sinyal
non-stasioner, karakteristik berbeda pada skala yang berbeda. Pada dasarnya,
cara kerja dari wavelet dalam mengekstraksi data adalah dengan menggunakan
proses penguraian (\emph{decomposition}) atau ekspansi deret, yaitu dengan cara
ekspansi tak berhingga dari wavelet yang diulur atau \emph{dilated} dan digeser
atau \emph{translated}.

Transformasi wavelet dikembangkan sebagi suatu alternatif pendekatan pada
Transformasi Fourier Waktu Pendek (\emph{Short Time Fourier Transform = STFT})
dalam mengatasi masalah resolusi. Pada STFT, skala dari jendela yang
digunakan bersifat konstan, sedangkan pada Transformasi wavelet, lebar dari
jendela akan berubah-ubah selama proses transformasi dilakukan dalam menghitung
masing-masing komponen spektrum. Dimana hal ini merupakan ciri khas dari
Transformasi Wavelet. Dengan karakteristik seperti itu, dengan transformasi
wavelet akan dapat diperoleh resolusi waktu dan frekuensi yang jauh lebih baik
daripada metode-metode yang lain.

Persamaan transformasi wavelet (WT) kontinu pada signal $f(x)$ dapat
didefinisikan sebagai berikut;
\begin{align}
\label{eq:wavelet_transform}
	W_{s}f(x) &= f(x) * \Psi_{s}(x) = \frac{1}{s}\int^{+\infty}_{-\infty}f(t)
	\Psi(\cfrac{x - t}{s})dt
\end{align}

\noindent dimana $s$ adalah faktor skala,  $\Psi_{s}(x) =
\cfrac{1}{s}\Psi(\cfrac{x}{s})$ adalah dilatasi dari fungsi jendela, yang
kemudian dikenal dengan wavelet penganalisis $\Psi(x)$, $t$ adalah  faktor
translasi/pergeseran dari fungsi jendela tersebut. Dari persamaan
\ref{eq:wavelet_transform} ditunjukkan bahwa fungsi jendela tersebut terdilatasi
maupun termampatkan berdasarkan faktor skala. Dengan skala yang rendah, maka
frekuensi tinggi akan terlokalisasi, sedangkan pada skala tinggi, yang
terlokalisasi adalah frekuensi rendah. 

\addFigure{width=1\textwidth}{pics/waveletcwt.png}{fig:cwt}{Contoh sinyal
nonstasioner dan hasil dari WT (kontinu)} 

Pada \pic~\ref{fig:cwt} ditunjukkan contoh sinyal nonstasioner dengan kandungan
frekuensi 250, 500, 750 dan 1000Hz dengan tambahan noise yang kemudian
ditransformasi menggunakan WT. Dari gambar tersebut terlihat bahwa frekuensi
250Hz muncul pertama kali yang diikuti dengan frekuensi yang lain. Ketika kita
akan menganalisis menggunakan wavelet, terdapat tradeoff resolusi antara waktu
dan frekuensi(pada dasarnya pada WT, direpresetnasikan dengan waktu-skala,
dimana skala dan frekuensi berbanding terbalik, semakin tinggi skala, maka
dapat melokalisasi frekuensi rendah dan sebaliknya). Jika skala tinggi
maka maka semakin tidak jelas resolusi frekuensi, namun
semakin jelas resolusi waktunya, dan sebaliknya.

\subsection{Descrete Wavelet Transform}
Pada dasarnya, nilai koefisien dari wavelet untuk jendela wavelet dan
signal tertentu menunjukkan seberapa dekat korelasi antara wavelet 
dengan bagian tertentu dari signal tersebut. semakin tinggi koefisien, maka akan
semakin mirip dimana hasilnya akan sangat tergantung dari bentuk wavelet yang
dipilih\cite{wavelet:matlab}. Namun jika perhitungan nilai koefisien dilakukan
untuk semua skala dan posisi (kontinu), maka akan dihasilkan jumlah data yang
sangat besar. Mallat \cite{Mallat:1989} mengembangkan suatu cara untuk
menghitung koefisien wavelet dengan mengambil sebagian saja dari skala dan
posisi berdasarkan pangkat-dua, yang kemudian dikenal dengan istilah
\emph{dyadic WT}. Hal ini juga dikenal dengan \emph{Discrete Wavelet
Transform (DWT)}. Dengan menggunakan \emph{dyadic WT} maka analisis akan lebih
efisien dan cukup  akurat. Algoritma Mallat ini menggunakan filter seperti dalam
skema \emph{two channel subband coder} dan dikenal juga dengan Transformasi
Wavelet Cepat (\emph{Fast Wavelet Transform}).

Dalam banyak kasus pemrosesan sinyal, kandungan frekuensi rendah adalah hal yang
penting karena memberikan indentitas dari sinyal yang bersangkutan. Kandungan
frekuensi tinggi hanya sebagai ``nuansa sinyal'' tambahan. Hal ini dapat
dianalogikan seperti sinyal suara manusia, jika komponen frekuensi tinggi
dihilangkan, maka suara akan berubah, namun masih mampu untuk mengetahui apa
yang diucapkan \cite{wavelet:matlab}. Hal inilah yang mendasari mengapa dalam
analisis berbasis wavelet banyak digunakan istilah aproksimasi dan detail.

Diberikan $s = 2^j$ dengan $j \in Z, Z$ set integer, maka dyadic WT dari sinyal
$f(x)$  dapat dihitung menggunakan algoritma Mallat sebagai berikut;

\begin{align}
	\label{eq:sfn}
	S_{2^{j}}f(x) &= \sum_{k \in~Z} h_{k}S_{2^{j-1}}f(x - 2^{j-1}k) \\
	\label{eq:wfn}
	W_{2^{j}}f(x) &= \sum_{k \in~Z} g_{k}S_{2^{j-1}}f(x - 2^{j-1}k)
\end{align}

\noindent dimana $S_{2^{j}}$ merupakan operator penghalus, $S_{2^{j}} f(n) =
a_{j}$. $a_{j}$  adalah koefisien dari frekuensi rendah (skala tinggi) yang
mengaproksimasi sinyal yang asli, sedangkan $W_{2^{j}} f(n) = d_{j}$, $d_{j}$ 
adalah koefisien frekuensi tinggi (skala rendah) yang merepresentasikan detail
dari sinyal asli.

\addFigure{width=0.8\textwidth}{pics/dwtdecompos.png}{fig:decompos}{Ilustrasi
proses dekomposisi, (a) Dekomposisi satu tingkat, (b) dekomposisi multi tingkat}

Pada \pic~\ref{fig:decompos}a dapat dilihat proses filtering wavelet dimana
$f(x)$ adalah sinyal asli, kemudian dilewatkan ke filter lolos-rendah
(\emph{lowpass})  lolos-tinggi (\emph{highpass}) dan menghasilkan dua sinyal,
\textbf{A}-proksimasi dan \textbf{D}-etail. 

Jika dekomposisi sinyal diteruskan secara iteratif untuk bagian aproksimasinya,
sehingga suatu sinyal dapat dibagi-bagi kedalam banyak komponen resolusi rendah,
maka proses ini dinamakan sebagai dekomposisi banyak tingkat atau
\emph{multilple-level decomposition} seperti yang ditunjukkan pada
\pic~\ref{fig:decompos}b.
 
Pemilihan fungsi jendela \emph{mother wavelet} dan jumlah level dekomposisi
dalam analisis menggunakan transformasi wavelet adalah sangat penting.
Pemilihan jumlah level dekomposisi berdasarkan atas komponen frekuensi
yang dominan dalam sinyal. Pemilihan yang tepat bertujuan untuk menjaga agar
bagian signal yang memiliki korelasi yang baik dengan frekuensi yang dibutuhkan
dalam proses pengenalan tetap ada pada koefisien wavelet. Terdapat berbagai
macam \emph{mother wavelet} yang dapat digunakan dalam analisis wavelet. Menurut
Senhadji \cite{Senhadji:1995}, proses dekomposisi dengan menggunakan wavelet
yang orthonormal akan memberikan informasi yang \emph{non-redundant}. Kelompok
wavelet orthonormal dan juga \emph{compactly supported} diantaranya adalah
Daubechies, Symlet dan Coiflet \cite{Agfi:2006}.


     
 
% calculated with Mallat algorithm as follows:
% 
% Informasi yang dapat
% diekstraksi menggunakan wavelet adalah 
% 
% 
% wavelet digunakan untuk melakukan filter frekuensi tinggi,
% dimana setelah didekomposisi, kemudian signal direkonstruksi kembali.
% 
% pada penelitian ini, wavelet digunakan untuk mereduksi fitur dari setiap beat
% dengan menggunakan coefisien aproksimasi-nya sebagai fitur, sesuai dengan
% tingkat level dekomposisi dan mengabaikan coefisien detail, dimana mengandung
% signal frekuensi tinggi.
% 
% 
% Transformasi Fourier
% Sampai sekarang transformasi Fourier mungkin masih menjadi transformasi yang
% paling populer di area pemrosesan sinyal digital (PSD). Transformasi Fourier
% memberitahu kita informasi frekuensi dari sebuah sinyal, tapi tidak informasi
% waktu (kita tidak dapat tahu di mana frekuensi itu terjadi). Karena itulah
% transformasi Fourier hanya cocok untuk sinyal stationari (sinyal yang informasi
% frekuensinya tidak berubah menurut waktu). Untuk menganalisa sinyal yang
% frekuensinya bervariasi di dalam waktu, diperlukan suatu transformasi yang dapat
% memberikan resolusi frekuensi dan waktu disaat yang bersamaan, biasa disebut
% analisis multi resolusi (AMR). AMR dirancang untuk memberika resolusi waktu yang
% baik dan resolusi frekuensi yang buruk pada frekuensi tinggi suatu sinyal, serta
% resolusi frekuensi yang baik dan resolusi waktu yang buruk pada frekuensi rendah
% suatu sinyal. Pendekatan ini sangat berguna untuk menganalisa sinyal dalam
% aplikasi-aplikasi praktis yang memang memiliki lebih banyak frekuensi rendah.
% Transformasi wavelet adalah suatu AMR yang dapat merepresentasikan informasi
% waktu dan frekuensi suatu sinyal dengan baik. Transformasi wavelet menggunakan
% sebuah jendela modulasi yang fleksibel, ini yang paling membedakannya dengan
% transformasi Fourier waktu-singkat (STFT), yang merupakan pengembangan dari
% transformasi Fourier. STFT menggunakan jendela modulasi yang besarnya tetap, ini
% menyebabkan dilema karena jendela yang sempit akan memberikan resolusi frekuensi
% yang buruk dan sebaliknya jendela yang lebar akan menyebabkan resolusi waktu
% yang buruk.
% -----------------------------------------------------------------------------%
\section{Himpunan Fuzzy}

Himpunan Fuzzy adalah suatu himpunan dimana elemennya memiliki derajat
keanggotaan. Himpunan Fuzzy merupakan generalisasi dari teori himpunan 
klasik. Secara formal, definisi dari himpunan fuzzy adalah sebagai berikut;

\begin{quotation}
	Suatu himpunan fuzzy $A$ pada semesta $X$ dengan $A \subset X$ dikarakterik
	oleh suatu fungsi keanggotaan (\emph{membership function}) $f_A(x)$  dimana
	setiap titik di $X$ dipetakan ke suatu bilangan real $[0,1]$, dengan nilai
	$f_A(x)$ menunjukkan derajat keanggotaan dari $x$ pada himpunan $A$. Sehingga
	semakin dekat nilai $f_A(x)$ dengan pusat (\emph{unity}), semakin tinggi
	derajatnya.
	\cite{Zadeh:1965}
\end{quotation}
  
Suatu himpunan fuzzy adalah kosong (\emph{empty}) jika dan hanya jika nilai
fungsi keanggotaannya adalah 0 pada $X$. Dua himpunan fuzzy $A$ dan $B$ adalah
sama, $A = B$ jika dan hanya jika $f_A(x) = f_B(x)$ untuk semua $x \in X$.
Sedangkan komplemen dari suatu himpunan fuzzy $A$ dinotasikan dengan $A'$
didefinisikan sebagai $f_{A'} = 1 - f_A$.

Seperti yang diuraikan diatas, suatu himpunan fuzzy dikarakterisasi dengan suatu
fungsi keanggotaan. Terdapat beberapa fungsi keanggotaan yang sering digunakan
dalam berbagai aplikasi sebagai berikut;

\begin{enumerate}
  \item Fungsi Keanggotaan segitiga.\\
  Fungsi keanggotaan segituga sangat umum dan banyak digunakan dalam pembuatan
  suatu sistem, karena kesederhanaannya dan untuk menyusunnya hanya membutuhkan
  3parameter semisal nilai minimum, rata-rata dan maksimum. Definisi dari fungsi
  keanggotaan segitiga adalah sebagai berikut;
  \begin{align}
	\label{eq:mftrim}
	f(x, a, b, c) = \left\{ 
	\begin{array}{ll}
	0 & , x \leq a\\
	\frac{x - a}{b - a} & , a < x \leq b \\
	\frac{c - x}{c - b} & , b < x < c \\
	0 & , x \geq c
	\end{array}
  \end{align}
   
   Bentuk fungsi keanggotaan segitiga dapat dilihat pada \pic~\ref{fig:mftrim}.
   \addFigure{width=0.6\textwidth}{pics/mftrim.png}{fig:mftrim}{Bentuk fungsi
   keanggotaan segitiga}
   
   \item Fungsi Keanggotaan Trapesium.\\
   Fungsi keanggotaan trapesium didefinisikan sebagai berikut;
   \begin{align}
	\label{eq:mftrap}
	f(x, a, b, c, d) = \left\{ 
	\begin{array}{ll}
	0 & , x \leq a\\
	\frac{x - a}{b - a} & , a < x \leq b \\
	1 & , b < x \leq c \\
	\frac{d - x}{d - c} & , c < x < d \\
	0 & , x \geq c
	\end{array}
  \end{align}
  
  Bentuk fungsi keangotaan segitiga dapat dilihat pada \pic~\ref{fig:mftrap}.
  \addFigure{width=0.6\textwidth}{pics/mftrap.png}{fig:mftrap}{Bentuk fungsi
   keanggotaan trapesium}
   
   \item Fungsi keanggotaan Gaussian.\\
   Fungsi keanggotaan Gaussian didefinisikan sebagai berikut;
	\begin{align}
	\label{eq:mfgaus}
		f(x, m, \sigma) = \exp\left(\frac{(x-x)^2}{2\sigma^2}\right)
	\end{align}

	dimana parameter $m$ dan $\sigma$ adalah pusat dan lebar dari fungsi
	keanggotaan. Bentuk dari fungsi keanggotaan Gaussian dapat dilihat pada
	\pic~\ref{fig:mfgaus}.
	\addFigure{width=0.6\textwidth}{pics/mfgaus.png}{fig:mfgaus}{Bentuk fungsi
	keanggotaan Gaussian}
	    
\end{enumerate}
	

% -----------------------------------------------------------------------------%
\section{Analisis Kinerja secara statistik}
\label{sec:ujittest}
Studi perbandingan umumnya dilakukan berdasarkan eksperimen  dengan
menggunakan data simulasi maupun menggunakan data nyata. Untuk menunjukkan suatu
algoritma memiliki kinerja yang signifikan terhadap yang lain, dapat dilakukan
melalui beberapa ujicoba yang kemudian data yang dihasilkan dapat di analisis
secara statistik untuk menunjukkan apakah suatu algoritma lebih baik secara
signifikan atau tidak. Kuncheva pada bukunya \cite{Kuncheva:2004} menyebutkan
beberapa uji tersebut diantaranya;

\subsection{McNemar Test}
Pada uji test mcNemar, hasil ujicoba dianalisis dengan menghitung peluang
perbedaan kesalahan pengenalan antara algoritma (\emph{Type I Error}). Misal
terdapat dua algoritma $A$ dan $B$;\\
\begin{tabular}{llp{0.7\textwidth}}
$n_{00}$ &=& jumlah data yang salah dikenali oleh $A$ dan $B$ \\
$n_{01}$ &=& jumlah data yang salah dikenali oleh $A$ tapi tidak $B$ \\
$n_{10}$ &=& jumlah data yang salah dikenali oleh $B$ tapi tidak $A$ \\
$n_{11}$ &=& jumlah data yang benar dikenali oleh $A$ dan $B$ \\
\end{tabular}

\noindent Maka perbedaan antara yang diharapkan dengan hasil observasi dapat
dihitung secara statistik dengan persamaan \ref{eq:mcnemar};

\begin{align}
\label{eq:mcnemar}
x^2 = \frac{(\lvert n_{01} - n_{10}\rvert - 1)^2}{n_{01} + n_{10}}
\end{align}

\noindent dimana nilai dari $x^2$ mendekati distribusi dari $\chi^2$ dengan 1
degree of freedom. null hypothesis ($H_0$=algoritma memiliki tingkat error yang
sama) dapat ditolak jika nilai $\lvertx^2\rvert > z$ dimana $z$ adalah nilai
tabular pada level signifikan tertentu, misal 0.05 dengan 1 derajat kebebasan.
Hal ini berarti kedua algoritma memiliki perbedaan tingkat error yang
signifikan.

\subsection{Cross-Validation}
Validasi silang (\emph{cross-validation}) merupakan metode statistik yang
digunakan untuk uji coba dengan memperhitungkan tingkat variasi dari data
set. Variasi tersebut dilakukan dengan membagi data menjadi beberapa sub bagian
dan digunakan baik sebagai data pelatihan maupun pengujian berdasarkan porsi
tertentu. Terdapat beberapa variasi \emph{cross validation} yang dapat dilakukan
diataranya;

\subsubsection*{K-Hold Out paired t-test}
Uji K-Hold out sangat umum digunakan untuk membandingkan algoritma dalam machine
learning. Ujicoba dilakukan dengan menggunakan data set $Z$ dimana data tersebut
akan dipartisi menjadi data latih dan data uji dengan porsi biasanya
$\frac{2}{3}$ data latih dan $\frac{1}{3}$ untuk data uji (\emph{hold-out
method}), atau mungkin juga dengan membagi data sama rata $\frac{1}{2}$ untuk
pelatihan dan pengujian. Kemudian algoritma diujicobakan dengan kedua data
tersebut, pelatihan dengan data latih, dan di uji dengan data uji. 

Proses ujicoba akan dilakukan berulang sebanyak $K$ kali dimana umumnya nilai
 $K=30$.Jika akurasi dari hasil eksperimen dinotasikan $P_A$ untuk algoritma $A$
 dan $P_B$ untuk algoritma $B$, maka untuk $K$ ujicoba didapat $P^1 = P_A^1 -
 P_B^1$ sampai $P^K = P_A^K - P_B^K$. Asumsi yang dipakai disini adalah bahwa
 perbedaan akurasi disini adalah independent dari distribusi normal.
 
Untuk menguji tingkat signifikansi kinerja algoritma $A$ dan $B$, digunakan
\emph{paired t-test}, dimana null hypothesis ($H_0$ = akurasi tidak berbeda
secara signifikan), maka persamaan berikut memiliki \emph{t-distribution} dengan
$K-1$ \emph{degree of freedom}

\begin{align}
\label{eq:t-test}
t = \frac{\bar{P}\sqrt{K}}{\sqrt{\sum_{i=1}^K (P^i - \bar{P})^2 / (K - 1)}}
\end{align}

\noindent dimana $\bar{P} = \frac{1}{K} \sum_{i=1}^K P^i$. Untuk menarik
kesimpulan apakah null hypothesis di tolak atau diterima, nilai $t$ test yang
didapat dari persamaan \ref{eq:t-test} dibandingkan dengan nilai pada
tabel distribusi $t$ pada tingkat signifikansi yang diinginkan pada $K-1$ degree
of freedom. Umumnya pada perbandingan algoritma ini, tingkat signifikasi 0.05
digunakan. Jika nilai $\lvert t \rvert > z$ dengan $z$ adalah nilai tabular yang
didapat, maka $H_0$ dapat ditolak, dan menerima bahwa memang terdapat perbedaan
yang signifikan antara kedua algoritma yang dibandingkan.

\subsubsection*{K-Fold Cross Validation paired t-test}
Metode validasi silang \emph{K-Hold} berusaha untuk menghindari overlap pada
testing data, dimana pada metode ini, data sampel di partisi menjadi $K$ bagian
dengan jumlah yang hampir sama, dan setiap bagian digunakan untuk testing
algoritma yang telah dilatih menggunakan $K-1$ bagian yang lain-nya. Pada metode
ini diasumsikan perbedaan yang dihasilkan adalah independen dari distribusi
normal. Dengan menggunakan perhitungan statistik $t$ yang sama pada persamaan
\ref{eq:t-test}, kemudian nilai $t$ dibandingkan dengan nilai dari tabel
distribusi $t$ untuk menarik kesimpulan.

\subsubsection*{Dietterich�s 5 x 2-Fold Cross-Validation Paired t-Test}
Pada metode ini, prosedur ujicoba dilakukan dengan mengulang \emph{2-fold cross
validation} sebanyak 5 kali, dimana disetiap eksekusi validasi silang, data
sampel dipartisi menjadi dua sama banyak. Algoritma $A$ dan $B$ dilatih
menggunakan setengah data \#1 dan diuji dengan setengah data \#2, kemudian
dihitung akurasi $P_A^1$ dan $P_B^1$. kemudian data latih dan data uji di
silang, dan dihitung $P_A^2$ dan $P_B^2$. Perbedaannya kemudian dihitung sebagai
berikut;
\begin{align}
P^1 = P_A^1 - P_B^1 \nonumber \\
\text{dan} \nonumber \\
P^2 = P_A^2 - P_B^2 \nonumber 
\end{align}

\noindent estimasi dari rata-rata dan varian dari perbedaan diatas, untuk
tow-fold cross validation dihitung sebagai berikut;
\begin{align}
\bar{P} = \frac{P^1 + P^2}{2}; \qquad s^2 = (P^1 - \bar{P})^2 + (P^2 -
\bar{P})^2
\end{align}

\noindent dengan $P_i^1$ adalah selisih $P^1$ pada eksekusi ke-$i$, dan $s_i^2$
adalah estimasi varian dari setiap eksekusi $i, i = 1, \dots, 5$, maka
perhitungan $\tilde{t}$ adalah sebagai berikut;
\begin{align}
\tilde{t} = \frac{P_1^1}{\sqrt{\frac{1}{5}\sum_{i=1}^5 s_i^2}}
\end{align}

\noindent Hanya satu dari 10 nilai perbedaan yang didapat akan dipergunakan
dalam perhitungan statistik diatas. Pada metode ini, $\tilde{t}$ mendekati 
distribusi $t$ dengan 5 derajat kebebasan. Penarikan kesimpulan sama dengan
metode-metode sebelumnya.

% \newpage
% \section{Revision}
% \begin{itemize}
%   \item Cari paper yang menyatakan GLVQ menjamin konvergensi bobot
%   \item Cari paper yang menyatakan GLVQ tidak sensitif terhadap inisialisasi
%   bobot awal. ketemu -> \cite{Sato-1999}
%   
% \end{itemize} 

% \newpage
% \section{Hasil presentasi Prof, Rudi di seminar reboan}
% 
% differentiable method like sigmoid is used for possible to find the minimization
% of error.
% 
% error prediction
% \begin{align}
% \end{align}
% 
% cross entropy error function
% \begin{align}
% 	E(W,V) = - \sum d_i \log p_i + (1 - d_i) \log(1 - p_i)
% \end{align}
% $d_i$ is desired output
% 
% Optimization method
% \begin{itemize}
%   \item Gradient descent / error backprop
%   \item Conjugate gradient
%   \item Quasi newton method
%   \item Geneticalgorithm
% \end{itemize}
% 
% neural network considered as well trained if it can predict training data and
% cross validation separtely
% 
% 
% Network Pruning : the way to remove unrelevant connection, so at the end we have
% network with connection that is relevant
% try to set a weight = 0 and if the performance is not affected, then it is more
% likely tobe unrelevant
% 
% Rule extraction
% 
% Re-RX
% Discreate / continous varaible
% algorithm Re-Rx(S, D, C)
% 
% Card Dataset
% 
% 
% AUC , ACC, 
% \begin{align}
% =\sum_{i=1}^{m}\sum_{i=1}^{m}
% \end{align}
% 
% $AUC_d$ = $AUC$ = 1 - fp + tp / 2
% 
% %=============================================
% -----------------------------------------------------------------------------%
