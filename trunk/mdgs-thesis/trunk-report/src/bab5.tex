%-----------------------------------------------------------------------------%
\chapter{\babLima}
%-----------------------------------------------------------------------------%
Pada bab ini akan dijelaskan mengenai percobaan, simulasi berbagai skenario
terhadap sistem yang dikembangkan yang dilanjutkan dengan analisis terhadap
hasil yang diperoleh untuk mengukur kinerja dari sistem, dibandingkan dengan
beberapa metode lain yang telah dicoba dan memperhitungkan tingkat kevalidan
percobaan terhadap hasil yang diperoleh.

%-----------------------------------------------------------------------------%
\section{Percobaan algoritma terhadap data fitur}
%-----------------------------------------------------------------------------%
Pada subbab ini akan diuraikan mengenai beberapa skenario ujicoba yang dilakukan
untuk mengetahui pengaruh ekstraksi fitur yang sudah dilakukan dan kinerja dari
algoritma pengenalan yakni LVQ1, LVQ21, GLVQ, FPGLVQ dan MGLVQ.

%-----------------------------------------------------------------------------%
\subsection{Percobaan terhadap data hasil ekstraksi beat}
%-----------------------------------------------------------------------------%
Pada percobaan ini, data hasil ekstraksi beat akan dicoba digunakan untuk
melatih algoritma pengenalan yang ada dan 

%-----------------------------------------------------------------------------%
\subsection{Percobaan terhadap data tanpa outlier}
%-----------------------------------------------------------------------------%

%-----------------------------------------------------------------------------%
\subsection{Percobaan terhadap fitur yang direduksi}
%-----------------------------------------------------------------------------%

%-----------------------------------------------------------------------------%
\subsection{Percobaan variasi urutan data terhadap algoritma}
%-----------------------------------------------------------------------------%
Pada percobaan ini, data pelatihan yang akan digunakan diurutkan secara
round robin terhadap kelas. Variasi yang dilakukan adalah dengan
membedakan jumlah data tiap kelas yang akan di saling silang, diantaranya;
\begin{enumerate}
  \item Pola#1 : jumlah data tiap kelas yang di saling silang adalah 1 data
  \item Pola#3 : jumlah data tiap kelas yang di saling silang adalah 3 data
  \item Pola#5 : jumlah data tiap kelas yang di saling silang adalah 5 data  
\end{enumerate}

Tujuan dari percobaan ini adalah untuk mengetahui bagaimana pengaruh pola
pelatihan terhadap kinerja dari classifier. Untuk uji coba ini 

urutan
data pelatihan yang akan digunakan yang round robin divariasikan mulai dari round robin 1 data tiap kelas

%-----------------------------------------------------------------------------%
\subsection{Percobaan algoritma terhadap variasi parameter}
%-----------------------------------------------------------------------------%

%-----------------------------------------------------------------------------%
\section{Analisis Hasil}
%-----------------------------------------------------------------------------%
1. Analisis bisa menggunakan T-Test dari 
Buku Combining Pattern Classifier
Error Calculation (Counting estimator p8)

2. McNemar Test
	- tentukan bobot terbaik antara LVQ1m LVQ21, GLVQ dan FPGLVQ
	- train dengan data 86 fitur, 6 kelas, simpan bobot nya
	- testing dengan mencatat setiap single data, catat untuk tiap classifier,
	dikenali atau tidak.


Untuk memberikan ranking terhadap algoritma, bisa menggunakan amalisis ROC 
\newpage
\section{Revision}
\begin{itemize}
  \item Why? wavelet
  \item Why? daubhechies
  \item Why? proses Outlier , baik baseline wander, maupun Outlier removal,
  knapa pake interquartile range? bukan mahalanobis?
  \item 
  \item 
\end{itemize}