\section{Electrocardiogram (ECG)}
\label{sec:ecg}

\subsection{Apa itu ECG?}

Secara etimologi, istilah \textit{Electrocardiograph} disusun dari 3 kata
(yunani) yaitu electro; aktifitas elektrik, cardio; jantung dan graph;
menulis/mencatat. jadi \textit{Electrocardiograph} bisa diartikan mencatat
aktifitas elektrik dari jantung.

Menurut medicinenet.com \cite{medicinenet.1},\textit{Electrocardiography}
(ECG/EKG) merupakan pengujian yang bersifat noninvasif yang dilakukan untuk
mengetahui kondisi dari jantung dengan cara mengukur aktifitas elektrik dari
jantung tersebut.
 
\addFigure{height=0.3\textheight}{pics/ecgmeasure.png}{fig:ecglead}{Pasien
dengan 10 elektrode}

Pengukuran ECG dilakukan dengan menggunakan sadapan (lead), yaitu sepasang atau
lebih elektrode, yang dipasangkan pada bagian tubuh tertentu seperti dibagian
ujung tubuh (\textit{extremity}), yang digunakan untuk melakukan pengukuran
aktitas elektrik jantung dengan memperhatikan perubahan potensial  elektrik
diantara keduanya. Sejumlah kecil gel dioleskan pada
kulit,  yang memungkinkan impuls listrik dari jantung lebih mudah terkirim ke
elektrode sehingga didapatkan berbagai informasi tentang
kondisi jantung yang dapat dipelajari dengan melihat pola karakteristik dari
ECG. Ilustrasi dapat dilihat seperti pada gambar
\ref{fig:ecglead}\footnote{sumber: John G. Webster, Editor, Medical
Instrumentation Application and Design, 4$^{th}$ edition, John Wiley \& Sons
Inc., 2009}.

\noindent Jenis Sadapan ada 2, yaitu;
\begin{enumerate}
   \item \textit{unipolar}; yaitu sadapan dimana 1 elektrode ditempatkan di
   daerah sekitar dada dekat jantung atau salah satu anggota badan sedangkan
   yang lain diletakkan pada potensial 0 atau pada \textit{central terminal}.
   pada kasus ini hanya satu elektrode saja yang mentransmisikan potensial.

   \item \textit{bipolar};yaitu sadapan 2 elektrode dimana ditempatkan terpisah
   pada tubuh (dada atau bagian tubuh lain) dan keduanya memberikan kontribusi 
   potensial.
\end{enumerate}

\addFigure{height=0.3\textheight}{pics/einthoventriangle.jpg}{fig:einthoventri}{Notasi
standar dari lead vector membentuk \textit{einthovens triangle}.}

Ada bermacam variasi alat perekam ECG dari sadapan-3, sadapan-5 maupun
sadapan-12. standar ECG yang umum digunakan adalah ECG sadapan-12 (12-lead)
dimana terdiri dari sadapan I (LA ke RA),II (LL ke RA),III (LL ke LA), sadapan
tambahan (\textit{augmented lead}) yang bersifat unipolar; aVF, aVR, aVL, dan 
sadapan prekordial ($V_1 \dots V_6$), dapat dilihat pada \pic~\ref{fig:ecglead},
\ref{fig:einthoventri}\footnotemark[\value{footnote}]. Yang perlu dipahami
disini adalah sadapan-12 hanya memiliki jumlah elektrode sebanyak 10 buah
(sesuai aturan US), yaitu;
\begin{enumerate}
   \setlength{\itemsep}{1pt}
   \setlength{\parskip}{0pt}
   \setlength{\parsep}{0pt}
    \item RA (Right Arm) : dipasangkan di lengan kanan
	\item LA (Left Arm)  : dipasangkan di lengan kiri
	\item RL (Right Leg) : dipasangkan di kaki kanan
	\item LL (Left Leg)  : dipasangkan di kaki kiri
	\item $V_1,V_2,V_3,V_4,V_5,V_6$ : dipasangkan pada dada. 
\end{enumerate}

ECG dapat memberikan informasi mengenai irama jantung  secara keseluruhan dan
kelemahan diberbagai bagian dari otot jantung. Dengan menggunakan ECG, dapat
diketahui;

\begin{itemize}
   \setlength{\itemsep}{1pt}
   \setlength{\parskip}{0pt}
   \setlength{\parsep}{0pt}
   \item mekanisme laju dan irama jantung, 
   \item orientasi dari jantung didalam rongga dada, 
   \item gejala peningkatan ketebalan (\textit{hypertrophy}) dari otot jantung,
   \item gejala kerusakan dari berbagai bagian otot jantung, 
   \item gejala gangguan akut aliran darah ke otot jantung
   \item informasi pola-pola aktivitas elektrik yang tidak normal yang dapat
   mempengaruhi pasien ke arah gangguan irama jantung yang abnormal
   (\textit{Abnormal Cardiac Rhythm Disturbances}).
\end{itemize}

\noindent ECG dapat mendiagnosa kondisi-kondisi seperti dibawah ini;
\begin{itemize}
   \setlength{\itemsep}{1pt}
   \setlength{\parskip}{0pt}
   \setlength{\parsep}{0pt} 
   \item irama jantung cepat atau tidak teratur yang tidak normal.
   \item irama jantung lambat yang tidak normal.
	\item konduksi impuls jantung yang tidak normal, yang mungkin dapat memberikan
	saran terhadap gangguan jantung maupun metabolisme.
	\item petunjuk tentang kemunculan serangan jantung yang terjadi sebelumnya
	\item petunjuk yang berkembang ke arah serangan jantung akut.
	\item petunjuk kerusakan akut dari aliran darah ke  jantung selama episode
	ancaman serangan jantung (angina tidak stabil).
	\item Efek merugikan pada jantung dari berbagai  penyakit jantung atau penyakit
	sistemik (seperti tekanan darah tinggi, kondisi tiroid, dll).
	\item Efek merugikan pada jantung dari kondisi  paru-paru tertentu (seperti
	emfisema, paru embolus (gumpalan darah ke paru-paru), dll).
	\item Petunjuk elektrolit darah tidak normal (kalium, kalsium, magnesium).
	\item Petunjuk peradangan jantung atau lapisannya.
\end{itemize}

\noindent
Keterbatasan dari ECG adalah sebagai berikut;
\begin{itemize}
    \item ECG adalah gambaran statis dan mungkin  tidak menunjukkan permasalahan
    jantung (parah) ketika si pasien tidak menunjukkan gejala apapun. Contoh
    yang paling umum dari kasus ini adalah pada pasien dengan riwayat nyeri dada
    intermiten yang parah yang disebabkan oleh penyakit arteri koroner. Pasien
    ini mungkin memiliki ECG normal ketika pasien tidak mengalami gejala-gejala
    sakit. Namun mungkin saja pada ECG yang tercatat melalui proses
    \textit{stress test} dapat saja menunjukkan suatu kelainan, sedangkan ECG
    yang diambil pada kondisi yang lainnya terlihat normal.

	\item Banyak pola  abnormal yang tidak spesifik muncul pada ECG,  yang berarti
	bahwa ECG dapat diamati pada berbagai kondisi yang berbeda. Bahkan ECG mungkin
	menunjukkan varian yang normal dan tidak mencerminkan suatu kelainan apapun.
	Kondisi ini sering ditemukan oleh seorang dokter, dengan melakukan pemeriksaan
	yang lebih terperinci, dan kadang-kadang tes jantung lainnya (misalnya,
	\textit{echocardiogram, exercise stress test}) mungkin akan menemukan suatu
	kelainan.

	\item ECG tidak dapat mengukur kemampuan pompa jantung secara handal, dimana
	dalam kasus ini sering digunakan \textit{echocardiogram}.

	\item Dalam beberapa kasus, ECG dapat sepenuhnya normal meskipun kemunculan
	kondisi jantung yang normal akan tercermin dalam ECG. Dan hal ini sebagian besar
	tidak diketahui penyebabnya. Namun yang perlu diingat adalah dengan ECG yang
	normal tidak menutup kemungkinan munculnya penyakit jantung. Selain itu, seorang
	pasien dengan gejala-gejala jantung kadang kala memerlukan evaluasi dan
	pengujian tambahan.

\end{itemize}

\subsection{Kertas ECG} 
Interpretasi waktu dari ECG ditunjukkan dengan suatu kertas bertanda
(\textit{paper speed}) yaitu suatu  kertas milimeter block yang berkorelasi
dengan waktu pencatatan dari denyut jantung. Biasanya electrocardiograph bekerja
pada paper speed 25 mm/s, meskipun paper speed yang lebih cepat terkadang
digunakan. luasan block kecil pada paper speed berukuran 1mm2. Pada paper speed
berukuran 25mm/s, satu block kecil ECG diterjemahkan menjadi 40ms. 5 blok kecil
yang disusun membentuk 1 blok besar, diterjemahkan menjadi 200ms, oleh karena
itu ada 5 blok besar untuk setiap detik-nya. Kualitas diagnostik 12-lead ECG
dikalibrasi pada 10 m/V, sehingga 1 mm diterjemahkan menjadi 0,1 mV. Sebuah
sinyal kalibrasi harus disertakan untk setiap record. Sebuah sinyal standar 1 mV
harus menggerakkan jarum 1 cm secara vertikal, yaitu dua kotak besar di kertas
ECG. ilustrasi kertas ECG (paper speed) dapat dilihat pada
\pic~\ref{fig:ecgpaper}

\addFigure{height=0.4\textwidth}{pics/ecgpaper.jpg}{fig:ecgpaper}{representasi
kertas ECG}

\subsection{ECG Signal} 
Siklus dari denyut jantung yang ada pada ECG terdiri dari gelombang P (P-wave),
QRS Complex, T-wave, dan U-wave yang mana biasanya terlihat pada hampir
50\%-70\% dari keseluruhan ECG. tegangan dasar dari ECG biasa dikenal dengan
nama \textit{isoelectric line}. Biasanya  \textit{isoelectric line} diukur
sebagai bagian dari pelacakan yang mengikuti T-wave dan mendahului P-wave
berikutnya. Ilustrasi dapat dilihat pada gambar \ref{fig:ecgwave}.


\addFigure{width=1\textwidth}{pics/ecgwave.jpg}{fig:ecgwave}{representasi
skematik dari ECG Normal}

Secara singkat, gelombang P dibangkitkan oleh atrial depolarization, QRS Complex
dibangkitkan oleh ventricular depolarization, gelombang T dibangkitkan oleh ventricular
repolarization. Sedangkan atrial repolarization dibungkus oleh gelombang QRS Com-
plex. Secara normal interval P-R dan S-T berada pada potensial 0, dimana segmen
P-R disebabkan oleh delay pada AV node, sedangkan segmen S-T berhubungan dengan
durasi rata-rata daerah individual dari sel ventrikular. terdapat gelombang tambahan,
yangdisebut gelombang U, dimana gelombang ini tidak harus muncul. jika muncul, bi-
asanya tercatat setelah gelombang T terjadi. gelombang ini muncul akibat dari proses
repolarisasi otot ventrikular papillary yang lambat.

Seperti yang telah dijabarkan pada bagian sebelumnya, ada beberapa macam alat
rekam ECG berdasarkan lead yang digunakan seperti ECG \textit{12-leads}, ECG
\textit{5-leads} dan ECG \textit{3-leads}. ECG recorder yang standar, umum
digunakan dirumah sakit adalah ECG 12-leads dengan 10 elektrode. Masing-masing
lead  dari 10 elektrode/sensor tersebut akan menghasilkan gelombang ECG
tersendiri dimana dari gelombang inilah diagnosa kemudian dilakukan. setiap lead
dapat memberikan informasi yang saling mendukung dengan lead-lead yang lain.
Pada umumnya, alat perekam ECG akan menghasilkan gelombang untuk setiap lead
secara bergantian tiap interval tertentu seperti pada gambar
\ref{fig:ecg12lead}.

\addFigure{width=0.8\textwidth}{pics/ecg12lead.jpg}{fig:ecg12lead}{representasi
gelombang ECG 12-lead untuk \textit{Unstable Angina}}

Dalam usaha mendeteksi suatu ketidaknormalan pada jantung melalui ECG,  para
ahli melakukannya  dengan melihat beberapa ciri yang dapat dibandingkan  dengan
ECG kondisi pada kondisi normal diantaranya ; irama, Rate QRS, Aksis QRS,
Morfologi Gelombang P, Interval PR, Durasi QRS, Morfologi QRS, Deviasi Segmen
ST, Morfologi Gelombang T, Morfologi Gelombang U, Lain-lain (LVH,LV Strain,BBB,
QT interval).

Menurut dr. Jolanda Jonas\footnote{dr. Jolanda Jonas merupakan dokter umum yang
sempat bekerja di RS. Yayasan Jantung Indonesia, Padang, Sumatra barat.
Sekarang beliau bekerja (\emph{Internship}) di bagian Penyakit Jantung, St. Olav
Hospital, Trondheim, Norway}, pemeriksaan ECG adalah \textbf{pemeriksaan
penunjang} dalam menentukan seorang pasien menderita penyakit jantung, khususnya
Arrhytmia.sehingga hasil dari pengenalan ECG hanya bisa dimanfaatkan sebagai
data penunjang dalam diagnosa pasien. Misal dari
rekaman 5 menit pasien, terdapat 300 beat, muncul 1-5 beat yang tergolong
kelainan/penyakit, misal RBBB, belum tentu pasien tersebut menderita penyakit
RBBB, tapi juga belum tentu salah. Hal ini tergantung juga dengan hasil 
pemeriksaan  fisik dan analisa gejala-gejala klinis yang timbul. Terkadang
Kelainan beat hanya muncul sekali dari rekaman 5 menit pasien, dimana hal ini 
sering digunakan sebagai indikasi untuk pemeriksaan lanjutan seperti pemasangan
alat observasi ECG 24 jam (\emph{Holter ECG}).  Oleh karena itu, seandainya
terdapat kasus seperti diatas, suatu sistem yang dikembangkan hanya menjamin
bahwa seorang pasien �\emph{memiliki kecenderungan mengalami kelainan irama
jantung (Arrhytmia) yakni RBBB}�, dan tidak lebih dari itu.


