%-----------------------------------------------------------------------------%
\chapter*{\kataPengantar}
%-----------------------------------------------------------------------------%

Puji dan syukur penulis panjatkan kepada Ida SangHyang Widhi Wasa, Tuhan Yang
Maha Kuasa, atas Asung Kerta Wara Nugraha-Nya, \saya
dapat menyelesaikan pengerjaan tesis yang berjudul �\judul�. Penulisan tesis
ini dikerjakan guna memenuhi sebagian persyaratan dalam memperoleh gelar  Master
Ilmu Komputer pada Fakultas Ilmu Komputer Universitas Indonesia. 

Dalam penyelesaian tesis ini banyak pihak yang telah membantu \saya, baik
memberikan dukungan, bimbingan serta semangat,  oleh karena itu \saya ingin
mengucapkan terima kasih yang sebesar-besarnya.  Ucapan terima kasih ingin
\saya tujukan kepada pihak-pihak berikut:
\begin{enumerate}
  \item Bapak Ibu Teka Suriawan, Bapak Ibu Narka Wirawan, Ni Putu Ayu Myra
  Gerhana Putri, Ni Putu Kaisra Ashiyana Putri, kakak, adik dan seluruh anggota
  keluarga lainnya atas semua perhatian, dukungan, semangat serta doa yang \saya
  dapatkan selama \saya kuliah.
  \item Negara Republik Indonesia melalui Departemen Pendidikan Nasional yang
  telah memberikan kepercayaan berupa beasiswa kepada \saya untuk  meneruskan
  pendidikan di FASILKOM UI. 
  \item Bapak Wisnu Jatmiko selaku dosen pembimbing tesis.
  \item Prof. Dra. Belawati H. Widjaja, M.Sc., Ph.D selaku pembimbing akademis.
  \item Mba Elly Matul Imah sebagai teman diskusi selama \saya mengerjakan tesis
  ini.
  \item Ibu dr. Jolanda Jonas dan keluarga atas bantuannya yang membuat \saya
  lebih memahami tentang Cardiologi khususnya Aritmia.
  \item Rekan-rekan MIK 2009 UI yang luar biasa dan bagaikan keluarga, terima
   kasih untuk semua.
  \item Rekan-rekan anggota lab 1231, Terima kasih atas kerjasama , bantuan dan
   dukungannya selama ini.
   \item Rekan-rekan pengurus dan anggota PPI-Trondheim/Keluarga Trondheim,
   Norway atas dukungannya selama ini.
\end{enumerate}
Semoga Tuhan Yang Maha Esa membalas segala kebaikan semua pihak yang telah
membantu. \saya sadar bahwa penelitian ini memiliki banyak kekurangan dan
kelemahan yang \saya kerjakan. Oleh karena itu, \saya mengharapkan kritik
dan saran yang membangun agar penelitian ini menjadi lebih baik dan bermanfaat
bagi yang membacanya. 

\vspace*{0.1cm}
\begin{flushright}
Depok, \bulantahun\\[0.1cm]
\vspace*{1cm}
\penulis

\end{flushright}