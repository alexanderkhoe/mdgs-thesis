%
% Halaman Abstrak
%
% @author  I Made Agus Setiawan
% @version 1.00
%

\chapter*{Abstrak}

\vspace*{0.2cm}

\noindent \begin{tabular}{l l p{10cm}}
	Nama&: & \penulis \\
	Program Studi&: & \programstudi \\
	Judul&: & \judul \\
\end{tabular} \\ 

\vspace*{0.5cm}

\noindent

Aritmia atau cardiac Aritmia merupakan salah satu penyakit jantung yang dapat
didiagnosa menggunakan standar ECG. dengan menggunakan ECG, para dokter dapat
menganalisis aktifitas elektrik jantung dan menentukan tipe dari Aritmia yang
diderita oleh pasien. 
Pada penelitian ini, proses pengenalan Aritmia dilakukan secara
otomatis menggunakan pendekatan jaringan saraf tiruan. Proses ini dibagi menjadi
tiga tahap yaitu; pemrosesan data, ekstraksi fitur dan proses pengenalan
oleh jaringan saraf. Pada proses pengolahan data awal, sinyal ECG disegmentasi
menjadi satuan beat dengan menggunakan puncak gelombang R sebagai pivot, dan
dilanjutkan dengan proses baseline wander removal dan outlier removal.
Transformasi Wavelet kemudian dilakukan untuk mengekstraksi dan mereduksi fitur.
Setiap beat tunggal kemudian diklasifikasi menjadi 6 dan 12 kelas menggunakan
metode baru yang dikembangkan disebut Fuzzy-Neuro Learning Vector
Quantization (FNGLVQ) yang merupakan adaptasi metode Fuzzy-Neuro kedalam GLVQ
yang dikembangkan oleh A.Sato. 

Hasil dari penelitian ini menunjukkan bahwa rata-rata tingkat pengenalan beat
Aritmia 6 kelas menggunakan metode FNGLVQ sebesar 98.53\% dan untuk 12 kelas
sebesar 96.33\% dimana metode yang dikembangkan memberikan hasil yang lebih
baik daripada GLVQ sebesar 97.03\% dan 94.13\% untuk 6 kelas dan 12 kelas.
Disamping itu, FNGLVQ memberikan hasil yang lebih baik untuk data yang tidak
seimbang dengan nilai rata-rata \emph{recall} mencapai 86.23\% lebih baik dari
GLVQ sebesar 82.12\%.
\\

\vspace*{0.2cm}

\noindent Kata Kunci: \\ 
\noindent Fuzzy-Neuro Generalized Learning Vector Quantization, FNGLVQ,
GLVQ, sistem pengenalan beat Aritmia\\

\newpage