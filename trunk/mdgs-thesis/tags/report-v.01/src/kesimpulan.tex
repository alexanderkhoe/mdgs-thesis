%---------------------------------------------------------------
\chapter{\kesimpulan}
%---------------------------------------------------------------
% \todo{Tambahkan kesimpulan dan saran terkait dengan perkerjaan 
% 	yang dilakukan.}
Pada bab ini akan dijelaskan mengenai kesimpulan yang dapat diambil dari
penelitian yang telah dilakukan oleh \saya dan beberapa saran yang dapat
dipertimbangan dalam penelitian berikutnya.

%---------------------------------------------------------------
\section{Kesimpulan}
%---------------------------------------------------------------
Sebuah metode jaringan saraf tiruan berbasis kompetisi dengan nama
FN-GLVQ berhasil dikembangkan dan diuji-coba dengan menggunakan data arrhytmia
dari data MIT-BIH database. Pengujian dilakukan dengan menggunakan beberapa
skenario dan dilakukan uji statistik untuk menunjukkan keunggulan metode FNLVQ
terhadap GLVQ khususnya pada data kelainan arrhytmia. Dari hasil pengujian dan
analisis dapat ditarik kesimpulan sebagai berikut;
\begin{enumerate}
  \item Penelitian ini berhasil mengembangkan engine pengenalan arrhytmia
  dalam bentuk java library.
  \item Pengolahan data awal menggunakan wavelet daubechies order 8 (db8)
  memberikan hasil rata-rata terbaik dibandingkan dengan db2, db4 maupun
  db6, yakni 94.92\%. 
  \item Penghilangan Noise pada data awal arrhytmia dapat meningkatkan kinerja
  algoritma dalam proses pengenalan arrhytmia.
  \item Penggunaan mekanisme pengurutan data training menggunakan metode
  round-robin dapat meningkatkan level akurasi dengan nilai $i < 30$.
  \item Penggabungan teori fuzzy yang diadaptasi dari metode FNLVQ dengan
  GLVQ menjadi FN-GLVQ dapat meningkatkan kinerja dari pengenalan kelainan
  arrhytmia sampai 98.53\% untuk 6 kelas arrhytmia dan 96.33\% untuk 12 kelas.
  \item Metode FN-GLVQ masih sensitif terhadap inisialisasi bobot awal
  dibandingkan dengan metode GLVQ, namun dengan bobot yang tepat, misal bobot
  diambil dari data training, dapat memberikan hasil yang lebih baik.
  \item Penggunaan metode LVQ khususnya GLVQ dan FN-GLVQ pada data arrhytmia
  dapat memberikan tingkat pengenalan rata-rata diatas 97\% untuk 6 kelas dan
  sampai rata-rata 95\% untuk 12 kelas
\end{enumerate}

%---------------------------------------------------------------
\section{Saran}
%---------------------------------------------------------------
Pemrosesan data awal yang dilakukan pada penelitian ini masih menyisakan
beberapa kemungkinan-kemungkinan pelenitian lanjutan diataranya;
\begin{enumerate}
  \item Pada penelitian ini fitur diambil berdasarkan beat, kedepan, pengambilan
  data fitur ECG dapat dilakukan dengan mengambil fitur-fitur time  interval dan
  morfologi dengan menggunakan transformasi wavelet.
  \item Metode FN-GLVQ membutuhkan analisis lanjutan untuk mempelajari tingkat
  sensitifitas inisalisasi bobot awal yang belum bisa dicapai pada penelitian
  ini.
\end{enumerate}
