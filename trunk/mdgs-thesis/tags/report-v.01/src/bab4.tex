%-----------------------------------------------------------------------------%
\chapter{\babEmpat}
%-----------------------------------------------------------------------------%
Pada bab ini akan diuraikan mengenai sekilas metode pelatihan LVQ, metode
klasifikasi yang dikembangkan, modifikasi yang dilakukan serta perancangan dan
implementasi dari metode tersebut.

%-----------------------------------------------------------------------------%
\section{Metode Pelatihan LVQ}
%-----------------------------------------------------------------------------%
\subsection{Sistem Epoch}
Dalam penelitian ini, studi dilakukan terhadap algoritma LVQ yang dikembangkan
oleh Kohonen dkk \cite{Kohonen92lvqpak}. Dari hasil analisis terhadap
implementasi algoritma tersebut, yakni LVQ PAK package, \saya menyadari bahwa
mekanisme iterasi (epoch) yang penulis pahami dan kebanyakan digunakan
dalam JST dengan yang ada pada paket program tersebut sedikit berbeda.
\begin{enumerate}
  \item Iterasi pada LVQ PAK, iterasi yang dilakukan untuk sekali proses
  pembelajaran (satu kali proses update bobot) dengan menggunakan satu data
  sampel. Hal ini akan berelasi dengan jumlah maksimal iterasi yang ditentukan
  sebagai parameter pembelajaran. Artinya jika jumlah data sampel yang digunakan
  sebanyak 100, sedangkan maksimal iterasi ditentukan sebanyak 50 kali, maka
  hanya 50 data sampel saja yang akan digunakan untuk proses pelatihan, dengan
  pemilihan data yang random maupun sequensial dan hanya akan terjadi 50 kali
  proses update bobot.
  \item Iterasi pada sistem epoch, makna dari satu iterasi adalah
  proses pelatihan dilakukan untuk semua data sampel yang diberikan. Jika data
  sampel yang diberikan sebanyak 100, maka jaringan saraf akan dilatih sebanyak
  100 kali untuk satu iterasi. Jika maksimum iterasi ditentukan sebanyak 50
  kali, maka akan setara dengan 1500 iterasi pada metode sebelumnya, atau 1500
  kali proses update bobot.
\end{enumerate}

Selain sensitif terhadap inisial bobot awal, LVQ standar tersebut juga sangat
sensitif terhadap jumlah iterasi yang dilakukan dalam proses pembelajaran. Jika
terlalu banyak iterasi pelatihan dilakukan, maka kecenderungannya adalah bobot
yang dihasilkan menjauhi bobot optimal (divergen)\cite{Sato:1995}.

\noindent Dalam penelitian ini, sistem iterasi yang akan digunakan adalah
mengikuti cara yang kedua diatas.
% 
% 
\subsection{Iterasi dengan data Round Robin}
\label{ssec:round-robin}
Melihat karakteristik dari proses pembelajaran LVQ dimana aturan update pada
proses pembelajaran dilakukan secara sequensial, maka perilaku proses
pembelajaran akan dimodifikasi dari sisi urutan data training. Pada iterasi metode sebelumnya,
proses iterasi pembelajaran dalam satu epoch dilakukan sesuai  dengan urutan
data masukan yang diberikan, bisa berupa urutan sejumlah N sampel data kategori
1 diikuti dengan  M sampel data kategori 2 dan seterusnya, atau bisa juga berupa
urutan data yang diambil secara random terhadap kategori data. Pada penelitian ini, \saya
mencoba untuk menerapkan mekanisme round-robin dalam iterasi proses pembelajaran
LVQ dimana dalam mekanisme ini, urutan data sampel yang diberikan pada proses
pembelajaran jaringan saraf dipastikan selang seling untuk setiap kategori data
sejumlah $k$ kategori. Jadi urutan data sampel akan menjadi; $X_{1,c_1},
X_{1,c_2}, \dots, X_{1,c_k}, X_{2,c_1}, X_{2,c_2}, \dots, X_{2,c_k},
X_{n_1,c_1},X_{n_2,c_2}, \dots,X_{n_3,c_k}$ dimana $n_i$ adalah jumlah data
untuk kelas ke-$i$.

Dengan menggunakan mekanisme round-robin diharapkan dapat memperbaiki proses
pembelajaran secara keseluruhan.


% . Berikut pada algoritma \ref{alg:opsi3} dapat dilihat pseudocode dari
% mekanisme iterasi round-robin ini.
% 
% % \begin{algorithm}
% % \scriptsize
% % \caption{Mekanisme iterasi secara round robin}
% % \label{alg:opsi3}
% % \begin{algorithmic}                    % enter the algorithmic environment
% % 	\STATE \ldots
% % 	\STATE $max\_N \leftarrow findMaxNumOfDataInCategory()$
% % 	\FOR {$i=1 \to max\_N$}
% % 		\FORALL {$C$ in $Category$}
% % 			\IF {$i > nC$} continue \ENDIF
% % 
% % 			\STATE $sample \leftarrow X_{c,i}$
% % 			\STATE $train(codebook, sample)$
% % 			\STATE \ldots
% % 		\ENDFOR
% % 	\ENDFOR
% % 	\STATE \ldots
% % \end{algorithmic}
% % \end{algorithm}

%-----------------------------------------------------------------------------%
\section{Metode Fuzzy Neuro GLVQ}
%-----------------------------------------------------------------------------%
Metode Fuzzy-Neuro GLVQ yang dikembangkan merupakan metode yang
mengadaptasi mekanisme fuzzifikasi dari Fuzzy-Neuro LVQ dengan metode
Generalized LVQ, dengan harapan dapat meningkatkan tingkat akurasi pengenalan
kelainan arrhytmia dimana sebaran dari tiap kelas kategori arrhytmia overlapping
satu sama lain. Berikut akan diuraikan mengenai metode FNGLVQ.

% dimana hal ini  dimotivasi oleh Kelemahan dari FNLVQ yang
% sensitif terhadap inisialisasi  bobot awal dan keunggulan dari GLVQ yang
% menjamin konvergensi dari prototipe  selama proses pelatihan, dan juga tidak
% sensitif terhadap inisialisasi bobot awal. Sedangkan keunggulan dari FNLVQ
% adalah memiliki kemampuan untuk mengenali data \emph{unknown}.  Dengan
% penggabungan 2 konsep ini diharapkan dihasilkan metode yang tidak sensitif
% terhadap inisialisasi data awal dan juga memiliki kemampuan untuk mengenali
% \emph{unknown} data. 

\subsection{Konsep dasar}
Perbedaan yang mendasar dari karakteristik pengenalan arrhytmia berdasarkan beat
dengan pengenalan aroma adalah dari sisi data yang akan menjadi masukan
jaringan LVQ. Pada pengenalan aroma, data masukan merupakan data himpunan
fuzzy, dimana merupakan representasi dari ketidakpastian (\emph{fuzziness})
sensor dalam membaca informasi aroma, dimana hal ini berimplikasi pada model
dari vektor pewakil yang diimplementasikan juga dengan himpunan fuzzy. Sedangkan
pada pengenalan arrhytmia, distribusi data masing-masing kategori saling tumpang
tindih (\emph{overlaping}) satu sama lain, seperti yang dapat ditunjukkan pada
\pic~\ref{fig:overlap} sehingga diharapkan dengan menggunakan vektor pewakil
fuzzy, maka ketidakpastian pola suatu kelas dapat diturunkan. Selain itu, kunci
utama proses pengenalan \emph{unknown} data pada \gls{fnlvq} adalah terletak
pada model vektor pewakil dengan menggunakan fuzzy, dimana jika nilai
similarity antara vektor masukan dengan vektor pewakil adalah 0 (nol), maka
vektor masukkan tersebut polanya belum pernah diketahui oleh jaringan saraf
(\emph{unknown}).

\addFigure{width=0.6\textwidth}{pics/plotoverlap.png}{fig:overlap}{Ilustrasi
data arrhytmia yang tumpang tindih (\emph{overlap}) antar kategori.}

Pada pengenalan arrhytmia yang dilakukan pada penelitian ini, yang dikenali
adalah beat dalam ECG, dimana kemunculan suatu kelainan beat ditandai dengan
morfologinya. Menurut dr. Jolanda Jonas, tidak semua kemunculan kelainan beat
menandakan seorang pasien menderita arrhytmia. Terdapat kelainan beat yang hanya muncul
sekali dalam data ECG dimana hal ini sering digunakan sebagai indikasi untuk
pemeriksaan lanjutan, seperti pemasangan alat observasi ECG 24 jam
(\emph{Holter ECG}). Sehingga himpunan fuzzy sebagai masukan
sistem, seperti pada sistem pengenalan aroma, tidak cocok untuk digunakan. Oleh
karena itu pada pengenalan arrhytmia ini, digunakan masukan berupa data
\emph{crisp}.
 
Seperti yang sudah diuraikan pada sub-bab \ref{ssec:glvq}, fungsi
diskriminan yang digunakan \gls{glvq} adalah menggunakan \emph{\gls{metric}}, 
yakni jarak euclidean, sehingga semakin kecil jarak antara input dengan vektor
pewakil, maka kedua vektor akan semakin mirip. Pada metode yang dikembangkan,
fungsi diskriminan yang digunakan adalah dengan menggunakan \emph{fuzzy similarity},
seperti yang digunakan \gls{fnlvq} dalam aplikasi pengenalan aroma.

\addFigure{width=0.6\textwidth}{pics/fmtrim.png}{fig:fmtrim}{Ilustrasi
Perhitungan similarity crisp data dengan menggunakan fungsi keanggotaan
segitiga}

Pada fuzzy similarity, perhitungan kemiripan dilakukan dengan mencari derajat
keanggotaan setiap fitur ($x_i$) terhadap fungsi keanggotaannya ($h_{ij}(x)$), 
dengan $i=$ fitur ke-$i$ dan $j=$ vektor pewakil kategori ke-$j$ seperti yang
diilustrasikan pada \pic~\ref{fig:fmtrim}
\begin{align}
	\mu_{ij} = h_{ij}(x_i)
\end{align}

Kemudian nilai derajat keanggotaan ($\mu_{j}$) vektor pewakil
(\emph{cluster}) dipropagasi ke neuron keluaran dengan menggunakan operasi
rata-rata (\emph{mean}).

\begin{align}
	\mu_j = \text{mean}_{\substack{j}} [\mu_{ij}]
\end{align}

Untuk menentukan pemenang (\emph{winner-take-all}), dipilih vektor
pewakil dengan nilai similarity ($\mu_j$) terbesar (\emph{max}).  
\begin{align}
	w_p = \max_j ( \mu_j )
\end{align}

dimana vektor pewakil pemenang akan di-update selama proses pembelajaran 
tergantung dari vektor masukan yang diberikan. Namun pada metode ini, vektor
pewakil yang akan di-update tidak hanya berdasarkan vektor pemenang saja,
melainkan ditentukan oleh \emph{missclassification error} dengan menghitung
jarak relatif antara jarak vektor masukkan($x$) dengan vektor pewakil dari
kelas yang sama ($C_x = C_w$) dan jarak terbesar vektor masukan dengan vektor
pewakil yang tidak berasal dari kelas yang sama ($C_x \neq C_{\max_{j}(w_j)}$).
Lebih jelas dapat dilihat kembali pada sub-bab \ref{ssec:glvq}.

\subsection{Metode FN-GLVQ}
Pada fuzzy similarity, semakin besar nilainya, maka kedua vektor akan semakin
mirip. Untuk dapat menggunakan konsep similarity pada \gls{glvq}, maka dicari
nilai \emph{disimilarity}  dengan menggunakan persamaan $d = 1 - \mu$, dimana
$d$ adalah nilai jarak (\emph{disimilarity}). Kemudian disubstitusikan ke
\equ~\ref{eq:mce} didapat;

\begin{align}
\label{eq:}
	\varphi(x) &= \frac{(1 - \mu_1) - (1 - \mu_2)}{(1 - \mu_1) + (1 -
	\mu_2)}\nonumber\\
	&= \frac{\mu_2 - \mu_1}{2 - \mu_1 - \mu_2}
\end{align}

dengan $\varphi(x)$ adalah nilai \emph{miss-classification error (MCE)},
$\mu_1$ adalah nilai similarity vektor masukkan($x$) dengan vektor pewakil dari
kelas yang sama ($C_x = C_w$), dan $\mu_2$ adalah nilai similarity
terbesar antara vektor masukan dengan vektor pewakil yang tidak berasal dari
kelas yang sama ($C_x \neq C_{\max_{j}(w_j)}$).  Untuk dapat mengintegrasikan
teori fuzzy dengan \gls{glvq}, maka akan dilakukan penurunan  cost function
terhadap bobot $w$ sebagai berikut; (dengan mengacu pada cost function
\equ~\ref{eq:costS} dan update rule pada \equ~\ref{eq:genuprule})

\begin{align}
\label{eq:turunancostS}
	\frac{\delta S}{\delta w_i} =  
	\frac{\delta S}{\delta \varphi} . \frac{\delta \varphi}{\delta \mu}.
	\frac{\delta \mu}{\delta w_i}
\end{align}

% \begin{align}
% \label{eq:}
% 	\Psi(x) &= \frac{f(x)}{g(x)} \nonumber \\
% 	\frac{\delta \Psi}{\delta x} &=  
% 	\frac{f'(x)g(x) - f(x)g'(g)}{g(x)^2}
% \end{align}

\noindent Berikut adalah turunan dari masing-masing bagian turunan berantai pada
\equ~\ref{eq:turunancostS} adalah sebagai berikut; 
\begin{align}
\label{eq:turunanmce1}
	\frac{\delta \varphi}{\delta \mu_1} &= 
	\frac{\delta \left(\frac{\mu_2-\mu_1}{2-\mu_1-\mu_2}\right)}{\delta \mu_1} 
	\nonumber \\ \frac{\delta \varphi}{\delta \mu_1} &=  
	\frac{-1 . (2 - \mu_1 - \mu_2) - (-1).(\mu_2-\mu_1)}
	{(2 - \mu_1 - \mu_2)^2} \nonumber \\
	\frac{\delta \varphi}{\delta \mu_1} &=  
	-2.\frac{(1 - \mu_2)}{(2 - \mu_1 - \mu_2)^2}
\end{align}

\begin{align}
\label{eq:turunanmce2}
	\frac{\delta \varphi}{\delta \mu_2} &= 
	\frac{\delta \left(\frac{\mu_2-\mu_1}{2-\mu_1-\mu_2}\right)}{\delta \mu_2} 
	\nonumber \\ \frac{\delta \varphi}{\delta \mu_2} &=  
	\frac{1 . (2 - \mu_1 - \mu_2) - (-1).(\mu_2-\mu_1)}
	{(2 - \mu_1 - \mu_2)^2} \nonumber \\
	\frac{\delta \varphi}{\delta \mu_2} &=  
	2.\frac{(1 - \mu_1)}{(2 - \mu_1 - \mu_2)^2}
\end{align}

$\frac{\delta \varphi}{\delta \mu_1}$ dan $\frac{\delta \varphi}{\delta
\mu_2}$ merupakan turunan MCE terhadap nilai similarity $\mu_1$ dan $\mu_2$.
Untuk mencari $\frac{\delta \mu}{\delta w_i}$, maka perhitungannya tergantung
dari fungsi keanggotaan yang digunakan pada setiap vektor pewakil. Pada
penelitian ini, fungsi keanggotaan $h(x)$ yang dipakai adalah fungsi segitiga,
seperti yang juga digunakan pada algoritma \gls{fnlvq}, karena fungsi segitiga adalah
fungsi yang paling sederhana untuk diimplementasikan, dan untuk mendapatkan
parameter-nya hanya membutuhkan nilai minimum, rata-rata dan maksimum yang
dihitung dari sebaran data pelatihan. Oleh karena itu, elemen vektor pewakil
pada algoritma ini akan direpresentasikan sebagai berikut;

\begin{align}
\label{eq:trimbobot}
	w_{ij} = (w_{min,ij}, w_{mean,ij}, w_{max,ij})
\end{align}

\noindent dimana $w_{ij}$ adalah vektor pewakil untuk fitur ke-$i$ dengan
kategori $j$, dan $w_{min,ij}, w_{mean,ij}, w_{max,ij}$ secara
berturut-turut nilai minimum, rata-rata dan maksimum dari distribusi data sampel
fitur ke-$i$ dengan kategori $j$. Untuk lebih menyederhanakan notasi, akan
digunakan $w_{min}, w_{mean}, w_{max}$ untuk mewakili notasi diatas.

\noindent Jika fungsi keanggotaan segitiga didefinisikan sebagai;
\begin{align}
\label{eq:trim}
	\mu = h(x, w_{min}, w_{mean}, w_{max}) = \left\{ 
	\begin{array}{ll}
	0 & , x \leq w_{min}\\
	\frac{x - w_{min}}{w_{mean} - w_{min}} & , w_{min} < x \leq w_{mean} \\
	\frac{w_{max} - x}{w_{max} - w_{mean}} & , w_{mean} < x < w_{max} \\
	0 & , x \geq w_{max}
	\end{array}
\end{align}

\noindent maka turunan dari fungsi segitiga diatas, dalam hal ini akan
diturunkan terhadap nilai rata-rata bobot ($w_{mean}$), didapat sebagai berikut;
\begin{itemize}
  \item Untuk nilai $x$ dengan kondisi $w_{min} < x \leq w_{mean}$ :
  \begin{align}
	\label{eq:trim1}
		\mu &= \frac{x - w_{min}}{w_{mean} - w_{min}} \nonumber \\
			&= (x - w_{min}) . (w_{mean} - w_{min})^{-1} \nonumber \\
		\frac{\delta \mu}{\delta w_{mean}} &=
		(x - w_{min}).(-1).(w_{mean} - w_{min})^{-2}.(1) \nonumber \\
		 &=
		- \frac{x - w_{min}}{(w_{mean} - w_{min})^2}
	\end{align}

	\item Untuk nilai $x$ dengan kondisi  $w_{mean} < x < w_{max}$ :
	\begin{align}
	\label{eq:trim2}
		\mu &= \frac{w_{max} - x}{w_{max} - w_{mean}} \nonumber \\
			&= (w_{max} - x) . (w_{max} - w_{mean})^{-1} \nonumber \\
		\frac{\delta \mu}{\delta w_{mean}} &=
		(w_{max} - x).(-1).(w_{max} - w_{mean})^{-2}.(-1) \nonumber \\
		 &=
		+ \frac{w_{max} - x}{(w_{max} - w_{mean})^2}
	\end{align}
	
	\item Untuk nilai $x$ dengan kondisi $x <= w_{min}\ \text{AND}\ x >= w_{max}$ :
	\begin{align}
	\label{eq:trim3}
		\mu &= 0 \nonumber \\
		\frac{\delta \mu}{\delta w_{mean}} &= 0
	\end{align}
\end{itemize}

Dari \equ~\ref{eq:trim1}, \ref{eq:trim2} dan \ref{eq:trim3} kemudian
disubstitusikan ke \equ~\ref{eq:genuprule} didapat aturan update pada proses
pembelajaran sebagai berikut;
\begin{itemize}
  \item Untuk nilai $x$ dengan kondisi $w_{min} < x \leq w_{mean}$ :
  	\begin{align}
	\label{eq:uprule11}
		w_1(t+1) \leftarrow & w_1(t) - \alpha .  
		\frac{\delta f}{\delta \varphi} . \frac{2.(1 - \mu_2)}{(2 - \mu_1 - \mu_2)^2}.
		\Bigg(\frac{x - w_{min}}{(w_{mean} - w_{min})^2}\Bigg) \\
	\label{eq:uprule12}
		w_2(t+1) \leftarrow & w_2(t) + \alpha .  
		\frac{\delta f}{\delta \varphi} . \frac{2.(1 - \mu_1)}{(2 - \mu_1 - \mu_2)^2}.
		\Bigg(\frac{x - w_{min}}{(w_{mean} - w_{min})^2}\Bigg) 
	\end{align}
  \item Untuk nilai $x$ dengan kondisi  $w_{mean} < x < w_{max}$ :
	\begin{align}
	\label{eq:uprule21}
		w_1 \leftarrow & w_1 + \alpha .  
		\frac{\delta f}{\delta \varphi} . \frac{2.(1 - \mu_2)}{(2 - \mu_1 - \mu_2)^2}.
		\Bigg(\frac{w_{max} - x}{(w_{max} - w_{mean})^2}\Bigg) \\
	\label{eq:uprule22}
		w_2 \leftarrow & w_2 - \alpha .  
		\frac{\delta f}{\delta \varphi} . \frac{2.(1 - \mu_1)}{(2 - \mu_1 - \mu_2)^2}.
		\Bigg(\frac{w_{max} - x}{(w_{max} - w_{mean})^2})\Bigg)
	\end{align}  
  \item Untuk nilai $x$ dengan kondisi  $x \leq w_{min}$ dan $x \geq w_{max}$ :
  	\begin{align}
	\label{eq:}
		w_i(t+1) \leftarrow & w_i(t) \qquad, i=1, 2     
	\end{align}  
\end{itemize}

\noindent dengan $w_1$ adalah vektor pewakil dari kelas yang sama dengan vektor
masukkan $x$ ($C_x = C_w$), dan $w_2$ adalah vektor pewakil dari kelas yang
berbeda dengan vektor masukkan dengan nilai similarity terbesar ($C_x \neq
C_{\max_{j}(w_j)}$). Proses update pada persamaan diatas (\ref{eq:uprule11},
\ref{eq:uprule12},\ref{eq:uprule21},\ref{eq:uprule22}) dilakukan pada $w_{mean}$
sedangkan $w_{min}, w_{max}$ mengikuti pergeseran dari $w_{mean}$. 
\begin{align}
\label{eq:upruleminmax}
	w_{min} \leftarrow & w_{mean}(t+1) - (w_{mean}(t) - w_{min}(t)) \\
	w_{max} \leftarrow & w_{mean}(t+1) + (w_{max}(t) - w_{mean}(t)) \nonumber \\
\end{align} 

Fungsi Monoton naik yang dipakai pada algoritma ini akan tetap sama dengan yang
dipakai pada \glvq standar, yakni menggunakan fungsi sigmoid, sehingga
$\frac{\delta \varphi}{\delta \mu}$ akan sama seperti pada
\equ~\ref{eq:deltasigmoid}. Sedangkan nilai laju pembelajaran $\alpha$ yang
digunakan adalah berkisar [0, 1] dan menurun seiring bertambahnya iterasi proses
pembelajaran. 
\begin{align}
\label{eq:alpha}
	\alpha(t+1) = \alpha_0 \times (1 - \frac{t}{t_{max}})
\end{align}

Sebagai bagian dari karakteristik algoritma \gls{glvq} dimana terlepas dari
apakah jaringan saraf benar mengenali vektor masukan maupun tidak, vektor
pewakil $w_1, w_2$ keduanya secara simultan akan di-update. Namun pada algoritma
ini, selain melakukan penyesuaian $w_1$ dan $w_2$, dilakukan penyesuaian
tambahan seperti yang dilakukan pada \gls{fnlvq} yaitu proses penyesuaian
lebar interval dari fungsi keanggotaan tiap vektor pewakil. 
\begin{enumerate}
  \setlength{\itemsep}{1pt}
  \setlength{\parskip}{0pt}
  \setlength{\parsep}{0pt}
  \item Jika jaringan bisa mengenali dengan benar vektor masukan yang diberikan,
  maka fungsi keanggotaan akan diperlebar dengan harapan tingkat pengenalan-nya
  meningkat. 
  \item Sebaliknya jika jaringan salah mengenali vektor masukan, maka fungsi
  keanggotaan akan dipersempit dengan harapan tingkat pengenalan terhadap 
  vektor masukan menurun. 
\end{enumerate}

\noindent Kedua langkah diatas hanya akan dilakukan jika nilai $\mu_1 > 0$ atau
$\mu_2 > 0$. Jika  kedua nilai similarity adalah 0, $\mu_1=0$ dan $\mu_2=0$, 
maka hal ini berarti semua vektor pewakil sama sekali tidak mengenali vektor 
masukan yang diberikan. Terdapat 2 kemungkinan; 
\begin{enumerate}
  \setlength{\itemsep}{1pt}
  \setlength{\parskip}{0pt}
  \setlength{\parsep}{0pt}
  \item Vektor masukan memang berada diluar distribusi dari kategori yang
  dikenali, 
  \item Interval dari fungsi keanggotaan (fuzzy) dari vektor pewakil
  terlalu sempit, sehingga tingkat pengenalan-nya rendah. 
\end{enumerate}

\noindent Karena ini merupakan proses pelatihan, maka asumsi adalah
yang ke-2, sehingga untuk membuat jaringan mengenali vektor pewakil, semua
interval dari fungsi keanggotaan vektor pewakil diperlebar. Berikut adalah
aturan perlebaran/penyempitan fungsi keanggotaan vektor pewakil yang telah
dijelaskan pada uraian diatas;
\begin{itemize}
  \setlength{\itemsep}{1pt}
  \setlength{\parskip}{0pt}
  \setlength{\parsep}{0pt}
  \item Jika  $\mu_1 > 0$ atau $\mu_2 > 0$, minimal salah satu dari kedua
  vektor pewakil mengenali vektor masukan:
  \begin{itemize}
  \setlength{\itemsep}{1pt}
  \setlength{\parskip}{0pt}
  \setlength{\parsep}{0pt}
    \item Jika pengenalan-nya benar ($\varphi < 0$), maka interval
    ketidakpastian (\emph{fuzziness}) dari vektor pewakil diperlebar
    	\begin{align}
    	\label{eq:increasefuzzy}
    	w_{min} &\leftarrow w_{mean} - (w_{mean} - w_{min}) . (1 + (\beta .
    	\alpha)) \nonumber \\
    	w_{max} &\leftarrow w_{mean} + (w_{max} - w_{mean}) . (1 + (\beta .
    	\alpha)) 
    	\end{align}
    \item Jika pengenalan-nya salah ($\varphi \geq 0$), maka interval
    ketidakpastian (\emph{fuzziness}) dari vektor pewakil dipersempit
    	\begin{align}
    	\label{eq:decreasefuzzy}
    	w_{min} &\leftarrow w_{mean} - (w_{mean} - w_{min}) . (1 - (\beta .
    	\alpha)) \nonumber \\
    	w_{max} &\leftarrow w_{mean} + (w_{max} - w_{mean}) . (1 - (\beta .
    	\alpha)) 
    	\end{align}
  \end{itemize}
  
  Disini, nilai $\beta$ adalah diantara [0,1]. Pada studi kasus yang dilakukan
  disini, dipilih nilai $\beta = 0.00005$.
  
  \item Jika  $\mu_1=0$ dan $\mu_2=0$, yang artinya kedua vektor pewakil
  tidak mengenali vektor masukan, maka semua fungsi keanggotaan pada semua
  vektor pewakil pada jaringan diperlebar dengan aturan sebagai berikut:
  	\begin{align}
  	\label{eq:increaseall}
	w_{min} &\leftarrow w_{mean} - (w_{mean} - w_{min}) . (1 - (\alpha .
	\gamma)) \nonumber \\ 
	w_{max} &\leftarrow w_{mean} + (w_{max} - w_{mean})
	. (1 + (\alpha . \gamma)) 
	\end{align}
	
	Disini, nilai $\gamma$ adalah diantara [0,1]. Pada studi kasus yang dilakukan
	disini, dipilih nilai $\gamma = 0.1$.
\end{itemize}
  
\clearpage

\noindent Berikut ini adalah algoritma FNGLVQ secara keseluruhan dalam bentuk
pseudocode;

\begin{algorithm}  
\scriptsize 
\caption{Algoritma FNGLVQ}          
\label{alg:fnglvq}                           
\begin{algorithmic}                    % enter the algorithmic environment
	\REQUIRE $X$ in round-robin mode
	\STATE Initialize weight vector $W$
	\STATE Initialize learning rate $\alpha_0$
	\STATE Initialize maximum iteration $t_{max}$
	\STATE $t \leftarrow 0$
	\WHILE {$\alpha_t \neq 0$ or $t < t_{max}$}
		\FOR{$x_i \in X$}
% 			\STATE $x \leftarrow $ getNextSampleData()
			\STATE $\mu_1 \leftarrow $ findSimilarity1($x, W$) where $C_x = C_{w_i}$
			\STATE $\mu_2 \leftarrow $ findSimilarity2($x, W$) where $\mu_j = \max_{j
			\wedge j \neq i}(\mu_j) \wedge (C_x \neq C_{w_j})$
			\IF {$\mu_1 > 0$ OR $\mu_2 > 0$}
				\STATE $mce \leftarrow $ equation \ref{eq:mce}
				\STATE $factor1 \leftarrow $ equation \ref{eq:turunanmce1}
				\STATE $factor2 \leftarrow $ equation \ref{eq:turunanmce2}
				\STATE
				\STATE \COMMENT{adjust prototype 1}
				\IF {$ w_{1, min} < x \leq w_{1,mean}$}
					\STATE $w_{1,mean} \leftarrow $ equation \ref{eq:uprule11}
				\ELSIF {$ w_{1, mean} < x \leq w_{1,max}$}
					\STATE $w_{1,mean} \leftarrow $ equation \ref{eq:uprule21}
				\ENDIF
	
				\STATE $w_{1,min} \leftarrow $ equation \ref{eq:upruleminmax}
				\STATE $w_{1,max} \leftarrow $ equation \ref{eq:upruleminmax}
				\STATE
				\STATE \COMMENT{adjust prototype 2}
				\IF {$ w_{2, min} < x \leq w_{2,mean}$}
					\STATE $w_{2,mean} \leftarrow $ equation \ref{eq:uprule12}
				\ELSIF {$ w_{2, mean} < x \leq w_{2,max}$}
					\STATE $w_{2,mean} \leftarrow $ equation \ref{eq:uprule22}
				\ENDIF
	
				\STATE $w_{2,min} \leftarrow $ equation \ref{eq:upruleminmax}
				\STATE $w_{2,max} \leftarrow $ equation \ref{eq:upruleminmax}
	
				\STATE
				\IF {$mce < 0$}
					\STATE \COMMENT{Increase Fuzziness}
					\STATE $w_1 \leftarrow$ equation \ref{eq:increasefuzzy}
				\ELSE
					\STATE \COMMENT{Decrease Fuzziness}
					\STATE $w_2 \leftarrow$  equation \ref{eq:decreasefuzzy}
				\ENDIF
			\ELSE
				\STATE \COMMENT{Increase Fuzziness for all vector ref.}
				\FOR{$w_j \in W$}
					\STATE $w_j \leftarrow $ equation \ref{eq:increaseall}
				\ENDFOR
			\ENDIF
		\ENDFOR
	\ENDWHILE
	\STATE $\alpha \leftarrow $ equation \ref{eq:alpha} 
\end{algorithmic}
\end{algorithm}  

\clearpage
%-----------------------------------------------------------------------------%
\section{Implementasi Sistem}
%-----------------------------------------------------------------------------%
Secara garis besar, sistem pengenalan arrhytmia yang dibangun dapat dilihat
melalui diagram pada \pic~\ref{fig:diagramAll}. 

\addFigure{width=1\textwidth}{pics/block-diagram.png}{fig:diagramAll}{Blok
diagram sistem pengenal arrhytmia yang dikembangkan.}

Seperti yang ditunjukkan pada \pic~\ref{fig:diagramAll}, pengembangan sistem
terdiri dari empat tahap diantaranya;
\begin{enumerate}
  \item \emph{Data acquisition} : yaitu tahap perekaman data ECG. Pada tahap
  ini untuk sementara menggunakan data digital yang disediakan secara bebas di
  physionet.net khususnya data MIT-BIH arrhytmia database. Kedepannya, tahap ini dapat
  disubstitusi dengan penelitian yang telah dilakukan oleh peneliti lain dalam tim Arrhytmia.
  \item \emph{Data preprocessing} : yaitu tahap pengolahan data awal. Pada tahap
  ini, data dasar diolah sedemikian rupa sehingga dapat merepresentasikan data
  ECG dengan baik agar dapat meningkatkan kinerja proses pengenalan. Untuk
  sementara proses pengolahan masih menggunakan paket aplikasi MATLAB v2009a.
  Kedepannya, proses pengolahan agar dapat diimplementasikan di lingkungan java.
  \item \emph{Pattern Classification} : yaitu tahap pengembangan metode
  pengenalan. Pada tahap ini, dikembangkan metode baru FN-GLVQ seperti yang
  telah diuraikan pada bagian awal bab ini. Semua Metode yang dikembangkan telah
  di implementasikan di lingkungan java dan siap untuk digunakan dan
  dikembangkan lebih lanjut.
  \item \emph{Visualisation} : yaitu tahap visualisasi hasil proses pengenalan.
  Tahap ini sebenarnya tidak terlalu berbeda dengan tahap ke-3, hanya saja tahap
  ini diperlukan untuk mempermudah penggunaan aplikasi. Untuk sementara,
  visualisasi yang dikembangkan merupakan antar muka sederhana, dimana
  kedepannya dapat disempurnakan.
\end{enumerate}

Pada diagram \ref{fig:diagramAll}, langkah pengolahan data ECG paper tidak
dilakukan karena sementara data yang dibutuhkan sudah memadai, dan usaha untuk
melakukan langkah tersebut tidak-lah mudah, sehingga tidak  mendesak untuk
dikembangkan.

\subsection{Lingkungan pengembangan}
Seperti yang telah diuraikan sebelumnya, penelitian ini difokuskan untuk
menggali representasi data ECG yang memadai serta mengembangkan metode
pengenalan ECG arrhytmia (\emph{Data preprocessing} dan \emph{Pattern
Classification}). 
\begin{enumerate}
  \item Untuk tahap pengolahan data, dilakukan dengan menggunakan aplikasi
  Matlab v2009a. Beberapa tahap pengolahan menggunakan kode yang
  dikembangkan sendiri seperti Beat Extraction, Outlier Removal dan
  Feature Extraction. Sedangkan untuk tahap baseline wander removal (BWR)
  menggunakan kode yang disediakan oleh Gari Clifford \cite{clifford-2005} yang
  bersifat open source.
  \item Untuk tahap pengembangan sistem pengenal arrhytmia, difokuskan pada
  pengembangan engine tahap awal dimana tujuannya adalah agar lebih mudah untuk
  digunakan dan dikembangkan lebih lanjut (\emph{Usability} dan
  \emph{Modifiability}). Pengembangan engine dilakukan di lingkungan :
  \begin{itemize}
    \item Menggunakan Bahasa pemrograman Java dengan JDK v1.6.0\_02
    \item Dikembangkan dengan menggunakan Eclipse IDE for Java Developers
    versi: Helios SR2 dengan plugin tambahan Subclipse untuk mendukung
    Versioning System.
    \item Library yang digunakan diantaranya;
    \begin{itemize}
      \item JFreeChart-1.0.13 untuk dukungan charting, 
      \item common-lang-2.6, common-math-2.2 dari apache project untuk dukungan
      fungsi-fungsi IO dan math.
	\end{itemize}
  \end{itemize}
\end{enumerate}
Seperti disebutkan diatas, pengembangan dilakukan dengan menggunakan eclipse +
subversion agar pengembangan engine nantinya dapat dikembangkan oleh tim
peneliti yang lain. Oleh karena itu, untuk sementara kode sumber di-hosting di
\url{http://code.google.com/p/mdgs-thesis/source/browse/#svn\ trunk\
mdgs-thesis\ trunk-code} dengan Code license GNU Lesser GPL.

\subsection{Arsitektur}
Untuk mendukung kemudahan penggunaan dan pengembangan lebih lanjut, Engine
dirancang seperti yang ditunjukkan pada \pic~\ref{fig:packaged}.

\addFigure{width=0.8\textwidth}{pics/package-diagram.png}{fig:packaged}{Package
diagram dari engine yang dikembangkan}.

\noindent Seperti yang terlihat pada package diagram \pic~\ref{fig:packaged},
berikut adalah uraian dari masing-masing package
\begin{itemize}
  \item \textbf{master} : merupakan package untuk \texttt{interface} class
  dimana dapat digunakan untuk mengembangkan metode klasifikasi yang lain.
  \item \textbf{dataset} : merupakan package untuk menangani data yang akan
  digunakan untuk proses training maupun testing. Disini ada dua jenis
  \texttt{class} yang diimplementasikan untuk mengakomodasi jenis data Crisp
  (\texttt{Dataset}) dan data Fuzzy (\texttt{FCodeBook}), khususnya Fuzzy dengan
  membership function segitiga. \\ 
  Untuk memudahkan pengelolaan data, serta mengimplementasikan metode
  Round-Robin yang diuraikan pada sub-bab \ref{ssec:round-robin},
  diimplementasikan dengan konsep folding, yaitu pada class
  \texttt{FoldedDataset} dan untuk tujuan Cross Validation, diimplementasikan
  class \texttt{KFoldedDataset}. Disamping itu diimplementasikan juga mekanisme
  untuk normalisasi data (\texttt{DataNormalization}  dan
  \texttt{ZScoreNormalization}).
  \item \textbf{lvq, glvq} dan \textbf{fn-glvq} : merupakan package yang
  terdiri dari metode pengenalan lvq, glvq dan metode baru yang dikembangkan
  yaitu fn-glvq.
  \item \textbf{evaluation} : yaitu package yang digunakan untuk melakukan uji
  test statistik yang umum dilakukan dalam pengembangan suatu metode klasifikasi.
  \item \textbf{gui} : merupakan package yang digunakan untuk memonitor proses
  pelatihan yang dilakukan dimana dapat menampilkan perkembangan vektor pewakil
  tiap iterasi dan tingkat error yang dihasilkan.
  \item \textbf{utils} : merupakan package utility yang digunakan disebagian
  beasar paket engine yang dikembangkan.
\end{itemize} 

Untuk lebih jelasnya, pada \pic~\ref{fig:classdiag} ditunjukkan class diagam
dari engine yang dikembangkan.

\clearpage

\addFigure{width=1\textwidth}{pics/class-diagram.png}{fig:classdiag}{Class
diagram dari engine yang dikembangkan.}


Berikut ditunjukkan antarmuka sistem yang dapat digunakan untuk menunjukkan
kerja dari engine yang dikembangkan, pada \pic~\ref{fig:mainform}
\addFigure{width=0.8\textwidth}{pics/mainform.png}{fig:mainform}{Contoh Antar
muka yang dikembangkan.}

Seperti yang disebutkan diatas, berikut pada \pic~\ref{fig:codemonitor} adalah
tampilan antar muka untuk melakukan pengawasan perkembangan proses pembelajaran,
terutama untuk melihat perkembangan perubahan vektor pewakil untuk setiap  kali
proses iterasi epoch.
\addFigure{width=1\textwidth}{pics/codemonitor.png}{fig:codemonitor}{Form
Codebook Monitor yang digunakan untuk memonitor perkembangan proses
pembelajaran.}


